\section{Trees}

\subsection{Cut Edges}

\subsubsection*{Cut-edge and Bridge Definition}
An edge $e$ of graph $G$ is a \bld{cut-edge}, or \bld{bridge}, if $G-e$ has more components than $G$.

\subsubsection*{Cut-edges and Cycles Theorem}
An edge $e$ of a graph $G$ is a cut-edge if and only if $e$ lies on \itl{no} cycle in $G$.

\subsection{Trees}

\subsubsection*{Tree and Forest Definitions}
A \bld{tree} is an acyclic connected graph. A \bld{forest} is an acyclic graph, where each component is a \itl{tree}. A \bld{Rooted tree} is a tree with a specific vertex designated as a root and drawn down.

Every edge of a tree is a cut-edge.

\subsubsection*{Unique Path in Trees Theorem}
Graph $G$ is a tree if and only if every 2 vertices are connected by a unique path.

\subsubsection*{Leaf Theorem}
Every nontrivial tree has at least 2 leaves.

\subsubsection*{Autumn Theorem}
If tree $T$ has order $t \geq 1$, then $T-v$, where $v$ is a leaf, is a tree of order $t-1$.

\subsubsection*{Tree Size Theorem}
Every tree of order $n$ has size $n-1$.

\subsubsection*{Forest Size Theorem}
Every forest of order $n$ with $k$ components has size $n-k$.

\subsubsection*{Minimum Size of a Connected Graph Theorem}
The size of every connected graph of order $n$ is at least $n-1$. Trees has minimal size among connected graphs of given order.

\subsubsection*{Tree Requirements Graph}
Graph $G$ of order $n$ and size $m$. Then $G$ is a tree if it satisfies any 2 of these properties:
\begin{enumerate}
    \item $G$ is connected
    \item $G$ acyclic
    \item $m = n-1$
\end{enumerate}

\subsubsection*{Tree Isomorphic Subgraph Theorem}
Let $T$ be a tree of order $k$. Then for any graph $G$ with $\delta(G) \geq k-1$, $T$ is isomorphic to a subgraph of $G$.

\subsection{Minimum Spanning Tree}

\subsubsection*{Spanning Tree Definition}
Let $G$ be a connected graph. A spanning subgraph of $G$ that is a tree is called a \bld{spanning tree}.

\subsubsection*{Spanning Tree Existence Theorem}
Every connected graph contains a spanning tree.

\subsubsection*{Minimum Spanning Tree Definition}
A \bld{minimum spanning tree} is a spanning tree of minimum weight.

\subsection*{Algorithms For Constructing Minimum Spanning Trees}

\subsubsection*{Kruskal's Algorithm}
\begin{enumerate}
    \item Pick an edge of minimum weight.
    \item Repeat, never allowing the chosen edges to produce a cycle.
    \item Stop once you have a spanning tree.
\end{enumerate}

\subsubsection*{Prim's Algorithm}
\begin{enumerate}
    \item Choose any vertex $u \in V(G)$.
    \item Let $e$ be an edge of minimum weight incident with $u$.
    \item Continue picking edges of minimum weight weight from the set of edges having exactly one of its vertices incident with an already selected edge.
    \item Stop once you have a spanning tree.
\end{enumerate}

\subsection{Counting Labeled Trees}

\subsubsection*{Cayley's Theorem}
There are $n^{n-2}$ distinct labeled trees on $n$ vertices.

\subsubsection*{Prüfer Sequence}
Encoding a Tree to a Sequence
\begin{enumerate}
    \item Start with a labeled tree T, and $i = 1$.
    \item Let $b_i =$ smallest label on a leaf.
    \item Let $a_i =$ label of the adjacent vertex of $b_1$.
    \item Remove $b_i$ and record $a_i$ in the sequence.
    \item Repeat with $b_{i+1} \tand a_{i+1}$.
    \item Stop once only vertices remain.
\end{enumerate}
Decoding a Sequence to a Tree
\begin{enumerate}
    \item Start with $(a_1, \ldots, a_{n-2}) \tand i = 1$.
    \item Let $b_i =$ smallest element of $\{1,\ldots,n\}$ \bld{not} in the sequence.
    \item Draw edge $a_ib_i$.
    \item Remove $a_i$ from the sequence and $b_i$ from the set.
    \item Repeat with $b_{i+1} \tand a_{i+1}$.
    \item Stop once the sequence is empty, and draw an edge between the last two elements in the set.
\end{enumerate}