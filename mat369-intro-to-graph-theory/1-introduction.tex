\section{Introduction}

\subsection{Graphs and Graph Models}

\subsubsection*{Graph Definition}
A (simple) \bld{graph} is an ordered pair $(V,E)$ where
\begin{itemize}
    \item $V$ is a nonempty set of objects called "vertices"
    \item $E$ is a set containing some two-subsets of $V$ called "edges". $E$ may be empty.
\end{itemize}

Graphs are often represented pictorially. For example consider
\[
    G = (V,E) \twhere V = \{1,2,3,4,5\} \tand E = \{\{1,4\},\{2,3\},\{2,4\},\{3,4\}\}
\]

\begin{center}
    \begin{tikzpicture}
        \pgfmathsetmacro{\gsize}{2}
    
        \foreach \rank/\elements/\size in {0/{,1,2}/3, 0.5/{5}/3, 1/{,4,3}/3} {
        \foreach[count = \i] \element in \elements {\node (\element) at (\i * \gsize - \size / 2 * \gsize, -\rank * \gsize) {$\element$};}
        }
        \foreach \j/\l in {1/4,2/3,2/4,3/4} {\draw (\j) -- (\l);}
        \node (name) at (-1,0) {$G:$}; %name
    \end{tikzpicture}
\end{center}
\begin{itemize}
\item Vertices $1 \tand 4$ are \bld{adjacent} because they are joined by an edge.
    \item Vertex $2$ and edge $2-3$ are \bld{indicent}.
    \item Edges $2-3 \tand 3-4$ are \bld{adjacent}.
\end{itemize}

\subsubsection*{Order Definition}
The \bld{order} of a graph $G$ is $|V(G)|$, or the number of vertices.

\subsubsection*{Size Definition}
The \bld{size} of a graph $G$ is $|E(G)|$, or the number of edges.

The graph $G$ from above has order $5$ and size $4$.

\subsection{Connected Graphs}

\subsubsection*{Subgraph Definition}
Let $G \tand H$ be graphs. $H$ is a \bld{subgraph} of $G$, notated as $H \subseteq G$, if
\[
    V(H) \subseteq V(G) \tand E(H) \subseteq E(G).
\]

\subsubsection*{Proper Subgraph Definition}
$H$ is a \bld{proper subgraph} of $G$ if $H \subseteq G$ and either
\[
    V(H) \subsetneq V(G) \tor E(H) \subsetneq E(G).
\]

\subsubsection*{Spanning Subgraph Definition}
Graph $H$ is a \bld{spanning subgraph} if $H \subseteq G$ and $V(H) = V(G)$.

\subsubsection*{Induced Subgraph Definition}
Graph $H$ is a \bld{induced subgraph} if $H \subseteq G$ and if
\[
    u,v \in V(H) \tand u,v \in E(G) \implies u,v \in E(H).
\]
Essentially, $H$ contains all valid edges it can take from $G$. Notation for \bld{induced subgraph} is
\[
    G[S], \text{ where $S$ is a set of vertices from $G$}.
\]

\subsubsection*{Edge-induced Subgraph Definition}
$G[X]$ is an \bld{edge-induced subgraph} of G if $G[X]$ has edge set $X \subseteq E(G)$ and a vertex set of all vertices incident with at least one edge of $X$. Interesting fact: $G[E(G)]$ removes any isolated vertices.

\subsubsection*{More on Spanning and Induced Subgraphs}
Let $G$ be a graph with vertex $v$ and edge $e$. Then,
\begin{itemize}
    \item $G-e$ is the \itl{spanning subgraph} of $G$ whose edge set is $E(G)-\{e\}$. 
        \subitem This definition can be expanded to $G-X \tfor X \subseteq E(G)$.
    \item $G-v$ is the \itl{induced subgraph} of $G$ whose vertex set is $V(G) - \{v\}$ and edge set includes all edges of $G$ except those incident with $v$. 
        \subitem This definition can be expanded to $G-U \tfor U \subseteq V(G)$.
\end{itemize}
Let $G$ be a graph, $u,v \in V(G) \tand e = uv \notin E(G)$. Then $G+e$ is the graph with vertex set $V(G)$ and edge set $E(G) \cup \{e\}$. $G$ is a \itl{spanning subgraph} of $G+e$

\subsubsection*{Walk, Trail, Path, Circuit, and Cycle Definitions}
Let $u,v \in V(G)$. A $u-v$ \bld{walk} in $G$ is a sequence of vertices
\[
    (u = v_0, v_1, \ldots, v_k = v)
\]
beginning with $u$, ending with $v$, and consecutive vertices are adjacent. 

A \bld{trail} is a walk in which \itl{no edges} are repeated. A \bld{path} is a walk in which \itl{no vertices} are repeated. Every \itl{path} is a \itl{trail} is a \itl{walk}.

A \bld{circuit} is a closed trail of length $\ge 3$. A \bld{cycle} is a circuit with no repeated vertices, except for the first and the last, which are the same. A \bld{$k$-cycle} is a cycle of length $k$. Every \itl{cycle} is a \itl{circuit} is a \itl{walk}.

\subsubsection*{Closed and Open Walks}
A $u-v$ walk with $u=v$ is called a \bld{closed} walk. A $u-v$ walk with $u \neq v$ is called a \bld{open} walk.

\subsubsection*{Walk and Path Theorem}
If $G$ contains a $u-v$ walk of length $\ell$, then $G$ contains a $u-v$ path of length $\leq \ell$.

\subsubsection*{Connectivity Definition}
A graph $G$ is said to be \bld{connected} if $\forall u,v \in V(G)$, $G$ contains a $u-v$ path. If this is not true, i.e. $\exists u,v \in V(G)$ where there is no $u-v$ path, then $G$ is said to be \bld{disconnected}.

\subsubsection*{Component Definition}
A connected subgraph of $G$ that is not a proper subgraph of any other connected subgraph of $G$ is a \bld{component} of $G$. The number of components of a graph $G$ is denoted by $k(G)$. A graph $G$ is connected if and only if $k(G) = 1$. Additionally, a graph is the union of its components.

\subsubsection*{Components and Equivalence Relations Theorem}
Define a relation $R$ on the $V(G)$ so that $uRv$ if $G$ contains a $u-v$ walk. Then $R$ is an equivalence relation.

\subsubsection*{Subtractive Connectivity Theorem (weak)}
Let $G$ be a graph of order $\geq 3$. If $\exists u,v \in V(G)$ such that $G-u$ and $G-v$ are connected, then $G$ is connected.

\subsubsection*{Distance, Geodesic, Diameter, and Girth Definitions}
The \bld{distance} between vertices $u \tand v$, denoted as $d(u,v) \tor d_G(u,v)$ is the smallest length of any $u-v$ path in $G$. If $u \tand v$ are in different components, then $d(u,v)$ is undefined.

A $u-v$ path of shortest length $d(u,v)$ is called a \bld{geodesic}. The \bld{diameter} of a connected graph $G$, denoted as $\diam(G)$, is the largest \itl{geodesic} between any two vertices of $G$. The \bld{girth} of a connected graph $G$ is the length of the shortest cycle in $G$.

\subsubsection*{Subtractive Connectivity Theorem (strong)}
Let $G$ be a graph of order $\geq 3$. Then $G$ is connected if and only if $\exists u,v \in V(G)$ such that $G-u$ and $G-v$ are connected.

\subsection{Common Classes of Graphs}
\begin{center}
    \begin{tabular}{|c|c|c|c|}
        \hline
        \bld{Name} & \bld{Symbol} & \bld{Order} & \bld{Size} \\
        \hline
        Path & $P_n$ & $n$ & $n-1$ \\
        Cycle & $C_n$ & $n \geq 3$ & $n$ \\
        Complete & $K_n$ & $n$ & $\binom{n}{2}$ \\
        Complete Bipartite & $K_{s,t}$ & $s+t$ & $s \cdot t$ \\
        \hline
    \end{tabular}
\end{center}

\subsubsection*{Bipartite Graph Definition}
$G$ is bipartite if $V(G)$ can be partitioned into partite sets $U \tand W$ so that every edge joins a vertex of $U$ and a vertex of $W$.

\subsubsection*{Odd Cycle and Bipartite-ness}
$G$ is bipartite if and only if $G$ contains no odd cycles.

\subsubsection*{K-partite Definition}
$G$ is a \bld{$k$-partite} graph if $V(G)$ can be partitioned into partite sets $U_1, \ldots, U_k$ so that every edge joins a vertex from $U_i$ and a vertex of $U_j$ where $i \neq j$.

\subsection*{Constructing New Graphs from Old Graphs}
\subsubsection*{Disjoint Union}
For two graphs $G \tand H$, $G \cup H$ is defined as...
\begin{align*}
    V(G \cup H) & = V(G) \cup V(H) \\
    E(G \cup H) & = E(G) \cup E(H)
\end{align*}

\subsubsection*{Complement}
For one graph $G$, $\overline{G}$ is defined as...
\begin{align*}
    V(\overline{G}) & = V(G) \\
    E(\overline{G}) & = \{uv | u,v \in V(G), u \neq v, uv \notin E(G)\}
\end{align*}

\subsubsection*{Join}
For two graph, $G \tand H$, $G + H$ is defined as...
\begin{center}
    Start with $G \cup H$ and draw all edges join a vertex of $G$ and a vertex of $H$
\end{center}

\subsubsection*{Cartesian Product}
For two graphs, $G \tand H$, $G \times H$ is defined as...
\begin{align*}
    V(G \times H) & = \{(u,v) | u \in V(G) \tand v \in V(H)\} \\
    (u,v)-(x,y) & \tif u = x \tand vy \in E(H) \lor v = y \tand ux \in E(G)
\end{align*}
A cartesian product between two graphs has the practical effect of duplicating one graph, and connecting the duplicates in the way of the other graph.

\subsubsection*{Complement Connectivity Theorem}
If $G$ is disconnected, then $\overline{G}$ is connected.

\subsection{Multigraphs and Digraphs}

\subsubsection*{Multigraph Definition}
A \bld{multigraph} is a graph where a pair of vertices may be joined by any finite number of edges.
\begin{itemize}
    \item Multiple edges: OK
    \item Loops: NOT OK
\end{itemize}

\subsubsection*{Pseudograph Definition}
A \bld{pseudograph} is a \itl{multigraph} where loops are allowed
\begin{itemize}
    \item Multiple edges: OK
    \item Loops: OK
\end{itemize}

\subsubsection*{Digraph Definition}
A \bld{directed graph} is a graph where $E(G)$ is a set of ordered pairs (rather than sets) of distinct vertices called directed edges, or arcs.

\subsubsection*{Oriented Graph Definition}
An \bld{oriented graph} is a \itl{digraph} in $\forall u,v \in V(G)$, $(u,v) \tand (v,u)$ are not \underbar{both} edges.