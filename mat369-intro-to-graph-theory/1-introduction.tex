\section{Introduction}

\subsection{Graphs and Graph Models}

\subsubsection*{Graph Definition}
A (simple) \bld{graph} is an ordered pair $(V,E)$ where
\begin{itemize}
    \item $V$ is a nonempty set of objects called "vertices"
    \item $E$ is a set containing some two-subsets of $V$ called "edges". $E$ may be empty.
\end{itemize}

Graphs are often represented pictorially. For example consider
\[
    G = (V,E) \twhere V = \{1,2,3,4,5\} \tand E = \{\{1,4\},\{2,3\},\{2,4\},\{3,4\}\}
\]

\begin{center}
    \begin{tikzpicture}
        \pgfmathsetmacro{\gsize}{2}
    
        \foreach \rank/\elements/\size in {0/{,1,2}/3, 0.5/{5}/3, 1/{,4,3}/3} {
        \foreach[count = \i] \element in \elements {\node (\element) at (\i * \gsize - \size / 2 * \gsize, -\rank * \gsize) {$\element$};}
        }
        \foreach \j/\l in {1/4,2/3,2/4,3/4} {\draw (\j) -- (\l);}
        \node (name) at (-1,0) {$G:$}; %name
    \end{tikzpicture}
\end{center}
\begin{itemize}
\item Vertices $1 \tand 4$ are \bld{adjacent} because they are joined by an edge.
    \item Vertex $2$ and edge $2-3$ are \bld{indicent}.
    \item Edges $2-3 \tand 3-4$ are \bld{adjacent}.
\end{itemize}

\subsubsection*{Order Definition}
The \bld{order} of a graph $G$ is $|V(G)|$, or the number of vertices.

\subsubsection*{Size Definition}
The \bld{size} of a graph $G$ is $|E(G)|$, or the number of edges.

The graph $G$ from above has order $5$ and size $4$.

\subsection{Connected Graphs}

\subsubsection*{Subgraph Definition}
Let $G \tand H$ be graphs. $H$ is a \bld{subgraph} of $G$, notated as $H \subseteq G$, if
\[
    V(H) \subseteq V(G) \tand E(H) \subseteq E(G).
\]

\subsubsection*{Proper Subgraph Definition}
$H$ is a \bld{proper subgraph} of $G$ if $H \subseteq G$ and either
\[
    V(H) \subsetneq V(G) \tor E(H) \subsetneq E(G).
\]

\subsubsection*{Spanning Subgraph Definition}
Graph $H$ is a \bld{spanning subgraph} if $H \subseteq G$ and $V(H) = V(G)$.

\subsubsection*{Induced Subgraph Definition}
Graph $H$ is a \bld{induced subgraph} if $H \subseteq G$ and if
\[
    u,v \in V(H) \tand u,v \in E(G) \implies u,v \in E(H).
\]
Essentially, $H$ contains all valid edges it can take from $G$. Notation for \bld{induced subgraph} is
\[
    G[S], \text{ where $S$ is a set of vertices from $G$}.
\]

\subsubsection*{Edge-induced Subgraph Definition}
$G[X]$ is an \bld{edge-induced subgraph} of G if $G[X]$ has edge set $X \subseteq E(G)$ and a vertex set of all vertices incident with at least one edge of $X$. Interesting fact: $G[E(G)]$ removes any isolated vertices.

\subsubsection*{More on Spanning and Induced Subgraphs}
Let $G$ be a graph with vertex $v$ and edge $e$. Then,
\begin{itemize}
    \item $G-e$ is the \itl{spanning subgraph} of $G$ whose edge set is $E(G)-\{e\}$. 
        \subitem This definition can be expanded to $G-X \tfor X \subseteq E(G)$.
    \item $G-v$ is the \itl{induced subgraph} of $G$ whose vertex set is $V(G) - \{v\}$ and edge set includes all edges of $G$ except those incident with $v$. 
        \subitem This definition can be expanded to $G-U \tfor U \subseteq V(G)$.
\end{itemize}
Let $G$ be a graph, $u,v \in V(G) \tand e = uv \notin E(G)$. Then $G+e$ is the graph with vertex set $V(G)$ and edge set $E(G) \cup \{e\}$. $G$ is a \itl{spanning subgraph} of $G+e$