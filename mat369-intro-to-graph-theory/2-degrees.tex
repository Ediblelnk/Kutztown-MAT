\section{Degrees}

\subsection{Degree of a Vertex}

\subsubsection*{Vertex Degree Definition}
The \bld{degree} of a vertex $v$, denoted as $\deg v \tor \deg_G v$, is the number of edges incident with $v$. If the $\deg v = 0$, then $v$ is an \bld{isolated vertex}. If $\deg v = 1$, then $v$ is a \bld{leaf}.
\begin{itemize}
    \item $\delta(G) = \min\{\deg v ~|~ v \in V(G)\}$, the minimum degree of $G$
    \item $\Delta(G) = \max\{\deg v ~|~ v \in V(G)\}$, the maximum degree of $G$
\end{itemize}
For any graph $G \tand v \in V(G)$,
\[
    0 \leq \delta(G) \leq \deg v \leq \Delta(G) \leq n-1.
\]

\subsubsection*{Neighborhood of a Vertex Definition}
The \bld{neighborhood} of a vertex $v$, denoted as $N(v)$, is the set of all vertices adjacent to $v$. So $\lvert N(v) \rvert = \deg v$.

\subsubsection*{Handshaking Theorem}
For a graph $G$ of size $m$, the total degree of $G$ $\sum_{v \in V(G)} = 2m$.

\subsubsection*{Handshaking Corollary}
Every graph has an even number of odd degree vertices.
\begin{proof}
    By handshaking,
    \[
        2m = \sum_{v \in V(G)} \deg(v) = \sum_{v \in V_1(G)} \deg(v) + \sum_{v \in V_2(G)} \deg(v),~\twhere
    \]
    \begin{align*}
        V_1(G) & = ~\text{set of all odd degree vertices}  \\
        V_2(G) & = ~\text{set of all even degree vertices}
    \end{align*}
    \[
        2m - \sum_{v \in V_2(G)} \deg(v) ~= \sum_{v \in V_1(G)} \deg(v) \text{,  so $|V_1(G)|$ must be even}.
    \]
\end{proof}

\subsubsection*{Sum Degree and Connectivity Theorem}
Consider graph $G$ of order $n$. If $\deg u + \deg v \geq n-1$ for all non-adjacent $(u,v) \in V(G)$, then $G$ is connected.
\begin{proof}
    Let $x,y \in V(G)$.
    \begin{enumerate}[start=1,label={\bfseries Case \arabic*:},leftmargin=0.75in]
        \item If $x,y$ are adjacent, then $(x,y)$ is a walk in $G$.
        \item If $x,y$ are \uln{not} adjacent, then $\deg u + \deg v \geq n-1$, by assumption. Since there are only $n-2$ vertices in $G$ besides $x \tand y$, $x \tand y$ must have a common neighbor $w \in V(G)$. Then $(x,w,y)$ is a walk in $G$.
    \end{enumerate}
\end{proof}

\subsubsection*{Sum Degree and Connectivity Corollary}
If $G$ has order $n$ and $\delta(G) \geq \frac{n-1}{2}$, then $G$ is connected.
\begin{proof}
    If $u,v \in V(G)$ are not adjacent, then
    \[
        \deg u + \deg v \geq \delta(G) + \delta(G) = \frac{n-1}{2} + \frac{n-1}{2} = n-1.
    \]
    Hence, by the previous theorem, $G$ is connected.
\end{proof}

\subsection{Regular Graphs}

\subsubsection*{Regular Graph Definition}
Graph $G$ is \bld{regular} if every vertex has the same degree. Graph $G$ is \bld{$r$-regular} if every vertex has degree $r$.

\subsubsection*{Regular Graph Existence Theorem}
Let $r,n \in \bb{Z}$ such that $0 \leq r \leq n-1$. Then there exists an $r$-regular graph of order $n$ if and only if at least one of $r \tand n$ is even.

\subsubsection*{Harary Graph}
An Harary Graph, denoted as $H_{r,n}$, is an $r$-regular graph of order $n$.

\subsubsection*{Induced Regular Subgraph Theorem}
For every graph $G$, and every integer $r \geq \Delta(G)$, there exists on $r$-regular graph $H$, containing $G$ as an induced subgraph.

\subsection{Degree Sequences}

\subsubsection*{Degree Sequence Definition}
A \bld{degree sequence} is a sequence of the degree of the vertices of a graph, typically, written in largest to smallest order.

\subsubsection*{Graphical Degree Sequence Definition}
A finite sequence of non-negative integers is \bld{graphical} if it is the degree sequence of some graph.

\subsubsection*{Graphical Degree Sequence Theorem}
A non-increasing sequence $S: d_1, d_2, \ldots, d_n$, where $n \geq n$, of non-negative integers is graphical if and only if
\[
    S_1: d_2 - 1, d_3 - 1, \ldots, d_{d_1 + 1}, d_{d_1 + 2}, d_n
\]
is graphical.

\subsection{Graph and Matrices}

\subsubsection*{Adjacency Matrix Definition}
The \bld{adjacency matrix} of $G$ is the $n \times n$ matrix $A = [a_{ij}]$, where
\[
    a_{ij} = \left\{
    \begin{array}{ll}
        1, & \tif v_iv_j \in E(G) \\
        0  & \text{otherwise};
    \end{array}
    \right.
\]
The entry $a_{ij}$ in $A^n$ is the number of walks of length $n$ from $v_i \tto v_j$.