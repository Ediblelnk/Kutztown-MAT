\documentclass{article}
\documentclass{article}
\usepackage[margin=1in]{geometry}
\usepackage{amsmath, amsthm, amssymb, fancyhdr, tikz, circuitikz, graphicx}
\usepackage{centernot, xcolor, hhline, multirow, listings}
\usepackage{blkarray, booktabs, bigstrut, etoolbox}
\usepackage[normalem]{ulem}
\usepackage{bookmark}
\usetikzlibrary{math}
\usetikzlibrary{fit}

\pagestyle{fancy}

\usepackage{hyperref}
\hypersetup{
  colorlinks=true,
  linkcolor=black,
  filecolor=magenta,
  urlcolor=cyan,
}
%formatting
\newcommand{\bld}{\textbf}
\newcommand{\itl}{\textit}
\newcommand{\uln}{\underline}

%math word symbols
\newcommand{\bb}{\mathbb}
\DeclareMathOperator{\tif}{~\text{if}~}
\DeclareMathOperator{\tand}{~\text{and}~}
\DeclareMathOperator{\tbut}{~\text{but}~}
\DeclareMathOperator{\tor}{~\text{or}~}
\DeclareMathOperator{\tsuchthat}{~\text{such that}~}
\DeclareMathOperator{\tsince}{~\text{since}~}
\DeclareMathOperator{\twhen}{~\text{when}~}
\DeclareMathOperator{\twhere}{~\text{where}~}
\DeclareMathOperator{\tfor}{~\text{for}~}
\DeclareMathOperator{\tthen}{~\text{then}~}
\DeclareMathOperator{\tto}{~\text{to}~}

%display shortcut
\DeclareMathOperator{\dstyle}{\displaystyle}
\DeclareMathOperator{\sstyle}{\scriptstyle}

%linear algebra
\DeclareMathOperator{\id}{\bld{id}}
\DeclareMathOperator{\vecspan}{\text{span}}

%discrete math - integer properties
\DeclareMathOperator{\tdiv}{\text{div}}
\DeclareMathOperator{\tmod}{\text{mod}}
\DeclareMathOperator{\lcm}{\text{lcm}}

%augmented matrix environment
\newenvironment{apmatrix}[2]{%
  \left(\begin{array}{@{~}*{#1}{c}|@{~}*{#2}{c}}
    }{
  \end{array}\right)
}
\newenvironment{abmatrix}[2]{%
  \left[\begin{array}{@{~}*{#1}{c}|@{~}*{#2}{c}}
      }{
    \end{array}\right]
}

%lists
\newcommand{\bitem}[1]{\item[\bld{#1.}]}
\newcommand{\bbitem}[2]{\item[\bld{#1.}] \bld{#2}}
\newcommand{\biitem}[2]{\item[\bld{#1.}] \itl{#2}}
\newcommand{\iitem}[1]{\item[\itl{#1.}]}
\newcommand{\iiitem}[2]{\item[\itl{#1.}] \bld{#2}}
\newcommand{\btitem}[2]{\item[\bld{#1.}] \texttt{#2}}

%homework
\newcommand{\question}[2]{\noindent {\large\bld{#1}} #2 \qline}
\newcommand{\qitem}[3]{\item[\bld{#1.}] \itl{#2} #3 \qdash}

\newcommand{\qline}{~\newline\noindent\textcolor[RGB]{200,200,200}{\rule[0.5ex]{\linewidth}{0.2pt}}}
\newcommand{\qdash}{~\newline\noindent\textcolor[RGB]{200,200,200}{\hdashrule[0.5ex]{\linewidth}{0.2pt}{2pt}}}

\newcommand{\assignment}{Homework 1}

\lhead{CSC 237: Data Structures}
\chead{\assignment}
\rhead{Peter Schaefer}

\begin{document}
\section*{\assignment}

\question{1.}{Solve and then prove that your closed form solution is correct by performing an inductive proof.
  \[
    S_n = \{1,4,8,13,19,\ldots\}
  \]
  Start with subscript 0, i.e. Base Case: $T_0 = 1$
}
It seems that $S_n$ is just the triangular numbers minus two, offset by one index. The recurrence relation which follows $S_n$ is
\[
  T_n = T_{n-1} + (n+2).
\]
In order to find the closed form, we can expand the terms. Expansions: $n-1+0+1 = n$
\begin{align*}
  T_n & = T_{n-1} + (n+2)                              \\
      & = T_{n-2} + (n+1) + (n+2)                      \\
      & = T_{n-3} + n + (n+1) + (n+2)                  \\
      & ~~\vdots                                       \\
      & = T_{0} + 3 + 4 + \cdots + n + (n+1) + (n+2)   \\
      & = (1 + 2 + 3 + \cdots + n + (n+1) + (n+2)) - 2 \\
      & = \sum_{i=1}^{n+2} i ~- 2                      \\
  T_n & = \frac{(n+2)(n+3)}{2} - 2
\end{align*}
Now we have a candidate for a closed form solution. We will prove this candidate through mathematical induction.
\begin{proof}[Proof through Induction]
  Proof that $T_n = \frac{(n+2)(n+3)}{2} - 2$ for all $n \in \bb{Z}^{\geq 0}$. \\
  Base Case: $n = 0$
  \[
    T_0 = \frac{(0 + 2)(0 + 3)}{2} - 2 = \frac{6}{2} - 2 = 3 - 2 = 1~\checkmark
  \]
  Inductive Hypothesis: Assume that
  \[
    T_k = \frac{(k + 2)(k + 3)}{2} - 2~\text{for some}~k \in \bb{Z}^{\geq 0}.
  \]
  Inductive step: $n = k+1$
  \begin{align*}
    T_{k+1} & = T_k + ((k+1)+2)                                                                      \\
            & = T_k + (k+3)                                                                          \\
            & = \frac{(k + 2)(k + 3)}{2} - 2 + (k + 3)          &  & \text{via Inductive Hypothesis} \\
            & = \frac{(k + 2)(k + 3)}{2} + \frac{2(k+3)}{2} - 2                                      \\
            & = \frac{(k + 2)(k + 3) + 2(k+3)}{2} - 2                                                \\
            & = \frac{(k+3)(k+2+2)}{2} - 2                                                           \\
    T_{k+1} & = \frac{((k+1)+2)((k+1)+3)}{2} - 2
  \end{align*}
  The inductive step holds. Therefore, through mathematical induction, $T_n = \frac{(n+2)(n+3)}{2} - 2$ for all $n \in \bb{Z}^{\geq 0}$.
\end{proof}
\qdash

\question{2.}{Find a closed form solution. Extra credit (4 points): Perform and inductive proof.
  \[
    S_n = \{1,8,36,148,596,\ldots\}.
  \]
  Start with subscript 1, i.e. Base Case: $T_1 = 1$
}
The derived recurrence relation is
\[
  T_n = 4T_{n-1} + 4.
\]
In order to find the closed form, we can expand the terms. Expansions: $n-1-1+1 = n-1$
\begin{align*}
  T_n & = 4T_{n-1} + 4                     &  & = 4T_{n-1} + 4               \\
      & = 4(4T_{n-2} + 4) + 4              &  & = 4^2T_{n-2} + 4^2 + 4       \\
      & = 4(4(4T_{n-3} + 4) + 4) + 4       &  & = 4^3T_{n-3} + 4^3 + 4^2 + 4 \\
      & ~~\vdots                                                             \\
      & = 4^{n-1} + \sum_{i = 1}^{n-1} 4^i                                   \\
      & = 4^{n-1} + \frac{4^{n}-4}{3}
\end{align*}
Now we have a candidate for a closed form solution. We will prove this candidate through mathematical induction.
\begin{proof}[Proof through Induction]
  Proof that $T_n = 4^{n-1} + \frac{4^{n}-4}{3}$ for all $n \in \bb{Z}^{+}$. \\
  Base Case: $n = 1$
  \[
    T_1 = 4^{1-1} + \frac{4^1-4}{3} = 4^0 + \frac{0}{3} = 1~\checkmark
  \]
  Inductive Hypothesis: Assume that
  \[
    T_k = 4^{k-1} + \frac{4^k-4}{3}~\text{for some}~k \in \bb{Z}^{+}.
  \]
  Inductive step: $n = k+1$
  \begin{align*}
    T_{k+1} & = 4T_{k} + 4                                                                       \\
            & = 4(4^{k-1} + \frac{4^k-4}{3}) + 4            &  & \text{via Induction Hypothesis} \\
            & = 4^{k} + \frac{4^{k+1}-16}{3} + \frac{12}{3}                                      \\
            & = 4^{k} + \frac{4^{k+1}-16 + 12}{3}                                                \\
    T_{k+1} & = 4^{k} + \frac{4^{k+1}-4}{3}
  \end{align*}
  The inductive step holds. Therefore, through mathematical induction, $T_n = 4^{n-1} + \frac{4^{n}-4}{3}$ for all $n \in \bb{Z}^{+}$.
\end{proof}
\qdash

\end{document}