\documentclass{article}

\usepackage[margin=1in]{geometry}
\usepackage{amsmath, amssymb}
\usepackage{graphicx}

\usepackage{fancyhdr}
\pagestyle{fancy}

\usepackage{hyperref}
\hypersetup{
    colorlinks=true,
    linkcolor=blue,
    filecolor=magenta,      
    urlcolor=cyan,
}

\title{\LaTeX\ for Peter}
\author{Peter Schaefer}
\date{Freshman Fall}

\begin{document}

\maketitle

\tableofcontents

\newpage

\section{Logic}
\subsection{Propositions and Logical Operations}

\textbf{Proposition}: a statement that is either \underline{true} or \underline{false}

Some examples include: "It is raining today",  "\(3 \cdot 8 = 20 \)".

However, not all statements are propositions:"open the door"


\begin{tabular}{|c|c|c|}
  \hline
  \textbf{Name} & \textbf{Symbol} & \textbf{alternate name} \\
  \hline
  NOT & $\lnot$ & negation \\
  AND & $\land$ & conjunction \\
  OR & $\lor$ & dijunction \\
  XOR & $\oplus$ & exclusive or \\
  \hline
\end{tabular}
\quad
\begin{tabular}{|c|c||c|c|c|c|}
  \hline
  \textbf{$p$} & \textbf{$q$} & \textbf{$\lnot p$} & \textbf{$p \land q$} & \textbf{$p \lor q$} & \textbf{$p \oplus q$} \\
  \hline
  T & T & F & T & T & F \\
  T & F & F & F & T & T \\
  F & T & T & F & T & T \\
  F & F & T & F & F & F \\
  \hline
\end{tabular}

XOR is very useful for encryption and binary arithmetic.
  
\subsection{Evaluating Compound Propositions}

\begin{align*}
  p&: \text{The weather is bad.} \\
  q&: \text{The trip is cancelled.} \\
  r&: \text{The trip is delayed.}
\end{align*}

\begin{center}
  \textbf{then}
\end{center}

\begin{align*}
  p \land q &: \text{The weather is bad \textit{and} the trip is cancelled} \\
  p \lor q &: \text{The weather is bad \textit{or} the trip is cancelled} \\
  p \land (q \oplus r)&: \text{The weather is bad \textit{and} either the trip is cancelled \textit{or} delayed}
\end{align*}

\textbf{Order of Evaluation} \(\lnot\), then \(\land\), then \(\lor\), but parenthesis always help for clarity.

\subsection{Conditional Statements}
\subsection{Logical Equivalence}
\subsection{Laws of Propositional Logic}
\subsection{Predicates and Quantifiers}
\subsection{Quantified Statements}
\subsection{DeMorgan's law for Quantified Statements}
\subsection{Nested Quantifiers}
\subsection{More Nested Quantifiers}
\subsection{Logical Reasoning}
\subsection{Rules of Inference with Propositions}
\subsection{Rules of Interence with Quantifiers}

\end{document}