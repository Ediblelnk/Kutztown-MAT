\documentclass{article}

\usepackage[margin=1in]{geometry}
\usepackage{amsmath, amssymb}
\usepackage{graphicx}

\usepackage{fancyhdr}
\pagestyle{fancy}

\usepackage{hyperref}
\hypersetup{
    colorlinks=true,
    linkcolor=blue,
    filecolor=magenta,      
    urlcolor=cyan,
}

\title{\LaTeX\ for Peter}
\author{Peter Schaefer}
\date{Freshman Fall}

\begin{document}

\maketitle

\tableofcontents

\newpage

\section{Logic}
\subsection{Propositions and Logical Operations}

\textbf{Proposition}: a statement that is either \underline{true} or \underline{false}

Some examples include: "It is raining today",  "\(3 \cdot 8 = 20 \)".

However, not all statements are propositions:"open the door"

{}
\begin{flushright}
  \begin{tabular}{|c|c|c|}
    \hline
    \textbf{Name} & \textbf{Symbol} & \textbf{alternate name} \\
    \hline
    NOT & $\lnot$ & negation \\
    AND & $\land$ & conjunction \\
    OR & $\lor$ & dijunction \\
    XOR & $\oplus$ & exclusive or \\
    \hline
  \end{tabular}
\end{flushright}
  
\subsection{Evaluating Compound Propositions}
\subsection{Conditional Statements}
\subsection{Logical Equivalence}
\subsection{Laws of Propositional Logic}
\subsection{Predicates and Quantifiers}
\subsection{Quantified Statements}
\subsection{DeMorgan's law for Quantified Statements}
\subsection{Nested Quantifiers}
\subsection{More Nested Quantifiers}
\subsection{Logical Reasoning}
\subsection{Rules of Inference with Propositions}
\subsection{Rules of Interence with Quantifiers}

\end{document}