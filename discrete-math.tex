\documentclass{article}

\usepackage[margin=1in]{geometry}
\usepackage{amsmath, amssymb}
\usepackage{graphicx}

\usepackage{fancyhdr}
\pagestyle{fancy}

\usepackage{hyperref}
\hypersetup{
    colorlinks=true,
    linkcolor=black,
    filecolor=magenta,      
    urlcolor=cyan,
}

\title{Discrete Math for Computer Science}
\author{Peter Schaefer}
\date{Freshman Fall}

\begin{document}

\maketitle

\tableofcontents

\newpage

\section{Logic}
\subsection{Propositions and Logical Operations}

\textbf{Proposition}: a statement that is either \underline{true} or \underline{false}.


Some examples include "It is raining today" and "\(3 \cdot 8 = 20 \)".


However, not all statements are propositions, such as "open the door"

\begin{center}
  \begin{tabular}{c|c|c}
    \textbf{Name} & \textbf{Symbol} & \textbf{alternate name} \\
    \hline
    NOT           & $\lnot$         & negation                \\
    AND           & $\land$         & conjunction             \\
    OR            & $\lor$          & dijunction              \\
    XOR           & $\oplus$        & exclusive or            \\
  \end{tabular}
  \qquad
  \begin{tabular}{c|c|c|c|c|c}
    \textbf{$p$} & \textbf{$q$} & \textbf{$\lnot p$} & \textbf{$p \land q$} & \textbf{$p \lor q$} & \textbf{$p \oplus q$} \\
    \hline
    T            & T            & F                  & T                    & T                   & F                     \\
    T            & F            & F                  & F                    & T                   & T                     \\
    F            & T            & T                  & F                    & T                   & T                     \\
    F            & F            & T                  & F                    & F                   & F                     \\
  \end{tabular}
\end{center}

XOR is very useful for encryption and binary arithmetic.

\subsection{Evaluating Compound Propositions}

\begin{align*}
  p & : \text{The weather is bad.}    &                &  & p \land q            & : \text{The weather is bad \textit{and} the trip is cancelled}                            \\
  q & : \text{The trip is cancelled.} & \triangleright &  & p \lor q             & : \text{The weather is bad \textit{or} the trip is cancelled}                             \\
  r & : \text{The trip is delayed.}   &                &  & p \land (q \oplus r) & : \text{The weather is bad \textit{and} either the trip is cancelled \textit{or} delayed} \\
\end{align*}

\textbf{Order of Evaluation} \(\lnot\), then \(\land\), then \(\lor\), but parenthesis always help for clarity.

\begin{center}
  Example Truth Table:
  \qquad
  \begin{tabular}{c|c|c|c|c}
    \(p\) & \(q\) & \(p \land q\) & \(\lnot q\) & \((p \land q) \oplus \lnot q\) \\
    \hline
    T     & T     & T             & F           & T                              \\
    T     & F     & F             & T           & T                              \\
    F     & T     & F             & F           & F                              \\
    F     & F     & F             & T           & T                              \\
  \end{tabular}
\end{center}

\subsection{Conditional Statements}

\begin{center}
  \(p \rightarrow q\) \ where \(p\) is the \underline{hypothesis} and \(q\) is the \underline{conclusion}
\end{center}

\begin{center}
  \begin{tabular}{c|c}
    Format                          & Terminology   \\
    \hline
    \(p \rightarrow q\)             & given         \\
    \(\lnot q \rightarrow \lnot p\) & contrapostive \\
    \(q \rightarrow p\)             & converse      \\
    \(\lnot p \rightarrow \lnot q\) & inverse       \\
  \end{tabular}
  \qquad
  \begin{tabular}{ccccc}
    given   & \(p \rightarrow q\)             & \(\equiv\) & \(\lnot q \rightarrow \lnot p\) & contrapostive \\
    inverse & \(\lnot p \rightarrow \lnot q\) & \(\equiv\) & \(q \rightarrow p\)             & converse
  \end{tabular}
\end{center}

\begin{center}
  \begin{tabular}{c|c|c}
    \(p\) & \(q\) & \(p \rightarrow q\) \\
    \hline
    T     & T     & T                   \\
    T     & F     & F                   \\
    F     & T     & T                   \\
    F     & F     & T                   \\
  \end{tabular}
  \quad
  \begin{tabular}{c}
    \(p\) is a \underline{sufficient} condition for \(q\) \\
    \(q\) is a \underline{necessary} condition for \(p\)
  \end{tabular}
  \quad
  \begin{tabular}{c|c}
    Phrase                     & Logic                   \\
    \hline
    \(q\) if \(p\)             & \(p \rightarrow q\)     \\
    \(q\) only if \(p\)        & \(q \rightarrow p\)     \\
    \(q\) if and only if \(p\) & \(p \leftrightarrow q\)
  \end{tabular}
\end{center}

\begin{center}
  \textbf{Order of Operations}: \(p \land q \rightarrow r \equiv (p \land q) \rightarrow r\)
\end{center}

\subsection{Logical Equivalence}



\subsection{Laws of Propositional Logic}



\subsection{Predicates and Quantifiers}



\subsection{Quantified Statements}



\subsection{DeMorgan's law for Quantified Statements}



\subsection{Nested Quantifiers}



\subsection{More Nested Quantifiers}



\subsection{Logical Reasoning}



\subsection{Rules of Inference with Propositions}



\subsection{Rules of Interence with Quantifiers}



\end{document}