\documentclass{article}
\documentclass{article}
\usepackage[margin=1in]{geometry}
\usepackage{amsmath, amsthm, amssymb, fancyhdr, tikz, circuitikz, graphicx}
\usepackage{centernot, xcolor, hhline, multirow, listings}
\usepackage{blkarray, booktabs, bigstrut, etoolbox}
\usepackage[normalem]{ulem}
\usepackage{bookmark}
\usetikzlibrary{math}
\usetikzlibrary{fit}

\pagestyle{fancy}

\usepackage{hyperref}
\hypersetup{
  colorlinks=true,
  linkcolor=black,
  filecolor=magenta,
  urlcolor=cyan,
}
%formatting
\newcommand{\bld}{\textbf}
\newcommand{\itl}{\textit}
\newcommand{\uln}{\underline}

%math word symbols
\newcommand{\bb}{\mathbb}
\DeclareMathOperator{\tif}{~\text{if}~}
\DeclareMathOperator{\tand}{~\text{and}~}
\DeclareMathOperator{\tbut}{~\text{but}~}
\DeclareMathOperator{\tor}{~\text{or}~}
\DeclareMathOperator{\tsuchthat}{~\text{such that}~}
\DeclareMathOperator{\tsince}{~\text{since}~}
\DeclareMathOperator{\twhen}{~\text{when}~}
\DeclareMathOperator{\twhere}{~\text{where}~}
\DeclareMathOperator{\tfor}{~\text{for}~}
\DeclareMathOperator{\tthen}{~\text{then}~}
\DeclareMathOperator{\tto}{~\text{to}~}

%display shortcut
\DeclareMathOperator{\dstyle}{\displaystyle}
\DeclareMathOperator{\sstyle}{\scriptstyle}

%linear algebra
\DeclareMathOperator{\id}{\bld{id}}
\DeclareMathOperator{\vecspan}{\text{span}}

%discrete math - integer properties
\DeclareMathOperator{\tdiv}{\text{div}}
\DeclareMathOperator{\tmod}{\text{mod}}
\DeclareMathOperator{\lcm}{\text{lcm}}

%augmented matrix environment
\newenvironment{apmatrix}[2]{%
  \left(\begin{array}{@{~}*{#1}{c}|@{~}*{#2}{c}}
    }{
  \end{array}\right)
}
\newenvironment{abmatrix}[2]{%
  \left[\begin{array}{@{~}*{#1}{c}|@{~}*{#2}{c}}
      }{
    \end{array}\right]
}

%lists
\newcommand{\bitem}[1]{\item[\bld{#1.}]}
\newcommand{\bbitem}[2]{\item[\bld{#1.}] \bld{#2}}
\newcommand{\biitem}[2]{\item[\bld{#1.}] \itl{#2}}
\newcommand{\iitem}[1]{\item[\itl{#1.}]}
\newcommand{\iiitem}[2]{\item[\itl{#1.}] \bld{#2}}
\newcommand{\btitem}[2]{\item[\bld{#1.}] \texttt{#2}}

%homework
\newcommand{\question}[2]{\noindent {\large\bld{#1}} #2 \qline}
\newcommand{\qitem}[3]{\item[\bld{#1.}] \itl{#2} #3 \qdash}

\newcommand{\qline}{~\newline\noindent\textcolor[RGB]{200,200,200}{\rule[0.5ex]{\linewidth}{0.2pt}}}
\newcommand{\qdash}{~\newline\noindent\textcolor[RGB]{200,200,200}{\hdashrule[0.5ex]{\linewidth}{0.2pt}{2pt}}}

\newcommand{\assignment}{Test 3}

\lhead{Linear Algebra}
\chead{\assignment}
\rhead{Peter Schaefer}

\begin{document}
\section*{\assignment}

\question{Problem 1}{Find the determinant of the $6 \times 6$ matrix
  \[
    \begin{pmatrix}
      1 & 1 & 1 & 1 & 1 & 1 \\
      1 & 2 & 2 & 2 & 2 & 2 \\
      1 & 2 & 3 & 3 & 3 & 3 \\
      1 & 2 & 3 & 4 & 4 & 4 \\
      1 & 2 & 3 & 4 & 5 & 5 \\
      1 & 2 & 3 & 4 & 5 & 6
    \end{pmatrix}.
  \]
  Justify your answer.}
\begin{proof}[Work]
  We can use row operations and their effects on the determinant to help us solve the $6 \times 6$ matrix.
  \begin{align*}
     & \begin{determinant}{6}
         1 & 1 & 1 & 1 & 1 & 1 \\
         1 & 2 & 2 & 2 & 2 & 2 \\
         1 & 2 & 3 & 3 & 3 & 3 \\
         1 & 2 & 3 & 4 & 4 & 4 \\
         1 & 2 & 3 & 4 & 5 & 5 \\
         1 & 2 & 3 & 4 & 5 & 6
       \end{determinant} \overset{R_1 - R_2}{===}
    \begin{determinant}{6}
      0 & -1 & -1 & -1 & -1 & -1 \\
      1 & 2 & 2 & 2 & 2 & 2 \\
      1 & 2 & 3 & 3 & 3 & 3 \\
      1 & 2 & 3 & 4 & 4 & 4 \\
      1 & 2 & 3 & 4 & 5 & 5 \\
      1 & 2 & 3 & 4 & 5 & 6
    \end{determinant} \overset{R_2 - R_3}{===}
    \begin{determinant}{6}
      0 & -1 & -1 & -1 & -1 & -1 \\
      0 & 0 & -1 & -1 & -1 & -1 \\
      1 & 2 & 3 & 3 & 3 & 3 \\
      1 & 2 & 3 & 4 & 4 & 4 \\
      1 & 2 & 3 & 4 & 5 & 5 \\
      1 & 2 & 3 & 4 & 5 & 6
    \end{determinant} \overset{R_3 - R_4}{===}                                           \\
     & \begin{determinant}{6}
         0 & -1 & -1 & -1 & -1 & -1 \\
         0 & 0 & -1 & -1 & -1 & -1 \\
         0 & 0 & 0 & -1 & -1 & -1 \\
         1 & 2 & 3 & 4 & 4 & 4 \\
         1 & 2 & 3 & 4 & 5 & 5 \\
         1 & 2 & 3 & 4 & 5 & 6
       \end{determinant} \overset{R_4 - R_5}{===}
    \begin{determinant}{6}
      0 & -1 & -1 & -1 & -1 & -1 \\
      0 & 0 & -1 & -1 & -1 & -1 \\
      0 & 0 & 0 & -1 & -1 & -1 \\
      0 & 0 & 0 & 0 & -1 & -1 \\
      1 & 2 & 3 & 4 & 5 & 5 \\
      1 & 2 & 3 & 4 & 5 & 6
    \end{determinant} \overset{R_5 - R_6}{===}                                           \\
     & \begin{determinant}{6}
         0 & -1 & -1 & -1 & -1 & -1 \\
         0 & 0 & -1 & -1 & -1 & -1 \\
         0 & 0 & 0 & -1 & -1 & -1 \\
         0 & 0 & 0 & 0 & -1 & -1 \\
         0 & 0 & 0 & 0 & 0 & -1 \\
         1 & 2 & 3 & 4 & 5 & 6
       \end{determinant} \overunderset{R_6 + R_2 + R_1}{R_6 + R_5 + R_4 + R_3}{========}
    \begin{determinant}{6}
      0 & -1 & -1 & -1 & -1 & -1 \\
      0 & 0 & -1 & -1 & -1 & -1 \\
      0 & 0 & 0 & -1 & -1 & -1 \\
      0 & 0 & 0 & 0 & -1 & -1 \\
      0 & 0 & 0 & 0 & 0 & -1 \\
      1 & 0 & 0 & 0 & 0 & 1
    \end{determinant} \overset{R_1 - R_2}{===}                                           \\
     & \begin{determinant}{6}
         0 & -1 & 0 & 0 & 0 & 0 \\
         0 & 0 & -1 & -1 & -1 & -1 \\
         0 & 0 & 0 & -1 & -1 & -1 \\
         0 & 0 & 0 & 0 & -1 & -1 \\
         0 & 0 & 0 & 0 & 0 & -1 \\
         1 & 0 & 0 & 0 & 0 & 1
       \end{determinant} \overset{R_2 - R_3}{===}
    \begin{determinant}{6}
      0 & -1 & 0 & 0 & 0 & 0 \\
      0 & 0 & -1 & 0 & 0 & 0 \\
      0 & 0 & 0 & -1 & -1 & -1 \\
      0 & 0 & 0 & 0 & -1 & -1 \\
      0 & 0 & 0 & 0 & 0 & -1 \\
      1 & 0 & 0 & 0 & 0 & 1
    \end{determinant} \overset{R_3 - R_4}{===}                                           \\
     & \begin{determinant}{6}
         0 & -1 & 0 & 0 & 0 & 0 \\
         0 & 0 & -1 & 0 & 0 & 0 \\
         0 & 0 & 0 & -1 & 0 & 0 \\
         0 & 0 & 0 & 0 & -1 & -1 \\
         0 & 0 & 0 & 0 & 0 & -1 \\
         1 & 0 & 0 & 0 & 0 & 1
       \end{determinant} \overunderset{R_4 - R_5}{R_6 + R_5}{===}
    \begin{determinant}{6}
      0 & -1 & 0 & 0 & 0 & 0 \\
      0 & 0 & -1 & 0 & 0 & 0 \\
      0 & 0 & 0 & -1 & 0 & 0 \\
      0 & 0 & 0 & 0 & -1 & 0 \\
      0 & 0 & 0 & 0 & 0 & -1 \\
      1 & 0 & 0 & 0 & 0 & 0
    \end{determinant} \overset{R_6 \leftrightarrow R_5}{===}                             \\
     & -\begin{determinant}{6}
          0 & -1 & 0 & 0 & 0 & 0 \\
          0 & 0 & -1 & 0 & 0 & 0 \\
          0 & 0 & 0 & -1 & 0 & 0 \\
          0 & 0 & 0 & 0 & -1 & 0 \\
          1 & 0 & 0 & 0 & 0 & 0 \\
          0 & 0 & 0 & 0 & 0 & -1
        \end{determinant} \overset{R_5 \leftrightarrow R_4}{===}
    \begin{determinant}{6}
      0 & -1 & 0 & 0 & 0 & 0 \\
      0 & 0 & -1 & 0 & 0 & 0 \\
      0 & 0 & 0 & -1 & 0 & 0 \\
      1 & 0 & 0 & 0 & 0 & 0 \\
      0 & 0 & 0 & 0 & -1 & 0 \\
      0 & 0 & 0 & 0 & 0 & -1
    \end{determinant} \overset{R_4 \leftrightarrow R_3}{===}
  \end{align*}
  \begin{align*}
     & -\begin{determinant}{6}
          0 & -1 & 0 & 0 & 0 & 0 \\
          0 & 0 & -1 & 0 & 0 & 0 \\
          1 & 0 & 0 & 0 & 0 & 0 \\
          0 & 0 & 0 & -1 & 0 & 0 \\
          0 & 0 & 0 & 0 & -1 & 0 \\
          0 & 0 & 0 & 0 & 0 & -1
        \end{determinant} \overset{R_3 \leftrightarrow R_2}{===}
    \begin{determinant}{6}
      0 & -1 & 0 & 0 & 0 & 0 \\
      1 & 0 & 0 & 0 & 0 & 0 \\
      0 & 0 & -1 & 0 & 0 & 0 \\
      0 & 0 & 0 & -1 & 0 & 0 \\
      0 & 0 & 0 & 0 & -1 & 0 \\
      0 & 0 & 0 & 0 & 0 & -1
    \end{determinant} \overset{R_2 \leftrightarrow R_1}{===}      \\
     & -\begin{determinant}{6}
          1 & 0 & 0 & 0 & 0 & 0 \\
          0 & -1 & 0 & 0 & 0 & 0 \\
          0 & 0 & -1 & 0 & 0 & 0 \\
          0 & 0 & 0 & -1 & 0 & 0 \\
          0 & 0 & 0 & 0 & -1 & 0 \\
          0 & 0 & 0 & 0 & 0 & -1
        \end{determinant} = -(1)(-1)(-1)(-1)(-1)(-1) = (-1)^6 = 1
  \end{align*}
  Therefore, the determinant of the $6 \times 6$ matrix is 1:
  \[
    \begin{determinant}{6}
      1 & 1 & 1 & 1 & 1 & 1 \\
      1 & 2 & 2 & 2 & 2 & 2 \\
      1 & 2 & 3 & 3 & 3 & 3 \\
      1 & 2 & 3 & 4 & 4 & 4 \\
      1 & 2 & 3 & 4 & 5 & 5 \\
      1 & 2 & 3 & 4 & 5 & 6
    \end{determinant} = 1.
  \]
\end{proof}
\qdash

\question{Problem 2}{Prove that
  \[
    \det \begin{pmatrix}
      a     & b     & c     \\
      b + c & c + a & a + b \\
      a^2   & b^2   & c^2
    \end{pmatrix} = -(a-b)(b-c)(c-a)(a+b+c).
  \]}
\begin{proof}
  We can use row operations and their effects on the determinant to help us solve the $6 \times 6$ matrix.
  \begin{align*}
     & \begin{determinant}{3}
         a     & b     & c     \\
         b + c & c + a & a + b \\
         a^2   & b^2   & c^2
       \end{determinant} \overset{R_2 + R_1}{===}
    \begin{determinant}{3}
      a     & b     & c     \\
      a + b + c & a + b + c & a + b + c \\
      a^2   & b^2   & c^2
    \end{determinant} =                                                   \\
     & (a+b+c) \begin{determinant}{3}
                 a   & b   & c \\
                 1   & 1   & 1 \\
                 a^2 & b^2 & c^2
               \end{determinant} \overset{R_3 - R_1}{===}
    (a+b+c) \begin{determinant}{3}
              a   & b   & c \\
              1   & 1   & 1 \\
              a^2-1 & b^2-1 & c^2-1
            \end{determinant} =                                                       \\
     & (a+b+c) \begin{determinant}{3}
                 a & 1 & a^2-1 \\
                 b & 1 & b^2-1 \\
                 c & 1 & c^2-1
               \end{determinant} \overunderset{R_2 - R_1}{R_3 - R_1}{===}
    (a+b+c) \begin{determinant}{3}
              a   & 1 & a^2-1 \\
              b-a & 0 & b^2-1 - a^2-1 \\
              c-a & 0 & c^2-1 - a^2-1
            \end{determinant} =                                                     \\
     & (a+b+c) \begin{determinant}{3}
                 a   & 1 & a^2-1 \\
                 b-a & 0 & b^2-a^2 \\
                 c-a & 0 & c^2-a^2
               \end{determinant} =
    (a+b+c) \begin{determinant}{3}
              a   & 1 & a^2-1 \\
              b-a & 0 & (b+a)(b-a) \\
              c-a & 0 & (c+a)(c-a)
            \end{determinant} =                                                       \\
     & (b-a)(c-a)(a+b+c) \begin{determinant}{3}
                           a   & 1 & a^2-1 \\
                           1 & 0 & (b+a) \\
                           1 & 0 & (c+a)
                         \end{determinant} =
    (b-a)(c-a)(a+b+c)(1)(-1)[(c+a)-(b+a)] =                                             \\
     & (b-a)(c-a)(a+b+c)(1)(-1)(c-b) = (b-a)(c-a)(a+b+c)(b-c) = -(a-b)(b-c)(c-a)(a+b+c)
  \end{align*}
  Therefore, the determinant of the $3 \times 3$ matrix is $-(a-b)(b-c)(c-a)(a+b+c)$:
  \[
    \det \begin{pmatrix}
      a     & b     & c     \\
      b + c & c + a & a + b \\
      a^2   & b^2   & c^2
    \end{pmatrix} = -(a-b)(b-c)(c-a)(a+b+c).
  \]
\end{proof}
\qdash

\question{Problem 3}{Find the adjoint of the matrix
  \[
    A = \begin{pmatrix}
      2  & 1 & 1 \\
      -1 & 0 & 2 \\
      3  & 1 & 3
    \end{pmatrix}.
  \]
  Then use the adjoint to find $A^{-1}$.}
\begin{proof}[Work]
  To find the adjoint of matrix $A$, we have to find the cofactors for each element of the matrix, and then transpose the matrix.
  \begin{align*}
     & A = \begin{pmatrix}
             2  & 1 & 1 \\
             -1 & 0 & 2 \\
             3  & 1 & 3
           \end{pmatrix} \xrightarrow{\text{(cofactor)}}
    \begin{pmatrix}
      (0 \cdot 3 - 1 \cdot 2)  & -(-1 \cdot 3 - 3 \cdot 2)   & (-1 \cdot 1 - 3 \cdot 0)   \\
      -(1 \cdot 3 - 1 \cdot 1) & (2 \cdot 3 - 3 \cdot 1)     & -(2 \cdot 1 - 3 \cdot 1)   \\
      (1 \cdot 2 - 0 \cdot 1)  & -(2 \cdot 2 - (-1) \cdot 1) & (2 \cdot 0 - (-1) \cdot 1)
    \end{pmatrix} = \\
     & \begin{pmatrix}
         (-2) & -(-9) & (-1)  \\
         -(2) & (3)   & -(-1) \\
         (2)  & -(5)  & (1)
       \end{pmatrix} =
    \begin{pmatrix}
      -2 & 9  & -1 \\
      -2 & 3  & 1  \\
      2  & -5 & 1
    \end{pmatrix} \xrightarrow{\text{(transpose)}}
    \begin{pmatrix}
      -2 & -2 & 2  \\
      9  & 3  & -5 \\
      -1 & 1  & 1
    \end{pmatrix}
  \end{align*}
  We can then use the equation $A^{-1} = \frac{1}{\det(A)}\text{adj}(A)$ to find $A^{-1}$ using $\adj(A) \tand \det(A)$.
  \begin{align*}
     & \det(A) = \begin{determinant}{3}
                   2  & 1 & 1 \\
                   -1 & 0 & 2 \\
                   3  & 1 & 3
                 \end{determinant} \overset{R_1 + R_2}{===}
    \begin{determinant}{3}
      1  & 1 & 3 \\
      -1 & 0 & 2 \\
      3  & 1 & 3
    \end{determinant} \overunderset{R_3 - 3R_1}{R_2 + R_1}{===}
    \begin{determinant}{3}
      1  & 1 & 3 \\
      0 & 1 & 5 \\
      0  & -2 & -6
    \end{determinant} =                                   \\
     & (1)(1)(1 \cdot -6 - (-2) \cdot 5) = (-6 + 10) = 4
  \end{align*}
  Therefore, $A^{-1} = \frac{1}{4}\adj(A)$:
  \[
    A^{-1} = \frac{1}{4}\adj(A) = \frac{1}{4}\begin{pmatrix}
      -2 & -2 & 2  \\
      9  & 3  & -5 \\
      -1 & 1  & 1
    \end{pmatrix} = \begin{pmatrix}
      -\frac{1}{2} & -\frac{1}{2} & \frac{1}{2}  \\
      \frac{9}{4}  & \frac{3}{4}  & -\frac{5}{4} \\
      -\frac{1}{4} & \frac{1}{4}  & \frac{1}{4}
    \end{pmatrix}
  \]
\end{proof}
\qdash

\question{Problem 6}{Determine ALL values $a \tand b$ such that
  \[
    \left\{ \begin{pmatrix}
      a & b \\
      b & b
    \end{pmatrix}, \quad \begin{pmatrix}
      b & a \\
      b & b
    \end{pmatrix}, \quad \begin{pmatrix}
      b & b \\
      a & b
    \end{pmatrix}, \quad \begin{pmatrix}
      b & b \\
      b & a
    \end{pmatrix}\right\}
  \]
  is linearly independent.}
\begin{proof}[Work]
  Let $k_1,k_2,k_3,k_4 \in \bb{R}$ such that
  \[
    k_1\begin{pmatrix}
      a & b \\
      b & b
    \end{pmatrix} + k_2 \begin{pmatrix}
      b & a \\
      b & b
    \end{pmatrix} + k_3 \begin{pmatrix}
      b & b \\
      a & b
    \end{pmatrix} + k_4\begin{pmatrix}
      b & b \\
      b & a
    \end{pmatrix} = \begin{pmatrix}
      0 & 0 \\
      0 & 0
    \end{pmatrix}.
  \]
  From this, we can get a linear system of equations, and a matrix equation.
  \begin{align*}
    ak_1 + bk_2 + bk_3 + bk_4 & = 0 \\
    bk_1 + ak_2 + bk_3 + bk_4 & = 0 \\
    bk_1 + bk_2 + ak_3 + bk_4 & = 0 \\
    bk_1 + bk_2 + bk_3 + ak_4 & = 0
  \end{align*}
  \begin{align*}
    \begin{pmatrix}
      a & b & b & b \\
      b & a & b & b \\
      b & b & a & b \\
      b & b & b & a
    \end{pmatrix}
    \begin{pmatrix}
      k_1 \\ k_2 \\ k_3 \\ k_4
    \end{pmatrix} =
    \begin{pmatrix}
      0 \\ 0 \\ 0 \\ 0
    \end{pmatrix}
  \end{align*}
  Consider the determinant of the square matrix.
  \begin{align*}
     & \begin{determinant}{4}
         a & b & b & b \\
         b & a & b & b \\
         b & b & a & b \\
         b & b & b & a
       \end{determinant} \overunderset{R_1 + R_2 + R_3}{R_1 + R_4}{=====}
    \begin{determinant}{4}
      a+3b & a+3b & a+3b & a+3b \\
      b & a & b & b \\
      b & b & a & b \\
      b & b & b & a
    \end{determinant} =
    (a+3b) \begin{determinant}{4}
             1 & 1 & 1 & 1 \\
             b & a & b & b \\
             b & b & a & b \\
             b & b & b & a
           \end{determinant} \overunderset{R_2 - aR_1, R_4 - R_1}{R_3 - aR_1}{=====}      \\
     & (a+3b) \begin{determinant}{4}
                1 & 1 & 1 & 1 \\
                b-a & 0 & b-a & b-a \\
                b-a & b-a & 0 & b-a \\
                b-a & b-a & b-a & 0
              \end{determinant} =
    (a+3b)(b-a)^3 \begin{determinant}{4}
                    1 & 1 & 1 & 1 \\
                    1 & 0 & 1 & 1 \\
                    1 & 1 & 0 & 1 \\
                    1 & 1 & 1 & 0
                  \end{determinant} \overunderset{R_2 - R_1, R_4 - R_1}{R_3 - R_1}{=====} \\
     & (a+3b)(b-a)^3 \begin{determinant}{4}
                       1 & 1 & 1 & 1 \\
                       0 & -1 & 0 & 0 \\
                       0 & 0 & -1 & 0 \\
                       0 & 0 & 0 & -1
                     \end{determinant} =
    (a+3b)(b-a)^3(1)(-1)(-1)(-1) = -(a+3b)(b-a)^3
  \end{align*}
  In order to ensure that $-(a+3b)(b-a)^3 \neq 0$, $a+3b \neq 0$ and $b-a \neq 0$. These conditions will ensure that the determinant of this matrix will not be zero. By Theorem 4 of Lecture Notes 32, as long as the determinant of a square matrix is not zero, the inverse of the matrix exists. Then, by the Big Theorem, if the coefficient matrix is invertible, then $A\vec{x} = \vec{0}$ has only the trivial solution. This means that there does not exist $k_1, k_2, k_3, k_3 \in \bb{R}$ such that
  \[
    k_1\begin{pmatrix}
      a & b \\
      b & b
    \end{pmatrix} + k_2 \begin{pmatrix}
      b & a \\
      b & b
    \end{pmatrix} + k_3 \begin{pmatrix}
      b & b \\
      a & b
    \end{pmatrix} + k_4\begin{pmatrix}
      b & b \\
      b & a
    \end{pmatrix} = \begin{pmatrix}
      0 & 0 \\
      0 & 0
    \end{pmatrix}.
  \]
  This means that
  \[
    \left\{ \begin{pmatrix}
      a & b \\
      b & b
    \end{pmatrix}, \quad \begin{pmatrix}
      b & a \\
      b & b
    \end{pmatrix}, \quad \begin{pmatrix}
      b & b \\
      a & b
    \end{pmatrix}, \quad \begin{pmatrix}
      b & b \\
      b & a
    \end{pmatrix}\right\}
  \]
  are linearly independent when $a+3b \neq 0$ and $b-a \neq 0$.
\end{proof}
\qdash

\question{Problem 7}{Prove or disprove the following statement:
  \[
    \det(A^TA) \geq 0~\text{for all}~n \times n~\text{matrices}~A.
  \]}
\begin{proof}
  We can use Theorem 1 and Theorem 6 from Lecture notes 32 to help proof the statement.
  \begin{align*}
     & \text{Thm 1.}~~ \det A = \det A^T       \\
     & \text{Thm 6.}~~ \det AB = \det A \det B
  \end{align*}
  Case 1: $\det A \geq 0$. Through Theorem 1, this implies that $\det A^T \geq 0$, since $\det A = \det A^T$.
  \begin{align*}
    \det A                & \geq 0                            \\
    \det A^T \cdot \det A & \geq 0 \cdot 0 &  & \text{Thm 1.} \\
    \det (A^TA)           & \geq 0         &  & \text{Thm 6.}
  \end{align*}
  Case 2: $\det A < 0$. Through Theorem 1, this implies that $\det A^T < 0$, since $\det A = \det A^T$.
  \begin{align*}
    \det A                & < 0                                                                                             \\
    \det A^T \cdot \det A & > 0 \cdot 0 &  & \text{Thm 1.} &  & \text{(inequality switches because of mult. by a negative)} \\
    \det (A^TA)           & > 0         &  & \text{Thm 6.}                                                                  \\
    \det (A^TA)           & \geq 0      &  &               &  & \text{($>$ also implies $\geq$)}
  \end{align*}
  Since $\det A$ must either be greater than or equal to zero, or less than zero, and the statement is true for both possibilities, this means that
  \[
    \det(A^TA) \geq 0~\text{for all}~n \times n~\text{matrices}~A.
  \]
\end{proof}
\qdash

\question{Problem 8}{Let $V = \mathcal{M}_{nn}$, the vector space containing all $n \times n$ matrices. Prove or disprove:
  \[
    W = \{A \in V : \det(A) = 0\}~\text{forms a subspace of}~V.
  \]}
\begin{proof}
  To prove whether or not $W = \{A \in V : \det(A) = 0\}$ is a subspace of $V = \mathcal{M}_{nn}$, both Axioms 1 and Axioms 6 must hold.

  \begin{proof}
    Axiom 1: Closed under vector addition \\
    Consider $A_{n \times n} = \begin{pmatrix}
        1      & 0      & 0      & \cdots & 0      \\
        0      & 0      & 0      & \cdots & 0      \\
        0      & 0      & 0      & \cdots & 0      \\
        \vdots & \vdots & \vdots & \ddots & \vdots \\
        0      & 0      & 0      & \cdots & 0
      \end{pmatrix} \tand B_{n \times n} = \begin{pmatrix}
        0      & 0      & 0      & \cdots & 0      \\
        0      & 1      & 0      & \cdots & 0      \\
        0      & 0      & 1      & \cdots & 0      \\
        \vdots & \vdots & \vdots & \ddots & \vdots \\
        0      & 0      & 0      & \cdots & 1
      \end{pmatrix}$. Since $A \tand B$ are diagonal matrices, through Theorem 1 of Lecture Notes 31,
    \begin{align*}
      \det A & = 1 \cdot 0 \cdot 0 \cdots 0 = 0, \tand \\
      \det B & = 0 \cdot 1 \cdot 1 \cdots 1 = 0.
    \end{align*}
    This means that both $A \tand B \in W$. Now consider $C = A + B$:
    \[
      C_{n \times n} = \begin{pmatrix}
        1      & 0      & 0      & \cdots & 0      \\
        0      & 0      & 0      & \cdots & 0      \\
        0      & 0      & 0      & \cdots & 0      \\
        \vdots & \vdots & \vdots & \ddots & \vdots \\
        0      & 0      & 0      & \cdots & 0
      \end{pmatrix} + \begin{pmatrix}
        0      & 0      & 0      & \cdots & 0      \\
        0      & 1      & 0      & \cdots & 0      \\
        0      & 0      & 1      & \cdots & 0      \\
        \vdots & \vdots & \vdots & \ddots & \vdots \\
        0      & 0      & 0      & \cdots & 1
      \end{pmatrix} = \begin{pmatrix}
        1      & 0      & 0      & \cdots & 0      \\
        0      & 1      & 0      & \cdots & 0      \\
        0      & 0      & 1      & \cdots & 0      \\
        \vdots & \vdots & \vdots & \ddots & \vdots \\
        0      & 0      & 0      & \cdots & 1
      \end{pmatrix}.
    \]
    $C$ is also a diagonal matrix. Therefore, through Theorem 1 of Lecture Notes 31,
    \[
      \det(A + B) = \det(C) = 1 \cdot 1 \cdot 1 \cdots 1 = 1 \neq 0
    \]
    $W$ is not closed under addition, meaning that Axiom 1 does not hold.
  \end{proof}
  Since Axiom 1 does not hold for $W$, it cannot possibly be a subspace of $V = \mathcal{M}_{nn}$.
\end{proof}
\qdash

\end{document}