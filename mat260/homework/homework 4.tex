\documentclass{article}
\usepackage[margin=1in]{geometry}
\usepackage{amsmath, amsthm, amssymb, fancyhdr, tikz, circuitikz, graphicx}
\usepackage{centernot, xcolor, hhline, multirow, listings}
\usepackage{blkarray, booktabs, bigstrut, etoolbox}
\usepackage[normalem]{ulem}
\usepackage{bookmark}
\usetikzlibrary{math}
\usetikzlibrary{fit}

\pagestyle{fancy}

\usepackage{hyperref}
\hypersetup{
  colorlinks=true,
  linkcolor=black,
  filecolor=magenta,
  urlcolor=cyan,
}
%formatting
\newcommand{\bld}{\textbf}
\newcommand{\itl}{\textit}
\newcommand{\uln}{\underline}

%math word symbols
\newcommand{\bb}{\mathbb}
\DeclareMathOperator{\tif}{~\text{if}~}
\DeclareMathOperator{\tand}{~\text{and}~}
\DeclareMathOperator{\tbut}{~\text{but}~}
\DeclareMathOperator{\tor}{~\text{or}~}
\DeclareMathOperator{\tsuchthat}{~\text{such that}~}
\DeclareMathOperator{\tsince}{~\text{since}~}
\DeclareMathOperator{\twhen}{~\text{when}~}
\DeclareMathOperator{\twhere}{~\text{where}~}
\DeclareMathOperator{\tfor}{~\text{for}~}
\DeclareMathOperator{\tthen}{~\text{then}~}
\DeclareMathOperator{\tto}{~\text{to}~}

%display shortcut
\DeclareMathOperator{\dstyle}{\displaystyle}
\DeclareMathOperator{\sstyle}{\scriptstyle}

%linear algebra
\DeclareMathOperator{\id}{\bld{id}}
\DeclareMathOperator{\vecspan}{\text{span}}

%discrete math - integer properties
\DeclareMathOperator{\tdiv}{\text{div}}
\DeclareMathOperator{\tmod}{\text{mod}}
\DeclareMathOperator{\lcm}{\text{lcm}}

%augmented matrix environment
\newenvironment{apmatrix}[2]{%
  \left(\begin{array}{@{~}*{#1}{c}|@{~}*{#2}{c}}
    }{
  \end{array}\right)
}
\newenvironment{abmatrix}[2]{%
  \left[\begin{array}{@{~}*{#1}{c}|@{~}*{#2}{c}}
      }{
    \end{array}\right]
}

%lists
\newcommand{\bitem}[1]{\item[\bld{#1.}]}
\newcommand{\bbitem}[2]{\item[\bld{#1.}] \bld{#2}}
\newcommand{\biitem}[2]{\item[\bld{#1.}] \itl{#2}}
\newcommand{\iitem}[1]{\item[\itl{#1.}]}
\newcommand{\iiitem}[2]{\item[\itl{#1.}] \bld{#2}}
\newcommand{\btitem}[2]{\item[\bld{#1.}] \texttt{#2}}

%homework
\newcommand{\question}[2]{\noindent {\large\bld{#1}} #2 \qline}
\newcommand{\qitem}[3]{\item[\bld{#1.}] \itl{#2} #3 \qdash}

\newcommand{\qline}{~\newline\noindent\textcolor[RGB]{200,200,200}{\rule[0.5ex]{\linewidth}{0.2pt}}}
\newcommand{\qdash}{~\newline\noindent\textcolor[RGB]{200,200,200}{\hdashrule[0.5ex]{\linewidth}{0.2pt}{2pt}}}

\lhead{Linear Algebra}
\chead{Homework 4}
\rhead{Peter Schaefer}

\begin{document}

\section*{Homework 4}

\subsection*{Problem 9}
Let $V = \bb{R}^3$ be a vector space with standard addition and scalar multiplication. Use Theorem 3 of Lecture Note 8 to determine whether the following sets are subspaces of $V$.
\begin{enumerate}
  \biitem{b}{$W$ is the set of vectors of the form $(a,1,1)$}
  \begin{proof}
    Axiom 1: Consider $\vec{a},\vec{b} \in W$ where $\vec{a} = (a,1,1) \tand \vec{b} = (b,1,1) \tfor a,b \in \bb{R}$.
    \begin{align*}
      \vec{a} \oplus \vec{b} & = (a,1,1) \oplus (b,1,1) = (a+b,1+1,1+1) \\
                             & = (a+b,2,2) \not \in W
    \end{align*}
    Therefore $W$ is \bld{not} closed under addition. \\
    Since Axiom 1 does not hold for $W$, $W$ is not a subspace of $V$.
  \end{proof}
  \biitem{c}{$W$ is the set of vectors of the form $(a,b,c)$, where $b = a + c$}
  \begin{proof}
    Axiom 1: Consider $\vec{v},\vec{u} \in W$ where $\vec{v} = (a_1,b_1,c_1) \tand \vec{u} = (a_2,b_2,c_2) \tfor a_1,a_2,b_1,b_2,c_1,c_2 \in \bb{R}, \twhere b_1 = a_1 + c_1 \tand b_2 = a_2 + c_2$.
    \begin{align*}
      \vec{v} \oplus \vec{u} & = (a_1,b_1,c_1) \oplus (a_2,b_2,c_2) = (a_1+a_2,b_1+b_2,c_1+c_2)   \\
                             & \in W                                                              \\ \\
      b_1 + b_2              & = (a_1 + c_1) + (a_2 + c_2) = (a_1 + a_2) + (c_1 + c_2)~\checkmark
    \end{align*}
    Therefore $W$ is closed under addition. \\
    Axiom 6: Consider $\vec{v} = (a,b,c) \tsuchthat a,b,c \in \bb{R} \tand k \in \bb{R}$. Let $b = a+c$.
    \begin{align*}
      k \odot \vec{v} & = k \odot (a,b,c) = k \odot (a,b,c) = (ka, kb, kc) \\
                      & \in W                                              \\ \\
      kb              & = k(a+c) = ka + kc~\checkmark
    \end{align*}
    Therefore $W$ is closed under scalar multiplication. \\
    Since Axiom 1 and Axiom 6 hold for $W$, $\oplus \tand \odot$ are inherited from $V$, and $W \subseteq V$, through the use of Theorem 3, $W$ is a subspace of $V$.
  \end{proof}
  \biitem{d}{$W$ is the set of vectors of the form $(a,b,0)$}
  \begin{proof}
    Axiom 1: Consider $\vec{v},\vec{u} \in W$ where $\vec{v} = (a_1,b_1,0) \tand \vec{u} = (a_2,b_2,0) \tfor a_1,a_2,b_1,b_2 \in \bb{R}$.
    \begin{align*}
      \vec{v} \oplus \vec{u} & = (a_1,b_1,0) \oplus (a_2,b_2,0) = (a_1+a_2,b_1+b_2,0+0) = (a_1+a_2,b_1+b_2,0) \\
                             & \in W~\text{since it takes the form}~(a,b,0)
    \end{align*}
    Therefore $W$ is closed under addition. \\
    Axiom 6: Consider $\vec{v} = (a,b,0) \tsuchthat a,b \in \bb{R} \tand k \in \bb{R}$.
    \begin{align*}
      k \odot \vec{v} & = k \odot (a,b,0) = (ka,kb,k0) = (ka,kb,0)   \\
                      & \in W~\text{since it takes the form}~(a,b,0)
    \end{align*}
    Therefore $W$ is closed under scalar multiplication. \\
    Since Axiom 1 and Axiom 6 hold for $W$, $\oplus \tand \odot$ are inherited from $V$, and $W \subseteq V$, through the use of Theorem 3, $W$ is a subspace of $V$.
  \end{proof}
\end{enumerate}

\subsection*{Problem 10}
Let $V = P_3$ be the vector space of all polynomials with degree \itl{up to} 3, with standard addition and scalar multiplication. Use Theorem 3 of Lecture Note 8 to determine whether the following sets are subspaces of $V$.
\begin{enumerate}
  \biitem{b}{$W$ is the set of polynomials $a_0+a_1x+a_2x^2+a_3x^3$ for which $a_0+a_1+a_2+a_3=0$}
  \begin{proof}
    Axiom 1: Consider $\vec{a},\vec{b} \in W$ where $\vec{a} = a_0+a_1x+a_2x^2+a_3x^3 \tand \vec{b} = b_0+b_1x+b_2x^2+b_3x^3 \twhere a_{0-3}, b_{0-3} \in \bb{R}$. \\
    Let $a_0+a_1+a_2+a_3=0 \tand b_0+b_1+b_2+b_3=0$.
    \begin{align*}
      \vec{a} \oplus \vec{b} & = (a_0+a_1x+a_2x^2+a_3x^3) \oplus (b_0+b_1x+b_2x^2+b_3x^3) \\
                             & = a_0+a_1x+a_2x^2+a_3x^3 + b_0+b_1x+b_2x^2+b_3x^3          \\
                             & = (a_0+b_0) + (a_1+b_1)x + (a_2+b_2)x^2 + (a_3+b_3)x^3     \\
                             & \in W
    \end{align*}
    \begin{align*}
      (a_0+b_0) + (a_1+b_1) + (a_2+b_2) + (a_3+b_3) & = (a_0 + a_1 + a_2 + a_3) + (b_0 + b_1 + b_2 + b_3) \\
                                                    & = 0 + 0 = 0~\checkmark
    \end{align*}
    Therefore $W$ is closed under addition. \\
    Axiom 6: Consider $\vec{a} \in W$ where $\vec{a} = a_0+a_1x+a_2x^2+a_3x^3 \tsuchthat a_{0-3} \in \bb{R} \tand k \in \bb{R}$. \\
    Let $a_0+a_1+a_2+a_3=0$.
    \begin{align*}
      k \odot \vec{a} & = k \odot (a_0+a_1x+a_2x^2+a_3x^3) = k(a_0+a_1x+a_2x^2+a_3x^3)           \\
                      & = ka_0+ka_1x+ka_2x^2+ka_3x^3                                             \\
                      & \in W                                                                    \\ \\
      ka_0            & +ka_1x+ka_2x^2+ka_3x^3 = k(a_0+a_1x+a_2x^2+a_3x^3) = k(0) = 0~\checkmark
    \end{align*}
    Therefore $W$ is closed under scalar multiplication. \\
    Since Axiom 1 and Axiom 6 hold for $W$, $\oplus \tand \odot$ are inherited from $V$, and $W \subseteq V$, through the use of Theorem 3, $W$ is a subspace of $V$.
  \end{proof}
  \biitem{c}{$W$ is the set of polynomials $a_0+a_1x+a_2x^2+a_3x^3$ in which $a_0,a_1,a_2, \tand a_3$ are integers.}
  \begin{proof}
    Axiom 6: Consider $\vec{a} \in W$ where $\vec{a} = 1+1x+1x^2+1x^3 \tand k = 0.66$.
    \begin{align*}
      k \odot \vec{a} & = 0.66 \odot (1+1x+1x^2+1x^3) = 0.66(1+1x+1x^2+1x^3) \\
                      & = 0.66+0.66x+0.66x^2+0.66x^3                         \\
                      & \not \in W, \tsince 0.66 \not \in \bb{Z}
    \end{align*}
    Therefore $W$ is \bld{not} closed under scalar multiplication. \\
    Since Axiom 6 does not hold for $W$, $W$ is not a subspace of $V$.
  \end{proof}
\end{enumerate}

\subsection*{Problem 11}
Let $V = F(-\infty, \infty)$ be the vector space of all functions from $\bb{R}$ to $\bb{R}$, with standard addition and scalar multiplication. Use Theorem 3 of Lecture Note 8 to determine whether the following sets are subspaces of $V$.
\begin{enumerate}
  \biitem{b}{$W$ is the set of functions $f$ in $F(-\infty, \infty)$ for which $f(0) = 1$.}
  \begin{proof}
    Axiom 1: Consider $\vec{f},\vec{g} \in W$ where $\vec{f}(x) = e^x \tand \vec{g}(x) = e^x$.
    \begin{align*}
      (\vec{f} \oplus \vec{g})(x) & = \vec{f}(x) + \vec{g}(x) = e^x + e^x = 2e^x \not \in W \\
      \twhen x                    & = 0: 2e^0 = 2 \cdot 1 = 2 \neq 1
    \end{align*}
    Therefore $W$ is \bld{not} closed under addition. \\
    Since Axiom 1 does not hold for $W$, $W$ is not a subspace of $V$.
  \end{proof}
  \biitem{c}{$W$ is the set of functions $\vec{f}$ in $F(-\infty, \infty)$ for which $f(-x) = x$, $W$}
  \begin{proof}
    Axiom 1: Consider $\vec{f}, \vec{g} \in W$.
    \begin{align*}
      (\vec{f} \oplus \vec{g})(x) & = \vec{f}(x) + \vec{g}(x) \\
      \twhen \text{plugging in $-x$}:                         \\
      \vec{f}(-x) + \vec{g}(-x)   & = x + x = 2x              \\
                                  & \neq x \tif x \neq 0
    \end{align*}
    Therefore $W$ is \bld{not} closed under addition. \\
    Since Axiom 1 does not hold for $W$, $W$ is not a subspace of $V$.
  \end{proof}
\end{enumerate}

\subsection*{Problem 13}
Let $V$ be a vector space. Let $I$ be a nonempty set (often called the "index set"), and let $W_i$ be a subspace of $V$ for all $i \in I$. Prove that $\displaystyle \bigcap_{i \in I} W_i$, is a subspace of V.
\begin{proof}
  Axiom 1: Consider $\vec{u}, \vec{v} \in \displaystyle \bigcap_{i \in I} W_i$. This implies the following:
  \begin{align*}
    \vec{u} & \in W_i~~\forall~i \in I \\
    \vec{v} & \in W_i~~\forall~i \in I
  \end{align*}
  Since $\forall~i \in I,~W_i$ is a subspace of $V$, by Axiom 1 for $W_i$
  \[
    \forall i \in I,~\vec{u} \oplus \vec{v} \in W_i.
  \]
  This statement is equivalent to
  \[
    \vec{u} \oplus \vec{v} \in \bigcap_{i \in I} W_i
  \]
  Therefore $\displaystyle \bigcap_{i \in I} W_i$ is closed under addition. \\
  Axiom 6: Consider $\vec{v} \in \displaystyle \bigcap_{i \in I} W_i$. This implies the following:
  \[
    \vec{v} \in W_i~~\forall~i \in I
  \]
  Since $\forall~i \in I,~W_i$ is a subspace of $V$, by Axiom 6 for $W_i$
  \[
    \forall i \in I,~k \odot \vec{v} \in W_i.
  \]
  This statement is equivalent to
  \[
    k \odot \vec{v} \in \bigcap_{i \in I} W_i
  \]
  Therefore $\displaystyle \bigcap_{i \in I} W_i$ is closed under addition. \\
  Since Axiom 1 and Axiom 6 hold for $\displaystyle \bigcap_{i \in I} W_i$, through Theorem 3 $\displaystyle \bigcap_{i \in I} W_i$ is a subspace of $V$.
\end{proof}

\end{document}