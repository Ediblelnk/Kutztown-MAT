\documentclass{article}
\usepackage[margin=1in]{geometry}
\usepackage{amsmath, amsthm, amssymb, fancyhdr, tikz, circuitikz, graphicx}
\usepackage{centernot, xcolor, hhline, multirow, listings}
\usepackage{blkarray, booktabs, bigstrut, etoolbox}
\usepackage[normalem]{ulem}
\usepackage{bookmark}
\usetikzlibrary{math}
\usetikzlibrary{fit}

\pagestyle{fancy}

\usepackage{hyperref}
\hypersetup{
  colorlinks=true,
  linkcolor=black,
  filecolor=magenta,
  urlcolor=cyan,
}
%formatting
\newcommand{\bld}{\textbf}
\newcommand{\itl}{\textit}
\newcommand{\uln}{\underline}

%math word symbols
\newcommand{\bb}{\mathbb}
\DeclareMathOperator{\tif}{~\text{if}~}
\DeclareMathOperator{\tand}{~\text{and}~}
\DeclareMathOperator{\tbut}{~\text{but}~}
\DeclareMathOperator{\tor}{~\text{or}~}
\DeclareMathOperator{\tsuchthat}{~\text{such that}~}
\DeclareMathOperator{\tsince}{~\text{since}~}
\DeclareMathOperator{\twhen}{~\text{when}~}
\DeclareMathOperator{\twhere}{~\text{where}~}
\DeclareMathOperator{\tfor}{~\text{for}~}
\DeclareMathOperator{\tthen}{~\text{then}~}
\DeclareMathOperator{\tto}{~\text{to}~}

%display shortcut
\DeclareMathOperator{\dstyle}{\displaystyle}
\DeclareMathOperator{\sstyle}{\scriptstyle}

%linear algebra
\DeclareMathOperator{\id}{\bld{id}}
\DeclareMathOperator{\vecspan}{\text{span}}

%discrete math - integer properties
\DeclareMathOperator{\tdiv}{\text{div}}
\DeclareMathOperator{\tmod}{\text{mod}}
\DeclareMathOperator{\lcm}{\text{lcm}}

%augmented matrix environment
\newenvironment{apmatrix}[2]{%
  \left(\begin{array}{@{~}*{#1}{c}|@{~}*{#2}{c}}
    }{
  \end{array}\right)
}
\newenvironment{abmatrix}[2]{%
  \left[\begin{array}{@{~}*{#1}{c}|@{~}*{#2}{c}}
      }{
    \end{array}\right]
}

%lists
\newcommand{\bitem}[1]{\item[\bld{#1.}]}
\newcommand{\bbitem}[2]{\item[\bld{#1.}] \bld{#2}}
\newcommand{\biitem}[2]{\item[\bld{#1.}] \itl{#2}}
\newcommand{\iitem}[1]{\item[\itl{#1.}]}
\newcommand{\iiitem}[2]{\item[\itl{#1.}] \bld{#2}}
\newcommand{\btitem}[2]{\item[\bld{#1.}] \texttt{#2}}

%homework
\newcommand{\question}[2]{\noindent {\large\bld{#1}} #2 \qline}
\newcommand{\qitem}[3]{\item[\bld{#1.}] \itl{#2} #3 \qdash}

\newcommand{\qline}{~\newline\noindent\textcolor[RGB]{200,200,200}{\rule[0.5ex]{\linewidth}{0.2pt}}}
\newcommand{\qdash}{~\newline\noindent\textcolor[RGB]{200,200,200}{\hdashrule[0.5ex]{\linewidth}{0.2pt}{2pt}}}

\lhead{Linear Algebra}
\chead{Test 1}
\rhead{Peter Schaefer}

\begin{document}

\subsection*{Problem 1.}
Let $V = \bb{R}^+ \times \bb{R}$ be a set. In other words, every element of $V$ is in the form $(u_1, u_2), \twhere u_1$ is a positive real number and $u_2 \in \bb{R}$. For all $(u_1,u_2) \tand (v_1,v_2) \in V$, and for all $k \in \bb{R}$,
\[
  (u_1,u_2) \oplus (v_1,v_2) = (2u_1v_1, u_2 + v_2 - 3) \tand k \odot (u_1,v_1) = (u_1^k, ku_2).
\]
Verify the axioms 4, 5, and 7.

\begin{enumerate}
  \bitem{Ax4}
  \begin{proof}
    Consider $\vec{u}, \vec{v} \in V \tsuchthat \vec{u} = (u_1, u_2) \tand \vec{v} = (\frac{1}{2}, 3)$. ($u_1$ is positive real number).
    \begin{align*}
      \vec{u} \oplus \vec{v} & = (u_1, u_2) \oplus \left(\frac{1}{2}, 3\right) = \left(2u_1\frac{1}{2}, u_2 + 3 - 3\right) \\
                             & = (u_1, u_2) = \vec{u}                                                                      \\
      \vec{v} \oplus \vec{u} & = \left(\frac{1}{2}, 3\right) \oplus (u_1, u_2) = \left(2\frac{1}{2}u_1, 3 + u_2 - 3\right) \\
                             & = (u_1, u_2) = \vec{u}                                                                      \\
    \end{align*}
    Since $\vec{u} \oplus \vec{v} = \vec{u} \tand \vec{v} \oplus \vec{u} = \vec{u}$ for all $\vec{u} \in V$, therefore $\vec{v} = (\frac{1}{2}, 3)$ is the additive identity, $\id,\tfor V$.

    $\therefore$ additive identity exists for $V$.
  \end{proof}
  \bitem{Ax5}
  \begin{proof}
    Consider $\vec{u}, \vec{v} \in V \tsuchthat \vec{u} = (u_1, u_2) \tand \vec{v} = (\frac{1}{4u_1},6-u_2)$. Since by definition $u_1$ is a positive real number, $\frac{1}{4u_1}$ will always be defined and positive.
    \begin{align*}
      \vec{u} \oplus \vec{v} & = (u_1, u_2) \oplus \left(\frac{1}{4u_1}, 6-u_2\right) = \left(2u_1\frac{1}{4u_1}, u_2 + (6 - u_2) - 3\right) \\
                             & = \left(\frac{2}{4}\cdot\frac{u_1}{u_1}, u_2 - u_2 + 6 - 3\right) = \left(\frac{1}{2}, 3\right) = \id         \\
      \vec{v} \oplus \vec{u} & = \left(\frac{1}{4u_1}, 6-u_2\right) \oplus (u_1, u_2) = \left(2\frac{1}{4u_1}u_1, (6 - u_2) + u_2 - 3\right) \\
                             & = \left(\frac{2}{4}\cdot\frac{u_1}{u_1}, u_2 - u_2 + 6 - 3\right) = \left(\frac{1}{2}, 3\right) = \id
    \end{align*}
    $\therefore$ additive inverse exists for all $\vec{u} \in V$.
  \end{proof}
  \bitem{Ax7}
  \begin{proof}
    Consider $k \in \bb{R} \tand (u_1,u_2), (v_1,v_2) \in V$.
    \begin{align*}
      k \odot ((u_1,u_2) \oplus (v_1, v_2))       & = k \odot (2u_1v_1, u_2+v_2-3)                 \\
                                                  & = ((2u_1v_1)^k,k(u_2+v_2-3))                   \\
                                                  & = (4u_1^kv_1^k,ku_2+kv_2-3k)                   \\ \\
      k \odot (u_1,u_2) \oplus k \odot (v_1, v_2) & = (u_1^k,ku_2) \oplus (v_1^k,kv_2)             \\
                                                  & = (2u_1^kv_1^k,ku_2+kv_2-3)                    \\ \\
      (4u_1^kv_1^k,ku_2+kv_2-3k)                  & \neq (2u_1^kv_1^k,ku_2+kv_2-3) \twhen k \neq 1
    \end{align*}
    Since $k \odot ((u_1,u_2) \oplus (v_1, v_2))$ does not always equal $k \odot (u_1,u_2) \oplus k \odot (v_1, v_2)$, Axiom 7 does not hold for $V$.
  \end{proof}
\end{enumerate}
\newpage

\subsection*{Problem 2.}
Let $V$ be a set with a binary operator $\oplus$ defined, so that Axioms (1), (3), and (4) hold for $V$ (note that other axioms may not hold). Let $\vec{v} \in V$. Prove that \bld{if} $\vec{v}$ has an additive inverse, then this additive inverse is unique. (\itl{Hint}: Let $\vec{w} \tand \vec{x}$ be two different additive inverses of $\vec{v}$. Show that this will lead to a contradiction.)

\begin{proof}
  Let $\vec{w}, \vec{x}, \vec{v} \in V \tsuchthat \vec{w} \tand \vec{x}$ are two different additive inverses of $\vec{v}$, This implies that $\vec{w} \neq \vec{x}$.
  \begin{align*}
    \vec{v} \oplus \vec{w}                  & = \id                &  & \text{def. of additive inverse}  \\
    \vec{x} \oplus (\vec{v} \oplus \vec{w}) & = \vec{x} \oplus \id                                       \\
    (\vec{x} \oplus \vec{v}) \oplus \vec{w} & = \vec{x} \oplus \id &  & \text{axiom 3}                   \\
    \id \oplus \vec{w}                      & = \vec{x} \oplus \id &  & \text{def. of additive inverse}  \\
    \vec{w}                                 & = \vec{x}            &  & \text{def. of additive identity}
  \end{align*}
  However, $\vec{w} = \vec{x}$ contradicts our assertion that $\vec{w} \neq \vec{x}$. Therefore, through contradiction, if $\vec{v}$ has an additive inverse, then this additive inverse is unique.
\end{proof}

\subsection*{Problem 3.}
Let $V = P_3$, i.e., the set of all polynomials of degree up to 3, with standard addition and scalar multiplication. Let
\[
  W = \left\{a_0 + a_1x + a_2x^2 + a_3x^3 \in V : a_0 \cdot a_1 = 0\right\}.
\]
Verify whether $W$ is a subspace of $V$.

\begin{proof}
  According to Theorem 3 from Lecture 10, assuming that addition and scalar multiplication in $W$ are inherited from $V$, $W$ is a subspace of $V$ if and only if Axioms 1 and 6 hold for $W$.
  \begin{enumerate}
    \bitem{Ax1}
    \begin{proof}
      Let $\vec{a}, \vec{b} \in W \tsuchthat \vec{a} = 0 + 1x + 0x^2 + 0x^3 \tand \vec{b} = 1 + 0x + 0x^2 + 0x^3$. Now consider $\vec{a} \oplus \vec{b}$:
      \begin{align*}
        \vec{a} \oplus \vec{b} & = (0 + 1x + 0x^2 + 0x^3) \oplus (1 + 0x + 0x^2 + 0x^3) \\
                               & = 0 + 1x + 0x^2 + 0x^3 + 1 + 0x + 0x^2 + 0x^3          \\
                               & = (0+1) + (1+0)x + (0+0)x^2 + (0+0)x^3                 \\
                               & = 1 + 1x + 0x^2 + 0x^3                                 \\ \\
        1 \cdot 1              & = 1 \neq 0
      \end{align*}
      Therefore $\vec{a} \oplus \vec{b} \not \in W$, even though $\vec{a} \in W \tand \vec{b} \in W$.

      This means that $W$ is not closed under addition.
    \end{proof}
  \end{enumerate}
  Since Axiom 1 does not hold for $W$, $W$ cannot be a subspace of $V$.
\end{proof}

\newpage

\subsection*{Problem 5.}
Let
\[
  A = \begin{pmatrix}2 & 1 \\ 4 & 0\end{pmatrix}, B = \begin{pmatrix}1 & -1 \\ 3 & 4\end{pmatrix}, \tand C = \begin{pmatrix}3 & 2  \\ 5 & -4\end{pmatrix}.
\]
Express $\dstyle M = \begin{pmatrix}1 & 2 \\ 3 & 4\end{pmatrix}$ as a linear combination of $A,~B,\tand C$. Use Gauss-Jordan elimination.

\begin{proof}
  Let $k_1, k_2, k_3 \in \bb{R} \tsuchthat k_1A \oplus k_2B \oplus k_3C = M$.

  That is, $k_1\begin{pmatrix}2 & 1 \\ 4 & 0\end{pmatrix} \oplus k_2\begin{pmatrix}1 & -1 \\ 3 & 4\end{pmatrix} \oplus k_3\begin{pmatrix}3 & 2  \\ 5 & -4\end{pmatrix} = \begin{pmatrix}1 & 2 \\ 3 & 4\end{pmatrix}$. From this equation, we get a linear system of equations:
  \begin{align*}
    2k_1 + 1k_2 + 3k_3 = 1 \\
    1k_1 - 1k_2 + 2k_3 = 2 \\
    4k_1 + 3k_2 + 5k_3 = 3 \\
    0k_1 + 4k_2 - 4k_3 = 4 \\
  \end{align*}
  \[
    \begin{amatrix}{3}
      2 & 1  & 3  & 1 \\
      1 & -1 & 2  & 2 \\
      4 & 3  & 5  & 3 \\
      0 & 4  & -4 & 4
    \end{amatrix} \xrightarrow[(-4,-2,-6,-2)]{R_3 - 2R_1}
    \begin{pmatrix}
      2 & 1  & 3  & 1 \\
      1 & -1 & 2  & 2 \\
      0 & 1  & -1 & 1 \\
      0 & 4  & -4 & 4
    \end{pmatrix} \xrightarrow[(-2,2,-4,-4)]{R_1 - 2R_2}
    \begin{pmatrix}
      0 & 3  & -1 & -3 \\
      1 & -1 & 2  & 2  \\
      0 & 1  & -1 & 1  \\
      0 & 4  & -4 & 4
    \end{pmatrix} \xrightarrow[(0,1,-1,1)]{R_2 + R_3}
  \]
  \[
    \begin{pmatrix}
      0 & 3 & -1 & -3 \\
      1 & 0 & 1  & 3  \\
      0 & 1 & -1 & 1  \\
      0 & 4 & -4 & 4
    \end{pmatrix} \xrightarrow[(0,-3,3,-3)]{R_1 - 3R_3}
    \begin{pmatrix}
      0 & 0 & 2  & -6 \\
      1 & 0 & 1  & 3  \\
      0 & 1 & -1 & 1  \\
      0 & 4 & -4 & 4
    \end{pmatrix} \xrightarrow[(0,-4,4,-4)]{R_4 - 4R_1}
    \begin{pmatrix}
      0 & 0 & 2  & -6 \\
      1 & 0 & 1  & 3  \\
      0 & 1 & -1 & 1  \\
      0 & 0 & 0  & 0
    \end{pmatrix} \xrightarrow{\frac{1}{2}R_1}
  \]
  \[
    \begin{pmatrix}
      0 & 0 & 1  & -3 \\
      1 & 0 & 1  & 3  \\
      0 & 1 & -1 & 1  \\
      0 & 0 & 0  & 0
    \end{pmatrix} \xrightarrow[(0,0,-1,3)]{R_2 - R_1}
    \begin{pmatrix}
      0 & 0 & 1  & -3 \\
      1 & 0 & 0  & 6  \\
      0 & 1 & -1 & 1  \\
      0 & 0 & 0  & 0
    \end{pmatrix} \xrightarrow{R_3 + R_1}
    \begin{pmatrix}
      0 & 0 & 1 & -3 \\
      1 & 0 & 0 & 6  \\
      0 & 1 & 0 & -2 \\
      0 & 0 & 0 & 0
    \end{pmatrix} \xrightarrow[R_2 \leftrightarrow R_3]{R_1 \leftrightarrow R_2}
  \]
  \[
    \begin{amatrix}{3}
      1 & 0 & 0 & 6  \\
      0 & 1 & 0 & -2 \\
      0 & 0 & 1 & -3 \\
      0 & 0 & 0 & 0
    \end{amatrix}
  \]
  This augmented matrix represents the following equations:
  \begin{align*}
    k_1 = 6  \\
    k_2 = -2 \\
    k_3 = -3
  \end{align*}
  This means that $\begin{pmatrix}1 & 2 \\ 3 & 4\end{pmatrix}$ is a linear combination of $A,~B,\tand C$, when $k_1 = 6,~k_2 = -3, \tand k_3 = -3$
\end{proof}


\newpage

\subsection*{Problem 6.}
Decide whether
\[
  \vec{u} = 2 + x + 4x^2,~\vec{v} = 1 - x - 7x^2, \tand \vec{w} = 3 + 2x + 9x^2.
\]
spans $P_2$. Justify your answer using Gauss-Jordan elimination.

\begin{proof}
  Let $\vec{y} = y_0 + y_1x + y_2x^2$, and let $k_1,k_2,k_3 \in \bb{R} \tsuchthat k_1\vec{u} + k_2\vec{v} + k_3\vec{w} = \vec{y}$. In other words,
  \[
    k_1(2 + x + 4x^2) + k_2(1 - x - 7x^2) + k_3(3 + 2x + 9x^2) = y_0 + y_1x + y_2x^2.
  \]
  From this equation, we get the following system of linear equations.
  \begin{align*}
    2k_1 + 1k_2 + 3k_3 = y_0 \\
    1k_1 - 1k_2 + 2k_3 = y_1 \\
    4k_1 - 7k_2 + 9k_3 = y_2
  \end{align*}
  \[
    \begin{amatrix}{3}
      2 & 1  & 3 & y_0 \\
      1 & -1 & 2 & y_1 \\
      4 & -7 & 9 & y_2
    \end{amatrix} \xrightarrow[(-4,-2,-6,-2y_0)]{R_3 - 2R_1}
    \begin{pmatrix}
      2 & 1  & 3 & y_0       \\
      1 & -1 & 2 & y_1       \\
      0 & -9 & 3 & y_2 -2y_0
    \end{pmatrix} \xrightarrow[(-2,2,-4,-2y_1)]{R_1 - 2R_2}
    \begin{pmatrix}
      0 & 3  & -1 & y_0 -2y_1 \\
      1 & -1 & 2  & y_1       \\
      0 & -9 & 3  & y_2 -2y_0
    \end{pmatrix}
  \]
  \[
    \xrightarrow[(0,9,-3,2y_0 -4y_1)]{R_3 + 2R_3}
    \begin{pmatrix}
      0 & 3  & -1 & y_0 -2y_1               \\
      1 & -1 & 2  & y_1                     \\
      0 & 0  & 0  & y_2 - 2y_0 + 2y_0 -4y_1
    \end{pmatrix} =
    \begin{pmatrix}
      0 & 3  & -1 & y_0 -2y_1 \\
      1 & -1 & 2  & y_1       \\
      0 & 0  & 0  & y_2 -4y_1
    \end{pmatrix}
  \]
  The last row represents the equation $0 = y_2 -4y_1$. If $0 \neq y_2 -4y_1$, then there is no solution to the system of linear equations. Therefore, there exists $\vec{y} \in P_2$ that cannot be spanned by $\{\vec{u}, \vec{v}, \vec{w}\}$.
\end{proof}

\newpage

\subsection*{Problem 9.}
Let $V$ be a real vector space. Prove that $V$ cannot have exactly 3 elements.

\begin{proof}
  Let $V$ be a real vector space containing exactly 3 elements. Let the first element of $V$ be $\id$, which is required to be in $V$ through Axiom 4. Next, let the second element of $V$ be $\vec{v}$ (Note that $\vec{v}$ cannot be $\id$, since the additive identity is unique). Finally, let the second element of $V$ be $-\vec{v}$, the additive inverse of $\vec{v}$, which is required to be in $V$ through Axiom 5. Therefore, we have $V = \{\id, \vec{v}, -\vec{v}\}$, where $\id \neq \vec{v} \tand \id \neq -\vec{v} \tand \vec{v} \neq -\vec{v}$.

  Now consider $\vec{v} \oplus \vec{v}$:
  \begin{enumerate}
    \bitem{Case 1} $\vec{v} \oplus \vec{v} = \id$
    \begin{proof}
      Consider $\vec{v} \oplus \vec{v} \oplus -\vec{v}$.
      \begin{align*}
        \vec{v} \oplus \vec{v} \oplus -\vec{v} & = \vec{v} \oplus (\vec{v} \oplus -\vec{v}) &  & \text{axiom 3}                   \\
                                               & = \vec{v} \oplus \id                       &  & \text{def. of additive inverse}  \\
                                               & = \vec{v}                                  &  & \text{def. of additive identity} \\ \\
        \vec{v} \oplus \vec{v} \oplus -\vec{v} & = (\vec{v} \oplus \vec{v}) \oplus -\vec{v} &  & \text{axiom 3}                   \\
                                               & = \id \oplus -\vec{v}                      &  & \text{assertion}                 \\
                                               & = -\vec{v}                                 &  & \text{def. of additive identity}
      \end{align*}
      This implies that $\vec{v} = -\vec{v}$, which contradicts our assertion that $\vec{v} \neq -\vec{v}$. Therefore, $\vec{v} \oplus \vec{v} \neq \id$.
    \end{proof}
    \bitem{Case 2} $\vec{v} \oplus \vec{v} = \vec{v}$
    \begin{proof}
      Consider $\vec{v} \oplus \vec{v} \oplus -\vec{v}$.
      \begin{align*}
        \vec{v} \oplus \vec{v} \oplus -\vec{v} & = \vec{v} \oplus (\vec{v} \oplus -\vec{v}) &  & \text{axiom 3}                   \\
                                               & = \vec{v} \oplus \id                       &  & \text{def. of additive inverse}  \\
                                               & = \vec{v}                                  &  & \text{def. of additive identity} \\ \\
        \vec{v} \oplus \vec{v} \oplus -\vec{v} & = (\vec{v} \oplus \vec{v}) \oplus -\vec{v} &  & \text{axiom 3}                   \\
                                               & = \vec{v} \oplus -\vec{v}                  &  & \text{assertion}                 \\
                                               & = \id                                      &  & \text{def. of additive identity}
      \end{align*}
      This implies that $\id = \vec{v}$, which contradicts our assertion that $\id \neq \vec{v}$. Therefore, $\vec{v} \oplus \vec{v} \neq \vec{v}$
    \end{proof}
    \bitem{Case 3} $\vec{v} \oplus \vec{v} = -\vec{v}$
    \begin{proof}
      Consider the assertion that $\vec{v} \oplus \vec{v} = -\vec{v}$
      \begin{align*}
        \vec{v} \oplus \vec{v}                 & = -\vec{v}           &  & \text{assertion} \\
        1 \odot \vec{v} \oplus 1 \odot \vec{v} & = -\vec{v}           &  & \text{axiom 10}  \\
        1 \odot \vec{v} \oplus 1 \odot \vec{v} & = (-1) \odot \vec{v} &  & \text{theorem C} \\
        (1+1) \odot \vec{v}                    & = (-1) \odot \vec{v} &  & \text{axiom 8}
      \end{align*}
      This last equation implies that $1+1 = -1$, which is a contradiction. Therefore, $\vec{v} \oplus \vec{v} \neq \vec{v}$
    \end{proof}
  \end{enumerate}
  Every possible result in $V$ for $\vec{v} \oplus \vec{v}$ leads to a contradiction. Therefore, $\vec{v} \oplus \vec{v} \not \in V$, meaning that $V$ is not closed under addition. This means that $V$, containing exactly three elements, cannot be a real vector space.
\end{proof}
\newpage
\end{document}