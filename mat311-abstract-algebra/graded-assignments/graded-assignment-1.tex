\documentclass{article}
\usepackage[margin=1in]{geometry}
\usepackage{amsmath, amsthm, amssymb, fancyhdr, tikz, circuitikz, graphicx}
\usepackage{centernot, xcolor, hhline, multirow, listings, dashrule}
\usepackage{blkarray, booktabs, bigstrut, etoolbox, extarrows}
\usepackage[normalem]{ulem}
\usepackage{bookmark}
\usepackage{enumerate}          %These two package give custom labels to a list
\usepackage[shortlabels]{enumitem}
\usetikzlibrary{math}
\usetikzlibrary{fit}

\pagestyle{fancy}

\usepackage{hyperref}
\hypersetup{
    colorlinks=true,
    linkcolor=black,
    filecolor=magenta,
    urlcolor=cyan,
}
%formatting
\newcommand{\bld}{\textbf}
\newcommand{\itl}{\textit}
\newcommand{\uln}{\underline}

%math word symbols
\newcommand{\bb}{\mathbb}
\DeclareMathOperator{\tif}{~\text{if}~}
\DeclareMathOperator{\tand}{~\text{and}~}
\DeclareMathOperator{\tbut}{~\text{but}~}
\DeclareMathOperator{\tor}{~\text{or}~}
\DeclareMathOperator{\tsuchthat}{~\text{such that}~}
\DeclareMathOperator{\tsince}{~\text{since}~}
\DeclareMathOperator{\twhen}{~\text{when}~}
\DeclareMathOperator{\twhere}{~\text{where}~}
\DeclareMathOperator{\twith}{~\text{with}~}
\DeclareMathOperator{\tfor}{~\text{for}~}
\DeclareMathOperator{\tthen}{~\text{then}~}
\DeclareMathOperator{\tto}{~\text{to}~}
\DeclareMathOperator{\tin}{~\text{in}~}

%display shortcut
\DeclareMathOperator{\dstyle}{\displaystyle}
\DeclareMathOperator{\sstyle}{\scriptstyle}

%linear algebra
\DeclareMathOperator{\id}{\bld{id}}
\DeclareMathOperator{\vecspan}{\text{span}}
\DeclareMathOperator{\adj}{\text{adj}}

%discrete math - integer properties
\DeclareMathOperator{\tdiv}{\text{div}}
\DeclareMathOperator{\tmod}{\text{mod}}
\DeclareMathOperator{\lcm}{\text{lcm}}

%augmented matrix environment
\newenvironment{apmatrix}[2]{%
    \left(\begin{array}{@{~}*{#1}{c}|@{~}*{#2}{c}}
        }{
    \end{array}\right)
}
\newenvironment{abmatrix}[2]{%
    \left[\begin{array}{@{~}*{#1}{c}|@{~}*{#2}{c}}
            }{
        \end{array}\right]
}

\newenvironment{determinant}[1]{
    \left\lvert
    \begin{array}{@{~}*{#1}{c}}
        }{
    \end{array}
    \right\rvert
}

% graph theory
\DeclareMathOperator{\diam}{\text{diam}}
\newcommand{\comp}[1]{\overline{#1}}

% calculus iii
\newcommand{\vvec}{\overrightarrow}
\newcommand{\norm}[1]{\lVert #1 \rVert}

% abstract algebra
\newcommand{\struct}[1]{\langle #1 \rangle}

%lists
\newcommand{\bitem}[1]{\item[\bld{#1.}]}
\newcommand{\bbitem}[2]{\item[\bld{#1.}] \bld{#2}}
\newcommand{\biitem}[2]{\item[\bld{#1.}] \itl{#2}}
\newcommand{\iitem}[1]{\item[\itl{#1.}]}
\newcommand{\iiitem}[2]{\item[\itl{#1.}] \bld{#2}}
\newcommand{\btitem}[2]{\item[\bld{#1.}] \texttt{#2}}

%homework
\newcommand{\question}[2]{\noindent {\large\bld{#1}} #2 \qline}
\newcommand{\qitem}[3]{\item[\bld{#1.}] #2 \qdash \\ #3 \qdash}

\newcommand{\qline}{~\newline\noindent\textcolor[RGB]{200,200,200}{\rule[0.5ex]{\linewidth}{0.2pt}}}
\newcommand{\qdash}{~\newline\noindent\textcolor[RGB]{200,200,200}{\hdashrule[0.5ex]{\linewidth}{0.2pt}{2pt}}}

\newcommand{\assignment}{Graded Assignment \#1}

\lhead{MAT 311 Abstract Algebra}
\chead{\assignment}
\rhead{Peter Schaefer}

\begin{document}
\section*{\assignment}

\question{1}{[2 points each] Which of the following are binary operations on the given sets? If it is not an operation, explain why.}
\begin{enumerate}
    \qitem{(a)}{$S=\bb{R}^+ \twith a*b=a\ln b$}{
        A binary operation must be uniquely defined and closed.
        \begin{proof}
            Consider $a = 1 \tand b = \frac{1}{e}$. Both $a \tand b \in \bb{R}^+$, but
            \[
                a*b = a \ln b = 1 \ln \frac{1}{e} = 1 \cdot -1 = -1 \notin \bb{R}^+.
            \]
            Thus $S$ is not closed under $*$, and $*$ cannot be a binary operation.
        \end{proof}
    }
    \qitem{(b)}{$S=\bb{R} \twhere a*b$ is the root of the equation $x^2-a^2b^2=0$}{
        A binary operation must be uniquely defined and closed.
        \begin{proof}
            Consider $a = 2 \tand b = 1$.
            \begin{align*}
                a*b = x^2 - a^2b^2 & = 0     \\
                x^2 - 2^2 1^2      & =       \\
                x^2 - 4            & = 0     \\
                (x-2)(x+2)         & = 0     \\
                x                  & = \pm 2
            \end{align*}
            Since there are two solutions, $S$ is not uniquely defined under $*$, and $*$ cannot be a binary operation.
        \end{proof}
    }
\end{enumerate}

\question{2}{[2 points each] Consider the binary operation $*$ defined on $\bb{R}^+$ by $a*b=\frac{ab}{a+b+1}$}
\begin{enumerate}
    \qitem{(a)}{Is $*$ commutative? Explain.}{
        $*$ is commuatative.
        \begin{proof}
            Consider $a*b \tand b*a \tfor a,b \in \bb{R}^+$:
            \[
                a*b = \frac{ab}{a+b+1} = \frac{ba}{b+a+1} = b*a
            \]
            Since $a*b = b*a$ for all $a,b \in \bb{R}^+$, $*$ is commutative.
        \end{proof}
    }
    \qitem{(b)}{Is $*$ associative? Explain.}{
        $*$ is associative.
        \begin{proof}
            Consider $a,b,c \in \bb{R}^+$:
            \begin{align*}
                (a*b)*c & = \frac{ab}{a+b+1}*c = \frac{\frac{ab}{a+b+1}c}{\frac{ab}{a+b+1} +c+1} = \frac{abc}{(a+b+1)(\frac{ab}{a+b+1}+c+1)} \\
                        & = \frac{abc}{ab+ac+bc+a+b+c+1}                                                                                     \\ \\
                a*(b*c) & = a*\frac{bc}{b+c+1} = \frac{a\frac{bc}{b+c+1}}{a+\frac{bc}{b+c+1}+1} = \frac{abc}{(b+c+1)(a+\frac{bc}{b+c+1}+1)}  \\
                        & = \frac{abc}{ab+ac+bc+a+b+c+1}
            \end{align*}
            Thus $(a*b)*c = a*(b*c)$, and $*$ is associative.
        \end{proof}
    }
\end{enumerate}

\question{3}{[3 points] Let $E$ denote the set of all even integers. Prove that $\struct{\bb{Z},+} \simeq \struct{E,+}$.}
\begin{proof}
    An isomorphism must be one-to-one, onto, and operation preserving. Consider $\phi:  \bb{Z} \rightarrow E$ such that $\phi(n) = 2n$.
    \begin{enumerate}
        \item One-to-one: Assume $\phi(n_1) = \phi(n_2)$ for $n_1,n_2 \in \bb{Z}$.
              \begin{align*}
                  \phi(n_1) & = \phi(n_2) \\
                  2n_1      & = 2n_2      \\
                  n_1       & = n_2
              \end{align*}
              Thus $\phi$ is one-to-one.
        \item Onto: Let $m \in E$ Let us find $n \in \bb{Z}$ such that $m = \phi(n)$. Since $m$ is an even integer, it can be represented as $m = 2k$, where $k \in \bb{Z}$.
              \begin{align*}
                  m  & = \phi(n) \\
                  2k & = 2n      \\
                  k = n
              \end{align*}
              Choose $n = k$. Thus $\phi$ is onto.
        \item Operation Preserving: Need to show that $\phi(n + m) = \phi(n) + \phi(m)$
              \begin{align*}
                  \phi(n + m) & = 2(n+m)            \\
                              & = 2n + 2m           \\
                              & = \phi(n) + \phi(m)
              \end{align*}
              Thus $\phi$ is operation preserving.
    \end{enumerate}
    Since $\phi$ is one-to-one, onto, and operation preserving, thus $\phi$ is an isomorphism of $\struct{\bb{Z},+} \tand \struct{E,+}$, and $\struct{\bb{Z},+} \simeq \struct{E,+}$.
\end{proof}
\qdash

\question{4}{[3 points each] Prove that isomorphism is an equivalence relation among binary structures. To do this, you need to prove the following three properties:}
\begin{enumerate}
    \qitem{(a)}{Reflexive: Every binary structure is isomorphic to itself. Hint: let $\struct{S,*}$ be a binary structure and define $\phi: S \rightarrow S$ by $\phi(x)=x$. Prove that $\phi$ is an isomorphism.}{
        An isomorphism must be one-to-one, onto, and operation preserving.
        \begin{proof}
            Consider $\phi: S \rightarrow S$ such that $\phi(x)=x$.
            \begin{enumerate}[1.]
                \item One-to-one: Assume $\phi(x_1) = \phi(x_2)$ for some $x_1,x_2 \in S$.
                      \begin{align*}
                          \phi(x_1) & = \phi(x_2) \\
                          x_1       & = x_2
                      \end{align*}
                      Thus $\phi$ is one-to-one.
                \item Onto: Let $y \in S$. Let us find $x \in S$ such that $y = \phi(x)$.
                      \begin{align*}
                          y & = \phi(x) \\
                          y & = x
                      \end{align*}
                      Choose $x = y$. This $\phi$ is onto.
                \item Operation Preserving: Need to show that $\phi(x*y) = \phi(x)*\phi(y)$.
                      \begin{align*}
                          \phi(x*y) & = x*y               \\
                                    & = \phi(x) * \phi(y)
                      \end{align*}
                      Thus $\phi$ is operation preserving.
            \end{enumerate}
            Since $\phi$ is one-to-one, onto, and operation preserving, $\phi$ is an isomorphism of $\struct{S,*} \tand \struct{S,*}$, and $\struct{S,*} \simeq \struct{S,*}$.
        \end{proof}
    }
    \qitem{(b)}{Symmetric: For binary structures $\struct{S_1,*} \tand \struct{S_2,*'}$, if $S_1 \simeq S_2$ then $S_2 \simeq S_1$. Hint: assume $\phi: S_1 \rightarrow S_2$ is an isomorphism and prove that $\phi^{-1}: S_2 \rightarrow S_1$ is also an isomorphism.}{
        An isomorphism must be one-to-one, onto, and operation preserving.
        \begin{proof}
            Consider $\phi^{-1}: S_2 \rightarrow S_1$. Such an operation must exist, since $\phi$ is bijective by definition.
            \begin{enumerate}[1.]
                \item One-to-one: Assume $\phi^{-1}(x_1) = \phi^{-1}(x_2)$ for $x_1,x_2 \in S_2$.
                      \begin{align*}
                          \phi^{-1}(x_1)       & = \phi^{-1}(x_2)       \\
                          \phi(\phi^{-1}(x_1)) & = \phi(\phi^{-1}(x_2)) \\
                          x_1                  & = x_2
                      \end{align*}
                      Thus $\phi^{-1}$ is one-to-one.
                \item Onto: Let $y \in S_1$. Let us find $x \in S_2$ such that $y = \phi^{-1}(x)$.
                      \begin{align*}
                          y       & = \phi^{-1}(x)       \\
                          \phi(y) & = \phi(\phi^{-1}(x)) \\
                          \phi(y) & = x
                      \end{align*}
                      Choose $x = \phi(y)$. Thus $\phi^{-1}$ is onto.
                \item Operation Preserving: Need to show that $\phi^{-1}(x*'y) = \phi^{-1}(x) * \phi^{-1}(y)$. But first, consider $a,b \in S_1$.
                      \begin{align*}
                          \phi(a*b)          & = \phi(a) *' \phi(b)            &  & \text{since $\phi$ is an isomorphism and thus operation preserving} \\
                          \phi^{-1}\phi(a*b) & = \phi^{-1}(\phi(a) *' \phi(b))                                                                          \\
                          a*b                & = \phi^{-1}(\phi(a) *' \phi(b))
                      \end{align*}
                      We can use the equation $a*b = \phi^{-1}(\phi(a) *' \phi(b))$ to help us show that $\phi^{-1}(x*'y) = \phi^{-1}(x) * \phi^{-1}(y)$.
                      \begin{align*}
                          \phi^{-1}(x) * \phi^{-1}(y) & = \phi^{-1}(\phi(\phi^{-1}(x)) *' \phi(\phi^{-1}(y))) \\
                                                      & = \phi^{-1}(x *' y)
                      \end{align*}
                      Thus $\phi^{-1}$ is operation preserving.
            \end{enumerate}
            Since $\phi^{-1}$ is one-to-one, onto, and operation preserving, $\phi^{-1}$ is an isomorphism of $\struct{S_2,*'} \tand \struct{S_1,*}$, and $\struct{S_2,*'} \simeq \struct{S_1,*}$.
        \end{proof}
    }
    \qitem{(c)}{Transitive: For binary structures $\struct{S_1,*},~\struct{S_2,*'}, \tand \struct{S_3,*''}$, if $S_1 \simeq S_2 \tand S_2 \simeq S_3 \tthen S_1 \simeq S_3$. Hind: assume $\phi_1: S_1 \rightarrow S_2 \tand \phi_2: S_2 \rightarrow S_3$ are isomorphisms and prove that $\phi_2 \circ \phi_1: S_1 \rightarrow S_3$ is also an isomorphism.}{
        An isomorphism must be one-to-one, onto, and operation preserving.
        \begin{proof}
            Consider $\phi_2 \circ \phi_1: S_1 \rightarrow S_3$.
            \begin{enumerate}[1.]
                \item One-to-one: Assume $\phi_2 \circ \phi_1(x_1) = \phi_2 \circ \phi_1(x_2)$ for some $x_1,x_2 \in S_3$.
                      \begin{align*}
                          \phi_2 \circ \phi_1(x_1) & = \phi_2 \circ \phi_1(x_2)                                                                  \\
                          \phi_2(\phi_1(x_1))      & = \phi_2(\phi_1(x_2))                                                                       \\
                          \phi_1(x_1)              & = \phi_1(x_2)              &  & \text{since $\phi_2$ is an isomorphism and thus one-to-one} \\
                          x_1                      & = x_2                      &  & \text{since $\phi_1$ is an isomorphism and thus one-to-one}
                      \end{align*}
                      Thus $\phi_2 \circ \phi_1$ is one-to-one.
                \item Onto: Let $y \in S_3$. Let us find $x \in S_1$ such that $y = \phi_2 \circ \phi_1(x)$.
                      \begin{align*}
                          y                                & = \phi_2 \circ \phi_1(x)                                                                                                                           \\
                          \phi_2^{-1}(y)                   & = \phi_2^{-1} \circ \phi_2 \circ \phi_1(x)                   &  & \text{since $\phi_2$ is bijective, and has a well-defined inverse $\phi_2^{-1}$} \\
                          \phi_1^{-1} \circ \phi_2^{-1}(y) & = \phi_1^{-1} \circ \phi_2^{-1} \circ \phi_2 \circ \phi_1(x) &  & \text{since $\phi_1$ is bijective, and has a well-defined inverse $\phi_1^{-1}$} \\
                          \phi_1^{-1} \circ \phi_2^{-1}(y) & = \phi_1^{-1} \circ \phi_1(x)                                                                                                                      \\
                          \phi_1^{-1} \circ \phi_2^{-1}(y) & = x                                                                                                                                                \\
                      \end{align*}
                      Choose $x = \phi_1^{-1} \circ \phi_2^{-1}(y)$. Thus $\phi_2 \circ \phi_1$ is onto.
                \item Operation Preserving: Need to show that $\phi_2 \circ \phi_1(x*y)=\phi_2 \circ \phi_1(x)*''\phi_2 \circ \phi_1(y)$. Let us use the following equations to help with this. Let $a,b \in S_1$ and $\alpha,\beta \in S_2$
                      \begin{align*}
                          \phi_1(a*b)           & = \phi_1(a) *' \phi_1(b)           &  & \text{since $\phi_1$ is an isomorphism and operation preserving} \\
                          \phi_2(\alpha*'\beta) & = \phi_2(\alpha) *'' \phi_2(\beta) &  & \text{since $\phi_2$ is an isomorphism and operation preserving} \\
                      \end{align*}
                      We can compose these two equations to help us show $\phi_2 \circ \phi_1(x*y)=\phi_2 \circ \phi_1(x)*''\phi_2 \circ \phi_1(y)$:
                      \begin{align*}
                          \phi_2 \circ \phi_1(x*y) & = \phi_2(\phi_1(x) *' \phi_1(x))                  \\
                                                   & = \phi_2(\phi_1(x)) *'' \phi_2(\phi_1(y))         \\
                                                   & = \phi_2 \circ \phi_1(x)*''\phi_2 \circ \phi_1(y)
                      \end{align*}
                      Thus $\phi_2 \circ \phi_1$ is operation preserving.
            \end{enumerate}
            Since $\phi_2 \circ \phi_1$ is one-to-one, onto, and operation preserving, $\phi_2 \circ \phi_1$ is an isomorphism of $\struct{S_1,*}$ and $\struct{S_3,*''}$, thus $\struct{S_1,*} \simeq \struct{S_3,*''}$.
        \end{proof}
    }
\end{enumerate}

\end{document}