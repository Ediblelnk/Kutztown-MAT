\documentclass{article}
\usepackage[margin=1in]{geometry}
\usepackage{amsmath, amsthm, amssymb, fancyhdr, tikz, circuitikz, graphicx}
\usepackage{centernot, xcolor, hhline, multirow, listings, dashrule}
\usepackage{blkarray, booktabs, bigstrut, etoolbox, extarrows}
\usepackage[normalem]{ulem}
\usepackage{bookmark}
\usepackage{enumerate}          %These two package give custom labels to a list
\usepackage[shortlabels]{enumitem}
\usetikzlibrary{math}
\usetikzlibrary{fit}

\pagestyle{fancy}

\usepackage{hyperref}
\hypersetup{
    colorlinks=true,
    linkcolor=black,
    filecolor=magenta,
    urlcolor=cyan,
}
%formatting
\newcommand{\bld}{\textbf}
\newcommand{\itl}{\textit}
\newcommand{\uln}{\underline}

%math word symbols
\newcommand{\bb}{\mathbb}
\DeclareMathOperator{\tif}{~\text{if}~}
\DeclareMathOperator{\tand}{~\text{and}~}
\DeclareMathOperator{\tbut}{~\text{but}~}
\DeclareMathOperator{\tor}{~\text{or}~}
\DeclareMathOperator{\tsuchthat}{~\text{such that}~}
\DeclareMathOperator{\tsince}{~\text{since}~}
\DeclareMathOperator{\twhen}{~\text{when}~}
\DeclareMathOperator{\twhere}{~\text{where}~}
\DeclareMathOperator{\twith}{~\text{with}~}
\DeclareMathOperator{\tfor}{~\text{for}~}
\DeclareMathOperator{\tthen}{~\text{then}~}
\DeclareMathOperator{\tto}{~\text{to}~}
\DeclareMathOperator{\tin}{~\text{in}~}

%display shortcut
\DeclareMathOperator{\dstyle}{\displaystyle}
\DeclareMathOperator{\sstyle}{\scriptstyle}

%linear algebra
\DeclareMathOperator{\id}{\bld{id}}
\DeclareMathOperator{\vecspan}{\text{span}}
\DeclareMathOperator{\adj}{\text{adj}}

%discrete math - integer properties
\DeclareMathOperator{\tdiv}{\text{div}}
\DeclareMathOperator{\tmod}{\text{mod}}
\DeclareMathOperator{\lcm}{\text{lcm}}

%augmented matrix environment
\newenvironment{apmatrix}[2]{%
    \left(\begin{array}{@{~}*{#1}{c}|@{~}*{#2}{c}}
        }{
    \end{array}\right)
}
\newenvironment{abmatrix}[2]{%
    \left[\begin{array}{@{~}*{#1}{c}|@{~}*{#2}{c}}
            }{
        \end{array}\right]
}

\newenvironment{determinant}[1]{
    \left\lvert
    \begin{array}{@{~}*{#1}{c}}
        }{
    \end{array}
    \right\rvert
}

% graph theory
\DeclareMathOperator{\diam}{\text{diam}}
\newcommand{\comp}[1]{\overline{#1}}

% calculus iii
\newcommand{\vvec}{\overrightarrow}
\newcommand{\norm}[1]{\lVert #1 \rVert}

% abstract algebra
\newcommand{\struct}[1]{\langle #1 \rangle}

%lists
\newcommand{\bitem}[1]{\item[\bld{#1.}]}
\newcommand{\bbitem}[2]{\item[\bld{#1.}] \bld{#2}}
\newcommand{\biitem}[2]{\item[\bld{#1.}] \itl{#2}}
\newcommand{\iitem}[1]{\item[\itl{#1.}]}
\newcommand{\iiitem}[2]{\item[\itl{#1.}] \bld{#2}}
\newcommand{\btitem}[2]{\item[\bld{#1.}] \texttt{#2}}

%homework
\newcommand{\question}[2]{\noindent {\large\bld{#1}} #2 \qline}
\newcommand{\qitem}[3]{\item[\bld{#1.}] #2 \qdash \\ #3 \qdash}

\newcommand{\qline}{~\newline\noindent\textcolor[RGB]{200,200,200}{\rule[0.5ex]{\linewidth}{0.2pt}}}
\newcommand{\qdash}{~\newline\noindent\textcolor[RGB]{200,200,200}{\hdashrule[0.5ex]{\linewidth}{0.2pt}{2pt}}}

\newcommand{\assignment}{Graded Assignment \#2}

\lhead{MAT 311 Abstract Algebra}
\chead{\assignment}
\rhead{Peter Schaefer}

\begin{document}
\section*{\assignment}

\question{1}{[2 points each]}
\begin{enumerate}
    \qitem{a}{List the elements of $\struct{f}$ in $S_6$ where $f = \begin{pmatrix} 
        1 & 2 & 3 & 4 & 5 & 6 \\
        6 & 1 & 3 & 2 & 5 & 4
    \end{pmatrix}$}{
        $\group{f}$ consists of $\{f,f^2,f^3,\id\}$ where
        \begin{align*}
            f & = \begin{pmatrix} 
                1 & 2 & 3 & 4 & 5 & 6 \\
                6 & 1 & 3 & 2 & 5 & 4
            \end{pmatrix} & 
            f^2 & = \begin{pmatrix}
                1 & 2 & 3 & 4 & 5 & 6 \\
                4 & 6 & 3 & 1 & 5 & 2
            \end{pmatrix} \\
            f^3 & = \begin{pmatrix}
                1 & 2 & 3 & 4 & 5 & 6 \\
                2 & 4 & 3 & 6 & 5 & 1
            \end{pmatrix} & 
            \id = f^4 & = \begin{pmatrix}
                1 & 2 & 3 & 4 & 5 & 6 \\
                1 & 2 & 3 & 4 & 5 & 6
            \end{pmatrix}
        \end{align*}
    }
    \qitem{b}{If $f(x) = x+1$, describe the cyclic subgroup $\struct{f}$ of $\group{S_\bb{R}, \circ}$.}{
        $\group{f}$ consists of $\{\id = f^0, f^1, f^2, \ldots\}$, where for $x \in \bb{R}$
        \[
            f^n(x) = x + n
        \]
    }
    \qitem{c}{If $f(x) = x+1$, describe the cyclic subgroup $\struct{f}$ of $\group{F(\bb{R}), +}$.}{
        $\group{f}$ consists of $\{\id = f^0, f^1, f^2, \ldots\}$, where for $x \in \bb{R}$
        \[
            f^n(x) = nx + n
        \]
    }
\end{enumerate}

\question{2}{[4 points] The subgroup of $S_5$ generated by $f = \begin{pmatrix} 
    1 & 2 & 3 & 4 & 5 \\
    2 & 1 & 3 & 4 & 5 
\end{pmatrix}$ and $g = 
\begin{pmatrix} 
    1 & 2 & 3 & 4 & 5 \\
    1 & 2 & 4 & 5 & 3
\end{pmatrix}$ has six elements. List them, using $f$ and/or $g$ as appropriate, and write the operation table of this subgroup.}
Since each of these permutations are disjoint, the resulting subgroup generated by them will be Abelian. Let us first look at the permutations of $f$ and $g$ individually.
\begin{align*}
    f & = \begin{pmatrix} 
        1 & 2 & 3 & 4 & 5 \\
        2 & 1 & 3 & 4 & 5 
    \end{pmatrix} &
    g & = \begin{pmatrix} 
        1 & 2 & 3 & 4 & 5 \\
        1 & 2 & 4 & 5 & 3
    \end{pmatrix} \\
    \id = f^2 & = \begin{pmatrix}
        1 & 2 & 3 & 4 & 5 \\
        1 & 2 & 3 & 4 & 5
    \end{pmatrix} &
    g^2 & = \begin{pmatrix}
        1 & 2 & 3 & 4 & 5 \\
        1 & 2 & 5 & 3 & 4
    \end{pmatrix} \\
    && \id = g^3 & = \begin{pmatrix}
        1 & 2 & 3 & 4 & 5 \\
        1 & 2 & 3 & 4 & 5
    \end{pmatrix}
\end{align*}
Knowing these identities, let us determine the final two elements of this subgroup.
\begin{align*}
    fg = gf & = \begin{pmatrix}
        1 & 2 & 3 & 4 & 5 \\
        2 & 1 & 4 & 5 & 3
    \end{pmatrix} &
    fg^2 = g^2f & = \begin{pmatrix}
        1 & 2 & 3 & 4 & 5 \\
        2 & 1 & 5 & 3 & 4
    \end{pmatrix}
\end{align*}
The subgroup generated is the group $\{\id,f,g,g^2,fg,fg^2\}$, with operation table
\[
    \begin{array}{c|cccccc}
        \circ & \id & f & g & g^2 & fg & fg^2 \\
        \hline
        \id   & \id  & f    & g    & g^2  & fg   & fg^2 \\
        f     & f    & \id  & fg   & fg^2 & g    & g^2 \\
        g     & g    & fg   & g^2  & \id  & fg^2 & f \\
        g^2   & g^2  & fg^2 & \id  & g    & f    & fg \\
        fg    & fg   & g    & fg^2 & f    & g^2  & \id \\
        fg^2  & fg^2 & g^2  & f    & fg   & \id  & g
    \end{array}
\]

\question{3}{[2 points each] You must justify your answers for parts (b) and (c).}
\begin{enumerate}
    \qitem{a}{Compute the following product in $S_9$ and write your answers as a permutation tabular form: 
    
    $(1,4,7)(1,6,7,8)(7,4,1,3,2)$.}{answer}
    \qitem{b}{Determine whether the following is even or odd: $\begin{pmatrix}
        1 & 2 & 3 & 4 & 5 & 6 & 7 & 8 \\
        7 & 4 & 1 & 5 & 6 & 2 & 3 & 8
    \end{pmatrix}$.}{answer}
    \qitem{c}{Determine whether the following is even or odd: $(1,2,7,6)(3,2,4,1)(7,8,1,2)$.}{answer}
\end{enumerate}

\question{4}{[4 points] Let $H$ and $K$ be subgroups of a group $G$. Define $\sim$ on $G$ by $a \sim b$ if and only if $a = hbk$ for some $h \in H$ and some $k \in K$. Prove that $\sim$ is an equivalence relation on $G$. The equivalence classes of this equivalence relation are called \itl{double cosets}.}
answer

\end{document}