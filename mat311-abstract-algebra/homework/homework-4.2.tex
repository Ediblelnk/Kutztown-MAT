\documentclass{article}
\usepackage[margin=1in]{geometry}
\usepackage{amsmath, amsthm, amssymb, fancyhdr, tikz, circuitikz, graphicx}
\usepackage{centernot, xcolor, hhline, multirow, listings, dashrule}
\usepackage{blkarray, booktabs, bigstrut, etoolbox, extarrows}
\usepackage[normalem]{ulem}
\usepackage{bookmark}
\usepackage{enumerate}          %These two package give custom labels to a list
\usepackage[shortlabels]{enumitem}
\usetikzlibrary{math}
\usetikzlibrary{fit}

\pagestyle{fancy}

\usepackage{hyperref}
\hypersetup{
    colorlinks=true,
    linkcolor=black,
    filecolor=magenta,
    urlcolor=cyan,
}
%formatting
\newcommand{\bld}{\textbf}
\newcommand{\itl}{\textit}
\newcommand{\uln}{\underline}

%math word symbols
\newcommand{\bb}{\mathbb}
\DeclareMathOperator{\tif}{~\text{if}~}
\DeclareMathOperator{\tand}{~\text{and}~}
\DeclareMathOperator{\tbut}{~\text{but}~}
\DeclareMathOperator{\tor}{~\text{or}~}
\DeclareMathOperator{\tsuchthat}{~\text{such that}~}
\DeclareMathOperator{\tsince}{~\text{since}~}
\DeclareMathOperator{\twhen}{~\text{when}~}
\DeclareMathOperator{\twhere}{~\text{where}~}
\DeclareMathOperator{\tfor}{~\text{for}~}
\DeclareMathOperator{\tthen}{~\text{then}~}

%display shortcut
\DeclareMathOperator{\dstyle}{\displaystyle}
\DeclareMathOperator{\sstyle}{\scriptstyle}

%linear algebra
\DeclareMathOperator{\id}{\bld{id}}
\DeclareMathOperator{\vecspan}{\text{span}}

%augmented matrix environment
\newenvironment{amatrix}[1]{%
  \left(\begin{array}{@{}*{#1}{c}|c@{}}
    }{
  \end{array}\right)
}

%lists
\newcommand{\bitem}[1]{\item[\bld{#1.}]}
\newcommand{\bbitem}[2]{\item[\bld{#1.}] \bld{#2}}
\newcommand{\biitem}[2]{\item[\bld{#1.}] \itl{#2}}
\newcommand{\iitem}[1]{\item[\itl{#1.}]}
\newcommand{\iiitem}[2]{\item[\itl{#1.}] \bld{#2}}

%homework
\newenvironment*{question}[2]{
  \subsection*{#1} \itl{#2}
  \begin{enumerate}
    }{
  \end{enumerate}
}
\newcommand{\qitem}[2]{\item[\bld{#1}] \itl{#2}}

\newcommand{\assignment}{Section 4 Groups, p45 \#8,19,23,25,31,35}

\lhead{MAT 311 Abstract Algebra}
\chead{\assignment}
\rhead{Peter Schaefer}

\begin{document}
\section*{\assignment}

\begin{enumerate}
    \qitem{8}{We can also consider multiplication $\cdot_n$ modulo $n$ in $\bb{Z}_n$. For example, $5 \cdot_7 6 = 2$. The set $\{1,3,5,7\}$ with multiplication $\cdot_8$ modulo 8 is a group. Give the table for this group.}{
        Cayley Table:
        \[
            \begin{array}{c|cccc}
                \cdot_8 & 1 & 3 & 5 & 7 \\
                1 & 1 & 3 & 5 & 7 \\
                3 & 3 & 1 & 7 & 5 \\
                5 & 5 & 7 & 1 & 3 \\
                7 & 7 & 5 & 3 & 1
            \end{array}
        \]
    }
\end{enumerate}
In Exercise 11 through 18, determine whether the given set of matrices under the specified operation or multiplication, is a group.
\begin{enumerate}
    \qitem{19}{Let $S$ be the set of all real numbers except $-1$. Define $*$ on $S$ by
    \[
        a*b = a+b+ab
    \]}{
    Complete the following:
        \begin{enumerate}
            \qitem{a}{Show that $*$ give a binary operation on $S$.}{
                A binary operation is closed and uniquely defined.
                \begin{proof}
                    \begin{enumerate}
                        \item Closed: Consider if $a*b = -1$.
                        \begin{align*}
                            a*b & = -1 \\
                            a+b+ab & = -1 \\
                            b(1+a) & = -(1+a) \\
                            b(1+a)+(1+a) & = 0 \\
                            (b+1)(a+1) & = 0 \\
                            b & = -1 \tor a = -1
                        \end{align*}
                        Since we are only considering $a,b \in S$.
                        \item Uniquely defined:
                        \[
                            a*b = a+b+ab
                        \]
                        Since addition $+$ and multiplication $\cdot$ are uniquely defined, the result of $a+b+ab$, which consists of addition and multiplication, will always yield with a single result. Thus $*$ is uniquely defined.
                    \end{enumerate}
                    Since $*$ is uniquely defined and closed, thus it is a binary operation on $S$.
                \end{proof}
            }
            \qitem{b}{Show that $\group{S,*}$ is a group}{
                A group is closed, associative, has an identity, and has an inverse for every element.
                \begin{proof}
                    \begin{enumerate}
                        \item Associative: Consider $a,b \in S$.
                        \begin{align*}
                            a*(b*c) & = a*(b+c+bc) \\
                            & = a + (b+c+bc) + a(b+c+bc) \\
                            & = a+b+c+bc+ab+ac+abc \\ \\
                            (a*b)*c & = (a+b+ab)*c \\
                            & = (a+b+ab) + c + (a+b+ab)c \\
                            & = a+b+c+ab+ac+bc+abc
                        \end{align*}
                        Thus $*$ is associative for $S$.
                        \item Identity: Consider $e = 0$
                        \begin{align*}
                            a * e & = a + 0 + a0 = a \\
                            e * a & = 0 + a + 0a = a \\
                        \end{align*}
                        Thus $S$ has an identity element under $*$.
                        \item Inverse: Consider $a*a'=e$.
                        \begin{align*}
                            a*a' & = e \\
                            a + a' + aa' & = 0 \\
                            a' + aa' & = -a \\
                            a'(1+a) & = -a \\
                            a' & = \frac{-a}{1+a}
                        \end{align*}
                        Since $S$ does not include $-1$, this will always be a defined value, thus every element has an inverse.
                    \end{enumerate}
                    Since $\group{S, *}$ is closed, associative, has an identity, and has an inverse for every element, it is a group.
                \end{proof}
            }
            \qitem{c}{Find the solution of the equation $2*x*3=7$ in $S$}{
                We can use the properties of the group to help us solve this.
                \begin{align*}
                    2*x*3 & = 7 \\
                    2 + x + 3 + 2x + 6 + 3x + 6x & = 7 \\
                    11 + 12x & = 7 \\
                    12x & = -4 \\
                    x & = -\frac{1}{3}
                \end{align*}
            }
        \end{enumerate}
    }
\end{enumerate}
\bld{Concepts}
\begin{enumerate}
    \qitem{23}{The following "definitions" of a group are taken verbatim, including spelling and punctuation, from papers of students who write a bit too quickly and carelessly.}{
        Critizie them.
        \begin{enumerate}
            \qitem{a}{[Refer to book]}{answer}
            \qitem{b}{[Refer to book]}{answer}
            \qitem{c}{[Refer to book]}{answer}
            \qitem{d}{[Refer to book]}{answer}
        \end{enumerate}
    }
    \qitem{25}{Mark each of the following.}{
        True or False:
        \begin{enumerate}
            \qitem{a}{A group may have more than one identity element.}{false}
            \qitem{b}{Any two groups of three elements are ismorphic.}{true, because there is only one kind of group with three elements.}
            \qitem{c}{In a group, each linear equation has a solution.}{true, since a group is closed.}
            \qitem{d}{The proper attitude toward a definition is to memorize it so that you can reproduce it word for word as in the text.}{neither, whatever works best for the indivdual.}
            \qitem{e}{Any definition a person gives for a group is correct provided that everything that is a group by that person's definition is also a group by the definition in the text.}{false, it could be that something from the text is a group but not by the person's definition.}
            \qitem{f}{Any definition a person gives for a group is correct provided he or she can show that everything that satifies the definition satifies the one in the text and conversely.}{true}
            \qitem{g}{Every finite group of at most three elements is abelian.}{true, since the corresponding single, dual, and triple element groups are all commutative.}
            \qitem{h}{An equation of the form $a*x=c$ always has a unique solution in a group.}{true.}
            \qitem{i}{The empty set can be considered a group}{false, there is no identity element}
            \qitem{j}{Every group is a binary algebraic structure}{true, because a group requires a binary operation on a closed set.}
        \end{enumerate}
    }
\end{enumerate}
\bld{Theory}
\begin{enumerate}
    \qitem{31}{If $*$ is a binary operation on a set $S$, an element $x$ of $S$ is an \bld{idempotent for} $*$ if $x*x=x$. Prove that a group has exactly one idempotent element. (You may use any theorems proved so far in the text.)}{answer}
    \qitem{35}{Show that if $(a*b)^2 = a^2*b^2$ for $a \tand b$ in a group $G$, then $a*b = b*a$. See Exercise 33 for the meaning of $a^2$.}{answer}
\end{enumerate}

\end{document}