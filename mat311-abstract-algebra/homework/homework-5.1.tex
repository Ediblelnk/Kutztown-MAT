\documentclass{article}
\usepackage[margin=1in]{geometry}
\usepackage{amsmath, amsthm, amssymb, fancyhdr, tikz, circuitikz, graphicx}
\usepackage{centernot, xcolor, hhline, multirow, listings, dashrule}
\usepackage{blkarray, booktabs, bigstrut, etoolbox, extarrows}
\usepackage[normalem]{ulem}
\usepackage{bookmark}
\usepackage{enumerate}          %These two package give custom labels to a list
\usepackage[shortlabels]{enumitem}
\usetikzlibrary{math}
\usetikzlibrary{fit}

\pagestyle{fancy}

\usepackage{hyperref}
\hypersetup{
    colorlinks=true,
    linkcolor=black,
    filecolor=magenta,
    urlcolor=cyan,
}
%formatting
\newcommand{\bld}{\textbf}
\newcommand{\itl}{\textit}
\newcommand{\uln}{\underline}

%math word symbols
\newcommand{\bb}{\mathbb}
\DeclareMathOperator{\tif}{~\text{if}~}
\DeclareMathOperator{\tand}{~\text{and}~}
\DeclareMathOperator{\tbut}{~\text{but}~}
\DeclareMathOperator{\tor}{~\text{or}~}
\DeclareMathOperator{\tsuchthat}{~\text{such that}~}
\DeclareMathOperator{\tsince}{~\text{since}~}
\DeclareMathOperator{\twhen}{~\text{when}~}
\DeclareMathOperator{\twhere}{~\text{where}~}
\DeclareMathOperator{\tfor}{~\text{for}~}
\DeclareMathOperator{\tthen}{~\text{then}~}

%display shortcut
\DeclareMathOperator{\dstyle}{\displaystyle}
\DeclareMathOperator{\sstyle}{\scriptstyle}

%linear algebra
\DeclareMathOperator{\id}{\bld{id}}
\DeclareMathOperator{\vecspan}{\text{span}}

%augmented matrix environment
\newenvironment{amatrix}[1]{%
  \left(\begin{array}{@{}*{#1}{c}|c@{}}
    }{
  \end{array}\right)
}

%lists
\newcommand{\bitem}[1]{\item[\bld{#1.}]}
\newcommand{\bbitem}[2]{\item[\bld{#1.}] \bld{#2}}
\newcommand{\biitem}[2]{\item[\bld{#1.}] \itl{#2}}
\newcommand{\iitem}[1]{\item[\itl{#1.}]}
\newcommand{\iiitem}[2]{\item[\itl{#1.}] \bld{#2}}

%homework
\newenvironment*{question}[2]{
  \subsection*{#1} \itl{#2}
  \begin{enumerate}
    }{
  \end{enumerate}
}
\newcommand{\qitem}[2]{\item[\bld{#1}] \itl{#2}}

\newcommand{\assignment}{Section 5 Subgroups, p55 \#1-9,11,15,17}

\lhead{MAT 311 Abstract Algebra}
\chead{\assignment}
\rhead{Peter Schaefer}

\begin{document}
\section*{\assignment}

In Exercises 1 through 6, determine whether the given subset of the complex numbers is a subgroup of the group $\bb{C}$ of complex numbers under addition.
\begin{enumerate}
    \qitem{1}{$\bb{R}$}{This is a subgroup. A real number plus a real number will always yield a real number, and the inverse is just the negative of the real number.}
    \qitem{2}{$\bb{Q}^+$}{This is not a subgroup. Consider $1 \in \bb{Q}^+$. It has no inverse in $\bb{Q}^+$.}
    \qitem{3}{$7\bb{Z}$}{This is a subgroup. A multiple of 7 plus another multiple of 7 will always result in a multiple of 7, and the inverse is just he negative of the multiple of 7.}
    \qitem{4}{The set $i\bb{R}$ of pure imaginary numbers including $0$.}{This is a subgroup. An imaginary number plus an imaginary number will always yield an imaginary number, assuming zero is included. The inverse is just the negative of the imaginary number.}
    \qitem{5}{The set $\pi\bb{Q}$ of rational multiples of $\pi$.}{This is a subgroup. $\pi\frac{a}{b} + \pi\frac{c}{d} = \pi(\frac{a}{b}+\frac{c}{d})$. The inverse will for $\pi\frac{a}{b}$ is just $-\pi\frac{a}{b}$.}
    \qitem{6}{The set $\{\pi^n : n \in \bb{Z}\}$}{This is not a subgroup, because it is not closed. Consider $\pi + \pi = 2\pi$. This will not be an integer power of $\pi$.}
\end{enumerate}

\end{document}