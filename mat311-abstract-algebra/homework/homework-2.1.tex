\documentclass{article}
\usepackage[margin=1in]{geometry}
\usepackage{amsmath, amsthm, amssymb, fancyhdr, tikz, circuitikz, graphicx}
\usepackage{centernot, xcolor, hhline, multirow, listings, dashrule}
\usepackage{blkarray, booktabs, bigstrut, etoolbox, extarrows}
\usepackage[normalem]{ulem}
\usepackage{bookmark}
\usepackage{enumerate}          %These two package give custom labels to a list
\usepackage[shortlabels]{enumitem}
\usetikzlibrary{math}
\usetikzlibrary{fit}

\pagestyle{fancy}

\usepackage{hyperref}
\hypersetup{
    colorlinks=true,
    linkcolor=black,
    filecolor=magenta,
    urlcolor=cyan,
}
%formatting
\newcommand{\bld}{\textbf}
\newcommand{\itl}{\textit}
\newcommand{\uln}{\underline}

%math word symbols
\newcommand{\bb}{\mathbb}
\DeclareMathOperator{\tif}{~\text{if}~}
\DeclareMathOperator{\tand}{~\text{and}~}
\DeclareMathOperator{\tbut}{~\text{but}~}
\DeclareMathOperator{\tor}{~\text{or}~}
\DeclareMathOperator{\tsuchthat}{~\text{such that}~}
\DeclareMathOperator{\tsince}{~\text{since}~}
\DeclareMathOperator{\twhen}{~\text{when}~}
\DeclareMathOperator{\twhere}{~\text{where}~}
\DeclareMathOperator{\twith}{~\text{with}~}
\DeclareMathOperator{\tfor}{~\text{for}~}
\DeclareMathOperator{\tthen}{~\text{then}~}
\DeclareMathOperator{\tto}{~\text{to}~}
\DeclareMathOperator{\tin}{~\text{in}~}

%display shortcut
\DeclareMathOperator{\dstyle}{\displaystyle}
\DeclareMathOperator{\sstyle}{\scriptstyle}

%linear algebra
\DeclareMathOperator{\id}{\bld{id}}
\DeclareMathOperator{\vecspan}{\text{span}}
\DeclareMathOperator{\adj}{\text{adj}}

%discrete math - integer properties
\DeclareMathOperator{\tdiv}{\text{div}}
\DeclareMathOperator{\tmod}{\text{mod}}
\DeclareMathOperator{\lcm}{\text{lcm}}

%augmented matrix environment
\newenvironment{apmatrix}[2]{%
    \left(\begin{array}{@{~}*{#1}{c}|@{~}*{#2}{c}}
        }{
    \end{array}\right)
}
\newenvironment{abmatrix}[2]{%
    \left[\begin{array}{@{~}*{#1}{c}|@{~}*{#2}{c}}
            }{
        \end{array}\right]
}

\newenvironment{determinant}[1]{
    \left\lvert
    \begin{array}{@{~}*{#1}{c}}
        }{
    \end{array}
    \right\rvert
}

% graph theory
\DeclareMathOperator{\diam}{\text{diam}}
\newcommand{\comp}[1]{\overline{#1}}

% calculus iii
\newcommand{\vvec}{\overrightarrow}
\newcommand{\norm}[1]{\lVert #1 \rVert}

% abstract algebra
\newcommand{\struct}[1]{\langle #1 \rangle}

%lists
\newcommand{\bitem}[1]{\item[\bld{#1.}]}
\newcommand{\bbitem}[2]{\item[\bld{#1.}] \bld{#2}}
\newcommand{\biitem}[2]{\item[\bld{#1.}] \itl{#2}}
\newcommand{\iitem}[1]{\item[\itl{#1.}]}
\newcommand{\iiitem}[2]{\item[\itl{#1.}] \bld{#2}}
\newcommand{\btitem}[2]{\item[\bld{#1.}] \texttt{#2}}

%homework
\newcommand{\question}[2]{\noindent {\large\bld{#1}} #2 \qline}
\newcommand{\qitem}[3]{\item[\bld{#1.}] #2 \qdash \\ #3 \qdash}

\newcommand{\qline}{~\newline\noindent\textcolor[RGB]{200,200,200}{\rule[0.5ex]{\linewidth}{0.2pt}}}
\newcommand{\qdash}{~\newline\noindent\textcolor[RGB]{200,200,200}{\hdashrule[0.5ex]{\linewidth}{0.2pt}{2pt}}}

\newcommand{\assignment}{Section 2 Binary Operations, p25 7,9,11,17,19,21,23}

\lhead{MAT 311 Abstract Algebra}
\chead{\assignment}
\rhead{Peter Schaefer}

\begin{document}
\section*{\assignment}

In Exercises 7 through 11, determine whether the binary operation $*$ defined is commutative and whether $*$ is associative.
\begin{enumerate}
    \qitem{7}{$*$ defined on $\bb{Z}$ by letting $a * b = a - b$}{
        $*$ is neither commutative nor associative.
        \begin{enumerate}
            \item Commutative: Consider $1 \tand 2$. $1 - 2 = -1 \neq 2 - 1 = 1$. Thus $*$ is not commutative.
            \item Associative: Consider $1 - (4 - 3) = 0 \neq (1 - 3) - 3 = -5$. Thus $*$ is not commutative.
        \end{enumerate}
    }
    \qitem{9}{$*$ defined on $\bb{Q}$ by letting $a*b=ab/2$}{
        $*$ is both commutative and associative.
        \begin{enumerate}
            \item Commutative: $a*b = \frac{ab}{2} = \frac{ba}{2} = b*a$. Thus $*$ is commutative.
            \item Associative: Consider $a,b,c$.
            \begin{align*}
                a*(b*c) & = a * \frac{bc}{2} = \frac{a\frac{bc}{2}}{2} = \frac{1}{4} \cdot abc \\
                (a*b)*c & = \frac{ab}{2} * c = \frac{\frac{ab}{2}c}{2} = \frac{1}{4} \cdot abc
            \end{align*}
            Thus, $*$ is associative.
        \end{enumerate}
    }
    \qitem{11}{$*$ defined on $\bb{Z}^+$ by letting $a*b=a^b$}{
        $*$ is neither commutative nor associative.
        \begin{enumerate}
            \item Commutative: $2*3 = 2^3 = 8 \neq 3*2 = 3^2 = 9$. Thus $*$ is not commutative.
            \item Associative: Consider $2,3,3$.
            \begin{align*}
                2 * (3 * 3) & = 2 * 9 = 2^9 = 512 \\
                (2 * 3) * 3 & = 6 * 3 = 6^3 = 216
            \end{align*}
            Thus $*$ is not associative.
        \end{enumerate}
    }
\end{enumerate}

In Exercises 17 through 22, determine whether the definition of $*$ does give a binary operation on the set. In the event that $*$ is \itl{not} a binary operation, state whether condition 1 (uniquely defined), condition 2 (closed), or both of these conditions are violated.
\begin{enumerate}
    \qitem{17}{On $\bb{Z}^+$, define $*$ by letting $a*b = a-b$.}{
        $*$ is not a binary operation on $\bb{Z}^+$. Consider $a = 1 \tand b = 2$: $1*2 = 1-2 = -1 \notin \bb{Z}^+$. Thus $\bb{Z}^+$ is not closed under $*$. It is, however, uniquely defined for all $a,b \in \bb{Z}^+$.
    }
    \qitem{19}{On $\bb{R}$, define $*$ by letting $a*b = a-b$.}{
        $*$ is a binary operation on $\bb{R}$.
    }
    \qitem{21}{On $\bb{Z}^+$, define $*$ by letting $a*b=c$, where $c$ is at least 5 more than $a+b$.}{
        $*$ is not a binary operation on $\bb{Z}^+$. Consider $a = 1 \tand b = 1$: $1*1 = 7 \tand 8$. Thus $*$ is not uniquely defined. It is, however, closed.
    }
\end{enumerate}

\question{23}{Let $H$ be the subset of $M_2(\bb{R})$ consisting of all matrices of the form $\begin{bmatrix} a & -b \\ b & a\end{bmatrix}$ for $a,b \in \bb{R}$. Is $H$ closed under}
\begin{enumerate}
    \qitem{a}{matrix addition?}{
        Consider $\begin{bmatrix}
            a & -b \\
            b & a 
        \end{bmatrix} \tand \begin{bmatrix}
            c & -d \\
            d & c
        \end{bmatrix}$.
        \[
            \begin{bmatrix}
                a & -b \\
                b & a 
            \end{bmatrix} + \begin{bmatrix}
                c & -d \\
                d & c
            \end{bmatrix} = \begin{bmatrix}
                a+c & -(b+d) \\
                b+d & a+c
            \end{bmatrix}
        \]
        Since this take the form of $H$, $H$ is closed under matrix addition.
    }
    \qitem{b}{matrix multiplication?}{
        Consider $\begin{bmatrix}
            a & -b \\
            b & a 
        \end{bmatrix} \tand \begin{bmatrix}
            c & -d \\
            d & c
        \end{bmatrix}$.
        \[
            \begin{bmatrix}
                a & -b \\
                b & a 
            \end{bmatrix}\begin{bmatrix}
                c & -d \\
                d & c
            \end{bmatrix} = \begin{bmatrix}
                ac - bd & -ad - bc \\
                bc + ad & -bd + ac
            \end{bmatrix} = \begin{bmatrix}
                ac - bd & -(ad + bc) \\
                ad + bc & ac - bd
            \end{bmatrix}
        \]
        Since this takes the form of $H$, $H$ is closed under matrix multiplication.
    }
\end{enumerate}

\end{document}