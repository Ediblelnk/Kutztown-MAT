\documentclass{article}
\usepackage[margin=1in]{geometry}
\usepackage{amsmath, amsthm, amssymb, fancyhdr, tikz, circuitikz, graphicx}
\usepackage{centernot, xcolor, hhline, multirow, listings, dashrule}
\usepackage{blkarray, booktabs, bigstrut, etoolbox, extarrows}
\usepackage[normalem]{ulem}
\usepackage{bookmark}
\usepackage{enumerate}          %These two package give custom labels to a list
\usepackage[shortlabels]{enumitem}
\usetikzlibrary{math}
\usetikzlibrary{fit}

\pagestyle{fancy}

\usepackage{hyperref}
\hypersetup{
    colorlinks=true,
    linkcolor=black,
    filecolor=magenta,
    urlcolor=cyan,
}
%formatting
\newcommand{\bld}{\textbf}
\newcommand{\itl}{\textit}
\newcommand{\uln}{\underline}

%math word symbols
\newcommand{\bb}{\mathbb}
\DeclareMathOperator{\tif}{~\text{if}~}
\DeclareMathOperator{\tand}{~\text{and}~}
\DeclareMathOperator{\tbut}{~\text{but}~}
\DeclareMathOperator{\tor}{~\text{or}~}
\DeclareMathOperator{\tsuchthat}{~\text{such that}~}
\DeclareMathOperator{\tsince}{~\text{since}~}
\DeclareMathOperator{\twhen}{~\text{when}~}
\DeclareMathOperator{\twhere}{~\text{where}~}
\DeclareMathOperator{\tfor}{~\text{for}~}
\DeclareMathOperator{\tthen}{~\text{then}~}

%display shortcut
\DeclareMathOperator{\dstyle}{\displaystyle}
\DeclareMathOperator{\sstyle}{\scriptstyle}

%linear algebra
\DeclareMathOperator{\id}{\bld{id}}
\DeclareMathOperator{\vecspan}{\text{span}}

%augmented matrix environment
\newenvironment{amatrix}[1]{%
  \left(\begin{array}{@{}*{#1}{c}|c@{}}
    }{
  \end{array}\right)
}

%lists
\newcommand{\bitem}[1]{\item[\bld{#1.}]}
\newcommand{\bbitem}[2]{\item[\bld{#1.}] \bld{#2}}
\newcommand{\biitem}[2]{\item[\bld{#1.}] \itl{#2}}
\newcommand{\iitem}[1]{\item[\itl{#1.}]}
\newcommand{\iiitem}[2]{\item[\itl{#1.}] \bld{#2}}

%homework
\newenvironment*{question}[2]{
  \subsection*{#1} \itl{#2}
  \begin{enumerate}
    }{
  \end{enumerate}
}
\newcommand{\qitem}[2]{\item[\bld{#1}] \itl{#2}}

\newcommand{\assignment}{Section 4 Groups, p45 \#2,3,5,10,11-16 all}

\lhead{MAT 311 Abstract Algebra}
\chead{\assignment}
\rhead{Peter Schaefer}

\begin{document}
\section*{\assignment}

In Exercises 1 through 6, determine whether the binary operation $*$ gives a group structure on the given set. If no group results, give the first axiom in order $\mathfrak{G}_1,\mathfrak{G}_2,\mathfrak{G}_3$ from Definition 4.1 that does not hold.
\begin{enumerate}
    \qitem{2}{Let $*$ be defined on $\bb{Z}$ by letting $a*b = ab$.}{
        $\mathfrak{G}_2$ (identity) does not hold. One might consider $1$ to be the identity, but $1 \cdot 0 = 0$. In fact, $n \cdot 0 = 0$ for any such $n \in \bb{Z}$. So no identity can exist with this $*$ on $\bb{Z}$.
    }
    \qitem{3}{Let $*$ be defined on $2\bb{Z} = \{2n : n \in \bb{Z}\}$ by letting $a*b = ab$.}{
        $\mathfrak{G}_2$ (identity) does not hold. There is no such element $e$ where $e * n = n$, for any $n \in 2\bb{Z}$.
    }
    \qitem{5}{Let $*$ be defined on the set $\bb{R}^*$ of nonzero real numbers by letting $a*b = a/b$.}{
        $\mathfrak{G}_2$ (identity) is only partially held. $1$ is a right identity, as $x * 1 = x/1 = x$ for all $x \in \bb{R}^*$. However, this does not apply to the left as $1*x = 1/x \neq x$ unless $x = 1$. Since both are required for this axiom to apply, it is not satisfied.
    }
    \qitem{10}{Let $n$ be a positive integer and let $n\bb{Z} = \{nm | m \in \bb{Z}\}$.}{
        Show the following:
        \begin{enumerate}
            \qitem{a}{$\group{n\bb{Z}, +}$ is a group.}{
                A group must be closed, associative, have an identity, and have an inverse for every element.
                \begin{proof}
                    Consider $\group{n\bb{Z}, +}$.
                    \begin{enumerate}
                        \item Closed: Consider $nm \tand np$ for $m,p \in \bb{Z}$ and $n \in \bb{Z}^+$.
                        \[
                            nm + np = n(m+p)
                        \]
                        Thus $\group{n\bb{Z}, +}$ is closed under addition.

                        \item Associativity: Consider $nm,np,nq \in n\bb{Z}$.
                        \begin{align*}
                            (nm + np) + nq & = n(m+p) + nq \\
                            & = n(m+p+q) \\ \\
                            nm + (np + nq) & = nm + n(p+q) \\
                            & = n(m+p+q)
                        \end{align*}
                        Thus $\group{n\bb{Z}, +}$ is associative.

                        \item Identity: Consider $n0 \tand nm \in n\bb{Z}$.
                        \begin{align*}
                            n0 + nm & = n(0+m) = nm \\
                            nm + n0 & = n(m+0) = nm
                        \end{align*}
                        Thus $\group{n\bb{Z}, +}$ has an identity.

                        \item Inverse: Consider $nm + n\overline{m} = n0$
                        \begin{align*}
                            nm + n\overline{m} & = n0 \\
                            n\overline{m} & = n0 + (-nm) \\
                            n\overline{m} & = -nm
                        \end{align*}
                        Thus every element $nm$ has inverse $-nm$.
                    \end{enumerate}
                    Because $\group{n\bb{Z}, +}$ is closed, associative, has an identity, and an inverse for every element, it is a group.
                \end{proof}
            }
            \qitem{b}{$\group{n\bb{Z}, +} \simeq \group{\bb{Z},+}$.}{
                An isomorphism must be one-to-one, onto, and operation preserving.
                \begin{proof}
                    Consider $\phi: n\bb{Z} \rightarrow \bb{Z}$ such that $\phi(x) = x/n$. We know such a $\phi$ is closed because $x$ takes the form of $nz$ for some $z \in \bb{Z}$.
                    \begin{enumerate}
                        \item One-to-one: Consider $\phi(x_1) = \phi(x_2)$ for $x_1,x_2 \in n\bb{Z}$
                        \begin{align*}
                            \phi(x_1) & = \phi(x_2) \\
                            x_1/n & = x_2/n \\
                            x_1 & = x_2
                        \end{align*}
                        Thus $\phi$ is one-to-one
                        \item Onto: Let $y \in \bb{Z}$. Let us find $x \in n\bb{Z}$ such that $y = \phi(x)$.
                        \begin{align*}
                            y & = \phi(x) \\
                            y & = x/n \\
                            ny & = x
                        \end{align*}
                        Choose $x = ny$. Thus $\phi$ is onto.
                        \item Operation Preserving: Need to show that $\phi(x+y) = \phi(x) + \phi(y)$
                        \begin{align*}
                            \phi(x+y) & = \frac{x+y}{n} \\
                            & = \frac{x}{n} + \frac{y}{n} \\
                            & = \phi(x) + \phi(y)
                        \end{align*}
                        Thus $\phi$ is operation preserving.
                    \end{enumerate}
                    Since $\phi$ is one-to-one, onto, and operation preserving, thus it is an isomorphism. Further, this means that $\group{n\bb{Z}, +} \simeq \group{\bb{Z},+}$.
                \end{proof}
            }
        \end{enumerate}
    }
\end{enumerate}
In exercises 11 through 18, determine whether the given set of matrices under the specified operation, matrix addition or multiplication, is a group.
\begin{enumerate}
    \qitem{11}{All $n \times n$ diagonal matrices under matrix addition.}{
        Matrix addition is known to be associative. The identity matrix will simply be the zero matrix. A matrix's inverse will just the matrix with every entry negated. This is a group.
    }
    \qitem{12}{All $n \times n$ diagonal matrices under matrix multiplication.}{
        Since any matrix times the zero matrix will be the zero matrix, this means there is no identity matrix. This is not a group.
    }
    \qitem{13}{All $n \times n$ diagonal matricies with no zero diagonal entry under matrix multiplication.}{
        The identity matrix will be the traditional diagonal matrix will entries all 1, and since there are no zero diagonal entries, every matrix will be invertible. Matrix multiplication is known to be associative, and diagonal matrix multiplication is known to be closed. This is a group.
    }
    \qitem{14}{All $n \times n$ diagonal matricies with all diagonal entries 1 or -1 under matrix multiplication}{This is a subset of the previous group, so if it is closed, then it will be a group. Let $a_i, b_i \in \{1,-1\}$ for all $1 \leq i \leq n$
    \begin{align*}
        \begin{bmatrix}
            a_1 & 0 & \cdots & 0 \\
            0 & a_2 & \cdots & 0 \\
            \vdots & \vdots & \ddots & \vdots \\
            0 & 0 & \cdots & a_n
        \end{bmatrix}
        \begin{bmatrix}
            b_1 & 0 & \cdots & 0 \\
            0 & b_2 & \cdots & 0 \\
            \vdots & \vdots & \ddots & \vdots \\
            0 & 0 & \cdots & b_n
        \end{bmatrix}=
        \begin{bmatrix}
            a_1b_1 & 0 & \cdots & 0 \\
            0 & a_2b_2 & \cdots & 0 \\
            \vdots & \vdots & \ddots & \vdots \\
            0 & 0 & \cdots & a_nb_n
        \end{bmatrix}
    \end{align*}
    Since $1 \cdot -1 = -1 \cdot 1 = -1 \tand 1 \cdot 1 = -1 \cdot -1 = 1$, we know $a_i\cdot b_j$ will be in $\{1,-1\}$. Thus, this is closed, and a group.
    }
    \qitem{15}{All $n \times n$ upper-triangular matricies under matrix multiplication.}{
        Since any matrix times the zero matrix will be the zero matrix, this means there is no identity matrix. This is not a group.
    }
    \qitem{16}{All $n \times n$ upper-triangular matricies under matrix addition.}{
        Matrix addition is known to be associative. The identity matrix will simply be the zero matrix. A matrix's inverse will just the matrix with every entry negated. This is a group.
    }
\end{enumerate}

\end{document}