\documentclass{article}
\usepackage[margin=1in]{geometry}
\usepackage{amsmath, amsthm, amssymb, fancyhdr, tikz, circuitikz, graphicx}
\usepackage{centernot, xcolor, hhline, multirow, listings, dashrule}
\usepackage{blkarray, booktabs, bigstrut, etoolbox, extarrows}
\usepackage[normalem]{ulem}
\usepackage{bookmark}
\usepackage{enumerate}          %These two package give custom labels to a list
\usepackage[shortlabels]{enumitem}
\usetikzlibrary{math}
\usetikzlibrary{fit}

\pagestyle{fancy}

\usepackage{hyperref}
\hypersetup{
    colorlinks=true,
    linkcolor=black,
    filecolor=magenta,
    urlcolor=cyan,
}
%formatting
\newcommand{\bld}{\textbf}
\newcommand{\itl}{\textit}
\newcommand{\uln}{\underline}

%math word symbols
\newcommand{\bb}{\mathbb}
\DeclareMathOperator{\tif}{~\text{if}~}
\DeclareMathOperator{\tand}{~\text{and}~}
\DeclareMathOperator{\tbut}{~\text{but}~}
\DeclareMathOperator{\tor}{~\text{or}~}
\DeclareMathOperator{\tsuchthat}{~\text{such that}~}
\DeclareMathOperator{\tsince}{~\text{since}~}
\DeclareMathOperator{\twhen}{~\text{when}~}
\DeclareMathOperator{\twhere}{~\text{where}~}
\DeclareMathOperator{\tfor}{~\text{for}~}
\DeclareMathOperator{\tthen}{~\text{then}~}

%display shortcut
\DeclareMathOperator{\dstyle}{\displaystyle}
\DeclareMathOperator{\sstyle}{\scriptstyle}

%linear algebra
\DeclareMathOperator{\id}{\bld{id}}
\DeclareMathOperator{\vecspan}{\text{span}}

%augmented matrix environment
\newenvironment{amatrix}[1]{%
  \left(\begin{array}{@{}*{#1}{c}|c@{}}
    }{
  \end{array}\right)
}

%lists
\newcommand{\bitem}[1]{\item[\bld{#1.}]}
\newcommand{\bbitem}[2]{\item[\bld{#1.}] \bld{#2}}
\newcommand{\biitem}[2]{\item[\bld{#1.}] \itl{#2}}
\newcommand{\iitem}[1]{\item[\itl{#1.}]}
\newcommand{\iiitem}[2]{\item[\itl{#1.}] \bld{#2}}

%homework
\newenvironment*{question}[2]{
  \subsection*{#1} \itl{#2}
  \begin{enumerate}
    }{
  \end{enumerate}
}
\newcommand{\qitem}[2]{\item[\bld{#1}] \itl{#2}}

\newcommand{\assignment}{Section 4 Groups, p45 \#2,3,5,10,11-16 all}

\lhead{MAT 311 Abstract Algebra}
\chead{\assignment}
\rhead{Peter Schaefer}

\begin{document}
\section*{\assignment}

In Exercises 1 through 6, determine whether the binary operation $*$ gives a group structure on the given set. If no group results, give the first axiom in order $\mathfrak{G}_1,\mathfrak{G}_2,\mathfrak{G}_3$ from Definition 4.1 that does not hold.
\begin{enumerate}
    \qitem{2}{Let $*$ be defined on $\bb{Z}$ by letting $a*b = ab$.}{
        $\mathfrak{G}_2$ (identity) does not hold. One might consider $1$ to be the identity, but $1 \cdot 0 = 0$. In fact, $n \cdot 0 = 0$ for any such $n \in \bb{Z}$. So no identity can exist with this $*$ on $\bb{Z}$.
    }
    \qitem{3}{Let $*$ be defined on $2\bb{Z} = \{2n : n \in \bb{Z}\}$ by letting $a*b = ab$.}{
        $\mathfrak{G}_2$ (identity) does not hold. There is no such element $e$ where $e * n = n$, for any $n \in 2\bb{Z}$.
    }
    \qitem{5}{Let $*$ be defined on the set $\bb{R}^*$ of nonzero real numbers by letting $a*b = a/b$.}{
        $\mathfrak{G}_2$ (identity) is only partially held. $1$ is a right identity, as $x * 1 = x/1 = x$ for all $x \in \bb{R}^*$. However, this does not apply to the left as $1*x = 1/x \neq x$ unless $x = 1$. Since both are required for this axiom to apply, it is not satisfied.
    }
    \qitem{10}{Let $n$ be a positive integer and let $n\bb{Z} = \{nm | m \in \bb{Z}\}$.}{
        Show the following:
        \begin{enumerate}
            \qitem{a}{$\group{n\bb{Z}, +}$ is a group.}{
                A group must be closed, associative, have an identity, and have an inverse for every element.
                \begin{proof}
                    Consider $\group{n\bb{Z}, +}$.
                    \begin{enumerate}
                        \item Closed: Consider $nm \tand np$ for $m,p \in \bb{Z}$ and $n \in \bb{Z}^+$.
                        \[
                            nm + np = n(m+p)
                        \]
                        Thus $\group{n\bb{Z}, +}$ is closed under addition.

                        \item Associativity: Consider $nm,np,nq \in n\bb{Z}$.
                        \begin{align*}
                            (nm + np) + nq & = n(m+p) + nq \\
                            & = n(m+p+q) \\ \\
                            nm + (np + nq) & = nm + n(p+q) \\
                            & = n(m+p+q)
                        \end{align*}
                        Thus $\group{n\bb{Z}, +}$ is associative.

                        \item Identity: Consider $n0 \tand nm \in n\bb{Z}$.
                        \begin{align*}
                            n0 + nm & = n(0+m) = nm \\
                            nm + n0 & = n(m+0) = nm
                        \end{align*}
                        Thus $\group{n\bb{Z}, +}$ has an identity.
                        
                        \item Inverse: Consider $nm + n\overline{m} = n0$
                        \begin{align*}
                            nm + n\overline{m} & = n0 \\
                            n\overline{m} & = n0 + (-nm) \\
                            n\overline{m} & = -nm
                        \end{align*}
                        Thus every element $nm$ has inverse $-nm$.
                    \end{enumerate}
                    Because $\group{n\bb{Z}, +}$ is closed, associative, has an identity, and an inverse for every element, it is a group.
                \end{proof}
            }
            \qitem{b}{$\group{n\bb{Z}, +} \simeq \group{\bb{Z},+}$.}{answer}
        \end{enumerate}
    }
\end{enumerate}
In exercises 11 through 18, determine whether the given set of matrices under the specified operation, matrix addition or multiplication, is a group.
\begin{enumerate}
    \qitem{11}{All $n \times n$ diagonal matrices under matrix addition.}{answer}
    \qitem{12}{All $n \times n$ diagonal matrices under matrix multiplication.}{answer}
    \qitem{13}{All $n \times n$ diagonal matricies with no zero diagonal entry under matrix multiplication.}{answer}
    \qitem{14}{All $n \times n$ diagonal matricies with all diagonal entries 1 or -1 under matrix multiplication}{answer}
    \qitem{15}{All $n \times n$ upper-triangular matricies under matrix multiplication.}{answer}
    \qitem{16}{All $n \times n$ upper-triangular matricies under matrix addition.}{answer}
\end{enumerate}

\end{document}