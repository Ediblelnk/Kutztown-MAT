\documentclass{article}
\usepackage[margin=1in]{geometry}
\usepackage{amsmath, amsthm, amssymb, fancyhdr, tikz, circuitikz, graphicx}
\usepackage{centernot, xcolor, hhline, multirow, listings, dashrule}
\usepackage{blkarray, booktabs, bigstrut, etoolbox, extarrows}
\usepackage[normalem]{ulem}
\usepackage{bookmark}
\usepackage{enumerate}          %These two package give custom labels to a list
\usepackage[shortlabels]{enumitem}
\usetikzlibrary{math}
\usetikzlibrary{fit}

\pagestyle{fancy}

\usepackage{hyperref}
\hypersetup{
    colorlinks=true,
    linkcolor=black,
    filecolor=magenta,
    urlcolor=cyan,
}
%formatting
\newcommand{\bld}{\textbf}
\newcommand{\itl}{\textit}
\newcommand{\uln}{\underline}

%math word symbols
\newcommand{\bb}{\mathbb}
\DeclareMathOperator{\tif}{~\text{if}~}
\DeclareMathOperator{\tand}{~\text{and}~}
\DeclareMathOperator{\tbut}{~\text{but}~}
\DeclareMathOperator{\tor}{~\text{or}~}
\DeclareMathOperator{\tsuchthat}{~\text{such that}~}
\DeclareMathOperator{\tsince}{~\text{since}~}
\DeclareMathOperator{\twhen}{~\text{when}~}
\DeclareMathOperator{\twhere}{~\text{where}~}
\DeclareMathOperator{\twith}{~\text{with}~}
\DeclareMathOperator{\tfor}{~\text{for}~}
\DeclareMathOperator{\tthen}{~\text{then}~}
\DeclareMathOperator{\tto}{~\text{to}~}
\DeclareMathOperator{\tin}{~\text{in}~}

%display shortcut
\DeclareMathOperator{\dstyle}{\displaystyle}
\DeclareMathOperator{\sstyle}{\scriptstyle}

%linear algebra
\DeclareMathOperator{\id}{\bld{id}}
\DeclareMathOperator{\vecspan}{\text{span}}
\DeclareMathOperator{\adj}{\text{adj}}

%discrete math - integer properties
\DeclareMathOperator{\tdiv}{\text{div}}
\DeclareMathOperator{\tmod}{\text{mod}}
\DeclareMathOperator{\lcm}{\text{lcm}}

%augmented matrix environment
\newenvironment{apmatrix}[2]{%
    \left(\begin{array}{@{~}*{#1}{c}|@{~}*{#2}{c}}
        }{
    \end{array}\right)
}
\newenvironment{abmatrix}[2]{%
    \left[\begin{array}{@{~}*{#1}{c}|@{~}*{#2}{c}}
            }{
        \end{array}\right]
}

\newenvironment{determinant}[1]{
    \left\lvert
    \begin{array}{@{~}*{#1}{c}}
        }{
    \end{array}
    \right\rvert
}

% graph theory
\DeclareMathOperator{\diam}{\text{diam}}
\newcommand{\comp}[1]{\overline{#1}}

% calculus iii
\newcommand{\vvec}{\overrightarrow}
\newcommand{\norm}[1]{\lVert #1 \rVert}

% abstract algebra
\newcommand{\struct}[1]{\langle #1 \rangle}

%lists
\newcommand{\bitem}[1]{\item[\bld{#1.}]}
\newcommand{\bbitem}[2]{\item[\bld{#1.}] \bld{#2}}
\newcommand{\biitem}[2]{\item[\bld{#1.}] \itl{#2}}
\newcommand{\iitem}[1]{\item[\itl{#1.}]}
\newcommand{\iiitem}[2]{\item[\itl{#1.}] \bld{#2}}
\newcommand{\btitem}[2]{\item[\bld{#1.}] \texttt{#2}}

%homework
\newcommand{\question}[2]{\noindent {\large\bld{#1}} #2 \qline}
\newcommand{\qitem}[3]{\item[\bld{#1.}] #2 \qdash \\ #3 \qdash}

\newcommand{\qline}{~\newline\noindent\textcolor[RGB]{200,200,200}{\rule[0.5ex]{\linewidth}{0.2pt}}}
\newcommand{\qdash}{~\newline\noindent\textcolor[RGB]{200,200,200}{\hdashrule[0.5ex]{\linewidth}{0.2pt}{2pt}}}

\newcommand{\assignment}{Section 5 Subgroups, p55 \#21-25, 27-28, 43, 47, 55}

\lhead{MAT 311 Abstract Algebra}
\chead{\assignment}
\rhead{Peter Schaefer}

\begin{document}
\section*{\assignment}

\question{21}{Write at least 5 elements of each of the following cyclic groups.}
\begin{enumerate}
    \qitem{a}{$25\bb{Z}$ under addition}{$\{25,50,75,100,125,\ldots\}$}
    \qitem{b}{$\{(\frac{1}{2})^n : n \in \bb{Z}\}$ under multiplication}{$\{\frac{1}{2},\frac{1}{4},\frac{1}{8},\frac{1}{16},\frac{1}{32},
    \ldots\}$}
    \qitem{c}{$\{\pi^n : n \in \bb{Z}\}$ under multiplication}{$\{1, \pi, \pi^2, \pi^3, \pi^4,\ldots\}$}
\end{enumerate}
In Exercises 22 through 25, describe all the elements in the cyclic subgroup of $GL(2,\bb{R})$ generated by the given $2 \times 2$ matrix.
\begin{enumerate}
    \qitem{22}{$\begin{bmatrix} 0 & -1 \\ -1 & 0 \end{bmatrix}$}{
        $
            \begin{bmatrix}
                1 & 0 \\
                0 & 1
            \end{bmatrix}, 
            \begin{bmatrix}
                0 & -1 \\
                -1 & 0
            \end{bmatrix}
        $
    }
    \qitem{23}{$\begin{bmatrix} 1 & 1 \\ 0 & 1 \end{bmatrix}$}{
        $
            \ldots,
            \begin{bmatrix}
                1 & -2 \\
                0 & 1
            \end{bmatrix},
            \begin{bmatrix}
                1 & -1 \\
                0 & 1
            \end{bmatrix},
            \begin{bmatrix}
                1 & 0 \\
                0 & 1
            \end{bmatrix},
            \begin{bmatrix}
                1 & 1 \\
                0 & 1
            \end{bmatrix},
            \begin{bmatrix}
                1 & 2 \\
                0 & 1
            \end{bmatrix}, \ldots
        $
    }
    \qitem{24}{$\begin{bmatrix} 3 & 0 \\ 0 & 2 \end{bmatrix}$}{
        $
            \ldots,
            \begin{bmatrix}
                3^{-2} & 0 \\
                0 & 2^{-2}
            \end{bmatrix},
            \begin{bmatrix}
                3^{-1} & 0 \\
                0 & 2^{-1}
            \end{bmatrix},
            \begin{bmatrix}
                3^{0} & 0 \\
                0 & 2^{0}
            \end{bmatrix},
            \begin{bmatrix}
                3^{1} & 0 \\
                0 & 2^{1}
            \end{bmatrix},
            \begin{bmatrix}
                3^{2} & 0 \\
                0 & 2^{2}
            \end{bmatrix}, \ldots
        $
    }
    \qitem{25}{$\begin{bmatrix} 0 & -2 \\ -2 & 0 \end{bmatrix}$}{
        $
            \begin{bmatrix}
                0 & -2^{2k+1} \\
                -2^{2k+1} & 0
            \end{bmatrix},
            \begin{bmatrix}
                2^{2k} & 0 \\
                0 & 2^{2k}
            \end{bmatrix}, k \in \bb{Z}
        $
    }
\end{enumerate}
In Exercises 27 through 35, find the cyclic subgroup of the given group generated by the indicated element.
\begin{enumerate}
    \qitem{27}{The subgroup of $\bb{Z}_4$ generated by 3}{
        $
            \struct{3} = \{3, 3^2 = 2, 3^3 = 1, 3^4 = 0\}
        $
    }
    \qitem{28}{The subgroup of $V$ generated by $c$ 
    \[
        \begin{array}{c|cccc}
            V & e & a & b & c \\
            \hline
            e & e & a & b & c \\
            a & a & e & c & b \\
            b & b & c & e & a \\
            c & c & b & a & e
        \end{array}
    \]}{
        The subgroup:
        \[
            \struct{c} =
            \begin{array}{c|cc}
                & e & c \\
                \hline
                e & e & c \\
                c & c & e
            \end{array}
        \]
    }
\end{enumerate}
Theory
\begin{enumerate}
    \qitem{43}{Show that if $H \tand K$ are subgroups of an Abelian group $G$, then
    \[
        \{hk : h \in H \tand k \in K\}
    \]
    is a subgroup of G.}{
        To show this we must show closure and closure of inverses.
        \begin{proof}
            Let $L = \{hk : h \in H \tand k \in K\}$
            \begin{enumerate}
                \item Closure: Consider $h_1k_1, h_2k_2 \in L$.
                \[
                    h_1k_1h_2k_2 = h_1h_2k_1k_2 = (h_1h_2)(k_1k_2) \in L,
                \]
                since we know that $G$ is Abelian, and $H \tand K$ are closed. Thus this takes the form of an element in $L$, and $L$ is closed.
                \item Closure of Inverses: Consider $hk \in L$.
                \[
                    (hk)^{-1} = k^{-1}h^{-1} = h^{-1}k^{-1} \in L,
                \]
                since we know $G$ is Abelian, and $H \tand K$ are closed for inverses. Thus this takes the form of an element in $L$, and $L$ is closed for inverses.
            \end{enumerate}
            Since $L$ is closed and closed for inverses, it is a subgroup of $G$.
        \end{proof}
    }
    \qitem{47}{Prove that if $G$ is an Abelian group, written multiplicatively, with identity element $e$, then all elements $x$ of $G$ satisfying the equation $x^2 = e$ form a subgroup $H$ of $G$.}{
        To show this we must show closure and closure of inverses.
        \begin{proof}
            Assume the information above, with such a subset called $X$. Consider $x,y \in X$.
            \[
                (xy)^2 = xyxy = x^2y^2 = ee = e,
            \]
            since $G$ is Abelian. Thus the defining property of this set is satisfied, and $X$ is closed under the operation of $G$. Now consider $x^{-1} \in G$.
            \[
                (x^{-1})^2 = x^{-2} = (x^2)^{-1} = e^{-1} = e.
            \]
            Thus $x^{-1}$ satisfies the defining property of $X$, and $X$ is closed under inverses. Since $X$ is closed under the operation of $G$ and closed for inverses, $X$ is a subgroup of $G$.
        \end{proof}
    }
    \qitem{55}{Prove that every cylic group is Abelian.}{
        A cyclic group takes the form $\struct{a} = \{a^n : n \in \bb{Z}\}$.
        \begin{proof}
            We must prove that $ab = ba$ for any $a,b \in \struct{a}$. Consider $a^n \tand a^m \in \struct{a}$ for $n,m \in \bb{Z}$.
            \[
                a^na^m = a^{n+m} = a^{m+n} = a^ma^n
            \]
            Thus, $a^na^m = a^ma^n$, and any cyclic group is also Abelian.
        \end{proof}
    }
\end{enumerate}

\end{document}