\documentclass{article}
\usepackage[margin=1in]{geometry}
\usepackage{amsmath, amsthm, amssymb, fancyhdr, tikz, circuitikz, graphicx}
\usepackage{centernot, xcolor, hhline, multirow, listings, dashrule}
\usepackage{blkarray, booktabs, bigstrut, etoolbox, extarrows}
\usepackage[normalem]{ulem}
\usepackage{bookmark}
\usepackage{enumerate}          %These two package give custom labels to a list
\usepackage[shortlabels]{enumitem}
\usetikzlibrary{math}
\usetikzlibrary{fit}

\pagestyle{fancy}

\usepackage{hyperref}
\hypersetup{
    colorlinks=true,
    linkcolor=black,
    filecolor=magenta,
    urlcolor=cyan,
}
%formatting
\newcommand{\bld}{\textbf}
\newcommand{\itl}{\textit}
\newcommand{\uln}{\underline}

%math word symbols
\newcommand{\bb}{\mathbb}
\DeclareMathOperator{\tif}{~\text{if}~}
\DeclareMathOperator{\tand}{~\text{and}~}
\DeclareMathOperator{\tbut}{~\text{but}~}
\DeclareMathOperator{\tor}{~\text{or}~}
\DeclareMathOperator{\tsuchthat}{~\text{such that}~}
\DeclareMathOperator{\tsince}{~\text{since}~}
\DeclareMathOperator{\twhen}{~\text{when}~}
\DeclareMathOperator{\twhere}{~\text{where}~}
\DeclareMathOperator{\tfor}{~\text{for}~}
\DeclareMathOperator{\tthen}{~\text{then}~}

%display shortcut
\DeclareMathOperator{\dstyle}{\displaystyle}
\DeclareMathOperator{\sstyle}{\scriptstyle}

%linear algebra
\DeclareMathOperator{\id}{\bld{id}}
\DeclareMathOperator{\vecspan}{\text{span}}

%augmented matrix environment
\newenvironment{amatrix}[1]{%
  \left(\begin{array}{@{}*{#1}{c}|c@{}}
    }{
  \end{array}\right)
}

%lists
\newcommand{\bitem}[1]{\item[\bld{#1.}]}
\newcommand{\bbitem}[2]{\item[\bld{#1.}] \bld{#2}}
\newcommand{\biitem}[2]{\item[\bld{#1.}] \itl{#2}}
\newcommand{\iitem}[1]{\item[\itl{#1.}]}
\newcommand{\iiitem}[2]{\item[\itl{#1.}] \bld{#2}}

%homework
\newenvironment*{question}[2]{
  \subsection*{#1} \itl{#2}
  \begin{enumerate}
    }{
  \end{enumerate}
}
\newcommand{\qitem}[2]{\item[\bld{#1}] \itl{#2}}

\newcommand{\assignment}{Section 0, p8 12, 16, 17, 23, 25, 29, 31, 33}

\lhead{MAT 311 Abstract Algebra}
\chead{\assignment}
\rhead{Peter Schaefer}

\begin{document}
\section*{\assignment}

\question{12}{Let $A = \{1,2,3\} \tand B = \{2,4,6\}$. For each relation between $A \tand B$ given as a subset of $A \times B$, decide whether it is a function mapping $A$ into $B$. If it is a function, decide whether it is one to one and whether it is onto $B$.}
\begin{enumerate}
    \qitem{a}{$\{(1,4),(2,4),(3,6)\}$}{Since no two ordered pairs have the same first term, this \bld{is a function}. Since $(1,4)$ and $(2,4)$ share a second term, this function \bld{is \itl{not} one to one}. Since 2 does not appear in the relation in the second term, this function \bld{is \itl{not} onto $B$}.}
    \qitem{b}{$\{(1,4),(2,6),(3,4)\}$}{Since no two ordered pairs have the same first term, this \bld{is a function}. Since $(1,4)$ and $(3,4)$ share a second term, this function \bld{is \itl{not} one to one}. Since 2 does not appear in the relation in the second term, this function \bld{is \itl{not} onto $B$}.}
    \qitem{c}{$\{(1,6),(1,2),(1,4)\}$}{Since $(1,6)$ and $(1,2)$ share a first term, this \bld{is \itl{not} a function}.}
    \qitem{d}{$\{(2,2),(1,6),(3,4)\}$}{Since no two ordered pairs have the same first term, this \bld{is a function}. Since no two ordered pairs share the same second term, this function \bld{is one to one}. Since every element of $B$ appears in the range of the function, it \bld{is onto $B$}.}
    \qitem{e}{$\{(1,6),(2,6),(3,6)\}$}{Since no two ordered pairs have the same first term, this \bld{is a function}. Since $(1,6)$ and $(2,6)$ share a second term, this function \bld{is \itl{not} one to one}. Since 2 does not appear in the relation in the second term, this function \bld{is \itl{not} onto $B$}.}
    \qitem{f}{$\{(1,2),(2,6),(2,4)\}$}{Since $(2,6)$ and $(2,4)$ share a first term, this \bld{is \itl{not} a function}.}
\end{enumerate}

\question{16}{List the elements of the power set $\mathcal{P}$ of the given set and give the cardinality of the power set.}
\begin{enumerate}
    \qitem{a}{$\emptyset$}{$|\{\emptyset\}| = 1$}
    \qitem{b}{$\{a\}$}{$|\{\emptyset, \{a\}\}| = 2$}
    \qitem{c}{$\{a,b\}$}{$|\{\emptyset, \{a\}, \{b\}, \{a,b\}\}| = 4$}
    \qitem{d}{$\{a,b,c\}$}{$|\{\emptyset, \{a\}, \{b\}, \{c\}, \{a,b\}, \{a,c\}, \{b,c\}, \{a,c,b\}\}| = 8$}
\end{enumerate}

\question{17}{Let $A$ be a finite set, and let $|A| = s$. Based on the preceding exercise, make a conjecture about the value of $|\mathcal{P}(A)|$. Then try to prove your conjecture.}
Based on the previous exercise, I conjecture that $|\mathcal{P}(A)| = 2^s$.
\begin{proof}
    We shall conduct a \itl{proof through induction}.

    \bld{Base case}: Let $A = \emptyset \tand s = 0$. $\mathcal{P}(A) = \{\emptyset\}, \tand |\mathcal{P}(A)| = 1 = 2^0$. Thus the base case holds.

    \bld{Inductive hypothesis}: Let $|A| = k$ for some $k \in \bb{Z}^{\geq0}$, and assume that $|\mathcal{P}(A)| = 2^k$.

    \bld{Inductive case}: Consider $|A| = k$. Let us add an element $e$ to $A$ to create $A'$. To consider $\mathcal{P}(A')$, start with all of the sets in $\mathcal{P}(A)$, which has a cardinality of $2^k$ via the inductive hypothesis.

    \indent For each of these sets, we have a choice when we add it to $\mathcal{P}(A')$ to include element $e$ or to leave it alone, which are two choices. We will end with the original sets of the $\mathcal{P}(A)$ without $e$, along with these sets with $e$, in $\mathcal{P}(A')$. Each of these groups has a cardinality of $2^k$, so the cardinality of $\mathcal{P}(A')$ must be
    \[
        2^k + 2^k = 2 \cdot 2^k = 2^{k+1}.
    \]

    Thus, the inductive case holds, and the conjecture is correct.
\end{proof}

\noindent In Exercises 23 through 27, find the number of different partitions of a set having the given number of elements.
\begin{enumerate}
    \qitem{23}{1 element}{There is a single partition, $\{\{e\}\}$}
    \qitem{25}{3 elements}{There are 5 partitions, \[\{\{e_1\}, \{e_2\}, \{e_3\}\},~~~ \{\{e_1\}, \{e_2,e_3\}\},~~~ \{\{e_1, e_2\}, \{e_3\}\},~~~ \{\{e_1, e_3\}, \{e_2\}\},~~~ \{\{e_1, e_2, e_3\}\}\]}
\end{enumerate}

\noindent In Exercises 29 through 34, determine whether the given relation is an equivalence relation on the set. Describe the partition arising from each equivalence relation.
\begin{enumerate}
    \qitem{29}{$n\mathcal{R}m \tin \bb{Z} \tif nm > 0$}{Consider $0 \in \bb{Z}$. We know $0 \cdot 0 \ngtr 0$, thus $0$ is not related to itself. So the relation is not reflexive, and cannot be an equivalence relation.}
    \qitem{31}{$x\mathcal{R}y \tin \bb{R} \tif |x| = |y|$}{
        A equivalence relation must be reflexive, symmetric, and transitive.
        \begin{enumerate}
            \item Reflexive: Consider $x \in \bb{R}$. We know $|x|=|x|$, thus $\mathcal{R}$ is reflexive.
            \item Symmetric: Assume $x\mathcal{R}y$. We know $|x| = |y|$ also implies $|y| = |x|$, thus $y\mathcal{R}x$. So $\mathcal{R}$ is symmetric.
            \item Transitive: Assume $x\mathcal{R}y \tand y\mathcal{R}z$. We $|x| = |y| \tand |y| = |z|$ which implies $|x| = |z|$, thus $z\mathcal{R}x$. So $\mathcal{R}$ is transitive.
        \end{enumerate}
        The partition resulting from this equivalence relation will be sets $\overline{x} = \{x, -x\}$ for each $x \in \bb{R}$.
    }
    \qitem{33}{$n\mathcal{R}m \tin \bb{Z}^+ \tif n \tand m$ have the same number of digits in the usual base ten notation}{
        A equivalence relation must be reflexive, symmetric, and transitive.
        \begin{enumerate}
            \item Reflexive: Consider $x \in \bb{Z}^+$. We know a number has the same digits as itself, thus $\mathcal{R}$ is reflexive.
            \item Symmetric: Assume $x\mathcal{R}y$. We know $x \tand y$ have the same number of digits, implying $y \tand x$ have the same number of digits, thus $y\mathcal{R}x$. So $\mathcal{R}$ is symmetric.
            \item Transitive: Assume $x\mathcal{R}y \tand y\mathcal{R}z$. We know $x \tand y$ have the same number of digits, and that $y \tand z$ have the same number of digits, which implies $x \tand z$ have the same number of digits, thus $z\mathcal{R}x$. So $\mathcal{R}$ is transitive.
        \end{enumerate}
        The partition resulting from this equivalence relation will be classes where each member has the same number of digits, i.e. all single digit numbers will be in a class, all two digit numbers will be in a class, all three digit numbers will be in a class, etc.
    }
\end{enumerate}

\end{document}