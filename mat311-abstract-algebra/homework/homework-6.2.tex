\documentclass{article}
\usepackage[margin=1in]{geometry}
\usepackage{amsmath, amsthm, amssymb, fancyhdr, tikz, circuitikz, graphicx}
\usepackage{centernot, xcolor, hhline, multirow, listings, dashrule}
\usepackage{blkarray, booktabs, bigstrut, etoolbox, extarrows}
\usepackage[normalem]{ulem}
\usepackage{bookmark}
\usepackage{enumerate}          %These two package give custom labels to a list
\usepackage[shortlabels]{enumitem}
\usetikzlibrary{math}
\usetikzlibrary{fit}

\pagestyle{fancy}

\usepackage{hyperref}
\hypersetup{
    colorlinks=true,
    linkcolor=black,
    filecolor=magenta,
    urlcolor=cyan,
}
%formatting
\newcommand{\bld}{\textbf}
\newcommand{\itl}{\textit}
\newcommand{\uln}{\underline}

%math word symbols
\newcommand{\bb}{\mathbb}
\DeclareMathOperator{\tif}{~\text{if}~}
\DeclareMathOperator{\tand}{~\text{and}~}
\DeclareMathOperator{\tbut}{~\text{but}~}
\DeclareMathOperator{\tor}{~\text{or}~}
\DeclareMathOperator{\tsuchthat}{~\text{such that}~}
\DeclareMathOperator{\tsince}{~\text{since}~}
\DeclareMathOperator{\twhen}{~\text{when}~}
\DeclareMathOperator{\twhere}{~\text{where}~}
\DeclareMathOperator{\twith}{~\text{with}~}
\DeclareMathOperator{\tfor}{~\text{for}~}
\DeclareMathOperator{\tthen}{~\text{then}~}
\DeclareMathOperator{\tto}{~\text{to}~}
\DeclareMathOperator{\tin}{~\text{in}~}

%display shortcut
\DeclareMathOperator{\dstyle}{\displaystyle}
\DeclareMathOperator{\sstyle}{\scriptstyle}

%linear algebra
\DeclareMathOperator{\id}{\bld{id}}
\DeclareMathOperator{\vecspan}{\text{span}}
\DeclareMathOperator{\adj}{\text{adj}}

%discrete math - integer properties
\DeclareMathOperator{\tdiv}{\text{div}}
\DeclareMathOperator{\tmod}{\text{mod}}
\DeclareMathOperator{\lcm}{\text{lcm}}

%augmented matrix environment
\newenvironment{apmatrix}[2]{%
    \left(\begin{array}{@{~}*{#1}{c}|@{~}*{#2}{c}}
        }{
    \end{array}\right)
}
\newenvironment{abmatrix}[2]{%
    \left[\begin{array}{@{~}*{#1}{c}|@{~}*{#2}{c}}
            }{
        \end{array}\right]
}

\newenvironment{determinant}[1]{
    \left\lvert
    \begin{array}{@{~}*{#1}{c}}
        }{
    \end{array}
    \right\rvert
}

% graph theory
\DeclareMathOperator{\diam}{\text{diam}}
\newcommand{\comp}[1]{\overline{#1}}

% calculus iii
\newcommand{\vvec}{\overrightarrow}
\newcommand{\norm}[1]{\lVert #1 \rVert}

% abstract algebra
\newcommand{\struct}[1]{\langle #1 \rangle}

%lists
\newcommand{\bitem}[1]{\item[\bld{#1.}]}
\newcommand{\bbitem}[2]{\item[\bld{#1.}] \bld{#2}}
\newcommand{\biitem}[2]{\item[\bld{#1.}] \itl{#2}}
\newcommand{\iitem}[1]{\item[\itl{#1.}]}
\newcommand{\iiitem}[2]{\item[\itl{#1.}] \bld{#2}}
\newcommand{\btitem}[2]{\item[\bld{#1.}] \texttt{#2}}

%homework
\newcommand{\question}[2]{\noindent {\large\bld{#1}} #2 \qline}
\newcommand{\qitem}[3]{\item[\bld{#1.}] #2 \qdash \\ #3 \qdash}

\newcommand{\qline}{~\newline\noindent\textcolor[RGB]{200,200,200}{\rule[0.5ex]{\linewidth}{0.2pt}}}
\newcommand{\qdash}{~\newline\noindent\textcolor[RGB]{200,200,200}{\hdashrule[0.5ex]{\linewidth}{0.2pt}{2pt}}}

\newcommand{\assignment}{Section 6, p66 \#9,11,17,19,23-25,27,35,37,50}

\lhead{MAT 311 Abstract Algebra}
\chead{\assignment}
\rhead{Peter Schaefer}

\begin{document}
\section*{\assignment}

In Exercises 8 through 11, find the number of generators of a cyclic group having the given order
\begin{enumerate}
    \qitem{9}{$8$}{$8$ has prime factorization of $2^3$, so generators are $\{1,3,5,7\}$.}
    \qitem{11}{$60$}{$60$ has prime factorization of $2^2\cdot 3 \cdot 5$. Generators are
        \[
            \{1,7,11,13,17,19,23,29,31,37,41,43,47,49,53,59\}
        \]
    }
\end{enumerate}
In Exercises 17 through 21, find the number of elements in the indicated cyclic group.
\begin{enumerate}
    \qitem{17}{The cyclic subgroup of $\bb{Z}_30$, generated by $25$}{$\struct{25} = \{25,20,15,10,5,0\}$}
    \qitem{19}{The cyclic subgroup $\struct{i}$ of the group $\bb{C}^*$ of nonzero complex numbers under multiplication}{$\{i,-1,-i,1\}$}
\end{enumerate}
In Exercises 22 through 24, find all the subgroups of the given group, and draw the subgroup diagram for the subgroups.
\begin{enumerate}
    \qitem{23}{$\bb{Z}_{36}$}{
        Subgroups and their generators:
        \begin{itemize}
            \item $\struct{1} = \struct{5} = \struct{7} = \struct{11} = \struct{13} = \struct{17} = \struct{19} = \struct{23} = \struct{25} = \struct{29} = \struct{31} = \struct{35} = \bb{Z}_{36}$
            \item $\struct{2} = \struct{10} = \struct{14} = \struct{22} = \struct{26} = \struct{34} = \{0,2,4,6,8,10,12,14,16,18,20,22,24,26,28,30,32,34\}$
            \item $\struct{3} = \struct{15} = \struct{21} = \struct{33} = \{0,3,6,9,12,15,18,21,24,27,30,33\}$
            \item $\struct{4} = \struct{8} = \struct{16} = \struct{20} = \struct{28} = \struct{32} = \{0,4,8,12,16,20,24,28,32\}$
            \item $\struct{6} = \struct{30} = \{0,6,12,18,24,30\}$
            \item $\struct{9} = \struct{27} = \{0,9,18,27\}$
            \item $\struct{12} = \struct{24} = \{0,12,24\}$
            \item $\struct{18} = \{0,18\}$
        \end{itemize}
    }
    \qitem{24}{$\bb{Z}_8$}{
        Subgroups and their generators:
        \begin{itemize}
            \item $\struct{1} = \struct{3} = \struct{5} = \struct{7} = \bb{Z}_8$
            \item $\struct{2} = \struct{6} = \{0,2,4,6\}$
            \item $\struct{4} = \{0,4\}$
        \end{itemize}
    }
\end{enumerate}
In Exercises 25 through 29, find all the orders of the subgroups of the given group.
\begin{enumerate}
    \qitem{25}{$\bb{Z}_6$}{
        $|\struct{1}| = |\struct{5}| = 6$. $|\struct{2}| = \struct{4} = 3$. $|\struct{3}| = 2$.
    }
    \qitem{27}{$\bb{Z}_{12}$}{
        $|\struct{1}| = |\struct{5}| = |\struct{7}| = |\struct{11}| = 12$. $|\struct{2}| = |\struct{10}| = 6$. $|\struct{3}| = |\struct{9}| = 4$. $|\struct{4}| = |\struct{8}| = 3$. $|\struct{6}| = 2$.
    }
\end{enumerate}
In Exercises 33 through 37, either give an example of a group with the property described, or explain why no example exists.
\begin{enumerate}
    \qitem{35}{A cyclic group having only one generator}{Consider $\bb{Z}_2 = \{0,1\}$. The only generator for this group is $\struct{1}$.}
    \qitem{37}{A finite cyclic group having four generators}{Consider $\bb{Z}_{8}$. This group has four generators: $\struct{1}, \struct{3}, \struct{5}, \struct{7}$.}
\end{enumerate}
Theory
\begin{enumerate}
    \qitem{50}{Let $G$ be a group and suppose $a \in G$ generates a cyclic subgroup of order $2$ and is the \itl{unique} such element. Show that $ax = xa$ for all $x \in G$. Hint: Consider $(xax^{-1})^2$}{
        Assume the above.
        \begin{proof}
            Consider $(xax^{-1})^2$.
            \begin{align*}
                (xax^{-1})^2 & = xax'xax'                                       \\
                             & = xa^2x'                                         \\
                             & = xx' = e  &  & \text{because by def. $a^2 = e$}
            \end{align*}
            Since $(xax^{-1})^2 = e$, but $a^2 =e$ and $a$ is the \itl{unique} such element to do so, it follows that
            \[
                xax^{-1} = a.
            \]
            We can now manipulate this equation.
            \begin{align*}
                xax^{-1}  & = a  \\
                xax^{-1}x & = ax \\
                xa        & = ax
            \end{align*}
            Thus we conclude that $xa = ax$, for all $x \in G$.
        \end{proof}
    }
\end{enumerate}

\end{document}