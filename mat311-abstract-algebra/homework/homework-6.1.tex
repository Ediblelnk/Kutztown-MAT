\documentclass{article}
\usepackage[margin=1in]{geometry}
\usepackage{amsmath, amsthm, amssymb, fancyhdr, tikz, circuitikz, graphicx}
\usepackage{centernot, xcolor, hhline, multirow, listings, dashrule}
\usepackage{blkarray, booktabs, bigstrut, etoolbox, extarrows}
\usepackage[normalem]{ulem}
\usepackage{bookmark}
\usepackage{enumerate}          %These two package give custom labels to a list
\usepackage[shortlabels]{enumitem}
\usetikzlibrary{math}
\usetikzlibrary{fit}

\pagestyle{fancy}

\usepackage{hyperref}
\hypersetup{
    colorlinks=true,
    linkcolor=black,
    filecolor=magenta,
    urlcolor=cyan,
}
%formatting
\newcommand{\bld}{\textbf}
\newcommand{\itl}{\textit}
\newcommand{\uln}{\underline}

%math word symbols
\newcommand{\bb}{\mathbb}
\DeclareMathOperator{\tif}{~\text{if}~}
\DeclareMathOperator{\tand}{~\text{and}~}
\DeclareMathOperator{\tbut}{~\text{but}~}
\DeclareMathOperator{\tor}{~\text{or}~}
\DeclareMathOperator{\tsuchthat}{~\text{such that}~}
\DeclareMathOperator{\tsince}{~\text{since}~}
\DeclareMathOperator{\twhen}{~\text{when}~}
\DeclareMathOperator{\twhere}{~\text{where}~}
\DeclareMathOperator{\twith}{~\text{with}~}
\DeclareMathOperator{\tfor}{~\text{for}~}
\DeclareMathOperator{\tthen}{~\text{then}~}
\DeclareMathOperator{\tto}{~\text{to}~}
\DeclareMathOperator{\tin}{~\text{in}~}

%display shortcut
\DeclareMathOperator{\dstyle}{\displaystyle}
\DeclareMathOperator{\sstyle}{\scriptstyle}

%linear algebra
\DeclareMathOperator{\id}{\bld{id}}
\DeclareMathOperator{\vecspan}{\text{span}}
\DeclareMathOperator{\adj}{\text{adj}}

%discrete math - integer properties
\DeclareMathOperator{\tdiv}{\text{div}}
\DeclareMathOperator{\tmod}{\text{mod}}
\DeclareMathOperator{\lcm}{\text{lcm}}

%augmented matrix environment
\newenvironment{apmatrix}[2]{%
    \left(\begin{array}{@{~}*{#1}{c}|@{~}*{#2}{c}}
        }{
    \end{array}\right)
}
\newenvironment{abmatrix}[2]{%
    \left[\begin{array}{@{~}*{#1}{c}|@{~}*{#2}{c}}
            }{
        \end{array}\right]
}

\newenvironment{determinant}[1]{
    \left\lvert
    \begin{array}{@{~}*{#1}{c}}
        }{
    \end{array}
    \right\rvert
}

% graph theory
\DeclareMathOperator{\diam}{\text{diam}}
\newcommand{\comp}[1]{\overline{#1}}

% calculus iii
\newcommand{\vvec}{\overrightarrow}
\newcommand{\norm}[1]{\lVert #1 \rVert}

% abstract algebra
\newcommand{\struct}[1]{\langle #1 \rangle}

%lists
\newcommand{\bitem}[1]{\item[\bld{#1.}]}
\newcommand{\bbitem}[2]{\item[\bld{#1.}] \bld{#2}}
\newcommand{\biitem}[2]{\item[\bld{#1.}] \itl{#2}}
\newcommand{\iitem}[1]{\item[\itl{#1.}]}
\newcommand{\iiitem}[2]{\item[\itl{#1.}] \bld{#2}}
\newcommand{\btitem}[2]{\item[\bld{#1.}] \texttt{#2}}

%homework
\newcommand{\question}[2]{\noindent {\large\bld{#1}} #2 \qline}
\newcommand{\qitem}[3]{\item[\bld{#1.}] #2 \qdash \\ #3 \qdash}

\newcommand{\qline}{~\newline\noindent\textcolor[RGB]{200,200,200}{\rule[0.5ex]{\linewidth}{0.2pt}}}
\newcommand{\qdash}{~\newline\noindent\textcolor[RGB]{200,200,200}{\hdashrule[0.5ex]{\linewidth}{0.2pt}{2pt}}}

\newcommand{\assignment}{Section 6, p66 \#1-4,45,46}

\lhead{MAT 311 Abstract Algebra}
\chead{\assignment}
\rhead{Peter Schaefer}

\begin{document}
\section*{\assignment}

In Exercises 1 through 4, find the quotient and remainder, according to the division algorithm, where $n$ is divided by $n$.
\begin{enumerate}
    \qitem{1}{$n = 42,~m=9$}{$42 = 4(9)+6$}
    \qitem{2}{$n = -42,~m = 9$}{$-42 = -5(9)+3$}
    \qitem{3}{$n = -50,~m = 8$}{$-50 = -7(8)+6$}
    \qitem{4}{$n = 50,~m = 8$}{$50 = 6(8)+2$}
\end{enumerate}
Theory
\begin{enumerate}
    \qitem{45}{Let $r \tand s$ be positive integers. Show that $X = \{nr+ms : n,m \in \bb{Z}\}$ is a subgroup of $\bb{Z}$}{
        A subgroup is closed under the operation of the group and closed under inverses.
        \begin{proof}
            Consider $a, \tand b \in X$, where $a = n_1r+m_1s$ and $b = n_2r+m_2s$.
            \begin{align*}
                a + b & = (n_1r+m_1s) + (n_2r+m_2s) \\
                & = (n_1r+n_2r) + (m_1s+m_2s) \\
                & = (n_1+n_2)r + (m_1+m_2)s
            \end{align*}
            Since $n_1 + n_2 \in \bb{Z}$ and $m_1 + m_2 \in \bb{Z}$, $a+b$ takes the form of elements in $X$. Now consider $a = nr + ms \in X$. We know $a^{-1}$ exists since $a$ is also in $\bb{Z}$.
            \begin{align*}
                a^{-1} & = (nr + ms)^{-1} \\
                & = -(nr + ms) && \text{since we are using $\bb{Z}$ with addition} \\
                & = (-n)r + (-m)s
            \end{align*}
            We know $-n \tand -m \in \bb{Z}$, thus $a^{-1} \in X$. Thus $X$ is a subgroup of $\bb{Z}$, $X < \bb{Z}$
        \end{proof}
    }
    \qitem{46}{Let $a \tand b$ be elements of a group $G$. Show that if $ab$ has finite order $n$, then $ba$ also has order $n$}{
        When we say $a$ has order $n$, it means that in the cyclic group that $a^n = e$.
        \begin{proof}
            Assume $\struct{ab}$ has order $n$. This means that
            \begin{align*}
                (ab)^n & = e \\
                a^nb^n & = e \\
            \end{align*}
            If we assume that $a^n = c$, for some $c \in G$, then it must follow that $b^n = c^{-1}$, since $a^nb^n = cc^{-1} = e$. Now let us consider the order of $\struct{ba}$.
            \begin{align*}
                (ba)^n & = b^na^n \\
                & = c^{-1}c && \text{since $a^n = c,~b^n = c^{-1}$}\\
                & = e
            \end{align*}
            Since $(ba)^n = e$, we can say that $ba$ has order $n$.
        \end{proof}
    }
\end{enumerate}

\end{document}