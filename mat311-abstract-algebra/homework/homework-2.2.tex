\documentclass{article}
\usepackage[margin=1in]{geometry}
\usepackage{amsmath, amsthm, amssymb, fancyhdr, tikz, circuitikz, graphicx}
\usepackage{centernot, xcolor, hhline, multirow, listings, dashrule}
\usepackage{blkarray, booktabs, bigstrut, etoolbox, extarrows}
\usepackage[normalem]{ulem}
\usepackage{bookmark}
\usepackage{enumerate}          %These two package give custom labels to a list
\usepackage[shortlabels]{enumitem}
\usetikzlibrary{math}
\usetikzlibrary{fit}

\pagestyle{fancy}

\usepackage{hyperref}
\hypersetup{
    colorlinks=true,
    linkcolor=black,
    filecolor=magenta,
    urlcolor=cyan,
}
%formatting
\newcommand{\bld}{\textbf}
\newcommand{\itl}{\textit}
\newcommand{\uln}{\underline}

%math word symbols
\newcommand{\bb}{\mathbb}
\DeclareMathOperator{\tif}{~\text{if}~}
\DeclareMathOperator{\tand}{~\text{and}~}
\DeclareMathOperator{\tbut}{~\text{but}~}
\DeclareMathOperator{\tor}{~\text{or}~}
\DeclareMathOperator{\tsuchthat}{~\text{such that}~}
\DeclareMathOperator{\tsince}{~\text{since}~}
\DeclareMathOperator{\twhen}{~\text{when}~}
\DeclareMathOperator{\twhere}{~\text{where}~}
\DeclareMathOperator{\twith}{~\text{with}~}
\DeclareMathOperator{\tfor}{~\text{for}~}
\DeclareMathOperator{\tthen}{~\text{then}~}
\DeclareMathOperator{\tto}{~\text{to}~}
\DeclareMathOperator{\tin}{~\text{in}~}

%display shortcut
\DeclareMathOperator{\dstyle}{\displaystyle}
\DeclareMathOperator{\sstyle}{\scriptstyle}

%linear algebra
\DeclareMathOperator{\id}{\bld{id}}
\DeclareMathOperator{\vecspan}{\text{span}}
\DeclareMathOperator{\adj}{\text{adj}}

%discrete math - integer properties
\DeclareMathOperator{\tdiv}{\text{div}}
\DeclareMathOperator{\tmod}{\text{mod}}
\DeclareMathOperator{\lcm}{\text{lcm}}

%augmented matrix environment
\newenvironment{apmatrix}[2]{%
    \left(\begin{array}{@{~}*{#1}{c}|@{~}*{#2}{c}}
        }{
    \end{array}\right)
}
\newenvironment{abmatrix}[2]{%
    \left[\begin{array}{@{~}*{#1}{c}|@{~}*{#2}{c}}
            }{
        \end{array}\right]
}

\newenvironment{determinant}[1]{
    \left\lvert
    \begin{array}{@{~}*{#1}{c}}
        }{
    \end{array}
    \right\rvert
}

% graph theory
\DeclareMathOperator{\diam}{\text{diam}}
\newcommand{\comp}[1]{\overline{#1}}

% calculus iii
\newcommand{\vvec}{\overrightarrow}
\newcommand{\norm}[1]{\lVert #1 \rVert}

% abstract algebra
\newcommand{\struct}[1]{\langle #1 \rangle}

%lists
\newcommand{\bitem}[1]{\item[\bld{#1.}]}
\newcommand{\bbitem}[2]{\item[\bld{#1.}] \bld{#2}}
\newcommand{\biitem}[2]{\item[\bld{#1.}] \itl{#2}}
\newcommand{\iitem}[1]{\item[\itl{#1.}]}
\newcommand{\iiitem}[2]{\item[\itl{#1.}] \bld{#2}}
\newcommand{\btitem}[2]{\item[\bld{#1.}] \texttt{#2}}

%homework
\newcommand{\question}[2]{\noindent {\large\bld{#1}} #2 \qline}
\newcommand{\qitem}[3]{\item[\bld{#1.}] #2 \qdash \\ #3 \qdash}

\newcommand{\qline}{~\newline\noindent\textcolor[RGB]{200,200,200}{\rule[0.5ex]{\linewidth}{0.2pt}}}
\newcommand{\qdash}{~\newline\noindent\textcolor[RGB]{200,200,200}{\hdashrule[0.5ex]{\linewidth}{0.2pt}{2pt}}}

\newcommand{\assignment}{Section 2 Binary Operations, p25 1,3,5,27,28,36}

\lhead{MAT 311 Abstract Algebra}
\chead{\assignment}
\rhead{Peter Schaefer}

\begin{document}
\section*{\assignment}

Exercises 1 through 4 concern the binary operation $*$ defined on $S = \{a,b,c,d,e\}$ by means of Table 2.26 (not shown).
\begin{enumerate}
    \qitem{1}{Compute $b*d$, $c*c$, and $[(a*c)*e]*a$}{
        Here are the computations:
        \begin{align*}
            b*d                         & = e \\
            c*c                         & = b \\
            [(a*c)*e]*a = [c*e]*a = a*a & = a
        \end{align*}
    }
    \qitem{3}{Compute $(b*d)*c \tand b*(d*c)$. Can you say on the basis of these computations whether $*$ is associative?}{
        Examples can only tell us if $*$ is not associative.
        \begin{align*}
            (b*d)*c & = e*c = a \\
            b*(d*c) & = b*b = c
        \end{align*}
        Since $a \neq c$, we know that $*$ is not associative.
    }
    \qitem{5}{Complete Table 2.27 so as to define a commutative binary operation $*$ on $S = \{a,b,c,d\}$.}{
        \bld{2.28 Table}
        \[
            \begin{array}{c|cccc}
                * & a & b       & c       & d       \\
                \hline
                a & a & b       & c       & \bld{d} \\
                b & b & d       & \bld{a} & c       \\
                c & c & a       & d       & b       \\
                d & d & \bld{c} & \bld{b} & a
            \end{array}
        \]
    }
\end{enumerate}

In Exercise 27 and 28, either prove the statement or give a counterexample.
\begin{enumerate}
    \qitem{27}{Every binary operation on a set consisting of a single element (is) commutative and associative.}{
        There is only one unique set consisting of a single element.
        \begin{proof}
            Consider $S = \{s\} \twhere s*s = s$.
            \begin{enumerate}
                \item Commutative: $s*s = s = s*s$. Thus $S$ is commutative under $*$.
                \item Associative: $s*(s*s) = s*s = s = s*s = (s*s)*s$. Thus $S$ is associative under $*$.
            \end{enumerate}
            Thus any binary operation on a set consisting of a single element is commutative and associative.
        \end{proof}
    }
    \qitem{28}{Every commutative binary operation on a set having just two elements is associative.}{
        We shall conduct a proof through counterexample.
        \begin{proof}
            Consider $S = \{a,b\}$ with $*$ such that $\begin{array}{c|cc}
                    * & a & b \\
                    \hline
                    a & b & a \\
                    b & a & a
                \end{array}$.
            \begin{align*}
                a*(a*b) = a * a & = b \\
                (a*a)*b = b * b & = a
            \end{align*}
            Since we assert that $b \neq a$, thus $a*(a*b) \neq (a*a)*b$, so $S$ is a binary operation, which is commutative, but not associative.
        \end{proof}
    }
\end{enumerate}

\question{36}{Suppose that $*$ is an \itl{associative binary operation} on a set $S$. Let $H = \{a \in S : a * x = x * a ~\text{for all}~ x \in S\}$. Show that $H$ is closed under $*$. (We think of $H$ as consisting of all elements of $S$ that \itl{commute} with every element in $S$.)}
\begin{proof}
    Consider $a,b,c,d \in H$. We need to show that $(a*b)*(c*d) = (c*d)*(a*b)$ for $H$ to be closed under $*$, since that is the defining property of $H$.
    \begin{align*}
        LHS & = (a*b)*(c*d) \\
        & = a*(b*c)*d && \text{(1) $*$ is associative} \\
        & = a*(c*b)*d && \text{(2) elements in $H$ are commutative} \\
        & = (a*c)*(b*d) && \text{(1)} \\
        & = (c*a)*(d*b) && \text{(2)} \\
        & = c*(a*d)*b && \text{(1)} \\
        & = c*(d*a)*b && \text{(2)} \\
        & = (c*d)*(a*b) && \text{1} \\
        & = RHS
    \end{align*}
    Thus we can conclude that $(a*b)*(c*d) = (c*d)*(a*b)$, and $H$ is closed under $*$.
\end{proof}

\end{document}