\documentclass{article}
\usepackage[margin=1in]{geometry}
\usepackage{amsmath, amsthm, amssymb, fancyhdr, tikz, circuitikz, graphicx}
\usepackage{centernot, xcolor, hhline, multirow, listings, dashrule}
\usepackage{blkarray, booktabs, bigstrut, etoolbox, extarrows}
\usepackage[normalem]{ulem}
\usepackage{bookmark}
\usepackage{enumerate}          %These two package give custom labels to a list
\usepackage[shortlabels]{enumitem}
\usetikzlibrary{math}
\usetikzlibrary{fit}

\pagestyle{fancy}

\usepackage{hyperref}
\hypersetup{
    colorlinks=true,
    linkcolor=black,
    filecolor=magenta,
    urlcolor=cyan,
}
%formatting
\newcommand{\bld}{\textbf}
\newcommand{\itl}{\textit}
\newcommand{\uln}{\underline}

%math word symbols
\newcommand{\bb}{\mathbb}
\DeclareMathOperator{\tif}{~\text{if}~}
\DeclareMathOperator{\tand}{~\text{and}~}
\DeclareMathOperator{\tbut}{~\text{but}~}
\DeclareMathOperator{\tor}{~\text{or}~}
\DeclareMathOperator{\tsuchthat}{~\text{such that}~}
\DeclareMathOperator{\tsince}{~\text{since}~}
\DeclareMathOperator{\twhen}{~\text{when}~}
\DeclareMathOperator{\twhere}{~\text{where}~}
\DeclareMathOperator{\tfor}{~\text{for}~}
\DeclareMathOperator{\tthen}{~\text{then}~}

%display shortcut
\DeclareMathOperator{\dstyle}{\displaystyle}
\DeclareMathOperator{\sstyle}{\scriptstyle}

%linear algebra
\DeclareMathOperator{\id}{\bld{id}}
\DeclareMathOperator{\vecspan}{\text{span}}

%augmented matrix environment
\newenvironment{amatrix}[1]{%
  \left(\begin{array}{@{}*{#1}{c}|c@{}}
    }{
  \end{array}\right)
}

%lists
\newcommand{\bitem}[1]{\item[\bld{#1.}]}
\newcommand{\bbitem}[2]{\item[\bld{#1.}] \bld{#2}}
\newcommand{\biitem}[2]{\item[\bld{#1.}] \itl{#2}}
\newcommand{\iitem}[1]{\item[\itl{#1.}]}
\newcommand{\iiitem}[2]{\item[\itl{#1.}] \bld{#2}}

%homework
\newenvironment*{question}[2]{
  \subsection*{#1} \itl{#2}
  \begin{enumerate}
    }{
  \end{enumerate}
}
\newcommand{\qitem}[2]{\item[\bld{#1}] \itl{#2}}

\newcommand{\assignment}{Section 3, p34 1-7odd, 17, 21, 25}

\lhead{MAT 311 Abstract Algebra}
\chead{\assignment}
\rhead{Peter Schaefer}

\begin{document}
\section*{\assignment}

Computations
\begin{enumerate}
    \qitem{1}{What three things must be check to determine whether a function $\phi: S \mapsto S'$ is an isomorphism of a binary structure $\langle S, * \rangle \twith \langle S', *' \rangle$?}{answer}
\end{enumerate}
In Exercises 2 through 10, determine whether the given map $\phi$ is an isomorphism of the first binary structure with the second. If it is not an isomorphism, why not?
\begin{enumerate}
    \qitem{3}{$\struct{\bb{Z},+} \twith \struct{\bb{Z},+} \twhere \phi(n) = 2n \tfor n \in \bb{Z}$}{answer}
    \qitem{5}{$\struct{\bb{Q},+} \twith \struct{\bb{Q},+} \twhere \phi(x) = x/2 \tfor x \in \bb{Q}$}{answer}
    \qitem{7}{$\struct{\bb{R},\cdot} \twith \struct{\bb{R},\cdot} \twhere \phi(x) = x^3 \tfor x \in \bb{R}$}{answer}
\end{enumerate}
\begin{enumerate}
    \qitem{17}{The map $\phi: \bb{Z} \mapsto \bb{Z}$ defined by $\phi(n) = n+1 \tfor n \in \bb{Z}$ is one to one and onto $\bb{Z}$.}{
        Give the definition of a binary operation $*$ on $\bb{Z}$ such that $\phi$ is an isomorphism mapping
        \begin{enumerate}
            \qitem{a}{$\struct{\bb{Z},\cdot}$ onto $\struct{\bb{Z},*}$}{answer}
            \qitem{b}{$\struct{\bb{Z},*}$ onto $\struct{\bb{Z},\cdot}$}{answer}
        \end{enumerate}
    }
\end{enumerate}
In Exercises 21 and 22, correct the definition of the italicized term without reference to the text, if correction is needed, so that it is in a form acceptable for publication.
\begin{enumerate}
    \qitem{21}{A function $\phi: S \mapsto S'$ is an \itl{isomorphism} if and only if $\phi(a*b) = \phi(a) *' \phi(b)$.}{answer}
\end{enumerate}
\begin{enumerate}
    \qitem{25}{Continuing the ideas of Exercise 24 can a binary structure have the left identity element $e_L$ and a right identity element $e_R$ where $e_L \neq e_R$? If so, given an example, using an operation on a finite set $S$. If not, prove that it is impossible.}{answer}
\end{enumerate}

\end{document}