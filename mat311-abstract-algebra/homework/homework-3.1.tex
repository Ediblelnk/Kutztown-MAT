\documentclass{article}
\usepackage[margin=1in]{geometry}
\usepackage{amsmath, amsthm, amssymb, fancyhdr, tikz, circuitikz, graphicx}
\usepackage{centernot, xcolor, hhline, multirow, listings, dashrule}
\usepackage{blkarray, booktabs, bigstrut, etoolbox, extarrows}
\usepackage[normalem]{ulem}
\usepackage{bookmark}
\usepackage{enumerate}          %These two package give custom labels to a list
\usepackage[shortlabels]{enumitem}
\usetikzlibrary{math}
\usetikzlibrary{fit}

\pagestyle{fancy}

\usepackage{hyperref}
\hypersetup{
    colorlinks=true,
    linkcolor=black,
    filecolor=magenta,
    urlcolor=cyan,
}
%formatting
\newcommand{\bld}{\textbf}
\newcommand{\itl}{\textit}
\newcommand{\uln}{\underline}

%math word symbols
\newcommand{\bb}{\mathbb}
\DeclareMathOperator{\tif}{~\text{if}~}
\DeclareMathOperator{\tand}{~\text{and}~}
\DeclareMathOperator{\tbut}{~\text{but}~}
\DeclareMathOperator{\tor}{~\text{or}~}
\DeclareMathOperator{\tsuchthat}{~\text{such that}~}
\DeclareMathOperator{\tsince}{~\text{since}~}
\DeclareMathOperator{\twhen}{~\text{when}~}
\DeclareMathOperator{\twhere}{~\text{where}~}
\DeclareMathOperator{\twith}{~\text{with}~}
\DeclareMathOperator{\tfor}{~\text{for}~}
\DeclareMathOperator{\tthen}{~\text{then}~}
\DeclareMathOperator{\tto}{~\text{to}~}
\DeclareMathOperator{\tin}{~\text{in}~}

%display shortcut
\DeclareMathOperator{\dstyle}{\displaystyle}
\DeclareMathOperator{\sstyle}{\scriptstyle}

%linear algebra
\DeclareMathOperator{\id}{\bld{id}}
\DeclareMathOperator{\vecspan}{\text{span}}
\DeclareMathOperator{\adj}{\text{adj}}

%discrete math - integer properties
\DeclareMathOperator{\tdiv}{\text{div}}
\DeclareMathOperator{\tmod}{\text{mod}}
\DeclareMathOperator{\lcm}{\text{lcm}}

%augmented matrix environment
\newenvironment{apmatrix}[2]{%
    \left(\begin{array}{@{~}*{#1}{c}|@{~}*{#2}{c}}
        }{
    \end{array}\right)
}
\newenvironment{abmatrix}[2]{%
    \left[\begin{array}{@{~}*{#1}{c}|@{~}*{#2}{c}}
            }{
        \end{array}\right]
}

\newenvironment{determinant}[1]{
    \left\lvert
    \begin{array}{@{~}*{#1}{c}}
        }{
    \end{array}
    \right\rvert
}

% graph theory
\DeclareMathOperator{\diam}{\text{diam}}
\newcommand{\comp}[1]{\overline{#1}}

% calculus iii
\newcommand{\vvec}{\overrightarrow}
\newcommand{\norm}[1]{\lVert #1 \rVert}

% abstract algebra
\newcommand{\struct}[1]{\langle #1 \rangle}

%lists
\newcommand{\bitem}[1]{\item[\bld{#1.}]}
\newcommand{\bbitem}[2]{\item[\bld{#1.}] \bld{#2}}
\newcommand{\biitem}[2]{\item[\bld{#1.}] \itl{#2}}
\newcommand{\iitem}[1]{\item[\itl{#1.}]}
\newcommand{\iiitem}[2]{\item[\itl{#1.}] \bld{#2}}
\newcommand{\btitem}[2]{\item[\bld{#1.}] \texttt{#2}}

%homework
\newcommand{\question}[2]{\noindent {\large\bld{#1}} #2 \qline}
\newcommand{\qitem}[3]{\item[\bld{#1.}] #2 \qdash \\ #3 \qdash}

\newcommand{\qline}{~\newline\noindent\textcolor[RGB]{200,200,200}{\rule[0.5ex]{\linewidth}{0.2pt}}}
\newcommand{\qdash}{~\newline\noindent\textcolor[RGB]{200,200,200}{\hdashrule[0.5ex]{\linewidth}{0.2pt}{2pt}}}

\newcommand{\assignment}{Section 3, p34 1-7 odd, 17, 21, 25}

\lhead{MAT 311 Abstract Algebra}
\chead{\assignment}
\rhead{Peter Schaefer}

\begin{document}
\section*{\assignment}

Computations
\begin{enumerate}
    \qitem{1}{What three things must be check to determine whether a function $\phi: S \mapsto S'$ is an isomorphism of a binary structure $\langle S, * \rangle \twith \langle S', *' \rangle$?}{$\phi$ must be \itl{one-to-one}, \itl{onto}, and \itl{operation preserving}.}
\end{enumerate}
In Exercises 2 through 10, determine whether the given map $\phi$ is an isomorphism of the first binary structure with the second. If it is not an isomorphism, why not?
\begin{enumerate}
    \qitem{3}{$\struct{\bb{Z},+} \twith \struct{\bb{Z},+} \twhere \phi(n) = 2n \tfor n \in \bb{Z}$}{
        An isomorphism must be one-to-one, onto, and operation preserving.
        \begin{proof}
            Onto: Let $y \in \bb{Z}$. Let us find $n \in \bb{Z}$ such that $y = \phi(x)$
            \begin{align*}
                y & = \phi(n) \\
                y & = 2n \\
                y/2 & = n
            \end{align*}
            $y/2$ is not always an integer, so we cannot say that $\phi$ is onto. Therefore, $\phi$ is not an isomorphism between $\struct{\bb{Z},+} \tand \struct{\bb{Z},+}$.
        \end{proof}
    }
    \qitem{5}{$\struct{\bb{Q},+} \twith \struct{\bb{Q},+} \twhere \phi(x) = x/2 \tfor x \in \bb{Q}$}{
        An isomorphism must be one-to-one, onto, and operation preserving.
        \begin{proof}
            \begin{enumerate}
                \item One-to-one: Assume $\phi(x_1) = \phi(x_2)$ for some $x_1,x_2 \in \bb{Q}$.
                \begin{align*}
                    \phi(x_1) & = \phi(x_2) \\
                    \frac{x_1}{2} & = \frac{x_2}{2} \\
                    x_1 & = x_2
                \end{align*}
                Thus $\phi$ is one-to-one.
                \item Onto: Let $y \in \bb{Q}$. Let us find $x \in \bb{Q}$ such that $y = \phi(x)$
                \begin{align*}
                    y & = \phi(x) \\
                    y & = \frac{x}{2} \\
                    2y & = x
                \end{align*}
                Choose $x = 2y$. Thus $\phi$ is onto.
                \item Operation Preserving: Need to show that $\phi(x+y) = \phi(x) + \phi(y)$.
                \begin{align*}
                    \phi(x+y) & = \frac{x+y}{2} \\
                    & = \frac{x}{2} + \frac{y}{2} \\
                    & = \phi(x) + \phi(y)
                \end{align*}
                Thus $\phi$ is operation preserving.
            \end{enumerate}
            Since $\phi$ is one-to-one, onto, and operation preserving, it is an isomorphism between $\struct{\bb{Q},+} \tand \struct{\bb{Q},+}$.
        \end{proof}
    }
    \qitem{7}{$\struct{\bb{R},\cdot} \twith \struct{\bb{R},\cdot} \twhere \phi(x) = x^3 \tfor x \in \bb{R}$}{
        An isomorphism must be one-to-one, onto, and operation preserving.
        \begin{proof}
            \begin{enumerate}
                \item One-to-one: Assume $\phi(x_1) = \phi(x_2)$ for some $x_1,x_2 \in \bb{R}$.
                \begin{align*}
                    \phi(x_1) & = \phi(x_2) \\
                    x_1^3 & = x_2^3 \\
                    x_1 & = x_2
                \end{align*}
                Thus $\phi$ is one-to-one.
                \item Onto: Let $y \in \bb{R}$. Let us find $x \in \bb{R}$ such that $y = \phi(x)$.
                \begin{align*}
                    y & = \phi(x) \\
                    y & = x^3 \\
                    \sqrt[3]{y} & = x
                \end{align*}
                Choose $x = \sqrt[3]{x}$. Thus $\phi$ is onto.
                \item Operation Preserving: Need to show that $\phi(x \cdot y) = \phi(x) \cdot \phi(y)$.
                \begin{align*}
                    \phi(x \cdot y) & = (xy)^3 \\
                    & = x^3 \cdot y^3 \\
                    & = \phi(x) \cdot \phi(y)
                \end{align*}
                Thus $\phi$ is operation preserving.
            \end{enumerate}
            Since $\phi$ is one-to-one, onto, and operation preserving, it is an isomorphism between $\struct{\bb{R},\cdot} \tand \struct{\bb{R},\cdot}$.
        \end{proof}
    }
\end{enumerate}
\begin{enumerate}
    \qitem{17}{The map $\phi: \bb{Z} \mapsto \bb{Z}$ defined by $\phi(n) = n+1 \tfor n \in \bb{Z}$ is one to one and onto $\bb{Z}$.}{
        Give the definition of a binary operation $*$ on $\bb{Z}$ such that $\phi$ is an isomorphism mapping
        \begin{enumerate}
            \qitem{a}{$\struct{\bb{Z},\cdot}$ onto $\struct{\bb{Z},*}$}{
                Since $\phi$ is already one-to-one and onto, we just need to define $*$ so that $\phi$ is operation preserving. Consider $a*b = ab-a-b+2$
                \begin{align*}
                    \phi(x) * \phi(y) & = (x+1) * (y+1) \\
                    & = (x+1)(y+1)-(x+1)-(y+1)+2 \\
                    & = xy+x+y+1-x-1-y-1+2 \\
                    & = xy+1 \\
                    & = \phi(x \cdot y)
                \end{align*}
                $\phi(x) * \phi(y) = \phi(x \cdot y)$, meaning $\phi$ is operation preserving. Along with being one-to-one and onto, $\phi$ is thus an isomorphism between $\struct{\bb{Z},\cdot} \tand \struct{\bb{Z},*}$.
            }
            \qitem{b}{$\struct{\bb{Z},*}$ onto $\struct{\bb{Z},\cdot}$}{
                Since $\phi$ is already one-to-one and onto, we just need to define $*$ so that $\phi$ is operation preserving. Consider $a*b = ab+a+b$
                \begin{align*}
                    \phi(x*y) & = \phi(xy+x+y) \\
                    & = xy+x+y+1 \\
                    & = (x+1)(y+1) \\
                    & = \phi(x) \cdot \phi(y)
                \end{align*}
                $\phi(x) * \phi(y) = \phi(x \cdot y)$, meaning $\phi$ is operation preserving. Along with being one-to-one and onto, $\phi$ is thus an isomorphism between $\struct{\bb{Z},\cdot} \tand \struct{\bb{Z},*}$.
            }
        \end{enumerate}
    }
\end{enumerate}
In Exercises 21 and 22, correct the definition of the italicized term without reference to the text, if correction is needed, so that it is in a form acceptable for publication.
\begin{enumerate}
    \qitem{21}{A function $\phi: S \mapsto S'$ is an \itl{isomorphism} if and only if $\phi(a*b) = \phi(a) *' \phi(b)$.}{
        A function $\phi: S \mapsto S'$ is an \itl{operation preserving} if and only if $\phi(a*b) = \phi(a) *' \phi(b)$.
    }
\end{enumerate}
\begin{enumerate}
    \qitem{25}{Continuing the ideas of Exercise 24 can a binary structure have the left identity element $e_L$ and a right identity element $e_R$ where $e_L \neq e_R$? If so, given an example, using an operation on a finite set $S$. If not, prove that it is impossible.}{
        We shall conduct a proof by contradiction
        \begin{proof}
            Let $e_L \tand e_R$ be left- and right-sided identities, respectively, such that $e_L \neq e_R$. Let us consider $e_L * e_R$.
            \begin{align*}
                e_L * e_R & = e_L && \text{since $e_R$ is an identity from the right} \\
                e_L * e_R & = e_R && \text{since $e_L$ is an identity from the left}
            \end{align*}
            Since a binary structure is uniquely defined, we must conclude that $e_L = e_R$. However, this contradicts our assertion that $e_L \neq e_L$. Therefore, there \itl{cannot} be a binary structure such that the left identity and right identity are distinct.
        \end{proof}
    }
\end{enumerate}

\end{document}