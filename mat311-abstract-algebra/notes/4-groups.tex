\section{Groups}

\subsubsection{Def. Group}
A \bld{group} $\group{G,*}$ is a set $G$ \itl{closed} under the binary operation $*$, such that the following axioms are satisfied:
\begin{enumerate}
    \item[$\mathfrak{G}_1$:] For all $a,c,b \in G$, we have
    \[
        (a*b)*c = a*(b*c). \qquad \bld{associativity of *}
    \]
    \item[$\mathfrak{G}_2$:] There is an element $e$ in $G$ such that for all $x \in G$,
    \[
        e*x=x*e=x. \qquad \bld{identity element $e$ for *}
    \]
    \item[$\mathfrak{G}_3$:] Corresponding to each $a \in G$, there is an element $a'$ in $G$ such that
    \[
        a*a'=a'*a=e. \qquad \bld{inverse $a'$ of $a$}
    \]
\end{enumerate}
Note: $G$ does not \itl{need} to be commutative.

\subsubsection{Def. Abelian Group}
A group $G$ is \bld{Abelian} if its binary operation is \bld{commutative}.

\subsubsection{Thm. Cancellation Laws}
If $\group{G, *}$ is a group, then the left and right cancellation laws hold in $G$.
\begin{itemize}
    \item \bld{Left}:
    \[
        \tif a*b = a*c \tthen b = c
    \]
    \item \bld{Right}:
    \[
        \tif b*a = c*a \tthen b = c
    \]
\end{itemize}
\begin{proof}[Proof for Left]
    Assume $\group{G,*}$ is a group and $a*b=a*c$:
    \begin{align*}
        a*b & = a*c \\
        \overline{a}*a*b & = \overline{a}*a*c && \mathfrak{G}_3 \\
        e*b & = e*c && \mathfrak{G}_3 \\
        b & = c && \mathfrak{G}_2
    \end{align*}
\end{proof}
The proof for right cancellation follows the same structure.

\subsubsection{Thm. Unique Solutions}
If $\group{G,*}$ is a group and if $a,b \in G$, then $a*x=b \tand y*a=b$ have unique solutions $x$ and $y$ in G.
\begin{proof}
    Assume $\group{G,*}$ is a group and consider $a*x=b$ for $a,b \in G$.
    \begin{align*}
        a*x & = b \\
        \overline{a} * (a*x) & = \overline{a} * b && \mathfrak{G}_3 \\
        (\overline{a} * a) *x & = \overline{a} * b && \mathfrak{G}_1 \\
        e * x & = \overline{a} * b && \mathfrak{G}_3 \\
        x & = \overline{a} * b && \mathfrak{G}_2
    \end{align*}
    Assume $x_1 \tand x_2$ are both solutions to the above equation.
    \[
        a*x_1 =b \tand a * x_2 = b
    \]
    Thus $a*x_1 = a*x_2$. By left cancellation,
    \[
        x_1 = x_2
    \]
    Thus the solution is unique.
\end{proof}
The $y*a=b$ proof follows the same structure.

\subsubsection{Thm. Unique Identity and Inverse}
If $\group{G,*}$ is a group, then th identity element and the inverse of each element are unique.

\subsubsection{Thm. Inverse of Two Elements}
Let $\group{G,*}$ be a group. Then for all $a,b \in G$, we have $(a*b)' = a' * b'$.
\begin{proof}
    \begin{align*}
        (a*b)*(a*b)' & = e && \text{by definition of $\mathfrak{G}_3$} \\
        a*b*(a*b)' & = e && \text{$\mathfrak{G}_1$, associativity} \\
        (a'*a)*b*(a*b)' & = a'*e && \mathfrak{G}_1 \\
        b*(a*b)' & = a'*e && \mathfrak{G}_3 \\
        b'*b(a*b)' & = b'*a'*e \\
        (a*b)' & = b'*a' && \mathfrak{G}_1,~\mathfrak{G}_3
    \end{align*}
\end{proof}

\subsection{Finite Groups and Group Tables}

\subsubsection*{Cayley Tables}
Let $\group{G,*}$ be a finite group.
\begin{enumerate}
    \item If $\norm{G} = 1$, then $G = \{e\}$, where $e$ is the identity.
    \[
        \begin{array}{c|c}
            * & e \\
            \hline
            e & e
        \end{array}
    \]
    This is known as the \bld{trivial group}.
    \item If $\norm{G} = 2$, then $G = \{e,a\}$.
    \[
        \begin{array}{c|cc}
            * & e & a \\
            \hline
            e & e & a \\
            a & a & e
        \end{array}
    \]
    Note: by $\mathfrak{G}_3$, $e$ must appear in every row and column of a group table, and exactly once.
    \item If $\norm{G} = 3$, then $G = \{e,a,b\}$
    \[
        \begin{array}{c|ccc}
            * & e & a & b \\
            \hline
            e & e & a & b \\
            a & a & b & e \\
            b & b & e & a
        \end{array}
    \]
    \bld{Claim}: No row or column of a Cayley Table may contain the same element twice.
    \begin{proof}
        Let $a,x,y \in G$ for $\group{G,*}$, where $x \neq y$. Consider the Cayley Table:
        \[
            \begin{array}{c|cccccc}
                * & e & a & \cdots & x & \cdots & y \\
                \hline
                e & e & a & \cdots & x & \cdots & y \\
                a & a & \_ & \cdots & a*x & \cdots & a*y
            \end{array}
        \]
        Suppose a row can have the same element twice, say $a*x = a*y$. By left cancellation $x=y$, a contradiction. Thus no row or column can have the same element twice.
    \end{proof}
    By the pigeon-hole principle, each element of a group must be represented in each row and column exactly once.
\end{enumerate}