\section{Cosets and the Theorem of Lagrange}

\subsubsection{Thm. Relation for Cosets}
Let $H \leq G$. Let the relation $\sim_L$ be defined on $G$ by $a \sim_L b$ if and only if $a^{-1}b \in H$ for all $a,b \in G$. Similarly, let the relation $\sim_R$ be defined on $G$ by $a \sim_R b$ if and only if $ab^{-1} \in H$ for all $a,b \in G$. Then $\sim_L \tand \sim_R$ are both equivalence relations on $G$.

\begin{proof}[Proof of $\sim_L$]
    Let $G$ be a group and $H \leq G$. Define $a\ sim_L b$ by $a^{-1}b \in H$.
    \begin{enumerate}
        \item Reflexive on $G$:
        \[
            a^{-1}a = e \in H
        \]
        Thus $\sim_L$ is reflexive.
        \item Symmetric on $G$: Assume $a \sim_L b$. Since $a^{-1}b \in H$,
        \begin{align*}
            (a^{-1}b)^{-1} & \in H \\
            b^{-1}(a^{-1})^{-1} & \in H \\
            b^{-1}a & \in H
        \end{align*}
        Thus $\sim_L$ is symmetric.
        \item Transitive on $G$ Assume $a \sim_L b$ and $b \sim_L c$. Since $a^{-1}b \in H$ and $b^{-1}c \in H$,
        \begin{align*}
            (a^{-1}b)(b^{-1}c) & \in H \\
            a^{-1}bb^{-1}c & \in H \\
            a^{-1}c \in H
        \end{align*}
        Thus $\sim_L$ is transitive.
    \end{enumerate}
    Therefore, $\sim_L$ is an equivalence relation.
\end{proof}
(The proof for $\sim-R$ is essentially the same.)

\subsubsection*{Note}
Recall that equivalence relations define a partition on a set. Let $a \in G$ be fixed. The partition cell containing $a$ consists of all arbitrary $x \in G$ such that $a \sim_L x$. This implies $a^{-1}x \in H$, so there exists $h \in H$ such that $a^{-1}x = h$. That is, there exists $h \in H$ such that $x = ah$. Therefore, the partition cell containing $a$ is $\{ah : h \in H\}$.

\subsubsection{Def. Coset}
Let $G$ be a group and $H \leq G$. For any element $a \in G$, the symbol $aH$ denotes the set of all products $ah$ as $a$ remains fixed and $h$ ranges over $H$. The set $aH$ is called the \bld{left coset} of $H$ in $G$. Similarly, $Ha = \{ha : h \in H\}$ is the \bld{right coset} of $H$ in $G$.

\subsubsection*{Notes}
Cosets of $G$ are subsets of $G$. If $G$ is Abelian, then the left and right cosets are the same. That is, $aH = Ha$ for all $a \in G$.

If $a \in Hb$, then $Ha = Hb$.
\begin{proof}
    Assume $a \in Hb$. We must show that $Ha \subseteq Hb$ and $Ha \supseteq Hb$.
    \begin{proof}[$Ha \subseteq Hb$]
        Let $x \in Ha$. We know $\exists~h_1 \in H$ such that $x = h_1a$.

        Since $a \in Hb$, we know $\exists~h_2 \in H$ such that $a = h_2b$.

        So $x = h_1a = h_1(h_2b) = (h_1h_2)b$. $h_1h_2 \in H$, so $x \in Hb$.
    \end{proof}
    \begin{proof}[$Ha \supseteq Hb$]
        Let $y \in Hb$. We know $\exists~h_3 \in H$ such that $y = h_3b$.

        Since $a \in Hb$, we know $\exists~h_2 \in H$ such that $a = h_2b \implies b = h_2^{-1}a$.

        So $y = h_3b = h_3(h_2^{-1}a) = (h_3h_2^{-1})a$. $h_3h_2^{-1} \in H$, so $y = Ha$.
    \end{proof}
    Thus $Ha \subseteq Hb$ and $Ha \supseteq Hb$ and therefore $Ha = Hb$.
\end{proof}

\subsubsection*{Note}
A consequence of above is that a given coset can be written in more than one way. if a coset of $H$ has $n$ elements, say $a_1,a_2,\ldots,a_n$, then it can be written $n$ different ways: $Ha_1,Ha_2,\ldots,Ha_n$.

\subsubsection*{Example}
Consider $D_4$, the symmetries of a square. Let $H = \{\rho_0,\mu_2\}$. List the right cosets of $H$ in $D_4$ and the elements of each coset. See table 8.12 (not shown).
\begin{align*}
    H\rho_0 & = \{\rho_0, \mu_2\} = H\mu_2 \\
    H\rho_1 & = \{\rho_1, \delta_1\} = H\delta_1 \\
    H\rho_2 & = \{\rho_2, \mu_1\} = H\mu_1 \\
    H\rho_3 & = \{\rho_3, \delta_2\} = H\delta_2
\end{align*}

\subsubsection{Thm. One-to-one Correspondence of Cosets}
If $Ha$ is any coset of $H$ in $G$, then there is a one-to-one correspondence from $H$ to $Ha$.
\begin{proof}
    Define $f: H \rightarrow Ha$ by $f(h) = ha$.
    \begin{enumerate}
        \item One-to-one: Let $f(h_1) = f(h_2)$.
        \begin{align*}
            h_1a & = h_2a \\
            h_1 & = h_2
        \end{align*}
        Thus $f$ is one-to-one.
        \item Onto: Let $ha \in Ha$. Choose $h$, and $f(h) = ha$. So $f$ is onto.
    \end{enumerate}
    Thus $f$ is a one-to-one correspondence from $H$ to $Ha$.
\end{proof}
\bld{Consequence}: Any coset $Ha$ of $H$ has the same number of elements as $H$ and thus all cosets of $H$ in $G$ have the same cardinality.

\subsubsection{Thm. Lagrange's Theorem}
Let $G$ be a finite group and let $H$ be a subgroup of $G$. The order of $G$ is a multiple of the order of $H$. Or, the order of $H$ is a divisor of the order of $G$.
\[
    |G| = |H| \cdot |G:H|
\]

\subsubsection{Def. Index of H in G}
The \bld{index of $H$ in $G$}, denoted as $(G:H) \tor |G:H|$, is the number of cosets of $H$ in $G$.

\subsubsection{Cor. Groups of Prime Order}
If $G$ is a group with a prime number $p$ elements, then $G$ is a cyclic group. Furthermore, any element $a \neq \id$ in $G$ is a generator in $G$.
\begin{proof}
    Let $G$ be a group having $p$ elements, where $p$ is prime. If $a \in G$ but $a \neq \id$, then the order of $a$ is some integer $m \neq 1$.

    Then $\group{a} = \{a,a^2,\ldots, a^m = \id\}$ has $m$ elements, $|\group{a}| = m$.

    By Lagrange's Theorem, the order of $G$ is a multiple of $|\group{a}|$. But $|G|$ is prime $p$, and $m \neq 1$, so $p = m$. So $|\group{a}| = |G|$, and thus $\group{a} = G$.
\end{proof}