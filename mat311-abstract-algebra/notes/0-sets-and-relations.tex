\section{Sets and Relations}

\subsubsection*{Definition: What \itl{is} Abstract Algebra}
\begin{itemize}
    \item Algebra: procedures for performing operations, i.e. $+,-,\times,\div$, and methods for solving equations. It uses bld{specific} operations on \bld{specific} objects.
    \item Abstract Algebra: discuss \bld{general} structures and the relationships between the elements of these structures.
\end{itemize}

\subsection{Sets}

\subsubsection*{Definition: Set}
A set is a collection of objects. These objects are called "elements". A set is typically uppercase, and elements are typically lowercase.

\subsubsection*{Set Notation}
\begin{enumerate}
    \item List Notation:
    \begin{align*}
        B & = \{\text{John}, \text{Paul}, \text{Ringo}, \text{George}\} \\
        \bb{N} & = \{1,2,3,\ldots\}
    \end{align*}
    \item Set-builder Notation:
    \[
        B = \{b : b ~\text{is a Beatle}\}
    \]
\end{enumerate}

\subsubsection*{Well-Defined Sets}
Sets must be \bld{well-defined}. That is, given set $S$ and any element $x$, either $x \in S \tor x \notin S$.

\subsubsection*{Definition: Subset}
A set $A$ is a subset of set $B$, written as $A \subseteq B$, if every element of $A$ is also in $B$.

Note: every non-empty set has at least two subsets:
\begin{itemize}
    \item The set itself
    \item $\emptyset$
\end{itemize}

\subsubsection*{Definition: Proper Subset}
If $A \subseteq B \tbut A \not = B$, then $A$ is a \bld{proper subset} of $B$, written $A \subset B \tor A \subsetneq B$.

Note: A set $B$ is an \itl{improper subset} of itself.

\subsubsection*{Definition: Cartesian Product}
Let $A \tand B$ be sets. The set $A \times B = \{(a,b) : a \in A \tand b \in B\}$ is the cartesian product of $A \tand B$.

Note: $A \times B = B \times A \iff A = B, \tor A \times B = \emptyset$.

\subsubsection*{Example}
Let $A = \{c : c ~\text{is a primary color}\}$ and let $B = \{\epsilon, \delta\}$. Find:
\begin{enumerate}
    \item $B \times B = \{(\epsilon, \epsilon),(\epsilon,\delta),(\delta,\epsilon),(\delta,\delta)\}$
    \item $A \times \emptyset = \emptyset$
\end{enumerate}

\subsection{Relations}

\subsubsection*{Definition: Relation}
A \bld{relation} between sets $A \tand B$ is a subset $\mathcal{R}$ of $A \times B$. It is a collection of ordered pairs. Note: $(a,b) \in \mathcal{R} \equiv a\mathcal{R}b$ means "$a$ is related to $b$".

\subsubsection*{Definition: Function}
A \bld{function} is a relation in which no two of the ordered pairs have the same first term. Note: if $f:\bb{R} \rightarrow \bb{R}$ is a function, then is passes the vertical-line test.

\subsubsection*{Definition: One-to-One}
A function is \bld{one-to-one}, or \bld{injective}, if no two ordered pairs have the same \uln{second} term.

To prove $f$ is one-to-one, first assume that $f(x_1) = f(x_2)$, then show that $x_1 = x_2$.

\subsubsection*{Definition: Onto}
A function $f: X \rightarrow Y$ is \bld{onto}, or \bld{surjective}, if the codomain is equal to the range, meaning every element $y \in Y$ has some $x \in X$ such that $f(x) = y$.

\subsubsection*{Definition: One-to-One Correspondence}
A function $f: X \rightarrow Y$ is a \bld{one-to-one correspondence}, or a \bld{bijection}, if it is both one-to-one and onto.

\subsection{Partitions and Equivalence Relations}

\subsubsection*{Definition: Partition}
A \bld{partition} of a set $S$ is a collection of non-empty subsets of $S$ such that:
\begin{enumerate}
    \item The union of these subsets is $S$.
    \item These subsets are pairwise disjoint.
\end{enumerate}
Note: these subsets are called \bld{cells} of the partition.

\subsubsection*{Definition: Equivalence Relation}
An \bld{equivalence relation} $\mathcal{R}$ on a set $S$ must be:
\begin{enumerate}
    \item Reflexive, meaning $x\mathcal{R}x \qquad \forall~x \in S$.
    \item Symmetric, meaning $\tif x\mathcal{R}y, \tthen y\mathcal{R}x$.
    \item Transitive, meaning $\tif x\mathcal{R}y \tand y\mathcal{R}z, \tthen x\mathcal{R}z$.
\end{enumerate}

\subsubsection*{Definition: Equivalence Class}
\[
    \overline{x} = \{y \in S : x\mathcal{R}y\} ~\text{is the equivalence class of $x$}
\]

\subsubsection*{Example}
Let $S = \bb{R}$. Define $x\mathcal{R}y$ iff $x \geq y$. Is $\mathcal{R}$ an equivalence relation on $S$?
\begin{enumerate}
    \item Is $\mathcal{R}$ reflexive? $\forall x \in S, x\mathcal{R}x$, so YES.
    \item Is $\mathcal{R}$ symmetric? Consider 5 and 1: $5 \geq 1 \tbut 1 \ngeq 5$, so NO.
    \item Is $\mathcal{R}$ transitive? If $x \geq y \tand y \geq z \tthen x \geq z$, so YES.
\end{enumerate}
Since $\mathcal{R}$ is not symmetric, it is not an equivalence relation on $S$.

\subsubsection*{Note on Partition Cells and Equivalence Classes}
Partitions give rise to equivalence relations and vice versa. The \itl{cells} of the partition are analogous to the \itl{equivalence classes} of the equivalence relation.