\section{Orbits, Cycles, and the Alternating Groups}

Consider the set $A = \{1,2,3,\ldots,8\}$ and let $\sigma \in S_8$ be defined by $\sigma = \begin{pmatrix}
    1 & 2 & 3 & 4 & 5 & 6 & 7 & 8 \\
    3 & 5 & 6 & 4 & 7 & 1 & 2 & 8
\end{pmatrix}$. How does $\sigma$ "move" elements in $A$?
\begin{align*}
    & 1 \mapsto 3 \mapsto 6 \mapsto 1 \\
    & 2 \mapsto 5 \mapsto 7 \mapsto 2 \\
    & 8 \mapsto 8
\end{align*}

\subsubsection{Def. Orbits}
The \bld{orbits} of a permutation $\sigma$ are the equivalence class of $A$ determined by $a\sim b$ if and only if $b = \sigma^n(a)$ for some $n \in \bb{Z}$.

\subsubsection{Def. Cycle}
A permutation is a cycle if it has \itl{at most} \bld{one} orbit containing more than one element.

\subsubsection*{Example}
Writing $\begin{pmatrix}
    1 & 2 & 3 & 4 & 5 \\
    3 & 2 & 5 & 1 & 4
\end{pmatrix}$ as a cycle.
\begin{align*}
    (1,3,5,4)
\end{align*}
Note: elements that are not moved by the permutation do \bld{not} appear in the cycle.

\subsubsection*{Example}
In $S_8$, perform $(1,3,6)(2,8)(4,7,5)$ and express the answer as a permutation.
\[
    \begin{pmatrix}
        1 & 2 & 3 & 4 & 5 & 6 & 7 & 8 \\
        3 & 8 & 6 & 7 & 4 & 1 & 5 & 2
    \end{pmatrix}
\]
In $S_6$, write $(1,4,5,6)(2,1,5)$ as a permutation. Does $(2,1,5)(1,4,5,6)$ result in the same permutation? No, they do not.
\[
    \begin{pmatrix}
        1 & 2 & 3 & 4 & 5 & 6 \\
        6 & 4 & 3 & 5 & 2 & 1
    \end{pmatrix} \neq
    \begin{pmatrix}
        1 & 2 & 3 & 4 & 5 & 6 \\
        4 & 1 & 3 & 2 & 6 & 5
    \end{pmatrix}
\]

\subsubsection*{Notes}
Disjoint cycles commute. Every permutation $\sigma$ of a finite set can be expressed asa product of disjoint cycles.

\subsubsection{Def. Transposition}
A cycle of length two (2) is called a \bld{transposition}.

\subsubsection*{Note}
Every cycle can be expressed as a product of one or more transpositions, although it is \itl{not} unique.

In $S_5$,
\begin{align*}
    (1,2,3,4,5) & = (1,5)(1,4)(1,3)(1,2) \\
    & = (5,4)(5,3)(5,2)(5,1) \\
    & = (5,4)(5,2)(5,1)(1,4)(3,2)(4,1)
\end{align*}

\subsubsection{Def. Even and Odd Permutations}
A permutation is \bld{even} if it can be expressed as a product of an even number of transpositions. A permutation is \bld{odd} if it can be expressed as a product of an odd number of transpositions.

\subsubsection*{Note}
If $i$ is the identity permutation, then $i$ is even.

\subsubsection{Thm. Permutations are either Even or Odd}
If $\sigma \in S_n$, then $\sigma$ cannot be both even and odd.
\begin{proof}
    Let $\sigma \in S_n$ and assume $\sigma$ can be both even and odd. Note that $\sigma^{-1}$ is also both even and odd. But, $i = \sigma\sigma^{-1}$ is even, while $\sigma$ is odd and $\sigma^{-1}$ is even, or $\sigma$ is even and $\sigma^{-1}$ is odd. This would imply that $i$ could be odd, which is a contradiction.
\end{proof}

\subsubsection*{Recall}
$S_n$ is the group of all permutations on $\{1,2,3,\ldots,n\}$. Each of these permutations can be expressed as a product of \itl{transpositions}. Even though this breakdown is not unique, the above theorem shows that every breakdown of a particular permutation must either be even or odd. All of the even permutations are given a special designation.

\subsubsection{Def. The Alternating Group}
The set of all even permutations in $S_n$ is called the \bld{alternating group} on $\{1,2,\ldots,n\}$, denoted as $A_n$.

\subsubsection*{Notes}
The alternating group $A_n$ is a subgroup of $S_n$. Additionally, recall that $|S_n| = n!$. Thus $|A_n| = \frac{n!}{2}$.