\section{Groups of Permutations}

\bld{IDEA}: A \itl{permutation} of a set can be thought of as a rearrangement of the elements of the set.

\subsubsection{Def. Permutation}
A permutation of a set $A$ is a function $\phi: A \rightarrow A$ that is both one-to-one and onto. This means $\phi$ is a bijection from $A$ to itself.

Note: We will use "tabular notation" for $\phi$.

\subsubsection*{Example}
Let $A = \{1,2,3,4,5,6\}$ and consider two permutations of $A$: 

$f = \begin{pmatrix}
    1 & 2 & 3 & 4 & 5 & 6 \\
    6 & 1 & 3 & 5 & 4 & 2
\end{pmatrix}$ and $g = \begin{pmatrix}
    1 & 2 & 3 & 4 & 5 & 6 \\
    2 & 3 & 1 & 6 & 5 & 4
\end{pmatrix}$. Note that the operation of \itl{permutation multiplication} is function composition. That is, $fg = f \circ g$.
\begin{enumerate}
    \item $fg = \begin{pmatrix}
        1 & 2 & 3 & 4 & 5 & 6 \\
        1 & 3 & 6 & 2 & 4 & 5
    \end{pmatrix}$
    \item $g^2f = \begin{pmatrix}
        1 & 2 & 3 & 4 & 5 & 6 \\
        6 & 3 & 2 & 5 & 4 & 1
    \end{pmatrix}$
    \item $f^{-1}g = \begin{pmatrix}
        1 & 2 & 3 & 4 & 5 & 6 \\
        6 & 3 & 2 & 1 & 4 & 5
    \end{pmatrix}$
\end{enumerate}

\subsubsection{Thm. Permutations Multiplication and Groups}
Let $A$ be a nonempty set and let $S_A$ be the collection of all permutations of $A$. Then $S_A$ is a group under permutation multiplication.
\begin{proof}
    Note Permutation Multiplication is a binary operation on $S_A$.
    \begin{itemize}
        \item[$\mathfrak{G}_1$] Let $f,g,h \in S_A$. Let $a \in A$
        \begin{align*}
            [f(gh)](a) & = [f \circ (g \circ h)](a) \\
            & = f((g\circ h)(a)) \\
            & = f(g(h(a))) = (f\circ g)h(a) = [(fg)h](a)
        \end{align*}
        $\therefore \group{S_A, +}$ is associative.
        \item[$\mathfrak{G}_2$] Let $i(a) = a$ for all $a \in A$. Then $i$ is the identity permutation.
        \item[$\mathfrak{G}_3$] Every permutation in $S_A$ is bijective, so every permutation has an inverse.
    \end{itemize}
    $\therefore S_A$ is a group.
\end{proof}

\subsubsection{Def. Symmetric Group}
Let $A$ be the finite set $A =\{1,2,3,\ldots,n\}$. The group of all permutations of $A$ is called the \bld{symmetric group}, denoted $S_n$.

Note: $|S_n| = n!$

\subsubsection*{Example}
Consider $S_3$, which would be the group of all permutations of the set $A = \{1,2,3\}$. This set is also known as $D_3$, the group of symmetries of an equilateral triangle, where a symmetry is a movement of a shape to make it coincide with its former position. The letter $D$ is used because this type of group is called a \itl{dihedral group}, which are the groups of symmetries of regular polygons that include rotations and reflections.

Labeling the vertices of the triangle $1,2, \tand 3$, we get the following, where $\rho$ are rotations and $\mu$ are reflections.
\begin{align*}
    \rho_0 & = \begin{pmatrix}
        1 & 2 & 3 \\
        1 & 2 & 3
    \end{pmatrix} & \mu_1 & = \begin{pmatrix}
        1 & 2 & 3 \\
        1 & 3 & 2
    \end{pmatrix} \\
    \rho_1 & = \begin{pmatrix}
        1 & 2 & 3 \\
        2 & 3 & 1
    \end{pmatrix} & \mu_2 & = \begin{pmatrix}
        1 & 2 & 3 \\
        3 & 2 & 1
    \end{pmatrix} \\
    \rho_2 & = \begin{pmatrix}
        1 & 2 & 3 \\
        3 & 1 & 2
    \end{pmatrix} & \mu_3 & = \begin{pmatrix}
        1 & 2 & 3 \\
        2 & 1 & 3
    \end{pmatrix}
\end{align*}
However, when we consider $D_4$, the dihedral group consisting of symmetries of a square, we notice that $S_4 \neq D_4$.

\subsubsection{Thm. Cayley's Theorem}
Every group is isomorphic to a group of permutations.

\begin{proof}
    Let $G$ be a group, and let $a \in G$ be fixed. Define $\pi_a: G \rightarrow G$ by
    \[
        \pi_a(x) = ax, \qquad \forall~x \in G
    \]
    First, we prove that $\pi_a$ is a permutation of $G$.
    \begin{proof}
        A permutation is one-to-one and onto.
        \begin{enumerate}
            \item One-to-one: Assume $\pi_a(x_1) = \pi_a(x_1)$ for $x_1,x_2 \in G$.
            \begin{align*}
                \pi_a(x_1) & = \pi_a(x_1) \\
                ax_1 & = ax_2 \\
                x_1 & = x_2 && \text{by left cancellation}
            \end{align*}
            Thus $\pi_a$ is one-to-one.
            \item Onto: Let $y \in G$. Show $\exists~x \in G$ such that $y = \pi_a(x)$.
            \begin{align*}
                y & = \pi_a(x) = ax \\
                a^{-1}y = x
            \end{align*}
            Choose $x = a^{-1}y$. Thus $\pi_a$ is onto.
        \end{enumerate}
        Thus $\pi_a$ is a permutation of $G$.
    \end{proof}
    Let $G^* = \{\pi_a : a \in G\}$. We must show that $G^*$ is a group (consisting of permutations). It suffices to show that $G^*$ is a subgroup of $S_G$, the group of all permutations of $G$. Note: $G^* \subseteq S_G$.
    \begin{proof}
        A subgroup is closed under the operation and inverses.
        \begin{enumerate}
            \item Closed under operation of $S_G$: Consider $\pi_a, \pi_b \in G^*$ for $a,b \in G$. For $x \in G$,
            \[
                (\pi_a \circ \pi_b)(x) = \pi_a(bx) = abx = \pi_{ab}(x)
            \]
            Since $ab \in G$, we know that $\pi_{ab} \in G^*$, so $G^*$ is closed under the operation.
            \item Closed under inverses: Let $\pi_a \in G^*$. Since $\pi_a$ is a bijection, we know $\pi_a$ has an inverse $(\pi_a)^{-1}$. Note: $\pi_e$ is the identity of $S_G$. Consider $(\pi_a)^{-1} = \pi_{a^{-1}}$. For $x \in G$,
            \begin{align*}
                (\pi_{a^{-1}} \circ \pi_a)(x) = a^{-1}ax = ex & = \pi_e(x) \\
                (\pi_a \circ \pi_{a^{-1}})(x) = aa^{-1}x = ex & = \pi_e(x) \\
            \end{align*}
            Thus $(\pi_a)^{-1} = \pi_{a^{-1}} \in G^*$, and $G^*$ is closed under inverses.
        \end{enumerate}
        Thus $G^* \leq S_G$.
    \end{proof}
    It remains to be proven that $G \simeq G^*$. Consider $\phi: G \rightarrow G6*$, by
    \[
        \pi(a) = \pi_a.
    \]
    \begin{proof}
        An isomorphism is onto-to-one, onto, and operation preserving.
        \begin{enumerate}
            \item One-to-one: Let $\phi(a) = \phi(b)$ for $a,b \in G$.
            \begin{align*}
                \phi(a) & = \phi(b) \\
                \pi_a & = \pi_b
            \end{align*}
            Using $x \in G$,
            \begin{align*}
                \pi_a(x) & = \pi_b(x) \\
                ax & = bx \\
                a & = b && \text{by right cancellation}
            \end{align*}
            Thus $\phi$ is one-to-one.
            \item Onto: Given any $\pi_a \in G^*$, $\exists~a \in G$, such that $\phi(a) = \pi_a$. Thus $\phi$ is onto.
            \item Operation Preserving: Show $\phi(ab) = \phi(a) \circ \phi(b)$, $\forall~a,b \in G$.
            \begin{align*}
                \phi(ab) & = \pi_{ab} \\
                & = \pi_a \circ \pi_b \\
                & = \phi(a) \circ phi(b)
            \end{align*}
            Thus $\phi$ is operation preserving.
        \end{enumerate}
        Thus $\phi$ is an isomorphism, and $G \simeq G^*$.
    \end{proof}
    Thus group $G$ is isomorphic to a group of permutations $G^*$.
\end{proof}