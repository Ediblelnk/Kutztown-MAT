\section{Subgroups}

\subsection{Notation}

\begin{enumerate}
    \item Usually we will not use $*$ to denote a binary operation and instead will use \itl{juxtaposition}. That is, we write $ab$ instead of $a*b$. If the binary operation is commutative, $a+b$ is often used.
    \item $0$ is often used to represent the identity for the operation $+$ and $1$ to represent the identity for $\cdot$. We will also continue to use $e$, and personally I will often use $\id$.
    \item Instead of $a'$ to represent $a$'s inverse, we will use the more common $a^{-1}$ when the operation is $\cdot$ and $-a$ when the operation is $+$.
    \item Exponentiation:
    \begin{align*}
        a^n & = aaa\cdots a \qquad (n~\text{copies}) \\
        a^{-n} & = a^{-1}a^{-1}\cdots a^{-1} \qquad (n~\text{copies}) \\
        a^0 & = e
    \end{align*}
\end{enumerate}

\subsubsection{Def. Order}
If $G$ is a group, then the \bld{order} of $G$, denoted as $|G|$, is the number of elements in $G$.

\subsubsection{Def. Subgroup}
Let $H$ be a subset of a group $G$. $H$ is a \bld{subgroup} of $G$ if $H$ itself is a group under the operation of $G$. Notation: $H \leq G$.

\subsubsection{Def. Improper and Proper Subgroups}
$G$ is an \bld{improper} subgroup of itself. All other subgroups of $G$ are \bld{proper} subgroups, denoted as $H < G$.

Fact: All groups have a trivial subgroup $\{e\}$.

\subsubsection{Thm. Proving that a Subset of a Group is a Subgroup}
Let $H$ be a subset of a group $G$. If:
\begin{enumerate}
    \item $H$ is closed with respect to the operation of $G$ and,
    \item $H$ is closed with respect to inverses,
\end{enumerate}
then $H$ is a subgroup of $G$.
\begin{proof}
    Let $H \subseteq G$ and assume (1) and (2).
    \begin{enumerate}
        \item By (1), $H$ is closed under the operation of $G$.
        \item Associativity: Let $a,b,c \in H$. Note that $a,b,c \in G$, since $H \subseteq G$. Since $G$ is a group, $a(bc) = (ab)c$. Thus associativity is "\itl{inherited}" from $G$.
        \item Identity: Let $a \in H$. By (2), $a^{-1} \in H$. By (1), $aa^{-1} = e \in H$.
        \item Inverse: Let $a \in H$. By (2), $a^{-1} \in H$.
    \end{enumerate}
    Thus $H$ is a group, and thus also a subgroup of $G$.
\end{proof}

\subsubsection*{Example}
Prove that $\group{E,+} \leq \group{\bb{Z},+}$.
\begin{proof}
    Check: Is $E \subseteq \bb{Z}$? $\checkmark$
    \begin{enumerate}
        \item Is $E$ closed w.r.t. $+$?
        Let $a,b \in E$. By definition, $\exists~k,j \in \bb{Z} \tsuchthat a = 2k \tand b = 2j$. So, $a+b = 2k+2j = 2(k+j) \in E$. Thus, $E$ is closed  w.r.t. $E$.
        \item is $E$ closed w.r.t. inverses?
        Let $a \in E$. By definition, $\exists~k \in \bb{Z} \tsuchthat a = 2k$. Multiplying both sides by $-1$ gives $-a = -2k = 2(-k) \in E$.
    \end{enumerate}
    $\therefore E \leq \bb{Z}$ under $+$.
\end{proof}

\subsubsection{Thm. Cyclic Subgroups}
Let $G$ be a group and let $a \in G$. Then $H = \{a^n : n \in \bb{Z}\}$ is a subgroup of $G$. This subgroup $H$ is called the \bld{cyclic subgroup} of $G$ generated by $a$ and is denoted $\struct{a}$.

\subsubsection{Def. Cyclic Group and Generator of a Cylic Group}
Let $G$ be a group and let $a \in G$. Then $G$ is \bld{cyclic} if
\[
    G = \{a^n : n \in \bb{Z}\} = \struct{a}.
\]
'$a$' is called the \bld{generator} of the cyclic group.