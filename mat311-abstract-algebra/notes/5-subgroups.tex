\section{Subgroups}

\subsection{Notation}

\begin{enumerate}
    \item Usually we will not use $*$ to denote a binary operation and instead will use \itl{juxtaposition}. That is, we write $ab$ instead of $a*b$. If the binary operation is commutative, $a+b$ is often used.
    \item $0$ is often used to represent the identity for the operation $+$ and $1$ to represent the identity for $\cdot$. We will also continue to use $e$, and personally I will often use $\id$.
    \item Instead of $a'$ to represent $a$'s inverse, we will use the more common $a^{-1}$ when the operation is $\cdot$ and $-a$ when the operation is $+$.
    \item Exponentiation:
    \begin{align*}
        a^n & = aaa\cdots a \qquad (n~\text{copies}) \\
        a^{-n} & = a^{-1}a^{-1}\cdots a^{-1} \qquad (n~\text{copies}) \\
        a^0 & = e
    \end{align*}
\end{enumerate}

\subsubsection{Def. Order}
If $G$ is a group, then the \bld{order} of $G$, denoted as $|G|$, is the number of elements in $G$.

\subsubsection{Def. Subgroup}
Let $H$ be a subset of a group $G$. $H$ is a \bld{subgroup} of $G$ if $H$ itself is a group under the operation of $G$. Notation: $H \leq G$.

\subsubsection{Def. Improper and Proper Subgroups}
$G$ is an \bld{improper} subgroup of itself. All other subgroups of $G$ are \bld{proper} subgroups, denoted as $H < G$.

Fact: All groups have a trivial subgroup $\{e\}$.