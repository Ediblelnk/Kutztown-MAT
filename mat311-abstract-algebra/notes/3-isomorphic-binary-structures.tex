\section{Isomorphic Binary Structures}

\subsubsection{Def. Binary Algebraic Structure}
A \bld{binary algebraic structure} $\langle S,*\rangle$ is a set $S$ together with a binary operation $*$.

\subsubsection{Def. Isomorphism}
Let $\langle S,*\rangle \tand \langle S',*' \rangle$ be binary structures. An \bld{isomorphism} of $S$ with $S'$ is a \itl{one-to-one} function $\phi: S \mapsto S'$ such that
\[
    \forall~~ x,y \in S, \qquad \phi(x*y) = \phi(x) *' \phi(y).
\]
Notation: $\langle S,*\rangle \simeq \langle S',*'\rangle$

\subsubsection*{Example 1}
Prove that $\langle \bb{R},+\rangle \simeq \langle \bb{R}^+, \cdot \rangle$.

\begin{proof}
    Consider $\phi: \bb{R} \mapsto \bb{R}^+, \twhere \phi(x) = e^x$.
    \begin{enumerate}
        \item One-to-one: Assume $\phi(x_1) = \phi(x_2)$ for some $x_1,x_2 \in \bb{R}$.
              \begin{align*}
                  \phi(x_1)   & = \phi(x_2)   \\
                  e^{x_1}     & = e^{x_2}     \\
                  \ln e^{x_1} & = \ln e^{x_2} \\
                  x_1         & = x_2
              \end{align*}
              Thus $\phi$ is one-to-one.
        \item Onto: Let $y \in \bb{R}^+$. Let us find $x \in \bb{R}$ such that $y = \phi(x)$.
              \begin{align*}
                  y     & = \phi(x) = e^x \\
                  \ln y & = \ln e^x = x
              \end{align*}
              Choose $x = \ln y$. Thus $\phi$ is onto.
        \item Operation Preserving: Need to show that $\phi(x + y) = \phi(x) \cdot \phi(y)$.
              \begin{align*}
                  \phi(x+y) & = e^{x+y}               \\
                            & = e^x \cdot e^y         \\
                            & = \phi(x) \cdot \phi(y)
              \end{align*}
              Thus $\phi$ is operation preserving.
    \end{enumerate}
    Since $\phi$ is one-to-one, onto, and operation preserving, thus $\phi$ is an isomorphism of $\langle \bb{R},+\rangle \tand \langle \bb{R}^+, \cdot \rangle$, and $\langle \bb{R},+\rangle \simeq \langle \bb{R}^+, \cdot \rangle$.
\end{proof}

\subsubsection{Def. Identity Element}
Let $\langle S, * \rangle$ be an algebraic structure. An element $e \in S$ is the identity element $\id$ for $*$ if for all $s \in S$:
\[
    \underbrace{\overbrace{e * s}^{\text{left}~\id} = \overbrace{s * e}^{\text{right}~\id}}_{\text{two-sided}~\id} = s
\]

\subsubsection{Thm. Identity Uniqueness}
A binary structure $\langle S, * \rangle$ has at most one identity element.

\begin{proof}
    Assume $e_1 \tand e_2$ are both identity elements for $\langle S, * \rangle$. Thus,
    \begin{align*}
        e_1 * e_2 & = e_1 &  & \text{since $e_1$ is $\id$} \\
        e_1 * e_2 & = e_2 &  & \text{since $e_2$ is $\id$}
    \end{align*}
    Since binary operations are uniquely defined, $e_1 = e_2$ must be true.
    $\therefore$ $\langle S, * \rangle$ has at most one identity element.
\end{proof}

\subsubsection{Thm. Isomorphism and Identity}
Suppose $\langle S, * \rangle$ has identity element $e$. If $\phi: S \mapsto S'$ is an isomorphism of $\langle S, * \rangle$ with $\langle S', *' \rangle$, then $\phi(e)$ is the identity element for $\langle S', *' \rangle$.

\begin{proof}
    Assume $\langle S, * \rangle$ has identity $e$ and $\phi: S \mapsto S'$ is an isomorphism. Let $s' \in S'$.
    \begin{align*}
        \phi(e) *' s' & = \phi(e) *' \phi(s)                                                  \\
                      & = \phi(e * s)        &  & \text{since $\phi$ is operation preserving} \\
                      & = \phi(s) = s'
    \end{align*}
    Thus $\phi(e) *' s' = s'$.
    \begin{align*}
        s' *' \phi(e) & = \phi(s) *' \phi(e)                                                  \\
                      & = \phi(s * e)        &  & \text{since $\phi$ is operation preserving} \\
                      & = \phi(s) = s'
    \end{align*}
    Thus $s' *' \phi(e) = s'$. So $\phi(e) *' s' = s' *' \phi(e) = s'$. Thus $\phi(e)$ is the identity of $\langle S', *' \rangle$.
\end{proof}

\subsubsection*{Showing Two Binary Structure are \itl{not} Isomorphic}
To show that two binary structures are \itl{not} isomorphic, you need to show that one binary structure has some property that other does not, meaning they are structurally distinct.

\subsubsection*{Example}
Is $\struct{\bb{Z},+} \simeq \struct{\bb{R}, \cdot}$? \bld{No}, because $\bb{Z}$ is countably infinite, whereas $\bb{R}$ are uncountably infinite. These two sets have different cardinalities.