\documentclass{article}
\documentclass{article}
\usepackage[margin=1in]{geometry}
\usepackage{amsmath, amsthm, amssymb, fancyhdr, tikz, circuitikz, graphicx}
\usepackage{centernot, xcolor, hhline, multirow, listings}
\usepackage{blkarray, booktabs, bigstrut, etoolbox, extarrows}
\usepackage[normalem]{ulem}
\usepackage{bookmark}
\usetikzlibrary{math}
\usetikzlibrary{fit}

\pagestyle{fancy}

\usepackage{hyperref}
\hypersetup{
  colorlinks=true,
  linkcolor=black,
  filecolor=magenta,
  urlcolor=cyan,
}
%formatting
\newcommand{\bld}{\textbf}
\newcommand{\itl}{\textit}
\newcommand{\uln}{\underline}

%math word symbols
\newcommand{\bb}{\mathbb}
\DeclareMathOperator{\tif}{~\text{if}~}
\DeclareMathOperator{\tand}{~\text{and}~}
\DeclareMathOperator{\tbut}{~\text{but}~}
\DeclareMathOperator{\tor}{~\text{or}~}
\DeclareMathOperator{\tsuchthat}{~\text{such that}~}
\DeclareMathOperator{\tsince}{~\text{since}~}
\DeclareMathOperator{\twhen}{~\text{when}~}
\DeclareMathOperator{\twhere}{~\text{where}~}
\DeclareMathOperator{\tfor}{~\text{for}~}
\DeclareMathOperator{\tthen}{~\text{then}~}

%display shortcut
\DeclareMathOperator{\dstyle}{\displaystyle}
\DeclareMathOperator{\sstyle}{\scriptstyle}

%linear algebra
\DeclareMathOperator{\id}{\bld{id}}
\DeclareMathOperator{\vecspan}{\text{span}}

%augmented matrix environment
\newenvironment{amatrix}[1]{%
  \left(\begin{array}{@{}*{#1}{c}|c@{}}
    }{
  \end{array}\right)
}

%lists
\newcommand{\bitem}[1]{\item[\bld{#1.}]}
\newcommand{\bbitem}[2]{\item[\bld{#1.}] \bld{#2}}
\newcommand{\biitem}[2]{\item[\bld{#1.}] \itl{#2}}
\newcommand{\iitem}[1]{\item[\itl{#1.}]}
\newcommand{\iiitem}[2]{\item[\itl{#1.}] \bld{#2}}

%homework
\newenvironment*{question}[2]{
  \subsection*{#1} \itl{#2}
  \begin{enumerate}
    }{
  \end{enumerate}
}
\newcommand{\qitem}[2]{\item[\bld{#1}] \itl{#2}}

\newcommand{\assignment}{Homework 8}

\lhead{Linear Algebra}
\chead{\assignment}
\rhead{Peter Schaefer}

\begin{document}
\section*{\assignment}

\subsection*{4.3}

\question{4}{Which of the following sets of vector in $P_2$ are linearly dependent?}
\begin{enumerate}
  \qitem{a}{$2-x+4x^2,~~3+6x+2x^2,~~2+10x-4x^2$}{
    \begin{proof}[Work]
      Let $k_1, k_2, k_3 \in \bb{R} \tsuchthat k_1(2-x+4x^2) + k_2(3+6x+2x^2) + k_3(2+10x-4x^2) = \id$. From this, we can get a linear system of equations, and an augmented matrix.
      \begin{align*}
        2k_1 + 3k_2 + 2k_3          & = 0 \\
        -xk_1 + 6xk_2 + 10xk_3      & = 0 \\
        4x^2k_1 + 2x^2k_2 - 4x^2k_3 & = 0
      \end{align*}
      \begin{align*}
         & \begin{abmatrix}{3}{1}
             2  & 3 & 2  & 0 \\
             -1 & 6 & 10 & 0 \\
             4  & 2 & -4 & 0
           \end{abmatrix} \xrightarrow[R_3 + 4R_2]{R_1 + 2R_2}
        \begin{abmatrix}{3}{1}
          0  & 15 & 22 & 0 \\
          -1 & 6  & 10 & 0 \\
          0  & 26 & 36 & 0
        \end{abmatrix} \xrightarrow{R_3 - 2R_1}
        \begin{abmatrix}{3}{1}
          0  & 15 & 22 & 0 \\
          -1 & 6  & 10 & 0 \\
          0  & -4 & -8 & 0
        \end{abmatrix} \xrightarrow[-\frac{1}{4}R_3]{R_1 + 4R_3}                         \\
         & \begin{abmatrix}{3}{1}
             0  & -1 & -10 & 0 \\
             -1 & 6  & 10  & 0 \\
             0  & 1  & 2   & 0
           \end{abmatrix} \xrightarrow[R_2 - 6R_3]{R_1 + R_3}
        \begin{abmatrix}{3}{1}
          0  & 0 & -8  & 0 \\
          -1 & 0 & -12 & 0 \\
          0  & 1 & 2  & 0
        \end{abmatrix} \xrightarrow[-R_2]{-\frac{1}{8}R_1}
        \begin{abmatrix}{3}{1}
          0 & 0 & 1  & 0 \\
          1 & 0 & 12 & 0 \\
          0 & 1 & 2  & 0
        \end{abmatrix} \xrightarrow[R_3 - 2R_1]{R_2 - 12R_1}                             \\
         & \begin{abmatrix}{3}{1}
             0 & 0 & 1 & 0 \\
             1 & 0 & 0 & 0 \\
             0 & 1 & 0 & 0
           \end{abmatrix} \xrightarrow[R_1 \leftrightarrow R_2]{R_1 \leftrightarrow R_3}
        \begin{abmatrix}{3}{1}
          1 & 0 & 0 & 0 \\
          0 & 1 & 0 & 0 \\
          0 & 0 & 1 & 0
        \end{abmatrix}
      \end{align*}
      Since the only solution is the trivial solution, therefore $\{2-x+4x^2,~~3+6x+2x^2,~~2+10x-4x^2\}$ are linearly independent.
    \end{proof}
  }
  \qitem{c}{$3+x+x^2,~~2-x+5x^2,~~4-3x^2$}{
    \begin{proof}[Work]
      Let $k_1, k_2, k_3 \in \bb{R} \tsuchthat k_1(3+x+x^2) + k_2(2-x+5x^2) + k_3(4-3x^2) = \id$. From this, we can get a linear system of equations, and an augmented matrix.
      \begin{align*}
        3k_1 + 2k_2 + 4k_3        & = 0 \\
        xk_1 - xk_2 + 0xk_3       & = 0 \\
        x^2k_1 - 5x^2k_2 -3x^3k_3 & = 0
      \end{align*}
      \begin{align*}
         & \begin{abmatrix}{3}{1}
             3 & 2  & 4  & 0 \\
             1 & -1 & 0  & 0 \\
             1 & -5 & -3 & 0
           \end{abmatrix} \xrightarrow[R_3 - R_2]{R_1 - 3R_2}
        \begin{abmatrix}{3}{1}
          0 & 5  & 4  & 0 \\
          1 & -1 & 0  & 0 \\
          0 & -4 & -3 & 0
        \end{abmatrix} \xrightarrow{R_3 + R_1}
        \begin{abmatrix}{3}{1}
          0 & 1  & 1  & 0 \\
          1 & -1 & 0  & 0 \\
          0 & -4 & -3 & 0
        \end{abmatrix} \xrightarrow{R_3 + 4R_1}               \\
         & \begin{abmatrix}{3}{1}
             0 & 1  & 1 & 0 \\
             1 & -1 & 0 & 0 \\
             0 & 0  & 1 & 0
           \end{abmatrix} \xrightarrow[R_2 + R_1]{R_1 - R_3}
        \begin{abmatrix}{3}{1}
          0 & 1 & 0 & 0 \\
          1 & 0 & 0 & 0 \\
          0 & 0 & 1 & 0
        \end{abmatrix} \xrightarrow{R_1 \leftrightarrow R_2}
        \begin{abmatrix}{3}{1}
          1 & 0 & 0 & 0 \\
          0 & 1 & 0 & 0 \\
          0 & 0 & 1 & 0
        \end{abmatrix}
      \end{align*}
      Since the only solution is the trivial solution, therefore $\{3+x+x^2,~~2-x+5x^2,~~4-3x^2\}$ are linearly independent.
    \end{proof}
  }
\end{enumerate}

\question{9}{For which real values of $\lambda$ do the following vectors form a linearly dependent set in $\bb{R}^3$?
  \[
    v_1 = \left(\lambda, -\frac{1}{2}, -\frac{1}{2}\right), \quad v_2 = \left(-\frac{1}{2}, \lambda, -\frac{1}{2}\right), \quad v_3 = \left(-\frac{1}{2}, -\frac{1}{2}, \lambda\right)
  \]}
\begin{proof}[Work]
  Let $k_1, k_2, k_3 \in \bb{R} \tsuchthat k_1\left(\lambda, -\frac{1}{2}, -\frac{1}{2}\right) + k_2\left(-\frac{1}{2}, \lambda, -\frac{1}{2}\right) + k_3\left(-\frac{1}{2}, -\frac{1}{2}, \lambda\right) = (0,0,0)$. From this, we can get a linear system of equations, and matrix equation.
  \begin{align*}
    \lambda k_1 - \frac{1}{2}k_2 - \frac{1}{2}k_3  & = 0 \\
    -\frac{1}{2}k_1 + \lambda k_2 -\frac{1}{2}k_3  & = 0 \\
    -\frac{1}{2}k_1 - \frac{1}{2}k_2 + \lambda k_3 & = 0
  \end{align*}
  \begin{align*}
    \begin{bmatrix}
      \lambda      & -\frac{1}{2} & -\frac{1}{2} \\
      -\frac{1}{2} & \lambda      & -\frac{1}{2} \\
      -\frac{1}{2} & -\frac{1}{2} & \lambda
    \end{bmatrix}
    \begin{bmatrix}
      k_1 \\ k_2 \\ k_3
    \end{bmatrix} =
    \begin{bmatrix}
      0 \\ 0 \\ 0
    \end{bmatrix}
  \end{align*}
  Consider the determinant of the square matrix.
  \begin{align*}
     & \left\lvert\begin{array}{ccc}
                    \lambda      & -\frac{1}{2} & -\frac{1}{2} \\
                    -\frac{1}{2} & \lambda      & -\frac{1}{2} \\
                    -\frac{1}{2} & -\frac{1}{2} & \lambda
                  \end{array}\right\rvert \overunderset{R_1 + R_2}{R_1 + R_3}{===}
    \left\lvert\begin{array}{ccc}
                 \lambda - 1  & \lambda - 1  & \lambda - 1  \\
                 -\frac{1}{2} & \lambda      & -\frac{1}{2} \\
                 -\frac{1}{2} & -\frac{1}{2} & \lambda
               \end{array}\right\rvert ===
    (\lambda - 1)\left\lvert\begin{array}{ccc}
                              1            & 1            & 1            \\
                              -\frac{1}{2} & \lambda      & -\frac{1}{2} \\
                              -\frac{1}{2} & -\frac{1}{2} & \lambda
                            \end{array}\right\rvert \overunderset{R_2 + \frac{1}{2}R_1}{R_3 + \frac{1}{2}R_1}{===} \\
     & (\lambda - 1)\left\lvert\begin{array}{ccc}
                                 1 & 1                   & 1                   \\
                                 0 & \lambda+\frac{1}{2} & 0                   \\
                                 0 & 0                   & \lambda+\frac{1}{2}
                               \end{array}\right\rvert =
    (\lambda - 1)(1)(\lambda+\frac{1}{2}) = (\lambda - 1)(\lambda+\frac{1}{2})^2
  \end{align*}
  When $\lambda = 1 \tor \lambda = -\frac{1}{2}$, the determinant of this matrix will be zero. By Theorem 4 of Lecture Notes 32, when the determinant is zero, the coefficient matrix is singular. By the Big Theorem, if the coefficient matrix is singular, then $A\vec{x} = \vec{0}$ does not only have the trivial solution, meaning that there exists $k_1,k_2,k_3 \in \bb{R} \tsuchthat k_1\left(\lambda, -\frac{1}{2}, -\frac{1}{2}\right) + k_2\left(-\frac{1}{2}, \lambda, -\frac{1}{2}\right) + k_3\left(-\frac{1}{2}, -\frac{1}{2}, \lambda\right) = (0,0,0)$, where not all $k_i$ are zero. This means that these vectors are linearly dependent when $\lambda = 1 \tor \lambda = -\frac{1}{2}$.
\end{proof}
\qdash

\question{13}{Show that if $S = \{v_1,v_2,\ldots,v_r\}$ is a linearly dependent set of vectors in a vector space $V$, and if $v_{r+1},\ldots,v_n$ are any vectors in $V$ that are not in $S$, then $\{v_1,v_2,\ldots,v_r,v_{r+1},\ldots,v_n\}$ is also linearly dependent.}
\begin{proof}[Work]
  By definition, if S is a linearly dependent set of vectors in $V$, then $\exists~ k_1,k_2,\ldots,k_r \in \bb{R}$ such that $k_1v_1 + k_2v_2 + \cdots + k_rv_r = \id$, where not all $k_i = 0$. Now consider $S' = S \cup \{v_{r+1},\ldots,v_n\}$, where $v_{r+1},\ldots,v_n$ are not in $S$. Consider $k_1,k_2,\ldots,k_r,k_{r+1},\ldots,k_n \in \bb{R}$ such that $k_1v_1 + k_2v_2 + \cdots + k_rv_r + k_{r+1}v_{r+1} + \cdots + k_nv_n = \id$.
  \begin{align*}
    k_1v_1 + k_2v_2 + \cdots + k_rv_r + k_{r+1}v_{r+1} + \cdots + k_nv_n & = \id &  & \lnot \forall k_i = 0 \\
    k_1v_1 + k_2v_2 + \cdots + k_rv_r + 0v_{r+1} + \cdots + 0v_n         & = \id &  & \lnot \forall k_i = 0 \\
    k_1v_1 + k_2v_2 + \cdots + k_rv_r + \id + \cdots + \id               & = \id &  & \lnot \forall k_i = 0 \\
    k_1v_1 + k_2v_2 + \cdots + k_rv_r                                    & = \id &  & \lnot \forall k_i = 0 \\
  \end{align*}
  This final equation is known to exist, since we asserted at the beginning that $S$ was linearly dependent. Therefore, $S'$ must also be linearly dependent. Therefore, if $S = \{v_1,v_2,\ldots,v_r\}$ is a linearly dependent set of vectors in a vector space $V$, and if $v_{r+1},\ldots,v_n$ are any vectors in $V$ that are not in $S$, then $\{v_1,v_2,\ldots,v_r,v_{r+1},\ldots,v_n\}$ is also linearly dependent.
\end{proof}
\qdash

\question{21}{The functions $f_1(x) = x \tand f_2(x) = \cos x$ are linearly independent in $F(-\infty, \infty)$ because neither function is a scalar multiple of the other. Confirm the linear independence using Wroński's test.}
\begin{proof}[Work]
  According to the Wroński's test, $f_1 \tand f_1$ are linearly independent if $W(x)$ is not identically zero.
  \begin{align*}
    W(x) = \left\lvert \begin{array}{cc}
                         x & \cos x  \\
                         1 & -\sin x
                       \end{array}\right\rvert = -x\sin x - \cos x
  \end{align*}
  When $x = 0$, $W(x) = -0\sin 0 - \cos 0 = 0 - 1 = -1 \neq 0$. Therefore $W$ is not identically zero. Therefore, $f_1(x) = x \tand f_2(x) = \cos x$ are linearly independent.
\end{proof}
\qdash

\subsection*{4.4}

\question{4}{Which of the following form bases for $P_2$?}
\begin{enumerate}
  \qitem{a}{$1-3x+2x^2,~~1+x+4x^2,~~1-7x$}{
    \begin{proof}[Work]
      In order for $\{1-3x+2x^2,~~1+x+4x^2,~~1-7x\}$ to be a basis of $P_2$, it must be linearly independent and it must span $P_2$. Let $k_1,k_2,k_3 \in \bb{R} \tsuchthat k_1(1-3x+2x^2) + k_2(1+x+4x^2) + k_3(1-7x) = (b_1 + b_2x + b_3x^2)$.
      \begin{align*}
        1k_1  + 1k_2 + 1k_3          & = b_1 \\
        -3xk_1 + 1xk_2 - 7xk_3       & = b_2 \\
        2x^2k_1  + 4x^2k_2 + 0x^2k_3 & = b_3
      \end{align*}
      Consider the augmented matrix of this linear system of equations.
      \begin{align*}
        \begin{abmatrix}{3}{1}
          1  & 1 & 1 & b_1 \\
          -3 & 1 & -7 & b_2\\
          2  & 4 & 0 & b_3
        \end{abmatrix} \xrightarrow[R_3 - 2R_1]{R_2 + 3R_1}
        \begin{bmatrix}
          1 & 1 & 1  & b_1      \\
          0 & 4 & -4 & b_2+3b_1 \\
          0 & 2 & -2 & b_3-2b_1
        \end{bmatrix} \xrightarrow{R_2 - 2R_3}
        \begin{abmatrix}{3}{1}
          1 & 1 & 1  & b_1               \\
          0 & 0 & 0  & b_2+3b_1-b_3+4b_1 \\
          0 & 2 & -2 & b_3-2b_1
        \end{abmatrix}
      \end{align*}
      The second row of the augmented matrix represents $0 = b_2+3b_1-b_3+4b_1$. If $0 \neq b_2+3b_1-b_3+4b_1$, then there is no solution to the linear system of equations, and thus no coefficients for $k_1,k_2,k_3$ that satisfy the above condition. Therefore $\{1-3x+2x^2,~~1+x+4x^2,~~1-7x\}$ does not span $P_2$, and thus cannot be a basis for $P_2$.
    \end{proof}
  }
\end{enumerate}

\question{7}{Find the coordinate vector of $\vec{w}$ relative to the basis $S = \{\vec{u}_1, \vec{u}_2\}$ for $\bb{R}^2$.}
\begin{enumerate}
  \qitem{b}{$\vec{u}_1 = (2,-4),~~\vec{u}_2 = (3,8);~~\vec{w} = (1,1)$}{
    \begin{proof}[Work]
      Let $k_1, k_2 \in \bb{R} \tsuchthat k_1(2,-4) + k_2(3,8) = (1,1)$.
      \begin{align*}
        2k_1 + 3k_2  & = 1 \\
        -4k_1 + 8k_3 & = 1
      \end{align*}
      \begin{align*}
        \begin{abmatrix}{2}{1}
          2  & 3 & 1 \\
          -4 & 8 & 1
        \end{abmatrix} \xrightarrow{R_2 + 2R_2}
        \begin{abmatrix}{2}{1}
          2  & 3 & 1 \\
          0 & 14 & 3
        \end{abmatrix} \xrightarrow{\frac{1}{14}R_2}
        \begin{abmatrix}{2}{1}
          2  & 3 & 1 \\
          0 & 1 & \frac{3}{14}
        \end{abmatrix} \xrightarrow{R_1 - 3R_2}
        \begin{abmatrix}{2}{1}
          2  & 0 & \frac{5}{14} \\
          0 & 1 & \frac{3}{14}
        \end{abmatrix} \xrightarrow{\frac{1}{2}R_1}
        \begin{abmatrix}{2}{1}
          1  & 0 & \frac{5}{28} \\
          0 & 1 & \frac{3}{14}
        \end{abmatrix}
      \end{align*}
      Therefore, $\frac{5}{28}(2,-4) + \frac{3}{14}(3,8) = (1,1)$, or in other terms $(1,1)_S = (\frac{5}{28},\frac{3}{14})$.
    \end{proof}
  }
  \qitem{c}{$\vec{u}_1 = (1,1),~~\vec{u}_2 = (0,2);~~\vec{w} = (a,b)$}{
    \begin{proof}[Work]
      Let $k_1, k_2 \in \bb{R} \tsuchthat k_1(1,1) + k_2(0,2) = (a,b)$.
      \begin{align*}
        1k_1 + 0k_2 & = a \\
        1k_1 + 2k_2 & = b
      \end{align*}
      \begin{align*}
        \begin{abmatrix}{2}{1}
          1 & 0 & a \\
          1 & 2 & b
        \end{abmatrix} \xrightarrow{R_2 - R_1}
        \begin{abmatrix}{2}{1}
          1 & 0 & a \\
          0 & 2 & b-a
        \end{abmatrix} \xrightarrow{\frac{1}{2}R_2}
        \begin{abmatrix}{2}{1}
          1 & 0 & a \\
          0 & 1 & \frac{b-a}{2}
        \end{abmatrix}
      \end{align*}
      Therefore, $a(1,1) + \frac{b-a}{2}(0,2) = (a,b)$, or in other terms $(a,b)_S = (a,\frac{b-a}{2})$.
    \end{proof}
  }
\end{enumerate}

\question{12}{Show that $\{A_1,A_2,A_3,A_4\}$ is a basis for $\mathcal{M}_{22}$, and express $A$ as a linear combination of the basis vectors.
  \[
    A_1 = \begin{bmatrix}
      1 & 0 \\
      1 & 0
    \end{bmatrix},~~
    A_2 = \begin{bmatrix}
      1 & 1 \\
      0 & 0
    \end{bmatrix},~~
    A_3 = \begin{bmatrix}
      1 & 0 \\
      0 & 1
    \end{bmatrix},~~
    A_4 = \begin{bmatrix}
      0 & 0 \\
      1 & 0
    \end{bmatrix};~~
    A = \begin{bmatrix}
      6 & 2 \\
      5 & 3
    \end{bmatrix}
  \]}
\begin{proof}[Proof of basis]
  Let $k_1,k_2,k_3,k_4 \in \bb{R} \tsuchthat k_1\begin{bmatrix} 1 & 0 \\ 1 & 0 \end{bmatrix} + k_2\begin{bmatrix} 1 & 1 \\ 0 & 0 \end{bmatrix} + k_3\begin{bmatrix} 1 & 0 \\ 0 & 1 \end{bmatrix} + k_4\begin{bmatrix}0 & 0 \\ 1 & 0 \end{bmatrix} = \begin{bmatrix} 6 & 2 \\ 5 & 3 \end{bmatrix}$.
  \begin{align*}
    1k_1 + 1k_2 + 1k_3 + 0k_4 & = 6 \\
    0k_1 + 1k_2 + 0k_3 + 0k_4 & = 2 \\
    1k_1 + 0k_2 + 0k_3 + 1k_4 & = 5 \\
    0k_1 + 0k_2 + 1k_3 + 0k_4 & = 3
  \end{align*}
  Now consider only the coefficient matrix.
  \begin{align*}
    \begin{bmatrix}
      1 & 1 & 1 & 0 \\
      0 & 1 & 0 & 0 \\
      1 & 0 & 0 & 1 \\
      0 & 0 & 1 & 0
    \end{bmatrix} \xrightarrow[R_3 + R_4,R_3 + R_2]{R_3 - R_1}
    \begin{bmatrix}
      1 & 1 & 1 & 0 \\
      0 & 1 & 0 & 0 \\
      0 & 0 & 0 & 1 \\
      0 & 0 & 1 & 0
    \end{bmatrix} \xrightarrow[R_1 - R_3]{R_1 - R_2}
    \begin{bmatrix}
      1 & 0 & 0 & 0 \\
      0 & 1 & 0 & 0 \\
      0 & 0 & 0 & 1 \\
      0 & 0 & 1 & 0
    \end{bmatrix} \xrightarrow{R_3 \leftrightarrow R_4}
    \begin{bmatrix}
      1 & 0 & 0 & 0 \\
      0 & 1 & 0 & 0 \\
      0 & 0 & 1 & 0 \\
      0 & 0 & 0 & 1
    \end{bmatrix}
  \end{align*}
  Since every column has a leading 1, $\{A_1,A_2,A_3,A_4\}$ are linearly independent. Since every row has a leading 1, $\{A_1,A_2,A_3,A_4\}$ spans $\mathcal{M}_{22}$. Therefore, since $\{A_1,A_2,A_3,A_4\}$ is linearly independent and spans $\mathcal{M}_{22}$, $\{A_1,A_2,A_3,A_4\}$ is a basis for $\mathcal{M}_{22}$.
\end{proof}
\begin{proof}[A as a linear combination]
  \begin{align*}
    1k_1 + 1k_2 + 1k_3 + 0k_4 & = 6 \\
    0k_1 + 1k_2 + 0k_3 + 0k_4 & = 2 \\
    1k_1 + 0k_2 + 0k_3 + 1k_4 & = 5 \\
    0k_1 + 0k_2 + 1k_3 + 0k_4 & = 3
  \end{align*}
  \begin{align*}
     & \begin{abmatrix}{4}{1}
         1 & 1 & 1 & 0 & 6 \\
         0 & 1 & 0 & 0 & 2 \\
         1 & 0 & 0 & 1 & 5 \\
         0 & 0 & 1 & 0 & 3
       \end{abmatrix} \xrightarrow[R_3 + R_4,R_3 + R_2]{R_3 - R_1}
    \begin{abmatrix}{4}{1}
      1 & 1 & 1 & 0 & 6 \\
      0 & 1 & 0 & 0 & 2 \\
      0 & 0 & 0 & 1 & 4 \\
      0 & 0 & 1 & 0 & 3
    \end{abmatrix} \xrightarrow[R_1 - R_3]{R_1 - R_2}              \\
     & \begin{abmatrix}{4}{1}
         1 & 0 & 0 & 0 & 1 \\
         0 & 1 & 0 & 0 & 2 \\
         0 & 0 & 0 & 1 & 4 \\
         0 & 0 & 1 & 0 & 3
       \end{abmatrix} \xrightarrow{R_3 \leftrightarrow R_4}
    \begin{abmatrix}{4}{1}
      1 & 0 & 0 & 0 & 1 \\
      0 & 1 & 0 & 0 & 2 \\
      0 & 0 & 1 & 0 & 3 \\
      0 & 0 & 0 & 1 & 4
    \end{abmatrix}
  \end{align*}
  Therefore, $1A_1 + 2A_2 + 3A_3 + 4A_4 = A$.
\end{proof}
\qdash

\subsubsection*{4.5}

\question{3}{Find a basis for the solution space of the homogeneous linear system, and find the dimension of that space.
  \begin{align*}
    1x_1 - 4x_2 + 3x_3 - 1x_4 & = 0 \\
    2x_1 - 8x_2 + 6x_3 - 2x_4 & = 0
  \end{align*}}
\begin{proof}[Work]
  \begin{align*}
    \begin{abmatrix}{4}{1}
      1 & -4 & 3 & -1 & 0 \\
      2 & -8 & 6 & -2 & 0
    \end{abmatrix} \xrightarrow{R_2 -2R_1}
    \begin{abmatrix}{4}{1}
      1 & -4 & 3 & -1 & 0 \\
      0 & 0  & 0 & 0  & 0
    \end{abmatrix}
  \end{align*}
  Let $x_2 = t_2,~x_3 = t_3,~x_4 = t_4$, where $t_2,t_3,t_4$ are free parameters. $x_1 = 4t_2 - 3t_3 + 1t_4$. Therefore a solution to the homogeneous linear system takes the form
  \begin{align*}
    (x_1,x_2,x_3,x_4) & = (4t_2 - 3t_3 + 1t_4, t_2,t_3,t_4)                \\
                      & = (4t_2,t_2,0,0) + (-3t_3,0,t_3,0) + (t_4,0,0,t_4) \\
                      & = t_2(4,1,0,0) + t_3(-3,0,1,0) + t_4(1,0,0,1)
  \end{align*}
  Therefore a basis of the solution space is $\{(4,1,0,0),(-3,0,1,0),(1,0,0,1)\}$. Since there are three basis vectors, the dimension is the solution space is 3.
\end{proof}
\qdash

\question{7}{Find bases for the following subspaces of $\bb{R}^3$.}
\begin{enumerate}
  \qitem{a}{The plane $3x-2y+5z = 0$.}{
    \begin{proof}[Work]
      \begin{align*}
        3x-2y+5z                    & = 0 \\
        x-\frac{2}{3}y+\frac{5}{3}z & = 0
      \end{align*}
      Let $y = t_y \tand z = t_z$, where $t_y \tand t_z$ are free parameters. $x = \frac{2}{3}t_y-\frac{5}{3}t_z$. Therefore a point on the plane $3x-2y+5z = 0$ takes the form
      \begin{align*}
        (x,y,z) & = (\frac{2}{3}t_y-\frac{5}{3}t_z,t_y,t_z)          \\
                & = (\frac{2}{3}t_y,t_y,0) + (-\frac{5}{3}t_z,0,t_z) \\
                & = t_y(\frac{2}{3},1,0) + t_z(-\frac{5}{3},0,1)
      \end{align*}
      Therefore a basis of the solution space is $\{(\frac{2}{3},1,0),(-\frac{5}{3},0,1)\}$.
    \end{proof}
  }
  \qitem{b}{The plane $x-y = 0$.}{
    \begin{proof}[Work]
      \begin{align*}
        x - y     & = 0 \\
        x - y + z & = z
      \end{align*}
      Let $y = t_y \tand z = t_z$, where $t_y \tand t_z$ are free parameters $x = t_y$. Therefore a point on the plane $x-y=0$ takes the form
      \begin{align*}
        (x,y,z) & = (t_y,t_y,t_z)           \\
                & = (t_y,t_y,0) + (0,0,t_z) \\
                & = t_y(1,1,0) + t_z(0,0,1)
      \end{align*}
      Therefore a basis of the solution space is $\{(1,1,0),(0,0,1)\}$.
    \end{proof}
  }
  \qitem{c}{The lines $x = 2t,~~y = -t,~~z = 4t$.}{
    \begin{proof}[Work]
      A point on the lines $x = 2t,~~y = -t,~~z = 4t$ take the form
      \begin{align*}
        (x,y,z) & = (2t,-t,4t) \\
                & = t(2,-1,4)
      \end{align*}
      Therefore a basis of the solution space is $\{(2,-1,4)\}$.
    \end{proof}
  }
  \qitem{c}{All the vectors of the form $(a,b,c), \twhere b = a + c$.}{
    \begin{proof}[Work]
      A vector which satisfies the conditions takes the form
      \begin{align*}
        (a,b,c) & = (a,a+c,c)           \\
                & = (a,a,0) + (0,c,c)   \\
                & = a(1,1,0) + c(0,1,1)
      \end{align*}
      Therefore a basis of the solution space is $\{(1,1,0),(0,1,1)\}$.
    \end{proof}
  }
\end{enumerate}

\question{11}{}
\begin{enumerate}
  \qitem{a}{Show that the set $W$ of all polynomials in $P_2$ such that $p(1) = 0$ is a subspace of $P_2$.}{
    \begin{proof}
      Axiom 1: Consider $p_a \tand p_b \in W$. Now consider $p_a + p_b$.
      \[
        (p_a + p_b)(x) = p_a(x) + p_b(x)
      \]
      When $x = 1$, $(p_a + p_b)(1) = p_a(1) + p_b(1) = 0 + 0 = 0$. Therefore Axiom 1 holds.

      Axiom 6: Consider $p \in W \tand k \in \bb{R}$. Now consider $k \cdot p$.
      \[
        (kp)(x) = kp(x)
      \]
      When $x = 1$, $(kp)(x) = kp(x) = k \cdot 0 = 0$. Therefore Axiom 6 holds. Since Axiom 1 and Axiom 6 hold for $W$, $W$ is a subspace of $P_2$.
    \end{proof}
  }
  \qitem{b}{Make a conjecture about the dimesion of $W$.}{
    \begin{proof}[Conjecture]
      I conjecture that the dimension of $W$ is two.
    \end{proof}
  }
  \qitem{c}{Confirm you conjecture by finding a basis for $W$.}{
    \begin{proof}[Work]
      A polynomial which satisfies the condition $p(1)=0$ must also satisfy the following identity.
      \begin{align*}
        p(1) & = p_0 + p_1(1) + p_2(1^2) = 0
             & = p_0 + p_1 + p_2 = 0
      \end{align*}
      Let $p_1 = t_1 \tand p_2 = t_2$, where $t_1 \tand t_2$ are free parameters. $p_0 = -t_1-t_2$. Therefore a vector of coefficients takes the form
      \begin{align*}
        (p_0,p_1,p_2) & = (-t_1-t_2,t_1,t_2)          \\
                      & = (-t_1,t_1,0) + (-t_2,0,t_2) \\
                      & = t_1(-1,1,0) + t_2(-1,0,1)
      \end{align*}
      Therefore a basis of the solution space is $\{-1+x,~-1+x^2\}$. There are two basis vectors, so the dimension of the subspace is two.
    \end{proof}
  }
\end{enumerate}

\question{18}{Let $S$ be a basis for an $n$-dimesional vector space $V$. show that if $\vec{v}_1,\vec{v}_2,\ldots,\vec{v}_r$ form a linearly independent set of vectors in $V$, then the coordinate vectors $(\vec{v}_1)_S,(\vec{v}_2)_S,\ldots,(\vec{v}_r)_S$ form a linearly independent set in $\bb{R}^n$, and conversely.}
\begin{proof}
  Let $S$ be a basis for an $n$-dimensional vector space $V$, and consider an linearly independent set of vectors in $V$, $\vec{v}_1,\vec{v}_2,\ldots,\vec{v}_r$. Then, by definition for $k_1,\ldots,k_r \in \bb{R}$,
  \[
    k_1\vec{v}_1 + k_2\vec{v}_2 + \cdots + k_r\vec{v}_r = \id
  \]
  only has the trivial solution. Now consider a change of basis into base $S$. The equation now becomes
  \[
    k_1(\vec{v}_1)_S + k_2(\vec{v}_2)_S + \cdots + k_r(\vec{v}_r)_S = (\id)_S.
  \]
  However, $(\id)_S = \id$ for all basis, since it can only be obtained by a trivial linear combination, which produces a sum of $\id$. Therefore we have
  \[
    k_1(\vec{v}_1)_S + k_2(\vec{v}_2)_S + \cdots + k_r(\vec{v}_r)_S = \id
  \]
  which only has the trivial solution. Therefore the coordinate vectors $(\vec{v}_1)_S,(\vec{v}_2)_S,\ldots,(\vec{v}_r)_S$ form a linearly independent set in $\bb{R}^n$. Conversely, if $S$ is changed to the original basis for $V$, then the converse can be achieved.
\end{proof}
\qdash

\end{document}