\documentclass{article}
\documentclass{article}
\usepackage[margin=1in]{geometry}
\usepackage{amsmath, amsthm, amssymb, fancyhdr, tikz, circuitikz, graphicx}
\usepackage{centernot, xcolor, hhline, multirow, listings}
\usepackage{blkarray, booktabs, bigstrut, etoolbox}
\usepackage[normalem]{ulem}
\usepackage{bookmark}
\usetikzlibrary{math}
\usetikzlibrary{fit}

\pagestyle{fancy}

\usepackage{hyperref}
\hypersetup{
  colorlinks=true,
  linkcolor=black,
  filecolor=magenta,
  urlcolor=cyan,
}
%formatting
\newcommand{\bld}{\textbf}
\newcommand{\itl}{\textit}
\newcommand{\uln}{\underline}

%math word symbols
\newcommand{\bb}{\mathbb}
\DeclareMathOperator{\tif}{~\text{if}~}
\DeclareMathOperator{\tand}{~\text{and}~}
\DeclareMathOperator{\tbut}{~\text{but}~}
\DeclareMathOperator{\tor}{~\text{or}~}
\DeclareMathOperator{\tsuchthat}{~\text{such that}~}
\DeclareMathOperator{\tsince}{~\text{since}~}
\DeclareMathOperator{\twhen}{~\text{when}~}
\DeclareMathOperator{\twhere}{~\text{where}~}
\DeclareMathOperator{\tfor}{~\text{for}~}
\DeclareMathOperator{\tthen}{~\text{then}~}
\DeclareMathOperator{\tto}{~\text{to}~}

%display shortcut
\DeclareMathOperator{\dstyle}{\displaystyle}
\DeclareMathOperator{\sstyle}{\scriptstyle}

%linear algebra
\DeclareMathOperator{\id}{\bld{id}}
\DeclareMathOperator{\vecspan}{\text{span}}

%discrete math - integer properties
\DeclareMathOperator{\tdiv}{\text{div}}
\DeclareMathOperator{\tmod}{\text{mod}}
\DeclareMathOperator{\lcm}{\text{lcm}}

%augmented matrix environment
\newenvironment{apmatrix}[2]{%
  \left(\begin{array}{@{~}*{#1}{c}|@{~}*{#2}{c}}
    }{
  \end{array}\right)
}
\newenvironment{abmatrix}[2]{%
  \left[\begin{array}{@{~}*{#1}{c}|@{~}*{#2}{c}}
      }{
    \end{array}\right]
}

%lists
\newcommand{\bitem}[1]{\item[\bld{#1.}]}
\newcommand{\bbitem}[2]{\item[\bld{#1.}] \bld{#2}}
\newcommand{\biitem}[2]{\item[\bld{#1.}] \itl{#2}}
\newcommand{\iitem}[1]{\item[\itl{#1.}]}
\newcommand{\iiitem}[2]{\item[\itl{#1.}] \bld{#2}}
\newcommand{\btitem}[2]{\item[\bld{#1.}] \texttt{#2}}

%homework
\newcommand{\question}[2]{\noindent {\large\bld{#1}} #2 \qline}
\newcommand{\qitem}[3]{\item[\bld{#1.}] \itl{#2} #3 \qdash}

\newcommand{\qline}{~\newline\noindent\textcolor[RGB]{200,200,200}{\rule[0.5ex]{\linewidth}{0.2pt}}}
\newcommand{\qdash}{~\newline\noindent\textcolor[RGB]{200,200,200}{\hdashrule[0.5ex]{\linewidth}{0.2pt}{2pt}}}

\newcommand{\assignment}{Homework 7}

\lhead{Linear Algebra}
\chead{\assignment}
\rhead{Peter Schaefer}

\begin{document}
\section*{\assignment}

\question{1.4}{Inverses; Algebraic Properties of Matrices}{}
\begin{enumerate}
  \qitem{28}{Show that if a square matrix $A$ satisfies $A^2-3A+1=0, \tthen A^{-1}=3I-A$.}{
    \begin{proof}[work]
      Consider $3I - A$.
      \[
        A(3I - A) = 3AI - A^2 = 3A - A^2
      \]
      If $A^2-3A+I=0$, then we can simplify further to determine exactly what $3A - A^2$ equals.
      \begin{align*}
        A^2-3A+I                 & =0               \\
        (3A - A^2) + A^2-3A+I    & = (3A - A^2) + 0 \\
        (3A + (-A^2 + A^2)-3A)+I & = (3A - A^2)     \\
        (3A - 3A) + I            & = 3A - A^2       \\
        I                        & = 3A - A^2       \\
        \therefore~I             & = A(3I-A)        \\
        \therefore~I             & = (3I-A)A
      \end{align*}
      Since $I = A(3I-A) \tand I = (3I-A)A$, therefore $A^{-1}=3I-A \tif A^2-3A+1=0$.
    \end{proof}
  }
  \qitem{31}{Assuming that all matrices are $n \times n$ and invertible, solve for $D$:
  \[
    C^TB^{-1}A^2BAC^{-1}DA^{-2}B^TC^{-2}=C^T.
  \]}{
  \begin{proof}[work]
    \begin{align*}
      C^TB^{-1}A^2BAC^{-1}DA^{-2}B^TC^{-2}                            & = C^T                               \\
      (C^TB^{-1}A^2BAC^{-1})^{-1}C^TB^{-1}A^2BAC^{-1}DA^{-2}B^TC^{-2} & = (C^TB^{-1}A^2BAC^{-1})^{-1}C^T    \\
      DA^{-2}B^TC^{-2}                                                & = CA^{-1}B^{-1}A^{-2}BC^{T^{-1}}C^T \\
    \end{align*}
  \end{proof}
  }
  \qitem{39}{Using Matrix Inversion, find the unique solution of the given linear system.
    \begin{align*}
      3x_1 - 2x_2 & = -1 \\
      4x_1 + 5x_2 & = 3
    \end{align*}}{}
  \qitem{53a}{Show that if $A,B$ and $A+B$ are invertible matrices with the same size, then
  \[
    A(A^{-1}+B^{-1})B(A+B)^{-1} = I.
  \]}{}
  \qitem{55}{Show that if $A$ is a square matrix such that $A^k=0$ for some positive integer $k$, then the matrix $(I-A)$ is invertible and
    \[
      (I-A)^{-1} = I + A + A^2 + \cdots + A^{k-1}.
    \]}{}
\end{enumerate}

\question{1.5}{Elementary Matrices and a Method for Finding A-1}
\begin{enumerate}
  \qitem{15}{Use the inverse algorithm to find the inverse of the given matrix, if the inverse exists.
    \[
      \begin{bmatrix}
        -1 & 2 & 0 \\
        2  & 1 & 2 \\
        0  & 2 & 1
      \end{bmatrix}.
    \]}{}
  \qitem{25}{Find the inverse of the following $4 \times 4$ matrices, where $k_1,k_2,k_3,k_4, \tand k$ are all non-zero.
    \begin{enumerate}
      \biitem{a}{$
          \begin{bmatrix}
            k_1 & 0   & 0   & 0   \\
            0   & k_2 & 0   & 0   \\
            0   & 0   & k_3 & 0   \\
            0   & 0   & 0   & k_4 \\
          \end{bmatrix}$.}
      \biitem{b}{$
          \begin{bmatrix}
            k & 1 & 0 & 0 \\
            0 & 1 & 0 & 0 \\
            0 & 0 & k & 1 \\
            0 & 0 & 0 & 1 \\
          \end{bmatrix}$.}
    \end{enumerate}}{}
  \qitem{27}{Find all values of $c$, if any, for which the given matrix is invertible.}{}
  \qitem{29}{Write the given matrix as a product of elementary matrices.
    \[
      \begin{bmatrix}
        -3 & 1 \\
        2  & 2
      \end{bmatrix}
    \]}{}
  \qitem{41}{Prove that if $A \tand B$ are $m \times n$ matrices, then $A \tand B$ are row equivalent if and only if $A \tand B$ have the same reduced row eschelon form.}{}
\end{enumerate}

\question{1.6}{More on Linear Systems and Invertible Matrices}
\begin{enumerate}
  \qitem{15}{Determine conditions on the $b_i$'s, if any, in order to guarantee that the linear system is consistent.
    \begin{align*}
      x_1 - 2x_2 + 5x_3   & = b_1 \\
      4x_1 - 5x_2 + 8x_3  & = b_2 \\
      -3x_1 + 3x_2 - 3x_3 & = b_3
    \end{align*}}{}
  \qitem{21}{Let $A\vec{x} = \vec{0}$ be a homogenous system of $n$ linear equations in $n$ unknown that has only the trivial solution. Show that if $k$ is any positive integer, then the system $A^k\vec{x} = \vec{0}$ also has only the trivial solution.}{}
\end{enumerate}

\end{document}