\documentclass{article}
\usepackage[margin=1in]{geometry}
\usepackage{amsmath, amsthm, amssymb, fancyhdr, tikz, circuitikz, graphicx}
\usepackage{centernot, xcolor, hhline, multirow, listings}
\usepackage{blkarray, booktabs, bigstrut, etoolbox, extarrows}
\usepackage[normalem]{ulem}
\usepackage{bookmark}
\usetikzlibrary{math}
\usetikzlibrary{fit}

\pagestyle{fancy}

\usepackage{hyperref}
\hypersetup{
  colorlinks=true,
  linkcolor=black,
  filecolor=magenta,
  urlcolor=cyan,
}
%formatting
\newcommand{\bld}{\textbf}
\newcommand{\itl}{\textit}
\newcommand{\uln}{\underline}

%math word symbols
\newcommand{\bb}{\mathbb}
\DeclareMathOperator{\tif}{~\text{if}~}
\DeclareMathOperator{\tand}{~\text{and}~}
\DeclareMathOperator{\tbut}{~\text{but}~}
\DeclareMathOperator{\tor}{~\text{or}~}
\DeclareMathOperator{\tsuchthat}{~\text{such that}~}
\DeclareMathOperator{\tsince}{~\text{since}~}
\DeclareMathOperator{\twhen}{~\text{when}~}
\DeclareMathOperator{\twhere}{~\text{where}~}
\DeclareMathOperator{\tfor}{~\text{for}~}
\DeclareMathOperator{\tthen}{~\text{then}~}

%display shortcut
\DeclareMathOperator{\dstyle}{\displaystyle}
\DeclareMathOperator{\sstyle}{\scriptstyle}

%linear algebra
\DeclareMathOperator{\id}{\bld{id}}
\DeclareMathOperator{\vecspan}{\text{span}}

%augmented matrix environment
\newenvironment{amatrix}[1]{%
  \left(\begin{array}{@{}*{#1}{c}|c@{}}
    }{
  \end{array}\right)
}

%lists
\newcommand{\bitem}[1]{\item[\bld{#1.}]}
\newcommand{\bbitem}[2]{\item[\bld{#1.}] \bld{#2}}
\newcommand{\biitem}[2]{\item[\bld{#1.}] \itl{#2}}
\newcommand{\iitem}[1]{\item[\itl{#1.}]}
\newcommand{\iiitem}[2]{\item[\itl{#1.}] \bld{#2}}

%homework
\newenvironment*{question}[2]{
  \subsection*{#1} \itl{#2}
  \begin{enumerate}
    }{
  \end{enumerate}
}
\newcommand{\qitem}[2]{\item[\bld{#1}] \itl{#2}}

\lhead{Linear Algebra}
\chead{Homework 4}
\rhead{Peter Schaefer}

\begin{document}

\section*{Homework 4}

\subsection*{9}
Let $V = \bb{R}^3$ be a vector space with standard addition and scalar multiplication. Use Theorem 3 of Lecture Note 8 to determine whether the following sets are subspaces of $V$.
\begin{enumerate}
  \biitem{b}{The set of vectors of the form $(a,1,1)$}
  \begin{proof}

  \end{proof}
  \biitem{c}{The set of vectors of the form $(a,b,c)$, where $b = a + c$}
  \begin{proof}

  \end{proof}
  \biitem{d}{The set of vectors of the form $(a,b,0)$}
  \begin{proof}

  \end{proof}
\end{enumerate}

\subsection*{10}
Let $V = P_3$ be the vector space of all polynomials with degree \itl{up to} 3, with standard addition and scalar multiplication. Use Theorem 3 of Lecture Note 8 to determine whether the following sets are subspaces of $V$.
\begin{enumerate}
  \biitem{b}{The set of polynomials $a_0+a_1x+a_2x^2+a_3x^3$ for which $a_0+a_1+a_2+a_3=0$.}
  \begin{proof}

  \end{proof}
  \biitem{c}{The set of polynomials $a_0+a_1x+a_2x^2+a_3x^3$ in which $a_0,a_1,a_2, \tand a_3$ are integers.}
  \begin{proof}

  \end{proof}
\end{enumerate}

\subsection*{11}
Let $V = F(-\infty, \infty)$ be the vector space of all functions from $\bb{R}$ to $\bb{R}$, with standard addition and scalar multiplication. Use Theorem 3 of Lecture Note 8 to determine whether the following sets are subspaces of $V$.
\begin{enumerate}
  \biitem{b}{The set of functions $f$ in $F(-\infty, \infty)$ for which $f(0) = 1$.}
  \begin{proof}

  \end{proof}
  \biitem{c}{The set of functions $f$ in $F(-\infty, \infty)$ for which $f(-x) = x$}
  \begin{proof}

  \end{proof}
\end{enumerate}

\subsection*{13}
Let $V$ be a vector space. Let $I$ be a nonempty set (often called the "index set"), and let $W_i$ be a subspace of $V$ for all $i \in I$. Prove that $\bigcap_{i \in I} W_i$, which addition and scalar multiplication inherited from $V$, is a subspace of V.
\begin{proof}

\end{proof}

\end{document}