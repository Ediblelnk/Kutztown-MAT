\documentclass{article}
\usepackage[margin=1in]{geometry}
\usepackage{amsmath, amsthm, amssymb, fancyhdr, tikz, circuitikz, graphicx}
\usepackage{centernot, xcolor, hhline, multirow, listings}
\usepackage{blkarray, booktabs, bigstrut, etoolbox}
\usepackage[normalem]{ulem}
\usepackage{bookmark}
\usetikzlibrary{math}
\usetikzlibrary{fit}

\pagestyle{fancy}

\usepackage{hyperref}
\hypersetup{
  colorlinks=true,
  linkcolor=black,
  filecolor=magenta,
  urlcolor=cyan,
}
%formatting
\newcommand{\bld}{\textbf}
\newcommand{\itl}{\textit}
\newcommand{\uln}{\underline}

%math word symbols
\newcommand{\bb}{\mathbb}
\DeclareMathOperator{\tif}{~\text{if}~}
\DeclareMathOperator{\tand}{~\text{and}~}
\DeclareMathOperator{\tbut}{~\text{but}~}
\DeclareMathOperator{\tor}{~\text{or}~}
\DeclareMathOperator{\tsuchthat}{~\text{such that}~}
\DeclareMathOperator{\tsince}{~\text{since}~}
\DeclareMathOperator{\twhen}{~\text{when}~}
\DeclareMathOperator{\twhere}{~\text{where}~}
\DeclareMathOperator{\tfor}{~\text{for}~}
\DeclareMathOperator{\tthen}{~\text{then}~}
\DeclareMathOperator{\tto}{~\text{to}~}

%display shortcut
\DeclareMathOperator{\dstyle}{\displaystyle}
\DeclareMathOperator{\sstyle}{\scriptstyle}

%linear algebra
\DeclareMathOperator{\id}{\bld{id}}
\DeclareMathOperator{\vecspan}{\text{span}}

%discrete math - integer properties
\DeclareMathOperator{\tdiv}{\text{div}}
\DeclareMathOperator{\tmod}{\text{mod}}
\DeclareMathOperator{\lcm}{\text{lcm}}

%augmented matrix environment
\newenvironment{apmatrix}[2]{%
  \left(\begin{array}{@{~}*{#1}{c}|@{~}*{#2}{c}}
    }{
  \end{array}\right)
}
\newenvironment{abmatrix}[2]{%
  \left[\begin{array}{@{~}*{#1}{c}|@{~}*{#2}{c}}
      }{
    \end{array}\right]
}

%lists
\newcommand{\bitem}[1]{\item[\bld{#1.}]}
\newcommand{\bbitem}[2]{\item[\bld{#1.}] \bld{#2}}
\newcommand{\biitem}[2]{\item[\bld{#1.}] \itl{#2}}
\newcommand{\iitem}[1]{\item[\itl{#1.}]}
\newcommand{\iiitem}[2]{\item[\itl{#1.}] \bld{#2}}
\newcommand{\btitem}[2]{\item[\bld{#1.}] \texttt{#2}}

%homework
\newcommand{\question}[2]{\noindent {\large\bld{#1}} #2 \qline}
\newcommand{\qitem}[3]{\item[\bld{#1.}] \itl{#2} #3 \qdash}

\newcommand{\qline}{~\newline\noindent\textcolor[RGB]{200,200,200}{\rule[0.5ex]{\linewidth}{0.2pt}}}
\newcommand{\qdash}{~\newline\noindent\textcolor[RGB]{200,200,200}{\hdashrule[0.5ex]{\linewidth}{0.2pt}{2pt}}}

\lhead{Linear Algebra}
\chead{Homework 3}
\rhead{Peter Schaefer}

\begin{document}
\subsection*{Problem 1}
\bld{Let $V$ be a vector space, and let $\vec{u}, \vec{v}, \vec{w} \in V$. Prove that $\tif \vec{u} \oplus \vec{w} = \vec{v} \oplus \vec{w} \tthen \vec{u} = \vec{v}$.}
\begin{proof}
  Consider $\vec{v}, \vec{u}, \vec{w} \in V$, and assume $\vec{u} \oplus \vec{w} = \vec{v} \oplus \vec{w}$.
  \begin{align*}
    \vec{u} \oplus \vec{w}                   & = \vec{v} \oplus \vec{w}                   &  & \text{Assertion}                     \\
    (\vec{u} \oplus \vec{w}) \oplus -\vec{w} & = (\vec{v} \oplus \vec{w}) \oplus -\vec{w} &  & \text{Axiom 5 states}~-\vec{w} \in V \\
    \vec{u} \oplus (\vec{w} \oplus -\vec{w}) & = \vec{v} \oplus (\vec{w} \oplus -\vec{w}) &  & \text{Axiom 3}                       \\
    \vec{u} \oplus \id                       & = \vec{v} \oplus \id                       &  & \text{Def. of additive inverse}      \\
    \vec{u}                                  & = \vec{v}                                  &  & \text{Def. of additive identity}
  \end{align*}
  $\therefore \tif \vec{u} \oplus \vec{w} = \vec{v} \oplus \vec{w} \tthen \vec{u} = \vec{v}$
\end{proof}

\subsection*{Problem 2}
\bld{Prove Theorem $B$.}
\begin{proof}
  Let $k \in \bb{R}$. Recall that Theorem $A$ implies that $0 \odot \id = \id$:
  \begin{align*}
    k \odot \id & = k \odot (0 \odot \id) &  & \text{Thm $A$} \\
                & = (k \cdot 0) \odot \id &  & \text{Axiom 9} \\
                & = 0 \odot \id                               \\
                & = \id                   &  & \text{Thm $A$}
  \end{align*}
  $\therefore \forall~k \in \bb{R},~k \odot \id = \id$
\end{proof}

\subsection*{Problem 3}
\bld{Prove Theorem $D$.} If $k \odot \vec{u} = \id, \tthen k=0 \tor \vec{u} = \id$
\begin{proof}
  Let $k \odot \vec{u} = \id$ and $k \neq 0 $. Since $\frac{1}{k} \neq \frac{1}{0}$, $\frac{1}{k}$ is well-defined.
  \begin{align*}
    k \odot \vec{u}                     & = \id                     &  & \text{Assertion} \\
    \frac{1}{k} \odot (k \odot \vec{u}) & = \frac{1}{k} \odot (\id)                       \\
    (\frac{1}{k} \cdot k) \odot \vec{u} & = \frac{1}{k} \odot \id   &  & \text{Axiom 9}   \\
    1 \odot \vec{u}                     & = \id                     &  & \text{Thm B}     \\
    \vec{u}                             & = \id                     &  & \text{Axiom 10}
  \end{align*}
  $\therefore \tif k \odot \vec{u} = \id, \tthen k=0 \tor \vec{u} = \id$
\end{proof}

\subsection*{Problem 4}
\bld{Prove that there does not exist a real vector space of size $2$.}
Show that there cannot be a vector space of size 2.
\begin{proof}
  Let $V = \{\vec{u}, \vec{v}\}$ be a vectorspace where $\vec{u} \neq \vec{v}$.Axiom 4 states that $\id$ exists, and is unique. Therefore either $\vec{u} = \id \tor \vec{v} = \id$. \\
  Without the loss of generality, let $\vec{u} = \id$.
  \begin{align*}
    \vec{u} \oplus \vec{v} & = \vec{v} & \text{Axiom 4} \\
    \id \oplus \vec{v}     & = \vec{v}
  \end{align*}
  Now consider Axiom 5: $-\vec{v} \in V$. Note that $\vec{v} \oplus \vec{u} = V$, so we know $-\vec{v} \neq \vec{u}$, so therefore $\vec{v} = -\vec{v}$.
  \begin{align*}
    \vec{v} \oplus \vec{v}                 & = \vec{u} = \id &  & \text{Def. of Additive Inverse} \\
    1 \odot \vec{v} \oplus 1 \odot \vec{v} & = \vec{u}       &  & \text{Axiom 10}                 \\
    (1+1) \odot \vec{v}                    & = \vec{u}       &  & \text{Axiom 8}                  \\
    2 \odot \vec{v}                        & = \vec{u} = \id
  \end{align*}
  By Theorem $A$, either $2 = 0 \tor \vec{v} = \id$. We know that $2 \neq 0 \tand \vec{v} \neq \id \tsince \vec{u} = \id$, a contradiction. \\
  $\therefore$ A vectorspace of size 2 cannot exist.

\end{proof}
\end{document}