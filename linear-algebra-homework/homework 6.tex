\documentclass{article}
\documentclass{article}
\usepackage[margin=1in]{geometry}
\usepackage{amsmath, amsthm, amssymb, fancyhdr, tikz, circuitikz, graphicx}
\usepackage{centernot, xcolor, hhline, multirow, listings}
\usepackage{blkarray, booktabs, bigstrut, etoolbox}
\usepackage[normalem]{ulem}
\usepackage{bookmark}
\usetikzlibrary{math}
\usetikzlibrary{fit}

\pagestyle{fancy}

\usepackage{hyperref}
\hypersetup{
  colorlinks=true,
  linkcolor=black,
  filecolor=magenta,
  urlcolor=cyan,
}
%formatting
\newcommand{\bld}{\textbf}
\newcommand{\itl}{\textit}
\newcommand{\uln}{\underline}

%math word symbols
\newcommand{\bb}{\mathbb}
\DeclareMathOperator{\tif}{~\text{if}~}
\DeclareMathOperator{\tand}{~\text{and}~}
\DeclareMathOperator{\tbut}{~\text{but}~}
\DeclareMathOperator{\tor}{~\text{or}~}
\DeclareMathOperator{\tsuchthat}{~\text{such that}~}
\DeclareMathOperator{\tsince}{~\text{since}~}
\DeclareMathOperator{\twhen}{~\text{when}~}
\DeclareMathOperator{\twhere}{~\text{where}~}
\DeclareMathOperator{\tfor}{~\text{for}~}
\DeclareMathOperator{\tthen}{~\text{then}~}
\DeclareMathOperator{\tto}{~\text{to}~}

%display shortcut
\DeclareMathOperator{\dstyle}{\displaystyle}
\DeclareMathOperator{\sstyle}{\scriptstyle}

%linear algebra
\DeclareMathOperator{\id}{\bld{id}}
\DeclareMathOperator{\vecspan}{\text{span}}

%discrete math - integer properties
\DeclareMathOperator{\tdiv}{\text{div}}
\DeclareMathOperator{\tmod}{\text{mod}}
\DeclareMathOperator{\lcm}{\text{lcm}}

%augmented matrix environment
\newenvironment{apmatrix}[2]{%
  \left(\begin{array}{@{~}*{#1}{c}|@{~}*{#2}{c}}
    }{
  \end{array}\right)
}
\newenvironment{abmatrix}[2]{%
  \left[\begin{array}{@{~}*{#1}{c}|@{~}*{#2}{c}}
      }{
    \end{array}\right]
}

%lists
\newcommand{\bitem}[1]{\item[\bld{#1.}]}
\newcommand{\bbitem}[2]{\item[\bld{#1.}] \bld{#2}}
\newcommand{\biitem}[2]{\item[\bld{#1.}] \itl{#2}}
\newcommand{\iitem}[1]{\item[\itl{#1.}]}
\newcommand{\iiitem}[2]{\item[\itl{#1.}] \bld{#2}}
\newcommand{\btitem}[2]{\item[\bld{#1.}] \texttt{#2}}

%homework
\newcommand{\question}[2]{\noindent {\large\bld{#1}} #2 \qline}
\newcommand{\qitem}[3]{\item[\bld{#1.}] \itl{#2} #3 \qdash}

\newcommand{\qline}{~\newline\noindent\textcolor[RGB]{200,200,200}{\rule[0.5ex]{\linewidth}{0.2pt}}}
\newcommand{\qdash}{~\newline\noindent\textcolor[RGB]{200,200,200}{\hdashrule[0.5ex]{\linewidth}{0.2pt}{2pt}}}

\newcommand{\assignment}{Homework 6}

\lhead{Linear Algebra}
\chead{\assignment}
\rhead{Peter Schaefer}

\begin{document}
\section*{\assignment}

\begin{question}{1.2}{Determine the values for which the system has no solutions, exactly one solution, or infinitely many solutions}
  \qitem{25}{\begin{tabular}{rcrcrcr}
      $x$  & $+$ & $2y$ & $-$ & $3z$        & $=$ & $4$   \\
      $3x$ & $-$ & $y$  & $+$ & $5z$        & $=$ & $2$   \\
      $4x$ & $+$ & $y$  & $+$ & $(a^2-14)z$ & $=$ & $a+2$
    \end{tabular}}
  \begin{proof}[work]
    \begin{align*}
      \begin{amatrix}{3}
        1 & 2 & -3 & 4 \\
        3 & -1 & 5 & 2 \\
        4 & 1 & a^2-14 & a+2
      \end{amatrix}     & \xrightarrow[(-4,-8,12,-16)]{R_3 - 4R_1}
      \begin{pmatrix}
        1 & 2  & -3    & 4    \\
        3 & -1 & 5     & 2    \\
        0 & -7 & a^2-2 & a-14
      \end{pmatrix} \xrightarrow[(-3,-6,9,-12)]{R_2 - 3R_1}                   \\
      \begin{pmatrix}
        1 & 2  & -3    & 4    \\
        0 & -7 & 14    & -10  \\
        0 & -7 & a^2-2 & a-14
      \end{pmatrix} & \xrightarrow[(0, 7,-14,10)]{R_3 - R_2, -\frac{1}{7}R_2}
      \begin{amatrix}{3}
        1 & 2 & -3     & 4            \\
        0 & 1 & -2     & \frac{10}{7} \\
        0 & 0 & a^2-16 & a-4
      \end{amatrix}
    \end{align*}
    $R_3$ represents the equation $(a^2-16)z = a-4$. If $a = 4$, there are infinitely many solutions, since $R_3$ leads to a full row of $0$'s. If $a = -4$, then there is no solution, since $R_3$ leads to $0 = -8$. If $a \neq \pm 4$, then there is exactly one solution to the system of equations.
  \end{proof}
  \qitem{27}{\begin{tabular}{rcrcr}
      $x$  & $+$ & $2y$       & $=$ & $1$   \\
      $2x$ & $+$ & $(a^2-5)y$ & $=$ & $a-1$
    \end{tabular}}
  \begin{proof}[work]
    \begin{align*}
      \begin{amatrix}{2}
        1 & 2     & 1 \\
        2 & a^2-5 & a-1
      \end{amatrix} \xrightarrow[(-2,-4,-2)]{R_2 - 2R_1}
      \begin{amatrix}{2}
        1 & 2     & 1 \\
        0 & a^2-9 & a-3
      \end{amatrix}
    \end{align*}
    $R_2$ represents the equation $(a^2-9)y = a-3$. If $a = 3$, there are infinitely many solutions, since $R_2$ leads to a full row of $0$'s. If $a = -3$, then there is no solution, since $R_2$ leads to $0 = -6$. If $a \neq \pm 3$, then there is exactly one solution to the system of equations.
  \end{proof}
  \qitem{32}{Reduce $\begin{bmatrix}
        2 & 1  & 3   \\
        0 & -2 & -29 \\
        3 & 4  & 5
      \end{bmatrix}$ to rref without introducing fractions at any intermediate stage.}
  \begin{proof}[work]
    \begin{align*}
      \begin{bmatrix}
        2 & 1  & 3   \\
        0 & -2 & -29 \\
        3 & 4  & 5
      \end{bmatrix} & \xrightarrow[(-2,-1,-3)]{R_3 - R_2}
      \begin{bmatrix}
        2 & 1  & 3   \\
        0 & -2 & -29 \\
        1 & 3  & 2
      \end{bmatrix} \xrightarrow[(-2,-6,-4)]{R_1 - 2R_3}
      \begin{bmatrix}
        0 & -5 & -1  \\
        0 & -2 & -29 \\
        1 & 3  & 2
      \end{bmatrix} \xrightarrow[(0,2,29)]{R_1 - R_2}     \\
      \begin{bmatrix}
        0 & -3 & 28  \\
        0 & -2 & -29 \\
        1 & 3  & 2
      \end{bmatrix} & \xrightarrow[R_3 + R_1]{R_2 - R_1}
      \begin{bmatrix}
        0 & -3 & 28  \\
        0 & 1  & -57 \\
        1 & 0  & 30
      \end{bmatrix} \xrightarrow[(0,3,-171)]{R_1 + 3R_2}
      \begin{bmatrix}
        0 & 0 & -143 \\
        0 & 1 & -57  \\
        1 & 0 & 30
      \end{bmatrix} \xrightarrow[R_2+57R_1,R_3-30R_1]{-\frac{1}{143}R_1, R_1 \leftrightarrow R_3}
      \begin{bmatrix}
        1 & 0 & 0 \\
        0 & 1 & 0 \\
        0 & 0 & 1
      \end{bmatrix}
    \end{align*}
  \end{proof}
\end{question}

\begin{question}{1.3}{}
  \qitem{5h}{Calculate $(C^TB)A^T$, where $A = \begin{pmatrix}
        3  & 0 \\
        -1 & 2 \\
        1  & 1
      \end{pmatrix},~B = \begin{pmatrix}
        4 & -1 \\
        0 & 2
      \end{pmatrix},~C = \begin{pmatrix}
        1 & 4 & 2 \\
        3 & 1 & 5
      \end{pmatrix}$.}
  \begin{proof}[work]
    \begin{align*}
       & \begin{pmatrix}
           1 & 4 & 2 \\
           3 & 1 & 5
         \end{pmatrix} \xrightarrow{C^T}
      \begin{pmatrix}
        1 & 3 \\
        4 & 1 \\
        2 & 5
      \end{pmatrix}  \xrightarrow{C^TB}
      \begin{pmatrix}
        1 \cdot 4 + 3 \cdot 0 & 1 \cdot -1 + 3 \cdot 2 \\
        4 \cdot 4 + 1 \cdot 0 & 4 \cdot -1 + 1 \cdot 2 \\
        2 \cdot 4 + 5 \cdot 0 & 2 \cdot -1 + 5 \cdot 2 \\
      \end{pmatrix} =
      \begin{pmatrix}
        4  & 5  \\
        16 & -2 \\
        8  & 8
      \end{pmatrix} \xrightarrow{C^TBA^T} \\
       & \begin{pmatrix}
           4  & 5  \\
           16 & -2 \\
           8  & 8
         \end{pmatrix}
      \begin{pmatrix}
        3 & -1 & 1 \\
        0 & 2  & 1
      \end{pmatrix}  \xrightarrow{C^TBA^T}
      \begin{pmatrix}
        4 \cdot 3 + 5 \cdot 0  & 4 \cdot -1 + 5 \cdot 2  & 4 \cdot 1 + 5 \cdot 1  \\
        16 \cdot 3 - 2 \cdot 0 & 16 \cdot -1 - 2 \cdot 2 & 16 \cdot 1 - 2 \cdot 1 \\
        8 \cdot 3 + 8 \cdot 0  & 8 \cdot -1 + 8 \cdot 2  & 8 \cdot 1 + 8 \cdot 1
      \end{pmatrix} =
      \begin{pmatrix}
        12 & 6   & 9  \\
        48 & -20 & 14 \\
        24 & 8   & 16
      \end{pmatrix}
    \end{align*}
  \end{proof}
  \qitem{10}{$A = \begin{bmatrix}
        3 & -2 & 7 \\
        6 & 5  & 4 \\
        0 & 4  & 9
      \end{bmatrix},~B = \begin{bmatrix}
        6 & -2 & 4 \\
        0 & 1  & 3 \\
        7 & 7  & 5
      \end{bmatrix},~AB = \begin{bmatrix}
        67 & 41 & 41 \\
        64 & 21 & 59 \\
        63 & 67 & 57 \\
      \end{bmatrix}, \tand BA = \begin{bmatrix}
        6  & -6 & 70  \\
        6  & 17 & 31  \\
        63 & 41 & 122
      \end{bmatrix}$.
    \begin{enumerate}
      \qitem{a}{express each column vector of $AB$ as a linear combination of the column vectors of $A$.}
      \begin{proof}[work] Since each column vector of $AB$ is computed using a row of $A$ and a column of $B$, the linear combination will simply be the corresponding column of $B$.
        \begin{enumerate}
          \item[1.] $6\begin{pmatrix}3 \\ 6 \\ 0\end{pmatrix} + 0\begin{pmatrix}-2 \\ 5 \\ 4\end{pmatrix} + 7\begin{pmatrix}7 \\ 4 \\ 9\end{pmatrix} = \begin{pmatrix}67 \\ 64 \\ 63\end{pmatrix}$.
          \item[2.] $-2\begin{pmatrix}3 \\ 6 \\ 0\end{pmatrix} + 1\begin{pmatrix}-2 \\ 5 \\ 4\end{pmatrix} + 7\begin{pmatrix}7 \\ 4 \\ 9\end{pmatrix} = \begin{pmatrix}41 \\ 21 \\ 67\end{pmatrix}$.
          \item[3.] $4\begin{pmatrix}3 \\ 6 \\ 0\end{pmatrix} + 3\begin{pmatrix}-2 \\ 5 \\ 4\end{pmatrix} + 5\begin{pmatrix}7 \\ 4 \\ 9\end{pmatrix} = \begin{pmatrix}41 \\ 59 \\ 57\end{pmatrix}$.
        \end{enumerate}
      \end{proof}
      \qitem{b}{express each column vector of $BA$ as a linear combination of the column vectors of $B$.}
      \begin{proof}[work] Since each column vector of $BA$ is computed using a row of $B$ and a column of $A$, the linear combination will simply be the corresponding column of $A$.
        \begin{enumerate}
          \item[1.] $3\begin{pmatrix}6 \\ 0 \\ 7\end{pmatrix} + 6\begin{pmatrix}-2 \\ 1 \\ 7\end{pmatrix} + 0\begin{pmatrix}4 \\ 3 \\ 5\end{pmatrix} = \begin{pmatrix}6 \\ 6 \\ 63\end{pmatrix}$.
          \item[2.] $-2\begin{pmatrix}6 \\ 0 \\ 7\end{pmatrix} + 5\begin{pmatrix}-2 \\ 1 \\ 7\end{pmatrix} + 4\begin{pmatrix}4 \\ 3 \\ 5\end{pmatrix} = \begin{pmatrix}-6 \\ 17 \\ 41\end{pmatrix}$.
          \item[3.] $7\begin{pmatrix}6 \\ 0 \\ 7\end{pmatrix} + 4\begin{pmatrix}-2 \\ 1 \\ 7\end{pmatrix} + 9\begin{pmatrix}4 \\ 3 \\ 5\end{pmatrix} = \begin{pmatrix}70 \\ 31 \\ 122\end{pmatrix}$.
        \end{enumerate}
      \end{proof}
    \end{enumerate}
  }
  \qitem{15}{Find all values of $k$, if any, that satisfy $\begin{bmatrix}k & 1 & 1\end{bmatrix}\begin{bmatrix}1 & 1 & 0 \\ 1 & 0 & 2 \\ 0 & 2 & -3\end{bmatrix}\begin{bmatrix}k \\ 1 \\ 1\end{bmatrix}$.}
  \begin{proof}[work]
    \begin{align*}
      \begin{bmatrix}
        k & 1 & 1
      \end{bmatrix}
      \begin{bmatrix}
        1 & 1 & 0  \\
        1 & 0 & 2  \\
        0 & 2 & -3
      \end{bmatrix}
      \begin{bmatrix}
        k \\ 1 \\ 1
      \end{bmatrix} =
      \begin{bmatrix}
        k+1 & k+2 & -1
      \end{bmatrix}
      \begin{bmatrix}
        k \\ 1 \\ 1
      \end{bmatrix} =
      \left[k(k + 1) + k + 2 - 1\right] = \left[(k+1)^2\right] & = \left[0\right] \\
      k + 1                                                    & = 0              \\
      k                                                        & = -1
    \end{align*}
  \end{proof}
  \qitem{22}{
    \begin{enumerate}
      \qitem{a}{Show that if $A$ has a row of zeros and $B$ is any matrix for which $AB$ is defined, then $AB$ also has a row of zeros.}
      \begin{proof}
        Consider
        \[
          A_{n \times m} = \begin{bmatrix}
            a_{11} & a_{12} & \cdots & a_{1m} \\
            a_{21} & a_{22} & \cdots & a_{2m} \\
            \vdots & \vdots & \ddots & \vdots \\
            0      & 0      & \cdots & 0      \\
            \vdots & \vdots & \ddots & \vdots \\
            a_{n1} & a_{n2} & \cdots & a_{nm}
          \end{bmatrix}
          B_{m \times k} = \begin{bmatrix}
            b_{11} & b_{12} & \cdots & b_{1k} \\
            b_{21} & b_{22} & \cdots & b_{2k} \\
            \vdots & \vdots & \ddots & \vdots \\
            b_{m1} & b_{m2} & \cdots & b_{mk}
          \end{bmatrix}.
        \]
        $AB$ can be expressed as a series of row vectors. These row vectors can be expressed as linear combinations from numbers of the row vectors from $A$, and the corresponding row of $B$. Here is an example:
        \begin{align}
          a_{11}\begin{bmatrix} b_{11} & b_{12} & \cdots & b_{1k}\end{bmatrix} +
          a_{12}\begin{bmatrix} b_{21} & b_{22} & \cdots & b_{2k}\end{bmatrix} + \cdots +
          a_{1m}\begin{bmatrix} b_{m1} & b_{m2} & \cdots & b_{mk}\end{bmatrix} & = \\
          = \begin{bmatrix}ab_{11} & ab_{11} & \cdots & ab_{1k} \end{bmatrix}  &
        \end{align}
        Since there is a row of zeros, one of these equations will be:
        \[
          0\begin{bmatrix} b_{11} & b_{12} & \cdots & b_{1k}\end{bmatrix} +
          0\begin{bmatrix} b_{21} & b_{22} & \cdots & b_{2k}\end{bmatrix} + \cdots +
          0\begin{bmatrix} b_{m1} & b_{m2} & \cdots & b_{mk}\end{bmatrix} =
          \begin{bmatrix} 0 & 0 & \cdots & 0 \end{bmatrix}.
        \]
        This represents a row of all zeros, which is contained within $AB$, at the same row at which it occurs in $A$. Therefore, if $A$ has a row of zeros and $B$ is any matrix for which $AB$ is defined, then $AB$ also has a row of zeros.
      \end{proof}
      \qitem{b}{Find a similar result involving a column of zeros.}
      \begin{proof}
        Consider
        \[
          A_{n \times m} = \begin{bmatrix}
            a_{11} & a_{12} & \cdots & a_{1m} \\
            a_{21} & a_{12} & \cdots & a_{2m} \\
            \vdots & \vdots & \ddots & \vdots \\
            a_{n1} & a_{n2} & \cdots & a_{nm}
          \end{bmatrix}
          B_{m \times k} = \begin{bmatrix}
            b_{11} & b_{12} & \cdots & 0      & \cdots & b_{1k} \\
            b_{21} & b_{22} & \cdots & 0      & \cdots & b_{2k} \\
            \vdots & \vdots & \ddots & \vdots & \ddots & \vdots \\
            b_{m1} & b_{m2} & \cdots & 0      & \cdots & b_{mk}
          \end{bmatrix}.
        \]
        $AB$ can be expressed as a series of column vectors. These column vectors can be expressed as linear combinations from numbers of the column vectors from $B$, and the corresponding column of $A$. Here is an example:
        \[
          b_{11}\begin{bmatrix} a_{11} \\ a_{21} \\ \vdots \\ a_{1m}\end{bmatrix} +
          b_{21}\begin{bmatrix} a_{12} \\ a_{22} \\ \vdots \\ a_{2m}\end{bmatrix} + \cdots +
          b_{m1}\begin{bmatrix} a_{1m} \\ a_{2m} \\ \vdots \\ a_{nm}\end{bmatrix} =
          \begin{bmatrix}ab_{11} \\ ab_{21} \\ \vdots \\ ab_{n1} \end{bmatrix}.
        \]
        Since there is a row of zeros, one of these equations will be:
        \[
          0\begin{bmatrix} a_{11} \\ a_{21} \\ \vdots \\ a_{1m}\end{bmatrix} +
          0\begin{bmatrix} a_{12} \\ a_{22} \\ \vdots \\ a_{2m}\end{bmatrix} + \cdots +
          0\begin{bmatrix} a_{1m} \\ a_{2m} \\ \vdots \\ a_{nm}\end{bmatrix} =
          \begin{bmatrix}0 \\ 0 \\ \vdots \\ 0 \end{bmatrix}.
        \]
        This represents a column of all zeros, which is contained within $AB$, at the same column at which it occurs in $A$. Therefore, if $B$ has a column of zeros and $A$ is any matrix for which $AB$ is defined, then $AB$ also has a column of zeros.
      \end{proof}
    \end{enumerate}}
  \qitem{24}{Find the $4 \times 4$ matrix $A = [a_{ij}]$ whose entries satisfy the stated condition.
  \begin{enumerate}
    \qitem{a}{$a_{ij}=i+j$}
    \begin{proof}[work]
      \[
        \begin{bmatrix}
          2 & 3 & 4 & 5 \\
          3 & 4 & 5 & 6 \\
          4 & 5 & 6 & 7 \\
          5 & 6 & 7 & 8
        \end{bmatrix}
      \]
    \end{proof}
    \qitem{b}{$a_{ij}=i^{j-1}$}
    \begin{proof}[work]
      \[
        \begin{bmatrix}
          1 & 1 & 1  & 1  \\
          1 & 2 & 4  & 8  \\
          1 & 3 & 9  & 27 \\
          1 & 4 & 16 & 64 \\
        \end{bmatrix}
      \]
    \end{proof}
    \qitem{c}{$a_{ij}=\left\{
        \begin{array}{rl}
          1  & \tif \left\lvert i-j\right\rvert > 1    \\
          -1 & \tif \left\lvert i-j\right\rvert \leq 1
        \end{array}\right.$}
    \begin{proof}[work]
      \[
        \begin{bmatrix}
          -1 & -1 & 1  & 1  \\
          -1 & -1 & -1 & 1  \\
          1  & -1 & -1 & -1 \\
          1  & 1  & -1 & -1 \\
        \end{bmatrix}
      \]
    \end{proof}
  \end{enumerate}}
  \qitem{27}{How many $3 \times 3$ matrices $A$ can you find such that
    \[
      A\begin{bmatrix}
        x \\ y \\ z
      \end{bmatrix} =
      \begin{bmatrix}
        x+y \\
        x-y \\
        0
      \end{bmatrix}
    \]
    for all choices of $x, y, \tand z$?}
\end{question}

\begin{question}{1.4}{}
  \qitem{17}{Use the given information to find $A$: $(I + 2A)^{-1}=\begin{bmatrix}-1 & 2 \\ 4 & 5\end{bmatrix}$.}
  \begin{proof}[work]
    \begin{align*}
      \begin{bmatrix}
        -1 & 2 \\
        4  & 5
      \end{bmatrix} \xrightarrow[I + 2A]{{(I + 2A)^{-1}}^{-1}}
      \frac{1}{-13}
      \begin{bmatrix}
        5  & -2 \\
        -4 & -1
      \end{bmatrix} =
      \begin{bmatrix}
        -\frac{5}{13} & \frac{2}{13} \\
        \frac{4}{13}  & \frac{1}{13}
      \end{bmatrix} \xrightarrow[2A]{I + 2A - I}
      \begin{bmatrix}
        -\frac{18}{13} & \frac{2}{13}   \\
        \frac{4}{13}   & -\frac{12}{13}
      \end{bmatrix} \xrightarrow[A]{\frac{1}{2} \cdot 2A}
      \begin{bmatrix}
        -\frac{9}{13} & \frac{1}{13}  \\
        \frac{2}{13}  & -\frac{6}{13}
      \end{bmatrix}
    \end{align*}
  \end{proof}
  \qitem{19e}{Given $A = \begin{bmatrix} 3 & 1 \\ 2 & 1\end{bmatrix}$, compute $p(A), \twhere p(x) = 2x^2-x+1$.}
  \begin{proof}[work]
    \begin{align*}
      p(A) = 2A^2 - A + I & = 2\begin{bmatrix}
                                 3 & 1 \\
                                 2 & 1
                               \end{bmatrix}^2 +
      \begin{bmatrix}
        -3 & -1 \\
        -2 & -1
      \end{bmatrix} +
      \begin{bmatrix}
        1 & 0 \\
        0 & 1
      \end{bmatrix}=
      2\begin{bmatrix}
         11 & 4 \\
         8  & 3
       \end{bmatrix} +
      \begin{bmatrix}
        -2 & -1 \\
        -2 & 0
      \end{bmatrix}                             \\
                          & =
      \begin{bmatrix}
        22 & 8 \\
        16 & 6
      \end{bmatrix} +
      \begin{bmatrix}
        -2 & -1 \\
        -2 & 0
      \end{bmatrix} =
      \begin{bmatrix}
        20 & 7 \\
        14 & 6
      \end{bmatrix}
    \end{align*}
  \end{proof}
  \qitem{27}{Consider the matrix
    \[
      A = \begin{bmatrix}
        a_{11} & 0      & \cdots & 0      \\
        0      & a_{22} & \cdots & 0      \\
        \vdots & \vdots & \ddots & \vdots \\
        0      & 0      & \cdots & a_{nn}
      \end{bmatrix}
    \]
    where $a_{11} \cdot a_{22} \cdots a_{nn} \neq 0$. Show that $A$ is invertible and find its inverse.}
  \begin{proof}
    Consider $B = \begin{bmatrix}
        \frac{1}{a_{11}} & 0                & \cdots & 0                \\
        0                & \frac{1}{a_{22}} & \cdots & 0                \\
        \vdots           & \vdots           & \ddots & \vdots           \\
        0                & 0                & \cdots & \frac{1}{a_{nn}}
      \end{bmatrix}$.
    \begin{align*}
      AB & = \begin{bmatrix}
               a_{11} & 0      & \cdots & 0      \\
               0      & a_{22} & \cdots & 0      \\
               \vdots & \vdots & \ddots & \vdots \\
               0      & 0      & \cdots & a_{nn}
             \end{bmatrix}\begin{bmatrix}
                            \frac{1}{a_{11}} & 0                & \cdots & 0                \\
                            0                & \frac{1}{a_{22}} & \cdots & 0                \\
                            \vdots           & \vdots           & \ddots & \vdots           \\
                            0                & 0                & \cdots & \frac{1}{a_{nn}}
                          \end{bmatrix} = \begin{bmatrix}
                                            1      & 0      & \cdots & 0      \\
                                            0      & 1      & \cdots & 0      \\
                                            \vdots & \vdots & \ddots & \vdots \\
                                            0      & 0      & \cdots & 1
                                          \end{bmatrix} = I \\
      BA & = \begin{bmatrix}
               \frac{1}{a_{11}} & 0                & \cdots & 0                \\
               0                & \frac{1}{a_{22}} & \cdots & 0                \\
               \vdots           & \vdots           & \ddots & \vdots           \\
               0                & 0                & \cdots & \frac{1}{a_{nn}}
             \end{bmatrix}\begin{bmatrix}
                            a_{11} & 0      & \cdots & 0      \\
                            0      & a_{22} & \cdots & 0      \\
                            \vdots & \vdots & \ddots & \vdots \\
                            0      & 0      & \cdots & a_{nn}
                          \end{bmatrix} = \begin{bmatrix}
                                            1      & 0      & \cdots & 0      \\
                                            0      & 1      & \cdots & 0      \\
                                            \vdots & \vdots & \ddots & \vdots \\
                                            0      & 0      & \cdots & 1
                                          \end{bmatrix} = I
    \end{align*}
    Since $AB = I = BA$, therefore $B = A^{-1}$, the inverse of $A$. This also proves that $A$ is invertible.
  \end{proof}
\end{question}

\end{document}