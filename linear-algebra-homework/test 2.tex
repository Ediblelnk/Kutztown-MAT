\documentclass{article}
\documentclass{article}
\usepackage[margin=1in]{geometry}
\usepackage{amsmath, amsthm, amssymb, fancyhdr, tikz, circuitikz, graphicx}
\usepackage{centernot, xcolor, hhline, multirow, listings}
\usepackage{blkarray, booktabs, bigstrut, etoolbox}
\usepackage[normalem]{ulem}
\usepackage{bookmark}
\usetikzlibrary{math}
\usetikzlibrary{fit}

\pagestyle{fancy}

\usepackage{hyperref}
\hypersetup{
  colorlinks=true,
  linkcolor=black,
  filecolor=magenta,
  urlcolor=cyan,
}
%formatting
\newcommand{\bld}{\textbf}
\newcommand{\itl}{\textit}
\newcommand{\uln}{\underline}

%math word symbols
\newcommand{\bb}{\mathbb}
\DeclareMathOperator{\tif}{~\text{if}~}
\DeclareMathOperator{\tand}{~\text{and}~}
\DeclareMathOperator{\tbut}{~\text{but}~}
\DeclareMathOperator{\tor}{~\text{or}~}
\DeclareMathOperator{\tsuchthat}{~\text{such that}~}
\DeclareMathOperator{\tsince}{~\text{since}~}
\DeclareMathOperator{\twhen}{~\text{when}~}
\DeclareMathOperator{\twhere}{~\text{where}~}
\DeclareMathOperator{\tfor}{~\text{for}~}
\DeclareMathOperator{\tthen}{~\text{then}~}
\DeclareMathOperator{\tto}{~\text{to}~}

%display shortcut
\DeclareMathOperator{\dstyle}{\displaystyle}
\DeclareMathOperator{\sstyle}{\scriptstyle}

%linear algebra
\DeclareMathOperator{\id}{\bld{id}}
\DeclareMathOperator{\vecspan}{\text{span}}

%discrete math - integer properties
\DeclareMathOperator{\tdiv}{\text{div}}
\DeclareMathOperator{\tmod}{\text{mod}}
\DeclareMathOperator{\lcm}{\text{lcm}}

%augmented matrix environment
\newenvironment{apmatrix}[2]{%
  \left(\begin{array}{@{~}*{#1}{c}|@{~}*{#2}{c}}
    }{
  \end{array}\right)
}
\newenvironment{abmatrix}[2]{%
  \left[\begin{array}{@{~}*{#1}{c}|@{~}*{#2}{c}}
      }{
    \end{array}\right]
}

%lists
\newcommand{\bitem}[1]{\item[\bld{#1.}]}
\newcommand{\bbitem}[2]{\item[\bld{#1.}] \bld{#2}}
\newcommand{\biitem}[2]{\item[\bld{#1.}] \itl{#2}}
\newcommand{\iitem}[1]{\item[\itl{#1.}]}
\newcommand{\iiitem}[2]{\item[\itl{#1.}] \bld{#2}}
\newcommand{\btitem}[2]{\item[\bld{#1.}] \texttt{#2}}

%homework
\newcommand{\question}[2]{\noindent {\large\bld{#1}} #2 \qline}
\newcommand{\qitem}[3]{\item[\bld{#1.}] \itl{#2} #3 \qdash}

\newcommand{\qline}{~\newline\noindent\textcolor[RGB]{200,200,200}{\rule[0.5ex]{\linewidth}{0.2pt}}}
\newcommand{\qdash}{~\newline\noindent\textcolor[RGB]{200,200,200}{\hdashrule[0.5ex]{\linewidth}{0.2pt}{2pt}}}

\newcommand{\assignment}{Test 2}

\lhead{Linear Algebra}
\chead{\assignment}
\rhead{Peter Schaefer}

\begin{document}
\section*{\assignment}

\question{Problem 1}{Let $A$ be an $n \times n$ matrix such that $A^4 + A^2 = O$. Show that
  \[
    (A^2-A+I)^{-1} = A^2 + A + I
  \]}
\begin{proof}
  Consider $A^2 + A + I$.
  \begin{align*}
    (A^2 + A + I)(A^2 - A + I) & = A^4 - A^3 + A^2 + A^3 - A^2 + A + A^2 - A + I^2 = A^4 + A^2 + I \\
    (A^2 - A + I)(A^2 + A + I) & = A^4 + A^3 + A^2 - A^3 - A^2 - A + A^2 + A + I^2 = A^4 + A^2 + I
  \end{align*}
  If $A^4 + A^2 = O$, we can simplify further:
  \begin{align*}
    A^4 + A^2 + I                         & = O + I &  & \tsince A^4 + A^2 = O \\
                                          & = I                                \\
    \therefore (A^2 + A + I)(A^2 - A + I) & = I                                \\
    \therefore (A^2 - A + I)(A^2 + A + I) & = I
  \end{align*}
  Since $(A^2 + A + I)(A^2 - A + I) = I \tand (A^2 - A + I)(A^2 + A + I) = I$, therefore if $A^4 + A^2 = O$, then $(A^2 - A + I)^{-1} = A^2 + A + I$.
\end{proof}
\qdash

\question{Problem 2}{In $\bb{R}^3$, let $\vec{v}_1 = (-3,1,4), \vec{v}_2 = (-4,2,5), \tand \vec{v}_3 = (-1,0,2)$. Express $\vec{u} = (5,-4,2)$ as a linear combination of $\vec{v}_1, \vec{v}_3, \tand \vec{v}_3$ by finding the MATRIX INVERSE.}
\begin{proof}[Work]
  Let $k_1,k_2,\tand k_3 \in \bb{R} \tsuchthat k_1\vec{v}_1 + k_2\vec{v}_2 + k_3\vec{v}_3 = \vec{u}$. That is, $k_1(-3,1,4) + k_2(-4,2,5) + k_3(-1,0,2) = (5,-4,2)$. This equation can be converted into a matrix equation.
  \[
    \begin{pmatrix}
      -3 & -4 & -1 \\
      1  & 2  & 0  \\
      4  & 5  & 2
    \end{pmatrix}
    \begin{pmatrix}
      k_1 \\
      k_2 \\
      k_3
    \end{pmatrix} =
    \begin{pmatrix}
      5  \\
      -4 \\
      2
    \end{pmatrix}
  \]
  In order to solve this equation, we must use the inverse algorithm.
  \begin{align*}
    \begin{apmatrix}{3}{3}
      -3 & -4 & -1 & 1 & 0 & 0 \\
      1  & 2  & 0  & 0 & 1 & 0 \\
      4  & 5  & 2  & 0 & 0 & 1
    \end{apmatrix} \xrightarrow[R_3 - 4R_2]{R_1 + 3R_2}
    \begin{apmatrix}{3}{3}
      0 & 2  & -1 & 1 & 3  & 0 \\
      1 & 2  & 0  & 0 & 1  & 0 \\
      0 & -3 & 2  & 0 & -4 & 1
    \end{apmatrix} \xrightarrow[R_2 - R_1]{R_3 + 2R_1} \\
    \begin{apmatrix}{3}{3}
      0 & 2 & -1 & 1  & 3  & 0 \\
      1 & 0 & -1 & -1 & -2 & 0 \\
      0 & 1 & 0  & 2  & 2  & 1
    \end{apmatrix} \xrightarrow{R_1 - 2R_3}
    \begin{apmatrix}{3}{3}
      0 & 0 & -1 & -3 & -1 & -2 \\
      1 & 0 & -1 & -1 & -2 & 0 \\
      0 & 1 & 0  & 2  & 2  & 1
    \end{apmatrix} \xrightarrow{R_2 + R_1}             \\
    \begin{apmatrix}{3}{3}
      0 & 0 & -1 & -3 & -1 & -2 \\
      1 & 0 & 0  & -4 & -3 & -2 \\
      0 & 1 & 0  & 2  & 2  & 1
    \end{apmatrix} \xrightarrow[R_1 \leftrightarrow R_2 \leftrightarrow R_3]{-R_1}
    \begin{apmatrix}{3}{3}
      1 & 0 & 0 & -4 & -3 & -2 \\
      0 & 1 & 0 & 2  & 2  & 1  \\
      0 & 0 & 1 & 3  & 1  & 2
    \end{apmatrix}
  \end{align*}
  Though the use of the inverse algorithm, we know that
  \[
    \begin{pmatrix}
      -3 & -4 & -1 \\
      1  & 2  & 0  \\
      4  & 5  & 2
    \end{pmatrix}^{-1} =
    \begin{pmatrix}
      -4 & -3 & -2 \\
      2  & 2  & 1  \\
      3  & 1  & 2
    \end{pmatrix}.
  \]
  We can use this identity to solve the original equation:
  \begin{align*}
    \begin{pmatrix}
      -3 & -4 & -1 \\
      1  & 2  & 0  \\
      4  & 5  & 2
    \end{pmatrix}
    \begin{pmatrix}
      k_1 \\
      k_2 \\
      k_3
    \end{pmatrix} & =
    \begin{pmatrix}
      5  \\
      -4 \\
      2
    \end{pmatrix}    \\
    \begin{pmatrix}
      -3 & -4 & -1 \\
      1  & 2  & 0  \\
      4  & 5  & 2
    \end{pmatrix}^{-1}
    \begin{pmatrix}
      -3 & -4 & -1 \\
      1  & 2  & 0  \\
      4  & 5  & 2
    \end{pmatrix}
    \begin{pmatrix}
      k_1 \\
      k_2 \\
      k_3
    \end{pmatrix} & =
    \begin{pmatrix}
      -3 & -4 & -1 \\
      1  & 2  & 0  \\
      4  & 5  & 2
    \end{pmatrix}^{-1}
    \begin{pmatrix}
      5  \\
      -4 \\
      2
    \end{pmatrix}    \\
    \begin{pmatrix}
      k_1 \\
      k_2 \\
      k_3
    \end{pmatrix} & =
    \begin{pmatrix}
      -4 & -3 & -2 \\
      2  & 2  & 1  \\
      3  & 1  & 2
    \end{pmatrix}
    \begin{pmatrix}
      5  \\
      -4 \\
      2
    \end{pmatrix}    \\
    \begin{pmatrix}
      k_1 \\
      k_2 \\
      k_3
    \end{pmatrix} & =
    \begin{pmatrix}
      -20 + 12 - 4 \\
      10 - 8 + 2   \\
      15 - 4 + 4   \\
    \end{pmatrix}    \\
    \begin{pmatrix}
      k_1 \\
      k_2 \\
      k_3
    \end{pmatrix} & =
    \begin{pmatrix}
      -12 \\
      4   \\
      15  \\
    \end{pmatrix}
  \end{align*}
  This means that when $k_1 = -12,~k_2 = 4, \tand k_3 = 15$, the equation $k_1\vec{v}_1 + k_2\vec{v}_2 + k_3\vec{v}_3 = \vec{u}$ is true. That is, $-12(-3,1,4) + 4(-4,2,5) + 15(-1,0,2) = (5,-4,2)$.
\end{proof}
\qdash

\question{Problem 3}{Solve for the matrix $A$ if
  \[
    (I - 2A)^{-1} = \begin{pmatrix}
      1  & -3 & 3  \\
      -2 & 2  & -5 \\
      3  & -8 & 9
    \end{pmatrix}.
  \]}
\begin{proof}[Work]
  \begin{align*}
    \begin{apmatrix}{3}{3}
      1  & -3 & 3  & 1 & 0 & 0 \\
      -2 & 2  & -5 & 0 & 1 & 0 \\
      3  & -8 & 9  & 0 & 0 & 1
    \end{apmatrix} \xrightarrow[R_3 - 3R_1]{R_2 + 2R_1}
    \begin{apmatrix}{3}{3}
      1 & -3 & 3 & 1  & 0 & 0 \\
      0 & -4 & 1 & 2  & 1 & 0 \\
      0 & 1  & 0 & -3 & 0 & 1
    \end{apmatrix} \xrightarrow[R_2 + 4R_3]{R_1 + 3R_3}
    \begin{apmatrix}{3}{3}
      1 & 0 & 3 & -8  & 0 & 3 \\
      0 & 0 & 1 & -10 & 1 & 4 \\
      0 & 1 & 0 & -3  & 0 & 1
    \end{apmatrix} \\
    \xrightarrow{R_1 - 3R_2}
    \begin{apmatrix}{3}{3}
      1 & 0 & 0 & 22  & -3 & -9 \\
      0 & 0 & 1 & -10 & 1  & 4  \\
      0 & 1 & 0 & -3  & 0  & 1
    \end{apmatrix} \xrightarrow{R_2 \leftrightarrow R_3}
    \begin{apmatrix}{3}{3}
      1 & 0 & 0 & 22  & -3 & -9 \\
      0 & 1 & 0 & -3  & 0  & 1  \\
      0 & 0 & 1 & -10 & 1  & 4
    \end{apmatrix}
  \end{align*}
  Therefore, through the inverse algorithm,
  \[
    \begin{pmatrix}
      1  & -3 & 3  \\
      -2 & 2  & -5 \\
      3  & -8 & 9
    \end{pmatrix}^{-1} =
    \begin{pmatrix}
      22  & -3 & -9 \\
      -3  & 0  & 1  \\
      -10 & 1  & 4
    \end{pmatrix}
  \]
  Using this fact, we can now solve the original equation.
  \begin{align*}
    (I - 2A)^{-1} & = \begin{pmatrix}
                        1  & -3 & 3  \\
                        -2 & 2  & -5 \\
                        3  & -8 & 9
                      \end{pmatrix}  & -2A & =  \begin{pmatrix}
                                                  22  & -3 & -9 \\
                                                  -3  & 0  & 1  \\
                                                  -10 & 1  & 4
                                                \end{pmatrix} - \begin{pmatrix}
                                                                  1 & 0 & 0 \\
                                                                  0 & 1 & 0 \\
                                                                  0 & 0 & 1
                                                                \end{pmatrix}               \\
    I - 2A        & = \begin{pmatrix}
                        1  & -3 & 3  \\
                        -2 & 2  & -5 \\
                        3  & -8 & 9
                      \end{pmatrix}^{-1} & -2A & = \begin{pmatrix}
                                                     21  & -3 & -9 \\
                                                     -3  & -1 & 1  \\
                                                     -10 & 1  & 3
                                                   \end{pmatrix}                            \\
    I - 2A        & = \begin{pmatrix}
                        22  & -3 & -9 \\
                        -3  & 0  & 1  \\
                        -10 & 1  & 4
                      \end{pmatrix}  & A   & = \begin{pmatrix}
                                                 -10\frac{1}{2} & 1\frac{1}{2} & 4\frac{1}{2}  \\
                                                 1\frac{1}{2}   & \frac{1}{2}  & -\frac{1}{2}  \\
                                                 5              & -\frac{1}{2} & -1\frac{1}{2}
                                               \end{pmatrix}
  \end{align*}
\end{proof}
\qdash

\question{Problem 4}{Let
  \[
    A = \begin{pmatrix}
      1  & 0 & 3  \\
      -2 & 0 & -5 \\
      0  & 2 & 0
    \end{pmatrix}.
  \]
  Write $A$ as a product of elementary matrices.}
\begin{proof}[Work]
  \begin{align*}
    \begin{pmatrix}
      1  & 0 & 3  \\
      -2 & 0 & -5 \\
      0  & 2 & 0
    \end{pmatrix} \xrightarrow{R_2 + 2R_1}
    \begin{pmatrix}
      1 & 0 & 3 \\
      0 & 0 & 1 \\
      0 & 2 & 0
    \end{pmatrix} \xrightarrow{R_1 - 3R_2}
    \begin{pmatrix}
      1 & 0 & 0 \\
      0 & 0 & 1 \\
      0 & 2 & 0
    \end{pmatrix} \xrightarrow{\frac{1}{2}R_3}
    \begin{pmatrix}
      1 & 0 & 0 \\
      0 & 0 & 1 \\
      0 & 1 & 0
    \end{pmatrix} \xrightarrow{R_2 \leftrightarrow R_3}
    \begin{pmatrix}
      1 & 0 & 0 \\
      0 & 1 & 0 \\
      0 & 0 & 1
    \end{pmatrix}
  \end{align*}
  Each of these row operations can be expressed as a left multiplication of an elementary matrix.
  \begin{align*}
    \begin{pmatrix}
      1 & 0 & 0 \\
      0 & 0 & 1 \\
      0 & 1 & 0
    \end{pmatrix}
    \begin{pmatrix}
      1 & 0 & 0           \\
      0 & 1 & 0           \\
      0 & 0 & \frac{1}{2}
    \end{pmatrix}
    \begin{pmatrix}
      1 & -3 & 0 \\
      0 & 1  & 0 \\
      0 & 0  & 1
    \end{pmatrix}
    \begin{pmatrix}
      1 & 0 & 0 \\
      2 & 1 & 0 \\
      0 & 0 & 1
    \end{pmatrix}
    \begin{pmatrix}
      1  & 0 & 3  \\
      -2 & 0 & -5 \\
      0  & 2 & 0
    \end{pmatrix} =
    \begin{pmatrix}
      1 & 0 & 0 \\
      0 & 1 & 0 \\
      0 & 0 & 1
    \end{pmatrix}     \\
    \begin{pmatrix}
      1  & 0 & 3  \\
      -2 & 0 & -5 \\
      0  & 2 & 0
    \end{pmatrix} = \left(
    \begin{pmatrix}
      1 & 0 & 0 \\
      0 & 0 & 1 \\
      0 & 1 & 0
    \end{pmatrix}
    \begin{pmatrix}
      1 & 0 & 0           \\
      0 & 1 & 0           \\
      0 & 0 & \frac{1}{2}
    \end{pmatrix}
    \begin{pmatrix}
      1 & -3 & 0 \\
      0 & 1  & 0 \\
      0 & 0  & 1
    \end{pmatrix}
    \begin{pmatrix}
      1 & 0 & 0 \\
      2 & 1 & 0 \\
      0 & 0 & 1
    \end{pmatrix}
    \right)^{-1}
    \begin{pmatrix}
      1 & 0 & 0 \\
      0 & 1 & 0 \\
      0 & 0 & 1
    \end{pmatrix}     \\
    \begin{pmatrix}
      1  & 0 & 3  \\
      -2 & 0 & -5 \\
      0  & 2 & 0
    \end{pmatrix} =
    \begin{pmatrix}
      1 & 0 & 0 \\
      2 & 1 & 0 \\
      0 & 0 & 1
    \end{pmatrix}^{-1}
    \begin{pmatrix}
      1 & -3 & 0 \\
      0 & 1  & 0 \\
      0 & 0  & 1
    \end{pmatrix}^{-1}
    \begin{pmatrix}
      1 & 0 & 0           \\
      0 & 1 & 0           \\
      0 & 0 & \frac{1}{2}
    \end{pmatrix}^{-1}
    \begin{pmatrix}
      1 & 0 & 0 \\
      0 & 0 & 1 \\
      0 & 1 & 0
    \end{pmatrix}^{-1} \\
    \begin{pmatrix}
      1  & 0 & 3  \\
      -2 & 0 & -5 \\
      0  & 2 & 0
    \end{pmatrix} =
    \begin{pmatrix}
      1  & 0 & 0 \\
      -2 & 1 & 0 \\
      0  & 0 & 1
    \end{pmatrix}
    \begin{pmatrix}
      1 & 3 & 0 \\
      0 & 1 & 0 \\
      0 & 0 & 1
    \end{pmatrix}
    \begin{pmatrix}
      1 & 0 & 0 \\
      0 & 1 & 0 \\
      0 & 0 & 2
    \end{pmatrix}
    \begin{pmatrix}
      1 & 0 & 0 \\
      0 & 0 & 1 \\
      0 & 1 & 0
    \end{pmatrix}
  \end{align*}
\end{proof}
\qdash

\question{Problem 5}{Find the conditions on $b_1, b_2, \tand b_3$ such that the system
  \begin{align*}
    1x_1 - 1x_2 + 1x_3  & = b_1 \\
    -4x_1 + 7x_2 + 2x_3 & = b_2 \\
    -2x_1 + 3x_2 + 0x_3 & = b_3
  \end{align*}
  is consistent.}
\begin{proof}[Work]
  The system of linear equations can be represented as an augmented matrix.
  \[
    \begin{apmatrix}{3}{1}
      1  & -1 & 1 & b_1 \\
      -4 & 7  & 2 & b_2 \\
      -2 & 3  & 0 & b_3
    \end{apmatrix}
  \]
  Reducing this augmented matrix will provide the constraints on $b_1,b_2,\tand b_3$ to make the system consistent.
  \begin{align*}
    \begin{apmatrix}{3}{1}
      1  & -1 & 1 & b_1 \\
      -4 & 7  & 2 & b_2 \\
      -2 & 3  & 0 & b_3
    \end{apmatrix} \xrightarrow[R_3 + 2R_1]{R_2 + 4R_1}
    \begin{pmatrix}
      1 & -1 & 1 & b_1        \\
      0 & 3  & 6 & b_2 + 4b_1 \\
      0 & 1  & 2 & b_3 + 2b_1
    \end{pmatrix} \\
    \xrightarrow{R_2 - 3R_3}
    \begin{pmatrix}
      1 & -1 & 1 & b_1                    \\
      0 & 0  & 0 & b_2 + 4b_1 -3b_3 -6b_1 \\
      0 & 1  & 2 & b_3 + 2b_1
    \end{pmatrix} \xrightarrow{R_2 \leftrightarrow R_3}
    \begin{pmatrix}
      1 & -1 & 1 & b_1               \\
      0 & 1  & 2 & b_3 + 2b_1        \\
      0 & 0  & 0 & b_2 - 2b_1 - 3b_3
    \end{pmatrix}
  \end{align*}
  The last row represents the equation $0 = b_2 - 2b_1 - 3b_3$. If $0 = b_2 - 2b_1 - 3b_3$, then the augmented matrix can be put into reduced row echelon form without any leading 1's in the final column, and the matrix will be consistent. However, if $0 \neq b_2 - 2b_1 - 3b_3$, then the last row will have a leading entry in the last column, and thus there will no solution, meaning the system is not consistent.
\end{proof}
\qdash

\question{Problem 7}{Prove that for all $n \times n$ matrices $A$, the matrix $A^TA + 2AA^T$ is symmetric.}
\begin{proof}
  Consider square matrix $A$, and the matrix $A^TA + 2AA^T$. Now consider the transpose of this matrix, $(A^TA + 2AA^T)^T$.
  \begin{align*}
    (A^TA + 2AA^T)^T & = (A^TA)^T + (2AA^T)^T     &  & \text{by properties of transpose} \\
                     & = A^T(A^T)^T + 2(A^T)^TA^T &  & \text{by properties of transpose} \\
    (A^TA + 2AA^T)^T & = A^TA + 2AA^T
  \end{align*}
  Since $(A^TA + 2AA^T)^T = A^TA + 2AA^T$, by definition, $A^TA + 2AA^T$ is a symmetric matrix, for all $A_{n \times n}$.
\end{proof}
\qdash

\end{document}