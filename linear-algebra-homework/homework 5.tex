\documentclass{article}
\usepackage[margin=1in]{geometry}
\usepackage{amsmath, amsthm, amssymb, fancyhdr, tikz, circuitikz, graphicx}
\usepackage{centernot, xcolor, hhline, multirow, listings}
\usepackage{blkarray, booktabs, bigstrut, etoolbox}
\usepackage[normalem]{ulem}
\usepackage{bookmark}
\usetikzlibrary{math}
\usetikzlibrary{fit}

\pagestyle{fancy}

\usepackage{hyperref}
\hypersetup{
  colorlinks=true,
  linkcolor=black,
  filecolor=magenta,
  urlcolor=cyan,
}
%formatting
\newcommand{\bld}{\textbf}
\newcommand{\itl}{\textit}
\newcommand{\uln}{\underline}

%math word symbols
\newcommand{\bb}{\mathbb}
\DeclareMathOperator{\tif}{~\text{if}~}
\DeclareMathOperator{\tand}{~\text{and}~}
\DeclareMathOperator{\tbut}{~\text{but}~}
\DeclareMathOperator{\tor}{~\text{or}~}
\DeclareMathOperator{\tsuchthat}{~\text{such that}~}
\DeclareMathOperator{\tsince}{~\text{since}~}
\DeclareMathOperator{\twhen}{~\text{when}~}
\DeclareMathOperator{\twhere}{~\text{where}~}
\DeclareMathOperator{\tfor}{~\text{for}~}
\DeclareMathOperator{\tthen}{~\text{then}~}
\DeclareMathOperator{\tto}{~\text{to}~}

%display shortcut
\DeclareMathOperator{\dstyle}{\displaystyle}
\DeclareMathOperator{\sstyle}{\scriptstyle}

%linear algebra
\DeclareMathOperator{\id}{\bld{id}}
\DeclareMathOperator{\vecspan}{\text{span}}

%discrete math - integer properties
\DeclareMathOperator{\tdiv}{\text{div}}
\DeclareMathOperator{\tmod}{\text{mod}}
\DeclareMathOperator{\lcm}{\text{lcm}}

%augmented matrix environment
\newenvironment{apmatrix}[2]{%
  \left(\begin{array}{@{~}*{#1}{c}|@{~}*{#2}{c}}
    }{
  \end{array}\right)
}
\newenvironment{abmatrix}[2]{%
  \left[\begin{array}{@{~}*{#1}{c}|@{~}*{#2}{c}}
      }{
    \end{array}\right]
}

%lists
\newcommand{\bitem}[1]{\item[\bld{#1.}]}
\newcommand{\bbitem}[2]{\item[\bld{#1.}] \bld{#2}}
\newcommand{\biitem}[2]{\item[\bld{#1.}] \itl{#2}}
\newcommand{\iitem}[1]{\item[\itl{#1.}]}
\newcommand{\iiitem}[2]{\item[\itl{#1.}] \bld{#2}}
\newcommand{\btitem}[2]{\item[\bld{#1.}] \texttt{#2}}

%homework
\newcommand{\question}[2]{\noindent {\large\bld{#1}} #2 \qline}
\newcommand{\qitem}[3]{\item[\bld{#1.}] \itl{#2} #3 \qdash}

\newcommand{\qline}{~\newline\noindent\textcolor[RGB]{200,200,200}{\rule[0.5ex]{\linewidth}{0.2pt}}}
\newcommand{\qdash}{~\newline\noindent\textcolor[RGB]{200,200,200}{\hdashrule[0.5ex]{\linewidth}{0.2pt}{2pt}}}

\lhead{Linear Algebra}
\chead{Homework 5}
\rhead{Peter Schaefer}

\begin{document}

\section{Homework 5}

\subsection*{Section 4.2}

\subsection*{7}
\itl{Which of the following are linear combinations of $\vec{u} = (0,-2,2) \tand \vec{v} = (1,3,-1)$?}
\begin{enumerate}
  \biitem{a}{$(2,2,2)$}
  \begin{proof}
    Let $k_1,k_2 \in \bb{R} \tsuchthat k_1\vec{u} + k_2\vec{v} = (2,2,2)$. That is, $k_1(0,-2,2) + k_2(1,3,-1) = (2,2,2)$. From this equation, we get a linear system of equations.
    \begin{align*}
      0k_1 + 1k_2  & = 2 \\
      -2k_1 + 3k_2 & = 2 \\
      2k_1 - 1k_2  & = 2
    \end{align*}
    \[
      \begin{amatrix}{2}
        0  & 1  & 2 \\
        -2 & 3  & 2 \\
        2  & -1 & 2
      \end{amatrix} \overset{-\frac{1}{2}R_2}{\longrightarrow}
      \begin{amatrix}{2}
        0 & 1 & 2 \\
        1 & -\frac{3}{2} & -1 \\
        2 & -1 & 2
      \end{amatrix} \overset{R_1 \leftrightarrow R_2}{\longrightarrow}
      \begin{amatrix}{2}
        1 & -\frac{3}{2} & -1 \\
        0 & 1 & 2 \\
        2 & -1 & 2
      \end{amatrix} \overunderset{R_3 - 2R_1}{(-2, 3, 2)}{\longrightarrow}
      \begin{amatrix}{2}
        1 & -\frac{3}{2} & -1 \\
        0 & 1 & 2 \\
        0 & 2 & 4
      \end{amatrix}
    \]
    \[
      \overunderset{R_3 - 2R_2}{(0,-2,-4)}{\longrightarrow}
      \begin{amatrix}{2}
        1 & -\frac{3}{2} & -1 \\
        0 & 1 & 2 \\
        0 & 0 & 0
      \end{amatrix} \overunderset{R_1 + \frac{3}{2}R_2}{(0,\frac{3}{2},3)}{\longrightarrow}
      \begin{amatrix}{2}
        1 & 0 & 2 \\
        0 & 1 & 2 \\
        0 & 0 & 0
      \end{amatrix}
    \]
    This augmented matrix represents the following equations:
    \begin{align*}
      k_1 + 0k_2 & = 2 & k_1 & = 2 \\
      0k_1 + k_2 & = 2 & k_2 & = 2 \\
      0 + 0      & = 0
    \end{align*}
    This means that $(2,2,2)$ is a linear combination of $\{\vec{u},\vec{v}\}$, when $k_1 = 2 \tand k_2 = 2$.
  \end{proof}
  \biitem{c}{$(0,4,5)$}
\end{enumerate}

\subsection*{8}
\begin{enumerate}
  \biitem{a}{description}
  \biitem{c}{description}
\end{enumerate}

\subsection*{9}
\begin{enumerate}
  \biitem{a}{description}
  \biitem{c}{description}
\end{enumerate}

\subsection*{10}
\begin{enumerate}
  \biitem{a}{description}
  \biitem{d}{description}
\end{enumerate}

\subsection*{Section 1.2}
\subsubsection*{4}

\end{document}