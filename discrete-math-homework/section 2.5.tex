\documentclass{article}
\usepackage[margin=1in]{geometry}
\usepackage{amsmath, amsthm, amssymb, fancyhdr, tikz, circuitikz, graphicx}
\usepackage{centernot, xcolor, hhline, multirow, listings}
\usepackage{blkarray, booktabs, bigstrut, etoolbox}
\usepackage[normalem]{ulem}
\usepackage{bookmark}
\usetikzlibrary{math}
\usetikzlibrary{fit}

\pagestyle{fancy}

\usepackage{hyperref}
\hypersetup{
  colorlinks=true,
  linkcolor=black,
  filecolor=magenta,
  urlcolor=cyan,
}
%formatting
\newcommand{\bld}{\textbf}
\newcommand{\itl}{\textit}
\newcommand{\uln}{\underline}

%math word symbols
\newcommand{\bb}{\mathbb}
\DeclareMathOperator{\tif}{~\text{if}~}
\DeclareMathOperator{\tand}{~\text{and}~}
\DeclareMathOperator{\tbut}{~\text{but}~}
\DeclareMathOperator{\tor}{~\text{or}~}
\DeclareMathOperator{\tsuchthat}{~\text{such that}~}
\DeclareMathOperator{\tsince}{~\text{since}~}
\DeclareMathOperator{\twhen}{~\text{when}~}
\DeclareMathOperator{\twhere}{~\text{where}~}
\DeclareMathOperator{\tfor}{~\text{for}~}
\DeclareMathOperator{\tthen}{~\text{then}~}
\DeclareMathOperator{\tto}{~\text{to}~}

%display shortcut
\DeclareMathOperator{\dstyle}{\displaystyle}
\DeclareMathOperator{\sstyle}{\scriptstyle}

%linear algebra
\DeclareMathOperator{\id}{\bld{id}}
\DeclareMathOperator{\vecspan}{\text{span}}

%discrete math - integer properties
\DeclareMathOperator{\tdiv}{\text{div}}
\DeclareMathOperator{\tmod}{\text{mod}}
\DeclareMathOperator{\lcm}{\text{lcm}}

%augmented matrix environment
\newenvironment{apmatrix}[2]{%
  \left(\begin{array}{@{~}*{#1}{c}|@{~}*{#2}{c}}
    }{
  \end{array}\right)
}
\newenvironment{abmatrix}[2]{%
  \left[\begin{array}{@{~}*{#1}{c}|@{~}*{#2}{c}}
      }{
    \end{array}\right]
}

%lists
\newcommand{\bitem}[1]{\item[\bld{#1.}]}
\newcommand{\bbitem}[2]{\item[\bld{#1.}] \bld{#2}}
\newcommand{\biitem}[2]{\item[\bld{#1.}] \itl{#2}}
\newcommand{\iitem}[1]{\item[\itl{#1.}]}
\newcommand{\iiitem}[2]{\item[\itl{#1.}] \bld{#2}}
\newcommand{\btitem}[2]{\item[\bld{#1.}] \texttt{#2}}

%homework
\newcommand{\question}[2]{\noindent {\large\bld{#1}} #2 \qline}
\newcommand{\qitem}[3]{\item[\bld{#1.}] \itl{#2} #3 \qdash}

\newcommand{\qline}{~\newline\noindent\textcolor[RGB]{200,200,200}{\rule[0.5ex]{\linewidth}{0.2pt}}}
\newcommand{\qdash}{~\newline\noindent\textcolor[RGB]{200,200,200}{\hdashrule[0.5ex]{\linewidth}{0.2pt}{2pt}}}

\lhead{Discrete Math II}
\chead{Section 2.4 Homework}
\rhead{Peter Schaefer}

\begin{document}

\subsection*{2.5.1}
\begin{enumerate}
  \bitem{b} For every integer $n$, if $n^3$ is even, then $n$ is even.
  \begin{proof}
    Assume that $n$ is odd. That is $n = 2k+1$, for some integer $k$.
    \begin{align*}
      n^3 = (2k+1)^3 & = (4k^2+4k+1)(2k+1)                \\
                     & = 8k^3 + 8k^2 + 2k + 4k^2 + 4k + 1 \\
                     & = 8k^3 + 12k^2 + 4k + 1            \\
                     & = 2(4k^3+6k^2+2k) + 1              \\
                     & = 2j+1 \tfor j = 4k^3+6k^2+2k
    \end{align*}
    Therefore $n^3$ takes the form of an odd number.
    Therefore, through contrapositive, for every integer $n$, if $n^3$ is even,
    then $n$ is even.
  \end{proof}
  \bitem{c} For every integer $n$, if $5n+3$ is even, then $n$ is odd.
  \begin{proof}
    Assume that $n$ is even. That is $n = 2k$, for some integer $k$.
    \begin{align*}
      5n+3 = 5(2k) + 3 & = 10k + 3             \\
                       & = 2(5k+1) + 1         \\
                       & = 2j+1 \tfor j = 5k+1
    \end{align*}
    Therefore $5n+3$ takes the form of an odd number.
    Therefore, through contrapositive, for every integer $n$, if $5n+3$ is even,
    then $n$ is odd.
  \end{proof}
\end{enumerate}

\subsection*{2.5.2}
\begin{enumerate}
  \bitem{a} If $x$ and $y$ are integers such that $3 \nmid xy$, then $3 \nmid x$.
  \begin{proof}
    Assume that $3 \nmid x$. That is, $x = 3k$, for some integer $k$.
    Let $y$ be an integer.
    \begin{align*}
      x         & = 3k              \\
      xy        & = 3ky             \\ \\
      3 \mid xy & \equiv 3 \mid 3ky
    \end{align*}
    Since $3$ divides $3$. Therefore, through contrapositive,
    if $x$ and $y$ are integer such that $3 \nmid xy$, then $3 \nmid x$
  \end{proof}
  \bitem{b} For any integers $x,y,\tand z$ if $x \mid y$ and $x \nmid z$, then $x \nmid (y+z)$.
  \begin{proof}
    Let $x \mid (y+z)$. That is, $y+z = kx$, for some integer $k$.
    Additionally, let $x \mid y$. That is, $y = jx$, for some integer $j$.
    \begin{align*}
      y + z               & = jx + z             \\ \\
      jx + z              & = kx                 \\
      z                   & = kx - jx = x(k-j)   \\ \\
      \therefore x \mid z & \equiv x \mid x(k-j)
    \end{align*}
    Since $x \mid x(k-j)$, therefore $x \mid z$. Therefore, through contrapositive,
    for any integers $x,y, \tand z$ if $x \mid y$ and $x \nmid z$, then $x \nmid (y+z)$.
  \end{proof}
\end{enumerate}

\end{document}