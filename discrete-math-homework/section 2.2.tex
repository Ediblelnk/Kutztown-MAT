\documentclass{article}
\usepackage[margin=1in]{geometry}
\usepackage{amsmath, amsthm, amssymb, fancyhdr, tikz, circuitikz, graphicx}
\usepackage{centernot, xcolor, hhline, multirow, listings}
\usepackage{blkarray, booktabs, bigstrut, etoolbox}
\usepackage[normalem]{ulem}
\usepackage{bookmark}
\usetikzlibrary{math}
\usetikzlibrary{fit}

\pagestyle{fancy}

\usepackage{hyperref}
\hypersetup{
  colorlinks=true,
  linkcolor=black,
  filecolor=magenta,
  urlcolor=cyan,
}
%formatting
\newcommand{\bld}{\textbf}
\newcommand{\itl}{\textit}
\newcommand{\uln}{\underline}

%math word symbols
\newcommand{\bb}{\mathbb}
\DeclareMathOperator{\tif}{~\text{if}~}
\DeclareMathOperator{\tand}{~\text{and}~}
\DeclareMathOperator{\tbut}{~\text{but}~}
\DeclareMathOperator{\tor}{~\text{or}~}
\DeclareMathOperator{\tsuchthat}{~\text{such that}~}
\DeclareMathOperator{\tsince}{~\text{since}~}
\DeclareMathOperator{\twhen}{~\text{when}~}
\DeclareMathOperator{\twhere}{~\text{where}~}
\DeclareMathOperator{\tfor}{~\text{for}~}
\DeclareMathOperator{\tthen}{~\text{then}~}
\DeclareMathOperator{\tto}{~\text{to}~}

%display shortcut
\DeclareMathOperator{\dstyle}{\displaystyle}
\DeclareMathOperator{\sstyle}{\scriptstyle}

%linear algebra
\DeclareMathOperator{\id}{\bld{id}}
\DeclareMathOperator{\vecspan}{\text{span}}

%discrete math - integer properties
\DeclareMathOperator{\tdiv}{\text{div}}
\DeclareMathOperator{\tmod}{\text{mod}}
\DeclareMathOperator{\lcm}{\text{lcm}}

%augmented matrix environment
\newenvironment{apmatrix}[2]{%
  \left(\begin{array}{@{~}*{#1}{c}|@{~}*{#2}{c}}
    }{
  \end{array}\right)
}
\newenvironment{abmatrix}[2]{%
  \left[\begin{array}{@{~}*{#1}{c}|@{~}*{#2}{c}}
      }{
    \end{array}\right]
}

%lists
\newcommand{\bitem}[1]{\item[\bld{#1.}]}
\newcommand{\bbitem}[2]{\item[\bld{#1.}] \bld{#2}}
\newcommand{\biitem}[2]{\item[\bld{#1.}] \itl{#2}}
\newcommand{\iitem}[1]{\item[\itl{#1.}]}
\newcommand{\iiitem}[2]{\item[\itl{#1.}] \bld{#2}}
\newcommand{\btitem}[2]{\item[\bld{#1.}] \texttt{#2}}

%homework
\newcommand{\question}[2]{\noindent {\large\bld{#1}} #2 \qline}
\newcommand{\qitem}[3]{\item[\bld{#1.}] \itl{#2} #3 \qdash}

\newcommand{\qline}{~\newline\noindent\textcolor[RGB]{200,200,200}{\rule[0.5ex]{\linewidth}{0.2pt}}}
\newcommand{\qdash}{~\newline\noindent\textcolor[RGB]{200,200,200}{\hdashrule[0.5ex]{\linewidth}{0.2pt}{2pt}}}

\lhead{Discrete Math II}
\chead{Section 2.2 Homework}
\rhead{Peter Schaefer}

\begin{document}

\subsection*{2.2.2 Prove each statement by exhaustion}
\begin{enumerate}
  \item[\bld{a.}] For every integer n such that $0 \leq n < 2$, $(n+1)^2 > n^3$
    \begin{proof}
      Let $n \in \mathbb{Z}$ such that $0 \leq n < 2$,
      \begin{align*}
        n & = 0: & (0+1)^2 = 1 & > 0 = 0^3 \checkmark \\
        n & = 1: & (1+1)^2 = 4 & > 1 = 1^3 \checkmark \\
        n & = 2: & (2+1)^2 = 9 & > 8 = 2^3 \checkmark \\
      \end{align*}
      \[
        \therefore \forall~ n \in \mathbb{Z}~\text{such that}~0 \leq n < 2: (n+1)^2 > n^3
      \]
    \end{proof}
\end{enumerate}

\subsection*{2.2.3 Find a counter example}
\begin{enumerate}
  \item[\bld{b.}] If $n$ is an integer and $n^2$ is divisible by 4, then n is divisible by 4.
  \item[] \itl{Counter example:} Consider $n=2$. $n^2,4$ is divisible by 4, but 2 is not.
  \item[\bld{e.}] The multiplicative inverse of $x \in \mathbb{R}$ is a real number $y$ such that
    $xy = 1$. Every real number has a multiplicative inverse.
  \item[] \itl{Counter example:} Consider $x=0$. $\forall~ y \in \mathbb{R}, xy \not = 1$. 0 has no multiplicative inverse.
\end{enumerate}

\subsection*{2.2.5 Proving existential statements}
\begin{enumerate}
  \item[\bld{a.}] There are positive integers $x$ and $y$ such that $\frac{1}{x} + \frac{1}{y}$ is an integer.
    \begin{proof}
      Consider $x=y=1$. $\frac{1}{x} = 1$ and $\frac{1}{y} = 1$ and $1+1 \in \mathbb{Z}$.
    \end{proof}
  \item[\bld{c.}] There are integers $m$ and $n$ such that $\sqrt{m+n} = \sqrt{m} + \sqrt{n}$.
    \begin{proof}
      Consider $m=n=0$. $\sqrt{0+0} = \sqrt{0} + \sqrt{0}$.
    \end{proof}
  \item[\bld{h.}] $\forall~ x,y \in \mathbb{R}, \exists~ z \in \mathbb{R}$ such that $x-z=z-y$.
    \begin{proof}
      Consider $z=\frac{x+y}{2}$,
      \begin{align*}
        x - \frac{x+y}{2} & = \frac{x+y}{2} - y & x +y & = 2\left(\frac{x+y}{2}\right) \\
        x + y             & = x+y               & 0    & = 0
      \end{align*}
      \[
        \therefore \forall~ x,y \in \mathbb{R}, \exists~ z \in \mathbb{R}~\text{such that}~x-z=z-y
      \]
    \end{proof}
\end{enumerate}

\end{document}
