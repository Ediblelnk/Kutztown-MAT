\documentclass{article}
\usepackage[margin=1in]{geometry}
\usepackage{amsmath, amsthm, amssymb, fancyhdr, tikz, circuitikz, graphicx}
\usepackage{centernot, xcolor, hhline, multirow, listings}
\usepackage{blkarray, booktabs, bigstrut, etoolbox}
\usepackage[normalem]{ulem}
\usepackage{bookmark}
\usetikzlibrary{math}
\usetikzlibrary{fit}

\pagestyle{fancy}

\usepackage{hyperref}
\hypersetup{
  colorlinks=true,
  linkcolor=black,
  filecolor=magenta,
  urlcolor=cyan,
}
%formatting
\newcommand{\bld}{\textbf}
\newcommand{\itl}{\textit}
\newcommand{\uln}{\underline}

%math word symbols
\newcommand{\bb}{\mathbb}
\DeclareMathOperator{\tif}{~\text{if}~}
\DeclareMathOperator{\tand}{~\text{and}~}
\DeclareMathOperator{\tbut}{~\text{but}~}
\DeclareMathOperator{\tor}{~\text{or}~}
\DeclareMathOperator{\tsuchthat}{~\text{such that}~}
\DeclareMathOperator{\tsince}{~\text{since}~}
\DeclareMathOperator{\twhen}{~\text{when}~}
\DeclareMathOperator{\twhere}{~\text{where}~}
\DeclareMathOperator{\tfor}{~\text{for}~}
\DeclareMathOperator{\tthen}{~\text{then}~}
\DeclareMathOperator{\tto}{~\text{to}~}

%display shortcut
\DeclareMathOperator{\dstyle}{\displaystyle}
\DeclareMathOperator{\sstyle}{\scriptstyle}

%linear algebra
\DeclareMathOperator{\id}{\bld{id}}
\DeclareMathOperator{\vecspan}{\text{span}}

%discrete math - integer properties
\DeclareMathOperator{\tdiv}{\text{div}}
\DeclareMathOperator{\tmod}{\text{mod}}
\DeclareMathOperator{\lcm}{\text{lcm}}

%augmented matrix environment
\newenvironment{apmatrix}[2]{%
  \left(\begin{array}{@{~}*{#1}{c}|@{~}*{#2}{c}}
    }{
  \end{array}\right)
}
\newenvironment{abmatrix}[2]{%
  \left[\begin{array}{@{~}*{#1}{c}|@{~}*{#2}{c}}
      }{
    \end{array}\right]
}

%lists
\newcommand{\bitem}[1]{\item[\bld{#1.}]}
\newcommand{\bbitem}[2]{\item[\bld{#1.}] \bld{#2}}
\newcommand{\biitem}[2]{\item[\bld{#1.}] \itl{#2}}
\newcommand{\iitem}[1]{\item[\itl{#1.}]}
\newcommand{\iiitem}[2]{\item[\itl{#1.}] \bld{#2}}
\newcommand{\btitem}[2]{\item[\bld{#1.}] \texttt{#2}}

%homework
\newcommand{\question}[2]{\noindent {\large\bld{#1}} #2 \qline}
\newcommand{\qitem}[3]{\item[\bld{#1.}] \itl{#2} #3 \qdash}

\newcommand{\qline}{~\newline\noindent\textcolor[RGB]{200,200,200}{\rule[0.5ex]{\linewidth}{0.2pt}}}
\newcommand{\qdash}{~\newline\noindent\textcolor[RGB]{200,200,200}{\hdashrule[0.5ex]{\linewidth}{0.2pt}{2pt}}}

\lhead{Discrete Math II}
\chead{Section 2.6}
\rhead{Peter Schaefer}

\begin{document}

\subsection*{2.6.1}
\begin{enumerate}
  \bitem{a} Prove that $\frac{\sqrt{2}}{2}$ is irrational.
  \begin{proof}
    Assume that $\frac{\sqrt{2}}{2}$ is rational. That is, it takes the form $\frac{p}{q}$, for some $p,q \in \bb{Z}$, where $q \nmid p$. This also means that $\sqrt{2}$ must be rational, that is it takes the form $\frac{m}{n}$, for some $m,n \in \bb{Z}$, where $n \nmid m$.
    \begin{align*}
      \sqrt{2}   & = \frac{m}{n}     \\
      \sqrt{2}^2 & = (\frac{m}{n})^2 \\
      2          & = \frac{m^2}{n^2} \\
      2n^2       & = m^2
    \end{align*}
    Since $2n^2 = m^2$, $m^2$ is an even number. Since the square of an even integer is also even, this means that $m$ is also even; it takes the form $2k$, for some $k \in \bb{Z}$. Therefore,
    \begin{align*}
      m^2  & = (2k)^2 = 4k^2 \\ \\
      2n^2 & = 4k^2
    \end{align*}
    This implies that there is a common factor between $n$ and $m$, and $m \mid n$. This contradicts the assertion that $m \nmid n$. Therefore $\sqrt{2}$ cannot be rational, and must be irrational.
  \end{proof}
  \bitem{b} Prove that $2-\sqrt{2}$ is irrational.
  \begin{proof}
    Assume that $2 - \sqrt{2}$ is rational. That is, it takes the form $\frac{p}{q}$, for some $p,q \in \bb{Z}$. This assertion also implies that $2$ is rational and $\sqrt{2}$ is rational. This was contradicted in problem 2.6.1a. Therefore $2 - \sqrt{2}$ cannot be rational, and must be irrational.
  \end{proof}
\end{enumerate}

\subsection*{2.6.2}
\begin{enumerate}
  \bitem{a} Prove that if $n$ is an integer such that $n^3$ is even, then $n$ is even.
  \begin{proof}
    Assume that $n^3$ is odd. That is, $n^3 = 2k+1$, for some $k \in \bb{Z}$. Consider $k = 4j^3+6j^2+2j$, for some $j \in \bb{Z}$
    \begin{align*}
      n^3 & = 2k+1 = 2(4j^3+6j^2+2j) + 1       \\
          & = 8j^3 + 12j^2 + 4j + 1            \\
          & = 8j^3 + 8j^2 + 2j + 4j^2 + 4j + 1 \\
          & = (4j^2+4j+1)(2j+1)                \\
          & = (2j+1)^3
    \end{align*}
    Therefore $n$ takes the form $2k+1$, an odd integer. This contradicts the conclusion that $n$ is even. Therefore, if $n$ is an integer such that $n^3$ is even, then $n$ is even.
  \end{proof}
\end{enumerate}

\subsection*{2.6.6}
\begin{enumerate}
  \bitem{a} If a group of 9 kids have won a total of 100 trophies, then at least one of the 9 kids has won as least 12 trophies.
  \begin{proof}
    Assume that if a group of 9 kids have won a total of 100 trophies, then all of the 9 kids have won at fewer than 12 trophies. This means that the total number of trophies must be:
    \begin{align*}
      \text{trophies} & \leq 9 \cdot 11 \\
                      & \leq 99
    \end{align*}
    This contradicts the assertion that 9 kids have won a total of 100 trophies. Therefore, if a group of 9 kids have won a total of 100 trophies, then at least one of the 9 kids has won as least 12 trophies
  \end{proof}
  \bitem{b} If a person buys at least 400 cups of coffee in a year, then there is at least one day in which the person has bought at least two cups of coffee.
  \begin{proof}
    Assume that if a persons buys at least 400 cups of coffee in a year, then there are no days in which the person has bought at least two cups of coffee. Therefore, the total number of coffees must be:
    \begin{align*}
      \text{cups of coffee} & \leq 1 \cdot 365 \\
                            & \leq 365
    \end{align*}
    This contradicts with the assertion that a person buys at least 400 cups of coffee in a year. Therefore, if a person buys at least 400 cups of coffee in a year, then there is at least one day in which the person has bought at least two cups of coffee
  \end{proof}
\end{enumerate}

\end{document}