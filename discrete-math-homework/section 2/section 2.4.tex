\documentclass{article}
\usepackage[margin=1in]{geometry}
\usepackage{amsmath, amsthm, amssymb, fancyhdr, tikz, circuitikz, graphicx}
\usepackage{centernot, xcolor, hhline, multirow, listings}
\usepackage{blkarray, booktabs, bigstrut, etoolbox}
\usepackage[normalem]{ulem}
\usepackage{bookmark}
\usetikzlibrary{math}
\usetikzlibrary{fit}

\pagestyle{fancy}

\usepackage{hyperref}
\hypersetup{
  colorlinks=true,
  linkcolor=black,
  filecolor=magenta,
  urlcolor=cyan,
}
%formatting
\newcommand{\bld}{\textbf}
\newcommand{\itl}{\textit}
\newcommand{\uln}{\underline}

%math word symbols
\newcommand{\bb}{\mathbb}
\DeclareMathOperator{\tif}{~\text{if}~}
\DeclareMathOperator{\tand}{~\text{and}~}
\DeclareMathOperator{\tbut}{~\text{but}~}
\DeclareMathOperator{\tor}{~\text{or}~}
\DeclareMathOperator{\tsuchthat}{~\text{such that}~}
\DeclareMathOperator{\tsince}{~\text{since}~}
\DeclareMathOperator{\twhen}{~\text{when}~}
\DeclareMathOperator{\twhere}{~\text{where}~}
\DeclareMathOperator{\tfor}{~\text{for}~}
\DeclareMathOperator{\tthen}{~\text{then}~}
\DeclareMathOperator{\tto}{~\text{to}~}

%display shortcut
\DeclareMathOperator{\dstyle}{\displaystyle}
\DeclareMathOperator{\sstyle}{\scriptstyle}

%linear algebra
\DeclareMathOperator{\id}{\bld{id}}
\DeclareMathOperator{\vecspan}{\text{span}}

%discrete math - integer properties
\DeclareMathOperator{\tdiv}{\text{div}}
\DeclareMathOperator{\tmod}{\text{mod}}
\DeclareMathOperator{\lcm}{\text{lcm}}

%augmented matrix environment
\newenvironment{apmatrix}[2]{%
  \left(\begin{array}{@{~}*{#1}{c}|@{~}*{#2}{c}}
    }{
  \end{array}\right)
}
\newenvironment{abmatrix}[2]{%
  \left[\begin{array}{@{~}*{#1}{c}|@{~}*{#2}{c}}
      }{
    \end{array}\right]
}

%lists
\newcommand{\bitem}[1]{\item[\bld{#1.}]}
\newcommand{\bbitem}[2]{\item[\bld{#1.}] \bld{#2}}
\newcommand{\biitem}[2]{\item[\bld{#1.}] \itl{#2}}
\newcommand{\iitem}[1]{\item[\itl{#1.}]}
\newcommand{\iiitem}[2]{\item[\itl{#1.}] \bld{#2}}
\newcommand{\btitem}[2]{\item[\bld{#1.}] \texttt{#2}}

%homework
\newcommand{\question}[2]{\noindent {\large\bld{#1}} #2 \qline}
\newcommand{\qitem}[3]{\item[\bld{#1.}] \itl{#2} #3 \qdash}

\newcommand{\qline}{~\newline\noindent\textcolor[RGB]{200,200,200}{\rule[0.5ex]{\linewidth}{0.2pt}}}
\newcommand{\qdash}{~\newline\noindent\textcolor[RGB]{200,200,200}{\hdashrule[0.5ex]{\linewidth}{0.2pt}{2pt}}}

\lhead{Discrete Math II}
\chead{Section 2.4 Homework}
\rhead{Peter Schaefer}

\begin{document}

\subsection*{2.4.1 Proving Statements about odd and even integers with direct proofs}
\begin{enumerate}
  \bitem{a} The sum of an odd and an even integer is odd
  \begin{proof}
    Let $x$ be an even integer and $y$ be an odd integer. $x=2k$ for some integer $k$ and $y=2j+1$ for some integer $j$.
    \begin{align*}
      x + y & = 2k + 2j +1 \\
            & = 2(k+j)+1
    \end{align*}
    Since $k$ and $j$ are integers, $k+j$ is an integer. Therefore, $2(k+j)+1$ is an odd integer. \\
    $\therefore$ the sum of an odd and an even integer is odd
  \end{proof}
  \bitem{e} If $x$ is an even integer and $y$ is an odd integer, then $x^2+y^2$ is odd
  \begin{proof}
    Since $x$ is an even integer, $x = 2k$ for some $k \in \bb{Z}$.
    Since $y$ is an odd integer, $x = 2j + 1$ for some $j \in \bb{Z}$.
    \begin{align*}
      x^2+y^2 & = (2k)^2 + (2j+1)^2       \\
              & = 4k^2 + 4j^2 + 4j + 1    \\
              & = 2(2k^2 + 2j^2 + 2j) + 1
    \end{align*}
    Since $k,j \in \bb{Z}$, $2k^2 + 2j^2 + 2j \in \bb{Z}$. Therefore $2(2k^2 + 2j^2 + 2j) + 1$ is an odd integer. \\
    $\therefore$ if $x$ is an even integer and $y$ is an odd integer, then $x^2+y^2$ is odd
  \end{proof}
\end{enumerate}

\subsection*{2.4.2 Proving statements about rational numbers with direct proofs}
\begin{enumerate}
  \bitem{c} If $x$ and $y$ are rational numbers, then $3x + 2y$ is also a rational number
  \begin{proof}
    Since $x \in \bb{Q}$, $x = \frac{a}{b}$ for some $a,b \in \bb{Z}$ with $b \neq 0$.
    Since $y \in \bb{Q}$, $y = \frac{c}{d}$ for some $c,d \in \bb{Z}$ with $d \neq 0$.
    \begin{align*}
      3x + 2y & = 3\frac{a}{b} + 2\frac{c}{d}              \\
              & = \frac{3ad}{bd} + \frac{2bc}{bd}          \\
              & = \frac{3ad + 2bc}{bd}, b \neq 0, d \neq 0
    \end{align*}
    Since both $b \neq 0$ and $d \neq 0$, $bd \neq 0$. $3ad + 2bc \in \bb{Z}$ by properties of $\bb{Z}$.
    $bd \in \bb{Z}$ by properties of $\bb{Z}$. Therefore $\frac{3ad + 2bc}{bd}$ takes the form of a rational number. \\
    $\therefore$ if $x$ and $y$ are rational numbers, then $3x + 2y$ is also a rational number
  \end{proof}
  \bitem{f} The average of two rational number is also rational
  \begin{proof}
    Let $x,y \in \bb{Q}$. \\
    Since $x \in \bb{Q}$, $x = \frac{a}{b}$ for some $a,b \in \bb{Z}$ with $b \neq 0$.
    Since $y \in \bb{Q}$, $y = \frac{c}{d}$ for some $c,d \in \bb{Z}$ with $d \neq 0$. \\
    The average of two numbers is found by $\frac{x + y}{2}$.
    \begin{align*}
      \frac{x+y}{2} & = \frac{\frac{a}{b} + \frac{c}{d}}{2}           \\
                    & = \frac{\frac{a}{b}}{2} + \frac{\frac{c}{d}}{2} \\
                    & = \frac{a}{2b} + \frac{c}{2d}                   \\
                    & = \frac{ad}{2bd} + \frac{bc}{2bd}               \\
                    & = \frac{ad + bc}{2bd}, b \neq 0, d \neq 0
    \end{align*}
    Since both $b \neq 0$ and $d \neq 0$, $bd \neq 0$. $ad + bc \in \bb{Z}$ by properties of $\bb{Z}$.
    $2bd \in \bb{Z}$ by properties of $\bb{Z}$. Therefore $\frac{ad + bc}{2bd}$ takes the form of a rational number. \\
    $\therefore$ The average of two rational number is also rational
  \end{proof}
\end{enumerate}

\subsection*{2.4.3 Proving algebraic statements with direct proofs}
\begin{enumerate}
  \bitem{a} For any positive real numbers $x$ and $y$, $(x+y)^2 \geq xy$
  \begin{proof}
    Let $x \in \bb{R}$ such that $x > 0$. Let $y \in \bb{R}$ such that $y > 0$.
    Since $x > 0$ and $y > 0$, $x^2 > 0$, $xy > 0$, and $y^2 > 0$. Therefore their sum is also greater than $0$.
    \begin{align*}
      x^2 + xy + y^2  & \geq 0  \\
      x^2 + xy + y^2  & \geq 0  \\
      x^2 + 2xy + y^2 & \geq xy \\
      (x+y)^2         & \geq xy \\
    \end{align*}
    $\therefore$ for any positive real numbers $x$ and $y$, $(x+y)^2 \geq xy$
  \end{proof}
  \bitem{b} If $x$ is a real number and $x \leq 3$, then $12 - 7x + x^2 \geq 0$
  \begin{proof}
    Let $x \in \bb{R}$ such that $x \leq 3$.
    \begin{align*}
      x-3           & \leq 0                                                               \\
      (x-3)(x-4)    & \geq 0,~\text{sign changes since $x \leq 3$ and therefore $x-4 < 0$} \\
      12 - 7x + x^2 & \geq 0
    \end{align*}
    $\therefore$ if $x$ is a real number and $x \leq 3$, then $12 - 7x + x^2 \geq 0$
  \end{proof}
  \bitem{c} If $n$ is a real number and $n > 1$, then $n^2 > n$
  \begin{proof}
    Let $n \in \bb{R}$ such that $n > 1$.
    \begin{align*}
      n         & > 1                                                  \\
      n \cdot n & > 1 \cdot n,~\text{sign stays since $n$ is positive} \\
      n^2       & > n                                                  \\
    \end{align*}
    $\therefore$ if $n$ is a real number and $n > 1$, then $n^2 > n$
  \end{proof}
  \bitem{d} If $x$ is a real number such that $0 < x < 1$, then $\frac{1}{x(1-x)} \geq 4$
  \begin{proof}
    Let $x \in \bb{R}$ such that $0 < x < 1$.
    \begin{align*}
      (x-\frac{1}{2})^2 & \geq 0,~\text{since $n^2 \geq 0$ for $n \in \bb{R}$} \\
      x^2-x+\frac{1}{4} & \geq 0                                               \\
      4x^2-4x+1         & \geq 0                                               \\
      1                 & \geq 4x - 4x^2                                       \\
      \frac{1}{x-x^2}   & \geq 4,~\text{sign stays when $0 < x < 1$}           \\
      \frac{1}{x(1-x)}  & \geq 4
    \end{align*}
    $\therefore$ if $x$ is a real number such that $0 < x < 1$, then $\frac{1}{x(1-x)} \geq 4$
  \end{proof}
\end{enumerate}

\end{document}