\documentclass{article}
\usepackage[margin=1in]{geometry}
\usepackage{amsmath, amsthm, amssymb, fancyhdr, tikz, circuitikz, graphicx}
\usepackage{centernot, xcolor, hhline, multirow, listings}
\usepackage{blkarray, booktabs, bigstrut, etoolbox}
\usepackage[normalem]{ulem}
\usepackage{bookmark}
\usetikzlibrary{math}
\usetikzlibrary{fit}

\pagestyle{fancy}

\usepackage{hyperref}
\hypersetup{
  colorlinks=true,
  linkcolor=black,
  filecolor=magenta,
  urlcolor=cyan,
}
%formatting
\newcommand{\bld}{\textbf}
\newcommand{\itl}{\textit}
\newcommand{\uln}{\underline}

%math word symbols
\newcommand{\bb}{\mathbb}
\DeclareMathOperator{\tif}{~\text{if}~}
\DeclareMathOperator{\tand}{~\text{and}~}
\DeclareMathOperator{\tbut}{~\text{but}~}
\DeclareMathOperator{\tor}{~\text{or}~}
\DeclareMathOperator{\tsuchthat}{~\text{such that}~}
\DeclareMathOperator{\tsince}{~\text{since}~}
\DeclareMathOperator{\twhen}{~\text{when}~}
\DeclareMathOperator{\twhere}{~\text{where}~}
\DeclareMathOperator{\tfor}{~\text{for}~}
\DeclareMathOperator{\tthen}{~\text{then}~}
\DeclareMathOperator{\tto}{~\text{to}~}

%display shortcut
\DeclareMathOperator{\dstyle}{\displaystyle}
\DeclareMathOperator{\sstyle}{\scriptstyle}

%linear algebra
\DeclareMathOperator{\id}{\bld{id}}
\DeclareMathOperator{\vecspan}{\text{span}}

%discrete math - integer properties
\DeclareMathOperator{\tdiv}{\text{div}}
\DeclareMathOperator{\tmod}{\text{mod}}
\DeclareMathOperator{\lcm}{\text{lcm}}

%augmented matrix environment
\newenvironment{apmatrix}[2]{%
  \left(\begin{array}{@{~}*{#1}{c}|@{~}*{#2}{c}}
    }{
  \end{array}\right)
}
\newenvironment{abmatrix}[2]{%
  \left[\begin{array}{@{~}*{#1}{c}|@{~}*{#2}{c}}
      }{
    \end{array}\right]
}

%lists
\newcommand{\bitem}[1]{\item[\bld{#1.}]}
\newcommand{\bbitem}[2]{\item[\bld{#1.}] \bld{#2}}
\newcommand{\biitem}[2]{\item[\bld{#1.}] \itl{#2}}
\newcommand{\iitem}[1]{\item[\itl{#1.}]}
\newcommand{\iiitem}[2]{\item[\itl{#1.}] \bld{#2}}
\newcommand{\btitem}[2]{\item[\bld{#1.}] \texttt{#2}}

%homework
\newcommand{\question}[2]{\noindent {\large\bld{#1}} #2 \qline}
\newcommand{\qitem}[3]{\item[\bld{#1.}] \itl{#2} #3 \qdash}

\newcommand{\qline}{~\newline\noindent\textcolor[RGB]{200,200,200}{\rule[0.5ex]{\linewidth}{0.2pt}}}
\newcommand{\qdash}{~\newline\noindent\textcolor[RGB]{200,200,200}{\hdashrule[0.5ex]{\linewidth}{0.2pt}{2pt}}}

\lhead{Discrete Math II}
\chead{Section 2.7}
\rhead{Peter Schaefer}

\begin{document}

\subsection*{2.7.1}
\begin{enumerate}
  \biitem{b}{For every integer $n$, $n^2 \geq n$}
  \begin{proof}
    Consider cases $n = 0$, $n > 0$, and $n < 0$. \\
    Case 1: $n=0$
    \begin{align*}
      n^2 & \geq n            \\
      0^2 & \geq 0            \\
      0   & \geq 0~\checkmark
    \end{align*}
    Case 2: $n > 0$
    \begin{align*}
      n^2           & \geq n                                                                  \\
      \frac{n^2}{n} & \geq \frac{n}{n} & \text{defined since $n > 0$}                         \\
      n             & \geq 1           & \text{equivalent to $n \geq 0$ since $n \in \bb{Z}$}
    \end{align*}
    Case 3: $n < 0$
    \begin{align*}
      n^2           & \geq n                                               \\
      \frac{n^2}{n} & \leq \frac{n}{n} & \text{sign changes since $n < 0$} \\
      n             & \leq 1           & \text{true since $n < 0$}
    \end{align*}
    Since the statement is true for all cases, and the cases completely cover the possibility space, therefore for every integer $n$, $n^2 \geq n$.
  \end{proof}
\end{enumerate}

\subsection*{2.7.2}
\begin{enumerate}
  \biitem{a}{If $x$ is an integer, then $x^2 + 5x - 1$ is odd}
  \begin{proof}
    Consider cases $x$ is even and $x$ is odd. \\
    Case 1: $x$ is odd, $x = 2k+1$, for some $k \in \bb{Z}$
    \begin{align*}
      (2k+1)^2 + 5(2k+1) - 1 & = 2k^2 + 4k + 1 + 10k + 5 - 1 \\
                             & = 2k^2 + 14k + 5              \\
      \text{odd form}        & = 2(k^2 + 7k + 2) + 1
    \end{align*}
    Case 2: $x$ is even, $x = 2j$, for some $j \in \bb{Z}$
    \begin{align*}
      (2j)^2 + 5(2k) - 1 & = 4j^2 + 10k - 1       \\
      \text{odd form}    & = 2(2j^2 + 5j - 1) + 1
    \end{align*}
    Since the statement is true for all cases, and the cases completely cover the possibility space, therefore if $x$ is an integer, then $x^2 + 5x - 1$ is odd.
  \end{proof}
\end{enumerate}

\subsection*{2.7.3}
\begin{enumerate}
  \biitem{a}{For any real number $x$, $\left\lvert x \right\rvert \geq 0$}
  \begin{proof}
    Consider cases $x = 0$, $x < 0$, $x > 0$. \\
    Case 1: $x = 0$
    \begin{align*}
      0                          & \geq 0            \\
      \left\lvert 0 \right\rvert & \geq 0~\checkmark
    \end{align*}
    Case 2: $x < 0$
    \begin{align*}
      x                          & \leq 0                                                  \\
      \left\lvert x \right\rvert & \geq 0 & \text{since $x = -\left\lvert x \right\rvert$}
    \end{align*}
    Case 3: $x > 0$
    \begin{align*}
      x                          & \geq 0                                                 \\
      \left\lvert x \right\rvert & \geq 0 & \text{since $x = \left\lvert x \right\rvert$}
    \end{align*}
    Since the statement is true for all cases, and the cases completely cover the possibility space, therefore for any real number $x$, $\left\lvert x \right\rvert \geq 0$.
  \end{proof}
\end{enumerate}

\end{document}