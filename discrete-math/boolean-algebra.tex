\section{Boolean Algebra}
\subsection{An introduction to Boolean Algebra}

\textbf{Boolean Algebra} is a set of rules/operations for working with variables whose values are either 0 or 1.
It corresponds highly to propositional logic.

\textbf{Boolean Multiplication}, denoted by $\cdot$.
\begin{align*}
  \text{Boolean } & \cdot & \text{Logic }           & \land      \\
  0 \cdot 0       & = 0   & \text{F} \land \text{F} & = \text{F} \\
  0 \cdot 1       & = 0   & \text{F} \land \text{T} & = \text{F} \\
  1 \cdot 0       & = 0   & \text{T} \land \text{F} & = \text{F} \\
  1 \cdot 1       & = 1   & \text{T} \land \text{T} & = \text{T}
\end{align*}

\textbf{Boolean Addition}, denoted by $+$.
\begin{align*}
  \text{Boolean } & \cdot & \text{Logic }          & \lor       \\
  0 + 0           & = 0   & \text{F} \lor \text{F} & = \text{F} \\
  0 + 1           & = 1   & \text{F} \lor \text{T} & = \text{T} \\
  1 + 0           & = 1   & \text{T} \lor \text{F} & = \text{T} \\
  1 + 1           & = 1   & \text{T} \lor \text{T} & = \text{T} \\
\end{align*}

\textbf{Boolean Complement}, denoted by $\bar{ }$.
\begin{align*}
  \text{Boolean } & \bar{ } & \text{Logic }  & \lnot      \\
  \bar{0}         & = 1     & \lnot \text{F} & = \text{T} \\
  \bar{1}         & = 0     & \lnot \text{T} & = \text{F} \\
\end{align*}

\begin{center}
  \begin{circuitikz}
    \draw (0,0) to[battery] (0,2)
    to[switch, l=$x$] (2,2)
    to[switch, l=$y$](4,2)
    to[lamp] (4,0) -- (0,0);
    \draw (2,-1) node[] {Shannon Circuit (AND $\cdot$)};
  \end{circuitikz}
  \qquad
  \begin{circuitikz}
    \draw (0,0) to[battery] (0,2) -- (1,2) -- (1,1.5)
    to[switch, l=$x$] (3,1.5) -- (3,2) -- (4,2) to[lamp] (4,0) -- (0,0)
    (1,2) -- (1,2.5)
    to[switch, l=$y$] (3,2.5) -- (3,2);
    \draw (2,-1) node[] {Switching Circuit (OR $+$)};
  \end{circuitikz}
\end{center}

Variables that can have a value of either $1$ or $0$ are called \textbf{Boolean Variables}.
Boolean expressions are made of boolean variables. There are also common shorthand ways of notating operations.
\begin{align*}
  x \cdot y + 1 \cdot \bar{z} & = xy + \bar{z}        \\
  x + z + \overline{0 + y}    & = x + z \cdot \bar{y}
\end{align*}

\begin{center}
  \begin{tabular}{r|c|c}
    \textbf{Law Name} & $+$ OR                                               & $\cdot$ AND                                          \\
    \hline
    Idempotent        & $x + x = x$                                          & $x \cdot x = x$                                      \\
    Associative       & $(x + y) + z = x + (y + z)$                          & $(x \cdot y) \cdot z = x \cdot (y \cdot z)$          \\
    Commutative       & $x + y = y + x$                                      & $x \cdot y = y \cdot x$                              \\
    Distributive      & $x + (y \cdot z) = (x + y) \cdot (x + z)$            & $x \cdot (y + z) = (x \cdot y) + (x \cdot z)$        \\
    Identity          & $x + 0 = x$                                          & $x \cdot 1 = x$                                      \\
    Domination        & $x + 1 = 1$                                          & $x \cdot 0 = 0$                                      \\
    Double Complement & $\overline{\overline{x}} = x$                                                                               \\
    Complement        & $x + \overline{x} =$ 1                               & $x \cdot \overline{x} =$ 0                           \\
    DeMorgan          & $\overline{x + y} = \overline{x} \cdot \overline{y}$ & $\overline{x \cdot y} = \overline{x} + \overline{y}$ \\
    Absorption        & $x + (x \cdot y) = x$                                & $x \cdot (x + y) = x$
  \end{tabular}
\end{center}

\subsection{Boolean functions}
\subsection{Disjunctive and conjunctive normal form}
\subsection{Functional completeness}
\subsection{Boolean satisfiability}
\subsection{Gates and circuits}