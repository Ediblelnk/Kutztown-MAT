\section{Induction and Recursion}
\subsection{Sequences}
A \textbf{sequence} is a special type of function in which the domain
is the set of consecutive integers.

When a function is specified as a sequence, using subscripts to denote input
is more common, so $g_k$ is used instead of $g(k)$

A value $g_k$ is called a \textbf{term}, and $k$ is the \textit{index} of $g_k$

For example:
\begin{align*}
  g_1 & = 3.67 & g_2 & = 2.88 \\
  g_3 & = 3.25 & g_4 & = 3.75
\end{align*}
\[
  g(k) = 3.67, 2.88, 3.25, 3.75
\]

An entire sequence is denoted by $\{gk\}$, whereas $g_k$ is used to denote a
single term in the sequence.

A sequence commonly starts with $0$ or $1$, but it could be \textit{any} integer.
\subsubsection*{Finite sequence}
A sequence with a finite domain is a \textbf{finite sequence}.
In a finite sequence, there is an \textit{initial index $m$} and a \textit{final index $n$}.
\subsubsection*{Infinite sequence}
A sequence with an infinite domain is a \textbf{infinite sequence}.
In an infinite sequence, there is an \textit{initial index m} and the sequence
is defined for indices $k \geq m$:
\[
  a_m, a_{m+1}, a_{m+2}, a_{m+3}, \ldots
\]
A sequence can be specified by an \textbf{explicit formula}, such as $d_k = 2^k$
for $k \geq 1$.
\[
  \{d_k\} = 2,4,8,16, \ldots
\]
\subsubsection*{Increasing and Decreasing Sequences}
\begin{itemize}
  \item a sequence is \textit{increasing} if for every two consecutive indices, $k$
        and $k+1$, $a_k < a_{k+1}$
  \item a sequence is \textit{non-decreasing} if for every two consecutive indices, $k$
        and $k+1$, $a_k \leq a_{k+1}$
\end{itemize}
For example,
\begin{align*}
  2 < 4 < 5 < 6          & ~\text{increasing \textit{and} non-decreasing}     \\
  2 \leq 4 \leq 5 \leq 6 & ~\text{non-decreasing \textit{but} not increasing}
\end{align*}
\textit{The same relationship can be said for \textbf{decreasing} and \textbf{non-increasing}}.
\subsubsection*{Geometric Sequences}
A \textbf{geometric sequence} is a sequence of real numbers where each term is found by taking
the previous term and multiplying it by a fixed number called the \textbf{common ratio}.

For example, with an \textit{initial term}: 4, and \textit{common ratio}: $\frac{1}{2}$,
\[
  4,2,\frac{1}{2},\frac{1}{4}, \ldots
\]
\subsubsection*{Arithmetic Sequence}
An \textbf{arithmetic sequence} is a sequence of real numbers where each term after the initial
term is found by taking the previous term and adding a fixed number called the \textbf{common difference}.

For example, with an \textit{initial value}: 2, and \textit{common difference}: 3,
\[
  2,5,8,11, \ldots
\]

\subsection{Recurrence relations}
A rule that defines a term $a_n$ as a function of previous terms in the sequence is called a
\textbf{recurrence relation}

For example,
\begin{align*}
  a_0 & = a~\text{initial value} \\
  a_n & = d + a_{n-1}
\end{align*}
Fibonacci Sequence:
\begin{align*}
  f_0 & = 0                                     \\
  f_1 & = 1                                     \\
  f_n & = f_{n-1} + f_{n-2}~\text{for}~n \geq 2 \\
\end{align*}
A \textbf{dynamical system} is a system that changes over time. The state of the system
at any point is determined by a set of well-defined rules that depend on the past states
of the system.

\subsection{Summations}
\textbf{Summation Notation} is used to express the sum of terms in a numerical sequence.
\[
  \sum_{i=s}^{t} a_i = a_{s} + a_{s+1} + \ldots + a_t
\]
\begin{center}
  \begin{tabular}{l}
    $i$ is the \textit{index}       \\
    $s$ is the \textit{lower limit} \\
    $t$ is the \textit{upper limit}
  \end{tabular}
\end{center}
Any variable name could be used for index, but $i,j,k$ and $\ell$
are most common.
\begin{align*}
  \sum_{j=1}^{4} j^3    & :~\text{summation form} \\
  1^3 + 2^3 + 3^3 + 4^3 & :~\text{expanded form}
\end{align*}
\[
  \sum_{k=m}^{n} a_k = \sum_{k=m}^{n-1} a_k + a_n,~\text{for}~n>m
\]
\[
  \sum_{j=1}^{n} (j+2)^3 = \sum_{k=1}^{n} (k+2)^3 = \sum_{i=3}^{n+2} i^3
\]
\subsubsection*{Closed Form}
A \textbf{closed form} for a mathematical sum expresses the value of the sum without
summation notation. For example,
\[
  \sum_{k=0}^{n-1} (a+kd) = an + \frac{d(n-1)n}{2}
\]
\textbf{Arithmetic Sequence Closed Form}:
\[
  \sum_{k=0}^{n-1} (a+kd) = an + \frac{d(n-1)n}{2}
\]
\textbf{Geometric Sequence Closed Form}:
\[
  \sum_{k=0}^{n-1} (a \cdot r^k) = \frac{a(r^n-1)}{r-1}
\]

\subsection{Mathematical induction}
Two Components of an inductive proof:
\begin{itemize}
  \item Base Case:
        \subitem establishes that the theorem is true for the first value in the sequence
  \item Inductive Step:
        \subitem establishes that if the theorem is true for $k$, then the theorem holds for $k+1$
\end{itemize}
\[
  \text{For a}~k \in \mathbb{Z}^+,~s(k) \implies s(k+1)
  \iff [s(1) \implies s(2) \implies s(3) \implies s(4) \implies \cdots]
\]
The supposition that $s(k)$ is true is called the \underline{inductive hypothesis}.

\subsection{More inductive proofs}
Explicit formula for a sequence defined by a recurrence relation
\begin{align*}
  \{g_n\}:~g_0 & =1,~g_n=3g_{n-1} + 2n~\text{then for any $n \geq 0$,} \\
  g_n          & = \frac{5}{2} \cdot 3^n - n - \frac{3}{2}
\end{align*}

\subsection{Strong induction and well-ordering}


\subsection{Loop invariants}


\subsection{Recursive definitions}


\subsection{Structural induction}


\subsection{Recursive algorithms}


\subsection{Induction and recursive algorithms}


\subsection{Analyzing the time complexity of recursive algorithms}


\subsection{Divide-and-conquer algorithms: Introduction and mergesort}


\subsection{Divide-and-conquer algorithms: Binary Search}


\subsection{Solving linear homogeneous recurrence relations}


\subsection{Solving linear non-homogeneous recurrence relations}


\subsection{Divide-and-conquer recurrence relations}
