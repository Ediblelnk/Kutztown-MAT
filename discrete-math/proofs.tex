\section{Proofs}
\subsection{Mathematical Definitions}
\begin{itemize}
  \item An integer $x$ is \textit{even} if there is an integer $k$ such that $x=2k$
  \item An integer $x$ is \textit{odd} if there is an integer $k$ such that $x=2k+1$
\end{itemize}
\subsubsection*{Parity}
The parity of a number is whether the number is odd or even.
\begin{itemize}
  \item If 2 numbers are both even or both odd, they have the \textit{same parity}.
  \item If 1 number is even and 1 number is odd, they have the \textit{opposite parity}.
\end{itemize}
\subsubsection*{Rational}
A number $r$ is rational if there exists $x$ and $y$ such that
\[
  r=\frac{x}{y},~y \not = 0
\]
where the choice of $x$ and $y$ are not necessarily unique.
\subsubsection*{Divides}
An integer $x$ \textbf{divides} an integer $y$ if and only if $x\not =0$ and $y = kx$,
for some integer $k$. $x$ divides $y$ is denoted as $x \mid  y$.
If $x$ does not divide $y$, it is denoted as $x \nmid y$.
If $x \mid y$, then $y$ is said to be a \underline{multiple} of $x$,
and $x$ is a \textbf{factor} or \textit{divisor} of $y$.
\subsubsection*{Primality}
An integer $n$ is \textbf{prime} if and only if $n > 1$ and the only positive integers
that divide $n$ and 1 and $n$.
\subsubsection*{Composite}
An integer $n$ is \textbf{composite} if and only if $n > 1$ and there is an integer $m$
such that $1 < m < n$ and $m \mid n$.
\subsubsection*{Properties of greater than  and less than}

If $x$ and $c$ are real numbers, then \underline{exactly} 1 of the following is true:
\begin{center}
  \begin{tabular}{ccc}
    $x<c$ & $x=c$ & $x>c$
  \end{tabular}
  \begin{tabular}{c}
    $\lnot (x < c) \Leftrightarrow x \geq c$ \\
    $\lnot (x > c) \Leftrightarrow x \leq c$ \\
    \hline
    $x > c \implies x \geq c$                \\
    $x < c \implies x \leq c$
  \end{tabular}
\end{center}

\subsection{Introduction to Proofs}
A \textbf{theorem} is a statement that can be proven to be true.

A \textbf{proof} consists of a series of steps, each of which follows
logically from assumptions, or from previously proven statements,
whose final step should result in the statement of the \textbf{theorem} being proven.

An \textbf{axiom} is a statement assumed to be true.
\subsubsection*{Example Theorem}
\textit{Every positive integer is than or equal to its square}.
\begin{proof}
  let $x$ be an integer, where $x>0$.

  Since $x$ is an integer and $x>0$, then $x \geq 1$.

  Since $x > 0$, we can multiple both sides of the inequality by $x$ to get
  \[
    (x \cdot x \geq 1 \cdot x) = (x^2 \geq x)
  \]
\end{proof}
\subsection{Writing Proofs: Best Practices}
\subsection{Writing Direct Proofs}
\subsection{Proof by Contrapositive}
\subsection{Proof by Contradiction}
\subsection{Proof by Cases}