\section{Proofs}
\subsection{Mathematical Definitions}
\begin{itemize}
  \item An integer $x$ is \textit{even} if there is an integer $k$ such that $x=2k$
  \item An integer $x$ is \textit{odd} if there is an integer $k$ such that $x=2k+1$
\end{itemize}
\subsubsection*{Parity}
The parity of a number is whether the number is odd or even.
\begin{itemize}
  \item If 2 numbers are both even or both odd, they have the \textit{same parity}.
  \item If 1 number is even and 1 number is odd, they have the \textit{opposite parity}.
\end{itemize}
\subsubsection*{Rational}
A number $r$ is rational if there exists $x$ and $y$ such that
\[
  r=\frac{x}{y},~y \not = 0
\]
where the choice of $x$ and $y$ are not necessarily unique.
\subsubsection*{Divides}
An integer $x$ \textbf{divides} an integer $y$ if and only if $x\not =0$ and $y = kx$,
for some integer $k$. $x$ divides $y$ is denoted as $x \mid  y$.
If $x$ does not divide $y$, it is denoted as $x \nmid y$.
If $x \mid y$, then $y$ is said to be a \underline{multiple} of $x$,
and $x$ is a \textbf{factor} or \textit{divisor} of $y$.
\subsubsection*{Primality}
An integer $n$ is \textbf{prime} if and only if $n > 1$ and the only positive integers
that divide $n$ and 1 and $n$.
\subsubsection*{Composite}
An integer $n$ is \textbf{composite} if and only if $n > 1$ and there is an integer $m$
such that $1 < m < n$ and $m \mid n$.
\subsubsection*{Properties of greater than  and less than}

If $x$ and $c$ are real numbers, then \underline{exactly} 1 of the following is true:
\begin{center}
  \begin{tabular}{ccc}
    $x<c$ & $x=c$ & $x>c$
  \end{tabular}
  \begin{tabular}{c}
    $\lnot (x < c) \Leftrightarrow x \geq c$ \\
    $\lnot (x > c) \Leftrightarrow x \leq c$ \\
    \hline
    $x > c \implies x \geq c$                \\
    $x < c \implies x \leq c$
  \end{tabular}
\end{center}

\subsection{Introduction to Proofs}
A \textbf{theorem} is a statement that can be proven to be true.

A \textbf{proof} consists of a series of steps, each of which follows
logically from assumptions, or from previously proven statements,
whose final step should result in the statement of the \textbf{theorem} being proven.

An \textbf{axiom} is a statement assumed to be true.
\subsubsection*{Example Theorem}
\textit{Every positive integer is than or equal to its square}.
\begin{proof}
  let $x$ be an integer, where $x>0$.

  Since $x$ is an integer and $x>0$, then $x \geq 1$.

  Since $x > 0$, we can multiple both sides of the inequality by $x$ to get
  \[
    (x \cdot x \geq 1 \cdot x) = (x^2 \geq x)
  \]
\end{proof}
\subsubsection*{Proof by Exhaustion}
\textit{if $n \in \{-1,0,1\}$, then $n^2 = \left\lvert n\right\rvert $}
\begin{proof}
  \begin{align*}
    n & = -1 & (-1)^2 = 1 & = \left\lvert -1\right\rvert \\
    n & = 0  & (0)^2 = 0  & = \left\lvert -1\right\rvert \\
    n & = 1  & (1)^2 = 1  & = \left\lvert 1\right\rvert
  \end{align*}
\end{proof}

\subsubsection*{Counter Example}
An assignment of values to variables that shows that a universal statement is false.

\subsection{Writing Proofs: Best Practices}
\subsubsection*{Allowed assumptions in proofs}
\begin{itemize}
  \item the rules of algebra
  \item the set of integers is closed under addition, multiplication, and subtraction
  \item every integer is either even or odd
  \item if $x$ is an integer, there is no integer between $x$ and $x+1$
  \item the relative order of any two real numbers, $x,y \in \mathbb{R}$
  \item the square of any real number is greater than or equal to 0
\end{itemize}
\subsubsection*{Best practices when writing proofs}
\begin{itemize}
  \item indicate when the proof starts and ends
  \item write proofs in complete sentences
  \item give the reader a road-map of what has been shown, what is assumed,
        and where the proof is going
  \item introduce each variable when the variable is used for the first time
  \item a block of equations should be introduced with English text and each
        step that does \underline{not} follow from algebra should be justified
\end{itemize}
\subsubsection*{Common mistakes in proofs}
\begin{itemize}
  \item generalizing from examples
  \item skipping steps
  \item circular reasoning
  \item assuming facts that have not yet been proven
\end{itemize}

\subsection{Writing Direct Proofs}
In a \textbf{direct proof} of a conditional statement, the hypothesis $p$ is assumed
to be true and the conclusion $c$ is proven as a direct result of the assumption.

After the assumptions are stated, a direct proof proceeds by proving the conclusion is true.

For example,
\begin{center}
  \begin{tabular}{c}
    The square of every odd integer is also odd. \\
    $\downarrow$                                 \\
    Let $n$ be an integer that is odd. We will show that $n^2$ is also odd.
  \end{tabular}
\end{center}
\subsubsection*{Direct Proof format}
\begin{center}
  \begin{tabular}{|c|}
    \hline
    Assume hypothesis \\
    $\vdots$          \\
    Derive conclusion \\
    \hline
  \end{tabular}
\end{center}

\subsection{Proof by Contrapositive}
A \textbf{proof by contrapositive} proves a conditional statement of the form $p \implies c$
by showing that the contrapositive $\lnot c \implies \lnot p$ is true.
In other words, $\lnot c$ is assumed to be true and $\lnot p$ is proven as a result of $\lnot c$.

For example,
\begin{center}
  \begin{tabular}{c}
    The square of every odd integer is also odd. \\
    $\downarrow$                                 \\
    Let $n^2$ be an integer that is \textit{not} odd. We will show that $n$ is also \textit{not} odd.
  \end{tabular}
\end{center}
\subsubsection*{Contrapositive Proof format}
\begin{center}
  \begin{tabular}{|c|}
    \hline
    Assume $\lnot$conclusion \\
    $\vdots$                 \\
    Show $\lnot$hypothesis   \\
    \hline
  \end{tabular}
\end{center}

\subsection{Proof by Contradiction}
A \textbf{proof by contradiction} starts by assuming that the theorem is false and then shows
that some logical inconsistency arises as a result of the assumption.
A proof by contradiction is sometimes called an \textbf{indirect proof}.
A proof by contradiction shows the only option is for a theorem to be true to avoid logical errors.

For example,
\begin{center}
  \begin{tabular}{c}
    The square of every odd integer is also odd. \\
    $\downarrow$                                 \\
    Assume there is an even square of an odd integer. We will show there is a logical inconsistency.
  \end{tabular}
\end{center}
\subsubsection*{Contradiction Proof format}
\begin{center}
  \begin{tabular}{|c|}
    \hline
    Assume $\lnot$theorem               \\
    $\vdots$                            \\
    Show \textit{logical inconsistency} \\
    \hline
  \end{tabular}
\end{center}

\subsection{Proof by Cases}
A \textbf{proof by cases} of a universal statement breaks the domain into different classes
and gives a different proof for each class. The proof for each class is called a \textbf{case}.
\textbf{Every} value in the domain \textit{must} be included in at least one class.

For example,
\begin{center}
  \begin{tabular}{c}
    The square of every odd integer is also odd.                                             \\
    $\downarrow$                                                                             \\
    Consider case $n$, where \textit{condition}. We will show theorem is true for this case. \\
    Consider case $n+1$, where \textit{condition}. We will show theorem is true for this case.
  \end{tabular}
\end{center}
\subsubsection*{Cases Proof format}
\begin{center}
  \begin{tabular}{|c|}
    \hline
    Assume hypothesis, and partition domain              \\
    $\vdots$                                             \\
    Show \textit{for each} case, the conclusion is true. \\
    \hline
  \end{tabular}
\end{center}