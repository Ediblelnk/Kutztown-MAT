\section{Relation and Digraphs}
\subsection{Introduction to binary relations}

A \textbf{Binary Relation} between two sets $A$ and $B$ is a subset $R$ of $A \times B$.
It is binary because it is between two sets.
\[
  \text{for } a \in A \land b \in B, (a,b) \in R \text{ is denoted as } a\text{R}b
\]

For example, consider the relation C between $\mathbb{R}$ and $\mathbb{Z}$:
\[
  x\text{C}y \text{ if } \left\lvert x-y\right\rvert \leq 1, \text{ where } x \in \mathbb{R} \text{ and } y \in \mathbb{Z}
\]
If $A$ and $B$ are finite, then relation R between $A$ and $B$ can be represented by a set of ordered pairs.

\subsubsection{Matrix Representation}
\begin{align*}
  P           & = \{\text{Sue}, \text{Harry}, \text{Sam}\}                       \\
  \text{File} & = \{\text{File A}, \text{File B}, \text{File C}, \text{File D}\}
\end{align*}
\[
  \bordermatrix{ & \text{File A} & \text{File B} & \text{File C} & \text{File D} \cr
    \text{Sue}   & 0 & 1 & 1 & 1 \cr
    \text{Harry} & 1 & 0 & 0 & 0 \cr
    \text{Sam}   & 0 & 0 & 0 & 0 \cr }
\]
\begin{center}
  An element is
  \begin{tabular}{c}
    1 if $p$R$f$ is true \\
    0 if $p$R$f$ is false
  \end{tabular}
\end{center}

\subsubsection{Arrow Diagram}
\begin{align*}
  A & = \{a,b,c,d,e\}                                \\
  R & \subseteq A \times A                           \\
  R & = \{(a,b), (b,c), (e,c), (c,e), (d,a), (d,d)\}
\end{align*}

\subsection{Properties of binary relations}
\subsection{Directed graphs, paths, and cycles}
\subsection{Composition of relations}
\subsection{Graph powers and the transitive closure}
\subsection{Matrix multiplication and graph powers}
\subsection{Partial orders}
\subsection{Strict orders and directed acyclic graphs}
\subsection{Equivalence relations}
\subsection{N-ary relations and relational databases}