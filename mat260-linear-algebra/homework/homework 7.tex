\documentclass{article}
\usepackage[margin=1in]{geometry}
\usepackage{amsmath, amsthm, amssymb, fancyhdr, tikz, circuitikz, graphicx}
\usepackage{centernot, xcolor, hhline, multirow, listings, dashrule}
\usepackage{blkarray, booktabs, bigstrut, etoolbox, extarrows}
\usepackage[normalem]{ulem}
\usepackage{bookmark}
\usetikzlibrary{math}
\usetikzlibrary{fit}

\pagestyle{fancy}

\usepackage{hyperref}
\hypersetup{
  colorlinks=true,
  linkcolor=black,
  filecolor=magenta,
  urlcolor=cyan,
}
%formatting
\newcommand{\bld}{\textbf}
\newcommand{\itl}{\textit}
\newcommand{\uln}{\underline}

%math word symbols
\newcommand{\bb}{\mathbb}
\DeclareMathOperator{\tif}{~\text{if}~}
\DeclareMathOperator{\tand}{~\text{and}~}
\DeclareMathOperator{\tbut}{~\text{but}~}
\DeclareMathOperator{\tor}{~\text{or}~}
\DeclareMathOperator{\tsuchthat}{~\text{such that}~}
\DeclareMathOperator{\tsince}{~\text{since}~}
\DeclareMathOperator{\twhen}{~\text{when}~}
\DeclareMathOperator{\twhere}{~\text{where}~}
\DeclareMathOperator{\twith}{~\text{with}~}
\DeclareMathOperator{\tfor}{~\text{for}~}
\DeclareMathOperator{\tthen}{~\text{then}~}
\DeclareMathOperator{\tto}{~\text{to}~}
\DeclareMathOperator{\tin}{~\text{in}~}

%display shortcut
\DeclareMathOperator{\dstyle}{\displaystyle}
\DeclareMathOperator{\sstyle}{\scriptstyle}

%linear algebra
\DeclareMathOperator{\id}{\bld{id}}
\DeclareMathOperator{\vecspan}{\text{span}}
\DeclareMathOperator{\adj}{\text{adj}}

%discrete math - integer properties
\DeclareMathOperator{\tdiv}{\text{div}}
\DeclareMathOperator{\tmod}{\text{mod}}
\DeclareMathOperator{\lcm}{\text{lcm}}

%augmented matrix environment
\newenvironment{apmatrix}[2]{%
    \left(\begin{array}{@{~}*{#1}{c}|@{~}*{#2}{c}}
        }{
    \end{array}\right)
}
\newenvironment{abmatrix}[2]{%
    \left[\begin{array}{@{~}*{#1}{c}|@{~}*{#2}{c}}
            }{
        \end{array}\right]
}

\newenvironment{determinant}[1]{
    \left\lvert
    \begin{array}{@{~}*{#1}{c}}
        }{
    \end{array}
    \right\rvert
}

% graph theory
\DeclareMathOperator{\diam}{\text{diam}}
\newcommand{\comp}[1]{\overline{#1}}

% calculus iii
\newcommand{\vvec}{\overrightarrow}
\newcommand{\norm}[1]{\lVert #1 \rVert}

% abstract algebra
\newcommand{\struct}[1]{\langle #1 \rangle}

%lists
\newcommand{\bitem}[1]{\item[\bld{#1.}]}
\newcommand{\bbitem}[2]{\item[\bld{#1.}] \bld{#2}}
\newcommand{\biitem}[2]{\item[\bld{#1.}] \itl{#2}}
\newcommand{\iitem}[1]{\item[\itl{#1.}]}
\newcommand{\iiitem}[2]{\item[\itl{#1.}] \bld{#2}}
\newcommand{\btitem}[2]{\item[\bld{#1.}] \texttt{#2}}

%homework
\newcommand{\question}[2]{\noindent {\large\bld{#1}} #2 \qline}
\newcommand{\qitem}[3]{\item[\bld{#1.}] #2 \qdash \\ #3 \qdash}

\newcommand{\qline}{~\newline\noindent\textcolor[RGB]{200,200,200}{\rule[0.5ex]{\linewidth}{0.2pt}}}
\newcommand{\qdash}{~\newline\noindent\textcolor[RGB]{200,200,200}{\hdashrule[0.5ex]{\linewidth}{0.2pt}{2pt}}}

\newcommand{\assignment}{Homework 7}

\lhead{Linear Algebra}
\chead{\assignment}
\rhead{Peter Schaefer}

\begin{document}
\section*{\assignment}

\question{1.4}{Inverses; Algebraic Properties of Matrices}{}
\begin{enumerate}
    \qitem{28}{Show that if a square matrix $A$ satisfies $A^2-3A+1=0, \tthen A^{-1}=3I-A$.}{
        \begin{proof}
            Consider $3I - A$.
            \[
                A(3I - A) = 3AI - A^2 = 3A - A^2
            \]
            If $A^2-3A+I=0$, then we can simplify further to determine exactly what $3A - A^2$ equals.
            \begin{align*}
                A^2-3A+I                 & =0               \\
                (3A - A^2) + A^2-3A+I    & = (3A - A^2) + 0 \\
                (3A + (-A^2 + A^2)-3A)+I & = (3A - A^2)     \\
                (3A - 3A) + I            & = 3A - A^2       \\
                I                        & = 3A - A^2       \\
                \therefore~I             & = A(3I-A)        \\
                \therefore~I             & = (3I-A)A
            \end{align*}
            Since $I = A(3I-A) \tand I = (3I-A)A$, therefore $A^{-1}=3I-A \tif A^2-3A+1=0$.
        \end{proof}
    }
    \qitem{31}{Assuming that all matrices are $n \times n$ and invertible, solve for $D$:
    \[
        C^TB^{-1}A^2BAC^{-1}DA^{-2}B^TC^{-2}=C^T.
    \]}{
    \begin{proof}[Work]
        \begin{align*}
            C^TB^{-1}A^2BAC^{-1}DA^{-2}B^TC^{-2}                            & = C^T                                                     \\
            (C^TB^{-1}A^2BAC^{-1})^{-1}C^TB^{-1}A^2BAC^{-1}DA^{-2}B^TC^{-2} & = (C^TB^{-1}A^2BAC^{-1})^{-1}C^T                          \\
            DA^{-2}B^TC^{-2}                                                & = CA^{-1}B^{-1}A^{-2}BC^{T^{-1}}C^T                       \\
            DA^{-2}B^TC^{-2}(A^{-2}B^TC^{-2})^{-1}                          & = CA^{-1}B^{-1}A^{-2}BC^{T^{-1}}C^T(A^{-2}B^TC^{-2})^{-1} \\
            D                                                               & = CA^{-1}B^{-1}A^{-2}BC^{T^{-1}}C^TC^2B^{T^{-1}}A^2       \\
            D                                                               & = CA^{-1}B^{-1}A^{-2}BC^2B^{T^{-1}}A^2
        \end{align*}
    \end{proof}
    }
    \qitem{39}{Using Matrix Inversion, find the unique solution of the given linear system.
        \begin{align*}
            3x_1 - 2x_2 & = -1 \\
            4x_1 + 5x_2 & = 3
        \end{align*}}{
        \begin{proof}[Work]
            \begin{align*}
                \begin{bmatrix}
                    3 & -2 \\
                    4 & 5
                \end{bmatrix}
                \begin{bmatrix}
                    x_1 \\
                    x_2
                \end{bmatrix} & =
                \begin{bmatrix}
                    -1 \\
                    3
                \end{bmatrix}                                     \\
                \begin{bmatrix}
                    x_1 \\
                    x_2
                \end{bmatrix} & =
                \begin{bmatrix}
                    3 & -2 \\
                    4 & 5
                \end{bmatrix}^{-1}
                \begin{bmatrix}
                    -1 \\
                    3
                \end{bmatrix}                                     \\
                \begin{bmatrix}
                    x_1 \\
                    x_2
                \end{bmatrix} & = \frac{1}{3 \cdot 5 - -2 \cdot 4}
                \begin{bmatrix}
                    5  & 2 \\
                    -4 & 3
                \end{bmatrix}
                \begin{bmatrix}
                    -1 \\
                    3
                \end{bmatrix}                                     \\
                \begin{bmatrix}
                    x_1 \\
                    x_2
                \end{bmatrix} & = \frac{1}{23}
                \begin{bmatrix}
                    5  & 2 \\
                    -4 & 3
                \end{bmatrix}
                \begin{bmatrix}
                    -1 \\
                    3
                \end{bmatrix}                                     \\
                \begin{bmatrix}
                    x_1 \\
                    x_2
                \end{bmatrix} & =
                \begin{bmatrix}
                    \frac{5}{23}  & \frac{2}{23} \\
                    -\frac{4}{23} & \frac{3}{23}
                \end{bmatrix}
                \begin{bmatrix}
                    -1 \\
                    3
                \end{bmatrix}                                     \\
                \begin{bmatrix}
                    x_1 \\
                    x_2
                \end{bmatrix} & =
                \begin{bmatrix}
                    \frac{1}{23} \\
                    \frac{13}{23}
                \end{bmatrix}                                     \\
            \end{align*}
        \end{proof}
    }
    \qitem{53a}{Show that if $A,B$ and $A+B$ are invertible matrices with the same size, then
    \[
        A(A^{-1}+B^{-1})B(A+B)^{-1} = I.
    \]}{
    \begin{proof}[Work]
        \begin{align*}
            A(A^{-1}+B^{-1})B(A+B)^{-1}            & = (I+AB^{-1})B(A+B)^{-1} \\
            = (B+A)(A+B)^{-1}                      & = (A+B)(A+B)^{-1}  = I   \\ \\
            \therefore~A(A^{-1}+B^{-1})B(A+B)^{-1} & = I
        \end{align*}
    \end{proof}
    }
    \qitem{55}{Show that if $A$ is a square matrix such that $A^k=0$ for some positive integer $k$, then the matrix $(I-A)$ is invertible and
        \[
            (I-A)^{-1} = I + A + A^2 + \cdots + A^{k-1}.
        \]}{
        \begin{proof}
            Consider square matrix $A \tsuchthat A^k=0$ for some positive integer $k$. Now consider the matrices $(I-A) \tand (I + A + A^2 + \cdots + A^{k-1})$.
            \begin{align*}
                (I-A)(I + A + A^2 + \cdots + A^{k-1}) & = (I + A + A^2 + \cdots + A^{k-1}) - (A + A^2 + \cdots + A^k) \\
                                                      & = I + A - A + A^2 - A^2 + \cdots + A^{k-1} - A^{k-1} + A^k    \\
                                                      & = I + A^k                                                     \\
                                                      & = I + \vec{0} = I
            \end{align*}
            Therefore $(I-A)(I + A + A^2 + \cdots + A^{k-1}) = I$.
            \begin{align*}
                (I + A + A^2 + \cdots + A^{k-1})(I-A) & = (I + A + A^2 + \cdots + A^{k-1}) - (A + A^2 + \cdots + A^k) \\
                                                      & = I + A - A + A^2 - A^2 + \cdots + A^{k-1} - A^{k-1} + A^k    \\
                                                      & = I + A^k                                                     \\
                                                      & = I + \vec{0} = I
            \end{align*}
            Therefore $(I + A + A^2 + \cdots + A^{k-1})(I-A) = I$. Since $(I + A + A^2 + \cdots + A^{k-1})(I-A) = I \tand (I-A)(I + A + A^2 + \cdots + A^{k-1}) = I$, therefore $(I-A)^{-1} = I + A + A^2 + \cdots + A^{k-1}$
        \end{proof}
    }
\end{enumerate}

\question{1.5}{Elementary Matrices and a Method for Finding A-1}
\begin{enumerate}
    \qitem{15}{Use the inverse algorithm to find the inverse of the given matrix, if the inverse exists.
        \[
            \begin{bmatrix}
                -1 & 3 & -4 \\
                2  & 4 & 1  \\
                -4 & 2 & -9
            \end{bmatrix}.
        \]}{
        \begin{proof}
            \begin{align*}
                \begin{bmatrix}
                    -1 & 3 & -4 \\
                    2  & 4 & 1  \\
                    -4 & 2 & -9
                \end{bmatrix} \xrightarrow[-R_1]{R_2 + 2R_1}
                \begin{bmatrix}
                    1  & -3 & 4  \\
                    0  & 10 & -7 \\
                    -4 & 2  & -9
                \end{bmatrix} \xrightarrow{R_4 + 4R_1}
                \begin{bmatrix}
                    1 & -3  & 4  \\
                    0 & 10  & -7 \\
                    0 & -10 & 7
                \end{bmatrix} \xrightarrow{R_3 + R_2}
                \begin{bmatrix}
                    1 & -3 & 4  \\
                    0 & 10 & -7 \\
                    0 & 0  & 0
                \end{bmatrix}
            \end{align*}
            The last matrix has a row of all zeros, which makes it impossible for rref$(A) = I$, thus rref$(A) \neq I$. Therefore, by the Big Theorem from Lecture Note 23, A is not invertible.
        \end{proof}
    }
    \qitem{25}{Find the inverse of the following $4 \times 4$ matrices, where $k_1,k_2,k_3,k_4, \tand k$ are all non-zero.
        \begin{enumerate}
            \biitem{a}{$
                    \begin{bmatrix}
                        k_1 & 0   & 0   & 0   \\
                        0   & k_2 & 0   & 0   \\
                        0   & 0   & k_3 & 0   \\
                        0   & 0   & 0   & k_4
                    \end{bmatrix}$.}
            \begin{proof}[Work]
                \begin{align*}
                    \begin{abmatrix}{4}{4}
                        k_1 & 0   & 0   & 0   & 1 & 0 & 0 & 0 \\
                        0   & k_2 & 0   & 0   & 0 & 1 & 0 & 0 \\
                        0   & 0   & k_3 & 0   & 0 & 0 & 1 & 0 \\
                        0   & 0   & 0   & k_4 & 0 & 0 & 0 & 1
                    \end{abmatrix} & \xrightarrow[\frac{1}{k_3}R_3,~\frac{1}{k_4}R_4]{\frac{1}{k_1}R_1,~\frac{1}{k_2}R_2}
                    \begin{abmatrix}{4}{4}
                        1 & 0 & 0 & 0 & \frac{1}{k_1} & 0 & 0 & 0 \\
                        0 & 1 & 0 & 0 & 0 & \frac{1}{k_2} & 0 & 0 \\
                        0 & 0 & 1 & 0 & 0 & 0 & \frac{1}{k_3} & 0 \\
                        0 & 0 & 0 & 1 & 0 & 0 & 0 & \frac{1}{k_4}
                    \end{abmatrix}                                                             \\
                    \therefore~
                    \begin{bmatrix}
                        k_1 & 0   & 0   & 0   \\
                        0   & k_2 & 0   & 0   \\
                        0   & 0   & k_3 & 0   \\
                        0   & 0   & 0   & k_4
                    \end{bmatrix}^{-1}                 & =
                    \begin{bmatrix}
                        \frac{1}{k_1} & 0             & 0             & 0             \\
                        0             & \frac{1}{k_2} & 0             & 0             \\
                        0             & 0             & \frac{1}{k_3} & 0             \\
                        0             & 0             & 0             & \frac{1}{k_4}
                    \end{bmatrix}
                \end{align*}
            \end{proof}
            \biitem{b}{$
                    \begin{bmatrix}
                        k & 1 & 0 & 0 \\
                        0 & 1 & 0 & 0 \\
                        0 & 0 & k & 1 \\
                        0 & 0 & 0 & 1 \\
                    \end{bmatrix}$.}
            \begin{proof}[Work]
                \begin{align*}
                    \begin{abmatrix}{4}{4}
                        k_1 & 1 & 0   & 0 & 1 & 0 & 0 & 0 \\
                        0   & 1 & 0   & 0 & 0 & 1 & 0 & 0 \\
                        0   & 0 & k_3 & 1 & 0 & 0 & 1 & 0 \\
                        0   & 0 & 0   & 1 & 0 & 0 & 0 & 1
                    \end{abmatrix} & \xrightarrow[R_3 - R_4]{R_1 - R_2}
                    \begin{abmatrix}{4}{4}
                        k & 0 & 0 & 0 & 1 & -1 & 0 &  0 \\
                        0 & 1 & 0 & 0 & 0 &  1 & 0 &  0 \\
                        0 & 0 & k & 0 & 0 &  0 & 1 & -1 \\
                        0 & 0 & 0 & 1 & 0 &  0 & 0 &  1
                    \end{abmatrix}                     \\ &\xrightarrow[\frac{1}{k}R_3]{\frac{1}{k}R_1}
                    \begin{abmatrix}{4}{4}
                        1 & 0 & 0 & 0 & \frac{1}{k} & -\frac{1}{k} & 0 &  0 \\
                        0 & 1 & 0 & 0 & 0 &  1 & 0 &  0 \\
                        0 & 0 & 1 & 0 & 0 &  0 & \frac{1}{k} & -\frac{1}{k} \\
                        0 & 0 & 0 & 1 & 0 &  0 & 0 &  1
                    \end{abmatrix} \\
                    \therefore~
                    \begin{bmatrix}
                        k & 1 & 0 & 0 \\
                        0 & 1 & 0 & 0 \\
                        0 & 0 & k & 1 \\
                        0 & 0 & 0 & 1 \\
                    \end{bmatrix}^{-1}                   & =
                    \begin{bmatrix}
                        \frac{1}{k} & -\frac{1}{k} & 0           & 0            \\
                        0           & 1            & 0           & 0            \\
                        0           & 0            & \frac{1}{k} & -\frac{1}{k} \\
                        0           & 0            & 0           & 1
                    \end{bmatrix}
                \end{align*}
            \end{proof}
        \end{enumerate}}{}
    \qitem{27}{Find all values of $c$, if any, for which the given matrix is invertible.
        \[
            \begin{bmatrix}
                c & c & c \\
                1 & c & c \\
                1 & 1 & c
            \end{bmatrix}
        \]}{
        \begin{proof}[Work]
            \begin{align*}
                \begin{abmatrix}{3}{3}
                    c & c & c & 1 & 0 & 0 \\
                    1 & c & c & 0 & 1 & 0 \\
                    1 & 1 & c & 0 & 0 & 1
                \end{abmatrix}  \xrightarrow[R_2 - R_3]{R_1 - R_2}
                \begin{abmatrix}{3}{3}
                    c-1 & 0 & 0 & 1 & -1 & 0 \\
                    0 & c-1 & 0 & 0 & 1 & -1 \\
                    1 & 1 & c & 0 & 0 & 1
                \end{abmatrix}                       \\
                \xrightarrow[\frac{1}{c-1}R_2]{\frac{1}{c-1}R_1}
                \begin{abmatrix}{3}{3}
                    1 & 0 & 0 & \frac{1}{c-1} & -\frac{1}{c-1} & 0 \\
                    0 & 1 & 0 & 0 & \frac{1}{c-1} & -\frac{1}{c-1} \\
                    1 & 1 & c & 0 & 0 & 1
                \end{abmatrix} \xrightarrow[R_3 - R_1]{R_3 - R_2}
                \begin{abmatrix}{3}{3}
                    1 & 0 & 0 & \frac{1}{c-1} & -\frac{1}{c-1} & 0 \\
                    0 & 1 & 0 & 0 & \frac{1}{c-1} & -\frac{1}{c-1} \\
                    0 & 0 & c & -\frac{1}{c-1} & 0 & \frac{c}{c-1}
                \end{abmatrix} \\
                \xrightarrow{\frac{1}{c}}
                \begin{abmatrix}{3}{3}
                    1 & 0 & 0 & \frac{1}{c-1} & -\frac{1}{c-1} & 0 \\
                    0 & 1 & 0 & 0 & \frac{1}{c-1} & -\frac{1}{c-1} \\
                    0 & 0 & 1 & -\frac{1}{c(c-1)} & 0 & \frac{1}{c-1}
                \end{abmatrix}
            \end{align*}
            The resulting inverse matrix is undefined when $c = 0$ or when $c = 1$, therefore $c \neq 0,~1$.
        \end{proof}
    }
    \qitem{29}{Write the given matrix as a product of elementary matrices.
        \[
            \begin{bmatrix}
                -3 & 1 \\
                2  & 2
            \end{bmatrix}
        \]}{
        \begin{proof}[Work]
            \begin{align*}
                \begin{bmatrix}
                    -3 & 1 \\
                    2  & 2
                \end{bmatrix} \xrightarrow{R_1 + 2R_2}
                \begin{bmatrix}
                    1 & 5 \\
                    2 & 2
                \end{bmatrix} \xrightarrow{R_2 - 2R_1}
                \begin{bmatrix}
                    1 & 5  \\
                    0 & -8
                \end{bmatrix} \xrightarrow{-\frac{1}{8}R_2}
                \begin{bmatrix}
                    1 & 5 \\
                    0 & 1
                \end{bmatrix} \xrightarrow{R_1 - 5R_2}
                \begin{bmatrix}
                    1 & 0 \\
                    0 & 1
                \end{bmatrix}
            \end{align*}
            Each of these row operations can be expressed as a left multiplication of a elementary matrix.
            \begin{align*}
                \begin{bmatrix}
                    1 & -5 \\
                    0 & 1
                \end{bmatrix}
                \begin{bmatrix}
                    1 & 0            \\
                    0 & -\frac{1}{8}
                \end{bmatrix}
                \begin{bmatrix}
                    1  & 0 \\
                    -2 & 1
                \end{bmatrix}
                \begin{bmatrix}
                    1 & 2 \\
                    0 & 1
                \end{bmatrix}
                \begin{bmatrix}
                    -3 & 1 \\
                    2  & 2
                \end{bmatrix} =
                \begin{bmatrix}
                    1 & 0 \\
                    0 & 1
                \end{bmatrix} \\
                \begin{bmatrix}
                    -3 & 1 \\
                    2  & 2
                \end{bmatrix} =
                \left(\begin{bmatrix}
                          1 & -5 \\
                          0 & 1
                      \end{bmatrix}
                \begin{bmatrix}
                    1 & 0            \\
                    0 & -\frac{1}{8}
                \end{bmatrix}
                \begin{bmatrix}
                    1  & 0 \\
                    -2 & 1
                \end{bmatrix}
                \begin{bmatrix}
                    1 & 2 \\
                    0 & 1
                \end{bmatrix}\right)^{-1}
                \begin{bmatrix}
                    1 & 0 \\
                    0 & 1
                \end{bmatrix} \\
                \begin{bmatrix}
                    -3 & 1 \\
                    2  & 2
                \end{bmatrix} =
                \begin{bmatrix}
                    1 & 2 \\
                    0 & 1
                \end{bmatrix}^{-1}
                \begin{bmatrix}
                    1  & 0 \\
                    -2 & 1
                \end{bmatrix}^{-1}
                \begin{bmatrix}
                    1 & 0            \\
                    0 & -\frac{1}{8}
                \end{bmatrix}^{-1}
                \begin{bmatrix}
                    1 & -5 \\
                    0 & 1
                \end{bmatrix}^{-1}
                \begin{bmatrix}
                    1 & 0 \\
                    0 & 1
                \end{bmatrix} \\
                \begin{bmatrix}
                    -3 & 1 \\
                    2  & 2
                \end{bmatrix} =
                \begin{bmatrix}
                    1 & -2 \\
                    0 & 1
                \end{bmatrix}
                \begin{bmatrix}
                    1 & 0 \\
                    2 & 1
                \end{bmatrix}
                \begin{bmatrix}
                    1 & 0  \\
                    0 & -8
                \end{bmatrix}
                \begin{bmatrix}
                    1 & 5 \\
                    0 & 1
                \end{bmatrix}
            \end{align*}
        \end{proof}
    }
    \qitem{41}{Prove that if $A \tand B$ are $m \times n$ matrices, then $A \tand B$ are row equivalent if and only if $A \tand B$ have the same reduced row echelon form.}{
        \begin{proof}
            First, if $A \tand B$ are row equivalent, there exists a sequence of elementary row operations that can transform $B$ into $A$. Now consider the $rref(A)$; it is by definition obtainable by row operations starting at $A$. Therefore there exists a sequence of row operations which transform $B$ into $A$, and then $A$ into $rref(A)$. The same can be said for $A$. Since reduced row echelon form is unique, thus $rref(A)$ is also the $rref(B)$. Therefore, $A \tand B$ have the same reduced row echelon form.

            Second, if $A \tand B$ have the same reduced row echelon form, then there exists a sequence of elementary row operations that can transform $A \tand B$ into the same reduced row echelon form. By definition there is a sequence of elementary row matrices to transform $A$ into $rref(A)$, which is also $rref(B)$. And also by definition there are elementary row matrices to transform $rref(B)$ into $B$. Therefore, there is a sequence of elementary row operations to transform $A$ into $B$. Therefore, by definition, $A \tand B$ are row equivalent.

            Combining both statements, we can conclude that $A \tand B$ are row equivalent if and only if they have the same reduced row echelon form.
        \end{proof}
    }
\end{enumerate}

\question{1.6}{More on Linear Systems and Invertible Matrices}
\begin{enumerate}
    \qitem{15}{Determine conditions on the $b_i$'s, if any, in order to guarantee that the linear system is consistent.
        \begin{align*}
            x_1 - 2x_2 + 5x_3   & = b_1 \\
            4x_1 - 5x_2 + 8x_3  & = b_2 \\
            -3x_1 + 3x_2 - 3x_3 & = b_3
        \end{align*}}{
        \begin{proof}[Work]
            \begin{align*}
                \begin{abmatrix}{3}{1}
                    1 & -2 & 5 & b_1 \\
                    4 & -5 & 8 & b_2 \\
                    -3 & 3 & -3 & b_3
                \end{abmatrix} \xrightarrow{R_3 - R_1}
                \begin{bmatrix}
                    1  & -2 & 5  & b_1     \\
                    4  & -5 & 8  & b_2     \\
                    -4 & 5  & -8 & b_3-b_1
                \end{bmatrix} \xrightarrow[R_2 - 4R_1]{R_3 + R_2}
                \begin{bmatrix}
                    1 & -2 & 5   & b_1         \\
                    0 & 3  & -12 & b_2-4b_1    \\
                    0 & 0  & 0   & b_3-b_1+b_2
                \end{bmatrix} \\
                \xrightarrow{\frac{1}{3}R_2}
                \begin{bmatrix}
                    1 & -2 & 5  & b_1                   \\
                    0 & 1  & -4 & \frac{1}{3}(b_2-4b_1) \\
                    0 & 0  & 0  & b_3-b_1+b_2
                \end{bmatrix} \xrightarrow{R_1 + 2R_2}
                \begin{bmatrix}
                    1 & 0 & -3 & b_1 + \frac{2}{3}(b_2-4b_1) \\
                    0 & 1 & -4 & \frac{1}{3}(b_2-4b_1)       \\
                    0 & 0 & 0  & b_3-b_1+b_2
                \end{bmatrix}
            \end{align*}
            The last row represents $0 = b_3-b_1+b_2$, which is $b_1=b_3+b_2$. If $b_1=b_3+b_2$, then reduced row echelon form is complete, and there are infinitely many solutions to the linear system. If $b_1 \neq b_3+b_2$, then the last row represents $0 = k \twhere k \neq 0$, and there is no solution to the linear system.
        \end{proof}
    }
    \qitem{21}{Let $A\vec{x} = \vec{0}$ be a homogenous system of $n$ linear equations in $n$ unknown that has only the trivial solution. Show that if $k$ is any positive integer, then the system $A^k\vec{x} = \vec{0}$ also has only the trivial solution.}{
        \begin{proof}
            Since $A\vec{x} = \vec{0}$ has only the trivial solution, through the Big Theorem of Lecture 23, this means that $A$ is invertible. This also means that $(A^k)^{-1}$ exists, as it is known that $(A^k)^{-1} = A^{-k} = (A^{-1})^k$. Now Consider the system $A^k\vec{x} = \vec{0}$.
            \begin{align*}
                A^k\vec{x}             & = \vec{0}             \\
                (A^{k})^{-1}A^k\vec{x} & = (A^{k})^{-1}\vec{0} \\
                I\vec{x}               & = \vec{0}             \\
                \vec{x}                & = \vec{0}
            \end{align*}
            Therefore $\vec{x} = \vec{0}$, and the only solution for the system is the trivial solution.
        \end{proof}
    }
\end{enumerate}

\end{document}