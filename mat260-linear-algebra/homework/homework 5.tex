\documentclass{article}
\usepackage[margin=1in]{geometry}
\usepackage{amsmath, amsthm, amssymb, fancyhdr, tikz, circuitikz, graphicx}
\usepackage{centernot, xcolor, hhline, multirow, listings, dashrule}
\usepackage{blkarray, booktabs, bigstrut, etoolbox, extarrows}
\usepackage[normalem]{ulem}
\usepackage{bookmark}
\usetikzlibrary{math}
\usetikzlibrary{fit}

\pagestyle{fancy}

\usepackage{hyperref}
\hypersetup{
  colorlinks=true,
  linkcolor=black,
  filecolor=magenta,
  urlcolor=cyan,
}
%formatting
\newcommand{\bld}{\textbf}
\newcommand{\itl}{\textit}
\newcommand{\uln}{\underline}

%math word symbols
\newcommand{\bb}{\mathbb}
\DeclareMathOperator{\tif}{~\text{if}~}
\DeclareMathOperator{\tand}{~\text{and}~}
\DeclareMathOperator{\tbut}{~\text{but}~}
\DeclareMathOperator{\tor}{~\text{or}~}
\DeclareMathOperator{\tsuchthat}{~\text{such that}~}
\DeclareMathOperator{\tsince}{~\text{since}~}
\DeclareMathOperator{\twhen}{~\text{when}~}
\DeclareMathOperator{\twhere}{~\text{where}~}
\DeclareMathOperator{\twith}{~\text{with}~}
\DeclareMathOperator{\tfor}{~\text{for}~}
\DeclareMathOperator{\tthen}{~\text{then}~}
\DeclareMathOperator{\tto}{~\text{to}~}
\DeclareMathOperator{\tin}{~\text{in}~}

%display shortcut
\DeclareMathOperator{\dstyle}{\displaystyle}
\DeclareMathOperator{\sstyle}{\scriptstyle}

%linear algebra
\DeclareMathOperator{\id}{\bld{id}}
\DeclareMathOperator{\vecspan}{\text{span}}
\DeclareMathOperator{\adj}{\text{adj}}

%discrete math - integer properties
\DeclareMathOperator{\tdiv}{\text{div}}
\DeclareMathOperator{\tmod}{\text{mod}}
\DeclareMathOperator{\lcm}{\text{lcm}}

%augmented matrix environment
\newenvironment{apmatrix}[2]{%
    \left(\begin{array}{@{~}*{#1}{c}|@{~}*{#2}{c}}
        }{
    \end{array}\right)
}
\newenvironment{abmatrix}[2]{%
    \left[\begin{array}{@{~}*{#1}{c}|@{~}*{#2}{c}}
            }{
        \end{array}\right]
}

\newenvironment{determinant}[1]{
    \left\lvert
    \begin{array}{@{~}*{#1}{c}}
        }{
    \end{array}
    \right\rvert
}

% graph theory
\DeclareMathOperator{\diam}{\text{diam}}
\newcommand{\comp}[1]{\overline{#1}}

% calculus iii
\newcommand{\vvec}{\overrightarrow}
\newcommand{\norm}[1]{\lVert #1 \rVert}

% abstract algebra
\newcommand{\struct}[1]{\langle #1 \rangle}

%lists
\newcommand{\bitem}[1]{\item[\bld{#1.}]}
\newcommand{\bbitem}[2]{\item[\bld{#1.}] \bld{#2}}
\newcommand{\biitem}[2]{\item[\bld{#1.}] \itl{#2}}
\newcommand{\iitem}[1]{\item[\itl{#1.}]}
\newcommand{\iiitem}[2]{\item[\itl{#1.}] \bld{#2}}
\newcommand{\btitem}[2]{\item[\bld{#1.}] \texttt{#2}}

%homework
\newcommand{\question}[2]{\noindent {\large\bld{#1}} #2 \qline}
\newcommand{\qitem}[3]{\item[\bld{#1.}] #2 \qdash \\ #3 \qdash}

\newcommand{\qline}{~\newline\noindent\textcolor[RGB]{200,200,200}{\rule[0.5ex]{\linewidth}{0.2pt}}}
\newcommand{\qdash}{~\newline\noindent\textcolor[RGB]{200,200,200}{\hdashrule[0.5ex]{\linewidth}{0.2pt}{2pt}}}

\lhead{Linear Algebra}
\chead{Homework 5}
\rhead{Peter Schaefer}

\begin{document}

\section*{Homework 5}

\subsection*{Section 4.2}

\subsection*{7}
\itl{Which of the following are linear combinations of $\vec{u} = (0,-2,2) \tand \vec{v} = (1,3,-1)$?}
\begin{enumerate}
    \biitem{a}{$(2,2,2)$}
    \begin{proof}
        Let $k_1,k_2 \in \bb{R} \tsuchthat k_1\vec{u} + k_2\vec{v} = (2,2,2)$. That is, $k_1(0,-2,2) + k_2(1,3,-1) = (2,2,2)$. From this equation, we get a linear system of equations.
        \begin{align*}
            0k_1 + 1k_2  & = 2 \\
            -2k_1 + 3k_2 & = 2 \\
            2k_1 - 1k_2  & = 2
        \end{align*}
        \[
            \begin{apmatrix}{2}{1}
                0  & 1  & 2 \\
                -2 & 3  & 2 \\
                2  & -1 & 2
            \end{apmatrix} \xrightarrow{-\frac{1}{2}R_2}
            \begin{pmatrix}
                0 & 1            & 2  \\
                1 & -\frac{3}{2} & -1 \\
                2 & -1           & 2
            \end{pmatrix} \xrightarrow{R_1 \leftrightarrow R_2}
            \begin{pmatrix}
                1 & -\frac{3}{2} & -1 \\
                0 & 1            & 2  \\
                2 & -1           & 2
            \end{pmatrix} \xrightarrow[(-2, 3, 2)]{R_3 - 2R_1}
            \begin{pmatrix}
                1 & -\frac{3}{2} & -1 \\
                0 & 1            & 2  \\
                0 & 2            & 4
            \end{pmatrix} \xrightarrow[(0,-2,-4)]{R_3 - 2R_2}
        \]
        \[
            \begin{pmatrix}
                1 & -\frac{3}{2} & -1 \\
                0 & 1            & 2  \\
                0 & 0            & 0
            \end{pmatrix} \xrightarrow[(0,\frac{3}{2},3)]{R_1 + \frac{3}{2}R_2}
            \begin{apmatrix}{2}
                1 & 0 & 2 \\
                0 & 1 & 2 \\
                0 & 0 & 0
            \end{apmatrix}
        \]
        This augmented matrix represents the following equations:
        \begin{align*}
            k_1 + 0k_2 & = 2 & k_1 & = 2 \\
            0k_1 + k_2 & = 2 & k_2 & = 2 \\
            0 + 0      & = 0
        \end{align*}
        This means that $(2,2,2)$ is a linear combination of $\{\vec{u},\vec{v}\}$, when $k_1 = 2 \tand k_2 = 2$.
    \end{proof}
    \biitem{c}{$(0,4,5)$}
    \begin{proof}
        Let $k_1,k_2 \in \bb{R} \tsuchthat k_1\vec{u} + k_2\vec{v} = (0,4,5)$. That is, $k_1(0,-2,2) + k_2(1,3,-1) = (0,4,5)$. From this equation, we get a linear system of equations.
        \begin{align*}
            0k_1 + 1k_2  & = 0 \\
            -2k_1 + 3k_2 & = 4 \\
            2k_1 - 1k_2  & = 5
        \end{align*}
        \[
            \begin{apmatrix}{2}{1}
                0 & 1 & 0 \\
                -2 & 3 & 4 \\
                2 & -1 & 5
            \end{apmatrix} \xrightarrow{-\frac{1}{2}R_2}
            \begin{pmatrix}
                0 & 1            & 0  \\
                1 & -\frac{3}{2} & -2 \\
                2 & -1           & 5
            \end{pmatrix} \xrightarrow[(-2,3,4)]{R_3 - 2R_2}
            \begin{pmatrix}
                0 & 1            & 0  \\
                1 & -\frac{3}{2} & -2 \\
                0 & 2            & 9  \\
            \end{pmatrix} \xrightarrow[(0,-2,0)]{R_3 - 2R_2}
            \begin{apmatrix}{2}{1}
                0 & 1 & 0 \\
                1 & -\frac{3}{2} & -2 \\
                0 & 0 & 9
            \end{apmatrix}
        \]
        The last row from this matrix provides the equation $0k_1 + 0k_2 = 9$, meaning $0 + 0 = 9$, which is impossible. Therefore, $(0,4,5)$ is not spanned by $\{\vec{u}, \vec{v}\}$.
    \end{proof}
\end{enumerate}

\subsection*{8}
\itl{Express the following combinations of $\vec{u} = (2,1,4), \vec{v} = (1,-1,3), \tand \vec{w} = (3,2,5)$}
\begin{enumerate}
    \biitem{a}{$(-9,-7,-15)$}
    \begin{proof}
        Let $k_1,k_2,k_3 \in \bb{R} \tsuchthat k_1\vec{u} + k_2\vec{v} + k_3\vec{w} = (-9,-7,-15)$. That is $k_1(2,1,4) + k_2(1,-1,3) + k_3(3,2,5) = (-9,-7,-15)$. From this equation, we get a linear system of equations.
        \begin{align*}
            2k_1 + 1k_2 + 3k_3 & = -9  \\
            1k_1 - 1k_2 + 2k_3 & = -7  \\
            4k_1 + 3k_2 + 5k_3 & = -15
        \end{align*}
        \[
            \begin{apmatrix}{3}
                2 & 1 & 3 & -9 \\
                1 & -1 & 2 & -7 \\
                4 & 3 & 5 & -15
            \end{apmatrix} \xrightarrow[(-4,-2,-6,18)]{R_3 - 2R_1}
            \begin{pmatrix}
                2 & 1  & 3  & -9 \\
                1 & -1 & 2  & -7 \\
                0 & 1  & -1 & 3
            \end{pmatrix} \xrightarrow[(-2,2,-4,14)]{R_1 - 2R_2}
            \begin{pmatrix}
                0 & 3  & -1 & 5  \\
                1 & -1 & 2  & -7 \\
                0 & 1  & -1 & 3
            \end{pmatrix} \xrightarrow{R_2 + R_3}
        \]
        \[
            \begin{pmatrix}
                0 & 3 & -1 & 5  \\
                1 & 0 & 1  & -4 \\
                0 & 1 & -1 & 3
            \end{pmatrix} \xrightarrow[(0,-3,3,-9)]{R_1 - 3R_3}
            \begin{pmatrix}
                0 & 0 & 2  & -4 \\
                1 & 0 & 1  & -4 \\
                0 & 1 & -1 & 3
            \end{pmatrix} \xrightarrow{\frac{1}{2}R_1}
            \begin{pmatrix}
                0 & 0 & 1  & -2 \\
                1 & 0 & 1  & -4 \\
                0 & 1 & -1 & 3
            \end{pmatrix} \xrightarrow[(0,0,-1,2)]{R_2 - R_1}
        \]
        \[
            \begin{pmatrix}
                0 & 0 & 1  & -2 \\
                1 & 0 & 0  & -2 \\
                0 & 1 & -1 & 3
            \end{pmatrix} \xrightarrow[(0,0,1,-2)]{R_3 + R_1}
            \begin{pmatrix}
                0 & 0 & 1 & -2 \\
                1 & 0 & 0 & -2 \\
                0 & 1 & 0 & 1
            \end{pmatrix} \xrightarrow[R_2 \leftrightarrow R_3]{R_1 \leftrightarrow R_2}
            \begin{apmatrix}{3}
                1 & 0 & 0 & -2 \\
                0 & 1 & 0 & 1 \\
                0 & 0 & 1 & -2
            \end{apmatrix}
        \]
        This augmented matrix represents the following equations:
        \begin{align*}
            k_1 & = -2 \\
            k_2 & = 1  \\
            k_3 & = -2
        \end{align*}
        This means that $(-9,-7,-15)$ is a linear combination of $\{\vec{u},\vec{v},\vec{w}\}$, when $k_1 = -2, k_2 = 1, \tand k_3 = -2$.
    \end{proof}
    \biitem{c}{$(0,0,0)$}
    \begin{proof}
        Let $k_1,k_2,k_3 \in \bb{R} \tsuchthat k_1\vec{u} + k_2\vec{v} + k_3\vec{w} = (0,0,0)$. That is $k_1(2,1,4) + k_2(1,-1,3) + k_3(3,2,5) = (0,0,0)$. From this equation, we get a linear system of equations.
        \begin{align*}
            2k_1 + 1k_2 + 3k_3 & = 0 \\
            1k_1 - 1k_2 + 2k_3 & = 0 \\
            4k_1 + 3k_2 + 5k_3 & = 0
        \end{align*}
        \[
            \begin{apmatrix}{3}
                2 & 1 & 3 & 0 \\
                1 & -1 & 2 & 0 \\
                4 & 3 & 5 & 0
            \end{apmatrix} \xrightarrow[(-4,-2,-6,0)]{R_3 - 2R_1}
            \begin{pmatrix}
                2 & 1  & 3  & 0 \\
                1 & -1 & 2  & 0 \\
                0 & 1  & -1 & 0
            \end{pmatrix} \xrightarrow[(-2,2,-4,0)]{R_1 - 2R_2}
            \begin{pmatrix}
                0 & 3  & -1 & 0 \\
                1 & -1 & 2  & 0 \\
                0 & 1  & -1 & 0
            \end{pmatrix} \xrightarrow{R_2 + R_3}
        \]
        \[
            \begin{pmatrix}
                0 & 3 & -1 & 0 \\
                1 & 0 & 1  & 0 \\
                0 & 1 & -1 & 0
            \end{pmatrix} \xrightarrow[(0,-3,3,0)]{R_1 - 3R_3}
            \begin{pmatrix}
                0 & 0 & 2  & 0 \\
                1 & 0 & 1  & 0 \\
                0 & 1 & -1 & 0
            \end{pmatrix} \xrightarrow{\frac{1}{2}R_1}
            \begin{pmatrix}
                0 & 0 & 1  & 0 \\
                1 & 0 & 1  & 0 \\
                0 & 1 & -1 & 0
            \end{pmatrix} \xrightarrow[(0,0,-1,0)]{R_2 - R_1}
        \]
        \[
            \begin{pmatrix}
                0 & 0 & 1  & 0 \\
                1 & 0 & 0  & 0 \\
                0 & 1 & -1 & 0
            \end{pmatrix} \xrightarrow[(0,0,1,0)]{R_3 + R_1}
            \begin{pmatrix}
                0 & 0 & 1 & 0 \\
                1 & 0 & 0 & 0 \\
                0 & 1 & 0 & 0
            \end{pmatrix} \xrightarrow[R_2 \leftrightarrow R_3]{R_1 \leftrightarrow R_2}
            \begin{apmatrix}{3}
                1 & 0 & 0 & 0 \\
                0 & 1 & 0 & 0 \\
                0 & 0 & 1 & 0
            \end{apmatrix}
        \]
        This augmented matrix represents the following equations:
        \begin{align*}
            k_1 & = 0 \\
            k_2 & = 0 \\
            k_3 & = 0
        \end{align*}
        This means that $(0,0,0)$ is a linear combination of $\{\vec{u},\vec{v},\vec{w}\}$, when $k_1 = 0, k_2 = 0, \tand k_3 = 0$.
    \end{proof}
\end{enumerate}

\subsection*{9}
\itl{Which of the following are linear combinations of
    \[
        A = \begin{bmatrix}
            4  & 0  \\
            -2 & -2
        \end{bmatrix}, \quad
        B = \begin{bmatrix}
            1 & -1 \\
            2 & 3
        \end{bmatrix}, \quad
        C = \begin{bmatrix}
            0 & 2 \\
            1 & 4
        \end{bmatrix}?
    \]
}
\begin{enumerate}
    \biitem{a}{$\dstyle \begin{bmatrix} 6 & -8 \\ -1 & -8 \end{bmatrix}$}
    \begin{proof}
        Let $k_1,k_2,k_3 \in \bb{R} \tsuchthat k_1A + k_2B + k_3C = \dstyle \begin{bmatrix} 6 & -8 \\ -1 & -8 \end{bmatrix}$. \\ That is, $\dstyle k_1\begin{bmatrix}4  & 0  \\-2 & -2\end{bmatrix} + k_2\begin{bmatrix}1 & -1 \\2 & 3\end{bmatrix} + k_3\begin{bmatrix}0 & 2 \\1 & 4\end{bmatrix} = \begin{bmatrix} 6 & -8 \\ -1 & -8 \end{bmatrix}$. From this equation, we get a linear system of equations.
        \begin{align*}
            4k_1 + 1k_2 + 0k_3  & = 6  \\
            0k_1 - 1k_2 + 2k_3  & = -8 \\
            -2k_1 + 2k_2 + 1k_3 & = -1 \\
            -2k_1 + 3k_2 + 4k_3 & = -8
        \end{align*}
        \[
            \begin{apmatrix}{3}
                4 & 1 & 0 & 6 \\
                0 & - 1 & 2 & -8 \\
                -2 & 2 & 1 & -1 \\
                -2 & 3 & 4 & -8
            \end{apmatrix} \xrightarrow[(-4,4,2,-2)]{R_1 + 2R_3}
            \begin{pmatrix}
                0  & 5   & 2 & 4  \\
                0  & - 1 & 2 & -8 \\
                -2 & 2   & 1 & -1 \\
                -2 & 3   & 4 & -8
            \end{pmatrix} \xrightarrow[(2,-2,-1,1)]{R_4 - R_3}
            \begin{pmatrix}
                0  & 5   & 2 & 4  \\
                0  & - 1 & 2 & -8 \\
                -2 & 2   & 1 & -1 \\
                0  & 1   & 3 & -7
            \end{pmatrix} \xrightarrow{-\frac{1}{2}R_3}
        \]
        \[
            \begin{pmatrix}
                0 & 5   & 2            & 4           \\
                0 & - 1 & 2            & -8          \\
                1 & -1  & -\frac{1}{2} & \frac{1}{2} \\
                0 & 1   & 3            & -7
            \end{pmatrix} \xrightarrow[R_2 + R_4]{R_3 + R_4}
            \begin{pmatrix}
                0 & 5 & 2            & 4             \\
                0 & 0 & 5            & -15           \\
                1 & 0 & 2\frac{1}{2} & -6\frac{1}{2} \\
                0 & 1 & 3            & -7
            \end{pmatrix} \xrightarrow[(0,-5,-15,35)]{R_1 - 5R_4}
            \begin{pmatrix}
                0 & 0 & -13          & 39            \\
                0 & 0 & 5            & -15           \\
                1 & 0 & 2\frac{1}{2} & -6\frac{1}{2} \\
                0 & 1 & 3            & -7
            \end{pmatrix} \xrightarrow[\frac{1}{5}R_2]{R_1 + 2R_2}
        \]
        \[
            \begin{pmatrix}
                0 & 0 & -3           & 9             \\
                0 & 0 & 1            & -3            \\
                1 & 0 & 2\frac{1}{2} & -6\frac{1}{2} \\
                0 & 1 & 3            & -7
            \end{pmatrix} \xrightarrow[R_1 \leftrightarrow R_4]{R_1 + 3R_2}
            \begin{pmatrix}
                0 & 1 & 3            & -7            \\
                0 & 0 & 1            & -3            \\
                1 & 0 & 2\frac{1}{2} & -6\frac{1}{2} \\
                0 & 0 & 0            & 0
            \end{pmatrix} \xrightarrow[(0,0,-3,9)]{R_1 - 3R_2}
            \begin{pmatrix}
                0 & 1 & 0            & 2             \\
                0 & 0 & 1            & -3            \\
                1 & 0 & 2\frac{1}{2} & -6\frac{1}{2} \\
                0 & 0 & 0            & 0
            \end{pmatrix} \xrightarrow[(0,0,-2\frac{1}{2}, 7\frac{1}{2})]{R_3 - 2\frac{1}{2}R_2}
        \]
        \[
            \begin{pmatrix}
                0 & 1 & 0 & 2  \\
                0 & 0 & 1 & -3 \\
                1 & 0 & 0 & 1  \\
                0 & 0 & 0 & 0
            \end{pmatrix} \rightarrow
            \begin{apmatrix}{3}
                1 & 0 & 0 & 1  \\
                0 & 1 & 0 & 2  \\
                0 & 0 & 1 & -3 \\
                0 & 0 & 0 & 0
            \end{apmatrix}
        \]
        This augmented matrix represents the following equations:
        \begin{align*}
            k_1 & = 1  \\
            k_2 & = 2  \\
            k_3 & = -3
        \end{align*}
        This means that $\dstyle \begin{bmatrix} 6 & -8 \\ -1 & -8 \end{bmatrix}$ is a linear combination of $\{A,B,C\}$, when $k_1 = 1, k_2 = 2, \tand k_3 = -3$.
    \end{proof}
    \biitem{c}{$\dstyle \begin{bmatrix} 6 & 0 \\ 3 & 8 \end{bmatrix}$}
    \begin{proof}
        Let $k_1,k_2,k_3 \in \bb{R} \tsuchthat k_1A + k_2B + k_3C = \dstyle \begin{bmatrix} 6 & 0 \\ 3 & 8 \end{bmatrix}$. \\ That is, $\dstyle k_1\begin{bmatrix}4  & 0  \\-2 & -2\end{bmatrix} + k_2\begin{bmatrix}1 & -1 \\2 & 3\end{bmatrix} + k_3\begin{bmatrix}0 & 2 \\1 & 4\end{bmatrix} = \begin{bmatrix} 6 & 0 \\ 3 & 8 \end{bmatrix}$. From this equation, we get a linear system of equations.
        \begin{align*}
            4k_1 + 1k_2 + 0k_3  & = 6 \\
            0k_1 - 1k_2 + 2k_3  & = 0 \\
            -2k_1 + 2k_2 + 1k_3 & = 3 \\
            -2k_1 + 3k_2 + 4k_3 & = 8
        \end{align*}
        \[
            \begin{apmatrix}{3}
                4 & 1 & 0 & 6 \\
                0 & - 1 & 2 & 0 \\
                -2 & 2 & 1 & 3 \\
                -2 & 3 & 4 & 8
            \end{apmatrix} \xrightarrow[(-4,4,2,6)]{R_1 + 2R_3}
            \begin{pmatrix}
                0  & 5   & 2 & 12 \\
                0  & - 1 & 2 & 0  \\
                -2 & 2   & 1 & 3  \\
                -2 & 3   & 4 & 8
            \end{pmatrix} \xrightarrow[(2,-2,-1,-3)]{R_4 - R_3}
            \begin{pmatrix}
                0  & 5   & 2 & 12 \\
                0  & - 1 & 2 & 0  \\
                -2 & 2   & 1 & 3  \\
                0  & 1   & 3 & 5
            \end{pmatrix} \xrightarrow{-\frac{1}{2}R_3}
        \]
        \[
            \begin{pmatrix}
                0 & 5   & 2            & 12            \\
                0 & - 1 & 2            & 0             \\
                1 & -1  & -\frac{1}{2} & -1\frac{1}{2} \\
                0 & 1   & 3            & 5
            \end{pmatrix} \xrightarrow[R_2 + R_4]{R_3 + R_4}
            \begin{pmatrix}
                0 & 5 & 2            & 12           \\
                0 & 0 & 5            & 5            \\
                1 & 0 & 2\frac{1}{2} & 3\frac{1}{2} \\
                0 & 1 & 3            & 5
            \end{pmatrix} \xrightarrow[(0,-5,-15,-25)]{R_1 - 5R_4}
            \begin{pmatrix}
                0 & 0 & -13          & -13          \\
                0 & 0 & 5            & 5            \\
                1 & 0 & 2\frac{1}{2} & 3\frac{1}{2} \\
                0 & 1 & 3            & 5
            \end{pmatrix} \xrightarrow[(0,0,10,10)]{R_1 + 2R_2}
        \]
        \[
            \begin{pmatrix}
                0 & 0 & -3           & -3           \\
                0 & 0 & 5            & 5            \\
                1 & 0 & 2\frac{1}{2} & 3\frac{1}{2} \\
                0 & 1 & 3            & 5
            \end{pmatrix} \xrightarrow[-\frac{1}{3}R_1]{R_4 + R_1}
            \begin{pmatrix}
                0 & 0 & 1            & 1            \\
                0 & 0 & 5            & 5            \\
                1 & 0 & 2\frac{1}{2} & 3\frac{1}{2} \\
                0 & 1 & 0            & 2
            \end{pmatrix} \xrightarrow[R_2 - 5R_1]{R_3 - 2\frac{1}{2}R_1}
            \begin{pmatrix}
                0 & 0 & 1 & 1 \\
                0 & 0 & 0 & 0 \\
                1 & 0 & 0 & 1 \\
                0 & 1 & 0 & 2
            \end{pmatrix} \rightarrow
        \]
        \[
            \begin{apmatrix}{3}
                1 & 0 & 0 & 1 \\
                0 & 1 & 0 & 2 \\
                0 & 0 & 1 & 1 \\
                0 & 0 & 0 & 0
            \end{apmatrix}
        \]
        This augmented matrix represents the following equations:
        \begin{align*}
            k_1 & = 1 \\
            k_2 & = 2 \\
            k_3 & = 1
        \end{align*}
        This means that $\dstyle \begin{bmatrix} 6 & 0 \\ 3 & 8 \end{bmatrix}$ is a linear combination of $\{A,B,C\}$, when $k_1 = 1, k_2 = 2, \tand k_3 = 1$.
    \end{proof}

\end{enumerate}

\subsection*{10}
\itl{In each part express the vector as a linear combination of $\vec{p}_1 = 2 + x + 4x^2, \vec{p}_2 = 1 - x + 3x^2, \tand \vec{p}_3 = 3 + 2x + 5x^2$.}
\begin{enumerate}
    \biitem{a}{$-9-7x-15x^2$}
    \begin{proof}
        Let $k_1,k_2,k_3 \in \bb{R} \tsuchthat k_1p_1 + k_2p_2 + k_3p_3 = -9-7x-15x^2$. That is, $k_1(2 + x + 4x^2) + k_2(1 - x + 3x^2) + k_3(3 + 2x + 5x^2) = -9-7x-15x^2$. From this equation, we get a linear system of equations.
        \begin{align*}
            2k_1 + 1k_2 + 3k_3 & = -9  \\
            1k_1 - 1k_2 + 2k_3 & = -7  \\
            4k_1 + 3k_2 + 5k_3 & = -15
        \end{align*}
        \[
            \begin{apmatrix}{3}
                2 & 1 & 3 & -9 \\
                1 & -1 & 2 & -7 \\
                4 & 3 & 5 & -15
            \end{apmatrix} \xrightarrow[(-4,-2,-6,18)]{R_3 - 2R_1}
            \begin{pmatrix}
                2 & 1  & 3  & -9 \\
                1 & -1 & 2  & -7 \\
                0 & 1  & -1 & 3
            \end{pmatrix} \xrightarrow[(-2,2,-4,14)]{R_1 - 2R_2}
            \begin{pmatrix}
                0 & 3  & -1 & 5  \\
                1 & -1 & 2  & -7 \\
                0 & 1  & -1 & 3
            \end{pmatrix} \xrightarrow{R_2 + R_3}
        \]
        \[
            \begin{pmatrix}
                0 & 3 & -1 & 5  \\
                1 & 0 & 1  & -4 \\
                0 & 1 & -1 & 3
            \end{pmatrix} \xrightarrow[(0,-3,3,-9)]{R_1 - 3R_3}
            \begin{pmatrix}
                0 & 0 & 2  & -4 \\
                1 & 0 & 1  & -4 \\
                0 & 1 & -1 & 3
            \end{pmatrix} \xrightarrow{\frac{1}{2}R_1}
            \begin{pmatrix}
                0 & 0 & 1  & -2 \\
                1 & 0 & 1  & -4 \\
                0 & 1 & -1 & 3
            \end{pmatrix} \xrightarrow[(0,0,-1,2)]{R_2 - R_1}
        \]
        \[
            \begin{pmatrix}
                0 & 0 & 1  & -2 \\
                1 & 0 & 0  & -2 \\
                0 & 1 & -1 & 3
            \end{pmatrix} \xrightarrow[(0,0,1,-2)]{R_3 + R_1}
            \begin{pmatrix}
                0 & 0 & 1 & -2 \\
                1 & 0 & 0 & -2 \\
                0 & 1 & 0 & 1
            \end{pmatrix} \xrightarrow[R_2 \leftrightarrow R_3]{R_1 \leftrightarrow R_2}
            \begin{apmatrix}{3}
                1 & 0 & 0 & -2 \\
                0 & 1 & 0 & 1 \\
                0 & 0 & 1 & -2
            \end{apmatrix}
        \]
        This augmented matrix represents the following equations:
        \begin{align*}
            k_1 & = -2 \\
            k_2 & = 1  \\
            k_3 & = -2
        \end{align*}
        This means that $-9-7x-15x^2$ is a linear combination of $\{\vec{p}_1,\vec{p}_2,\vec{p}_3\}$, when $k_1 = -2, k_2 = 1, \tand k_3 = -2$.
    \end{proof}
    \biitem{d}{$7+8x+9x^2$}
    \begin{proof}
        Let $k_1,k_2,k_3 \in \bb{R} \tsuchthat k_1p_1 + k_2p_2 + k_3p_3 = 7+8x+9x^2$. That is, $k_1(2 + x + 4x^2) + k_2(1 - x + 3x^2) + k_3(3 + 2x + 5x^2) = 7+8x+9x^2$. From this equation, we get a linear system of equations.
        \begin{align*}
            2k_1 + 1k_2 + 3k_3 & = 7 \\
            1k_1 - 1k_2 + 2k_3 & = 8 \\
            4k_1 + 3k_2 + 5k_3 & = 9
        \end{align*}
        \[
            \begin{apmatrix}{3}
                2 & 1 & 3 & 7 \\
                1 & -1 & 2 & 8 \\
                4 & 3 & 5 & 9
            \end{apmatrix} \xrightarrow[(-4,-2,-6,-14)]{R_3 - 2R_1}
            \begin{pmatrix}
                2 & 1  & 3  & 7  \\
                1 & -1 & 2  & 8  \\
                0 & 1  & -1 & -5
            \end{pmatrix} \xrightarrow[(-2,2,-4,-16)]{R_1 - 2R_2}
            \begin{pmatrix}
                0 & 3  & -1 & -9 \\
                1 & -1 & 2  & 8  \\
                0 & 1  & -1 & -5
            \end{pmatrix} \xrightarrow{R_2 + R_3}
        \]
        \[
            \begin{pmatrix}
                0 & 3 & -1 & -9 \\
                1 & 0 & 1  & 3  \\
                0 & 1 & -1 & -5
            \end{pmatrix} \xrightarrow[(0,-3,3,15)]{R_1 - 3R_3}
            \begin{pmatrix}
                0 & 0 & 2  & 6  \\
                1 & 0 & 1  & 3  \\
                0 & 1 & -1 & -5
            \end{pmatrix} \xrightarrow{\frac{1}{2}R_1}
            \begin{pmatrix}
                0 & 0 & 1  & 3  \\
                1 & 0 & 1  & 3  \\
                0 & 1 & -1 & -5
            \end{pmatrix} \xrightarrow[(0,0,-1,-3)]{R_2 - R_1}
        \]
        \[
            \begin{pmatrix}
                0 & 0 & 1  & 3  \\
                1 & 0 & 0  & 0  \\
                0 & 1 & -1 & -5
            \end{pmatrix} \xrightarrow[(0,0,1,3)]{R_3 + R_1}
            \begin{pmatrix}
                0 & 0 & 1 & 3  \\
                1 & 0 & 0 & 0  \\
                0 & 1 & 0 & -2
            \end{pmatrix} \xrightarrow[R_2 \leftrightarrow R_3]{R_1 \leftrightarrow R_2}
            \begin{apmatrix}{3}
                1 & 0 & 0 & 0  \\
                0 & 1 & 0 & -2 \\
                0 & 0 & 1 & 3
            \end{apmatrix}
        \]
        This augmented matrix represents the following equations:
        \begin{align*}
            k_1 & = 0  \\
            k_2 & = -2 \\
            k_3 & = 3
        \end{align*}
        This means that $-9-7x-15x^2$ is a linear combination of $\{\vec{p}_1,\vec{p}_2,\vec{p}_3\}$, when $k_1 = 0, k_2 = -2, \tand k_3 = 3$.
    \end{proof}
\end{enumerate}

\subsection*{Section 1.2}
\subsubsection*{4}
\itl{In each part, suppose that the augmented matrix for a system of linear equations has been reduced by row operations to the given reduced row echelon form. Solve the system.}
\begin{enumerate}
    \biitem{a}{$\dstyle \begin{bmatrix} 1 & 0 & 0 & -3 \\ 0 & 1 & 0 & 0 \\ 0 & 0 & 1 & 7 \end{bmatrix}$}
    \begin{tabular}{l}
        $k_1 = -3$ \\
        $k_2 = 0$  \\
        $k_3 = 7$
    \end{tabular}
    \biitem{b}{$\dstyle \begin{bmatrix} 1 & 0 & 0 & -7 & 8 \\ 0 & 1 & 0 & 3 & 2 \\ 0 & 0 & 1 & 1 & -5 \end{bmatrix}$}
    \begin{tabular}{ll}
        $k_1 - 7t = 8$ & $k_1 = 7t + 8$   \\
        $k_2 + 3t = 2$ & $k_2 = -3t + 2$  \\
        $k_3 + t = -5$ & $k_3 = -t -5$    \\
        $k_4 = t~(\text{free parameter})$ \\
    \end{tabular}
    \biitem{c}{$\dstyle \begin{bmatrix} 1 & -6 & 0 & 0 & 3 & -2 \\ 0 & 0 & 1 & 0 & 4 & 7 \\ 0 & 0 & 0 & 1 & 5 & 8 \\ 0 & 0 & 0 & 0 & 0 & 0 \end{bmatrix}$}
    \begin{tabular}{ll}
        $k_1 - 6t_1 + 3t_2 = -2$ & $k_1 = 6t_1 - 3t_2 - 2$ \\
        $k_2 = t_1~(\text{free parameter})$                \\
        $k_3 + 4t_2 = 7$         & $k_3 = -4t_2 + 7$       \\
        $k_4 + 5t_2 = 8$         & $k_4 = -5t_2 + 8$       \\
        $k_5 = t_2~(\text{free parameter})$                \\
    \end{tabular}
    \biitem{d}{$\dstyle \begin{bmatrix} 1 & -3 & 0 & 0 \\ 0 & 0 & 1 & 0 \\ 0 & 0 & 0 & 1 \end{bmatrix}$}
    $0k_1 + 0k_2 + 0k_3 = 1 \leftrightarrow 0 = 1$. No Solution.
\end{enumerate}

\end{document}