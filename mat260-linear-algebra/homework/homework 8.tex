\documentclass{article}
\usepackage[margin=1in]{geometry}
\usepackage{amsmath, amsthm, amssymb, fancyhdr, tikz, circuitikz, graphicx}
\usepackage{centernot, xcolor, hhline, multirow, listings, dashrule}
\usepackage{blkarray, booktabs, bigstrut, etoolbox, extarrows}
\usepackage[normalem]{ulem}
\usepackage{bookmark}
\usetikzlibrary{math}
\usetikzlibrary{fit}

\pagestyle{fancy}

\usepackage{hyperref}
\hypersetup{
  colorlinks=true,
  linkcolor=black,
  filecolor=magenta,
  urlcolor=cyan,
}
%formatting
\newcommand{\bld}{\textbf}
\newcommand{\itl}{\textit}
\newcommand{\uln}{\underline}

%math word symbols
\newcommand{\bb}{\mathbb}
\DeclareMathOperator{\tif}{~\text{if}~}
\DeclareMathOperator{\tand}{~\text{and}~}
\DeclareMathOperator{\tbut}{~\text{but}~}
\DeclareMathOperator{\tor}{~\text{or}~}
\DeclareMathOperator{\tsuchthat}{~\text{such that}~}
\DeclareMathOperator{\tsince}{~\text{since}~}
\DeclareMathOperator{\twhen}{~\text{when}~}
\DeclareMathOperator{\twhere}{~\text{where}~}
\DeclareMathOperator{\twith}{~\text{with}~}
\DeclareMathOperator{\tfor}{~\text{for}~}
\DeclareMathOperator{\tthen}{~\text{then}~}
\DeclareMathOperator{\tto}{~\text{to}~}
\DeclareMathOperator{\tin}{~\text{in}~}

%display shortcut
\DeclareMathOperator{\dstyle}{\displaystyle}
\DeclareMathOperator{\sstyle}{\scriptstyle}

%linear algebra
\DeclareMathOperator{\id}{\bld{id}}
\DeclareMathOperator{\vecspan}{\text{span}}
\DeclareMathOperator{\adj}{\text{adj}}

%discrete math - integer properties
\DeclareMathOperator{\tdiv}{\text{div}}
\DeclareMathOperator{\tmod}{\text{mod}}
\DeclareMathOperator{\lcm}{\text{lcm}}

%augmented matrix environment
\newenvironment{apmatrix}[2]{%
    \left(\begin{array}{@{~}*{#1}{c}|@{~}*{#2}{c}}
        }{
    \end{array}\right)
}
\newenvironment{abmatrix}[2]{%
    \left[\begin{array}{@{~}*{#1}{c}|@{~}*{#2}{c}}
            }{
        \end{array}\right]
}

\newenvironment{determinant}[1]{
    \left\lvert
    \begin{array}{@{~}*{#1}{c}}
        }{
    \end{array}
    \right\rvert
}

% graph theory
\DeclareMathOperator{\diam}{\text{diam}}
\newcommand{\comp}[1]{\overline{#1}}

% calculus iii
\newcommand{\vvec}{\overrightarrow}
\newcommand{\norm}[1]{\lVert #1 \rVert}

% abstract algebra
\newcommand{\struct}[1]{\langle #1 \rangle}

%lists
\newcommand{\bitem}[1]{\item[\bld{#1.}]}
\newcommand{\bbitem}[2]{\item[\bld{#1.}] \bld{#2}}
\newcommand{\biitem}[2]{\item[\bld{#1.}] \itl{#2}}
\newcommand{\iitem}[1]{\item[\itl{#1.}]}
\newcommand{\iiitem}[2]{\item[\itl{#1.}] \bld{#2}}
\newcommand{\btitem}[2]{\item[\bld{#1.}] \texttt{#2}}

%homework
\newcommand{\question}[2]{\noindent {\large\bld{#1}} #2 \qline}
\newcommand{\qitem}[3]{\item[\bld{#1.}] #2 \qdash \\ #3 \qdash}

\newcommand{\qline}{~\newline\noindent\textcolor[RGB]{200,200,200}{\rule[0.5ex]{\linewidth}{0.2pt}}}
\newcommand{\qdash}{~\newline\noindent\textcolor[RGB]{200,200,200}{\hdashrule[0.5ex]{\linewidth}{0.2pt}{2pt}}}

\newcommand{\assignment}{Homework 8}

\lhead{Linear Algebra}
\chead{\assignment}
\rhead{Peter Schaefer}

\begin{document}
\section*{\assignment}

\subsection*{2.1}

\question{3.}{Let
    \[
        A = \begin{bmatrix}
            4 & -1 & 1  & 6  \\
            0 & 0  & -3 & 3  \\
            4 & 1  & 0  & 14 \\
            4 & 1  & 3  & 2
        \end{bmatrix}
    \]
    Find the following:}
\begin{enumerate}
    \qitem{a}{$M_{13} \tand C_{13}$.}{
        \begin{proof}[Work]
            \begin{align*}
                M_{13} & = \left\lvert \begin{array}{ccc}
                                           0 & 0 & 3  \\
                                           4 & 1 & 14 \\
                                           4 & 1 & 2
                                       \end{array} \right\rvert = 3 \cdot (-1)^{1 + 3} \left\lvert \begin{array}{cc}
                                                                                                       4 & 1 \\
                                                                                                       4 & 1
                                                                                                   \end{array} \right\rvert = 3 \cdot 1 \cdot 0 = 0 \\
                C_{13} & = (-1)^{1+3} \cdot M_{13} = 1 \cdot 0 = 0
            \end{align*}
        \end{proof}
    }
    \qitem{b}{$M_{23} \tand C_{23}$.}{
        \begin{proof}[Work]
            \begin{align*}
                M_{23} & = \left\lvert \begin{array}{ccc}
                                           4 & -1 & 6  \\
                                           4 & 1  & 14 \\
                                           4 & 1  & 2
                                       \end{array} \right\rvert                                                                                                                                        \\
                       & = 4 \cdot (-1)^{1+1} \left\lvert \begin{array}{cc}
                                                              1 & 14 \\
                                                              1 & 2
                                                          \end{array} \right\rvert - 1 \cdot (-1)^{1+2} \left\lvert \begin{array}{cc}
                                                                                                                        4 & 14 \\
                                                                                                                        4 & 2
                                                                                                                    \end{array} \right\rvert + 6 \cdot (-1)^{1+3} \left\lvert \begin{array}{cc}
                                                                                                                                                                                  4 & 1 \\
                                                                                                                                                                                  4 & 1
                                                                                                                                                                              \end{array} \right\rvert \\
                       & = 4 \cdot -12 - (-1) \cdot (-48) + 6 \cdot 0                                                                                                                                  \\
                M_{23} & = -48 - 48 + 0 = -96                                                                                                                                                          \\
                C_{23} & = (-1)^{2+3} M_{23} = -1 \cdot -96 = 96
            \end{align*}
        \end{proof}
    }
\end{enumerate}

\question{11.}{Use the arrow technique to evaluate the determinant of the given matrix.
    \[
        \begin{bmatrix}
            -2 & 1 & 4  \\
            3  & 5 & -7 \\
            1  & 6 & 2
        \end{bmatrix}
    \]}
\begin{proof}[Work]
    \begin{align*}
        \left\lvert\begin{array}{ccc}
                       -2 & 1 & 4  \\
                       3  & 5 & -7 \\
                       1  & 6 & 2
                   \end{array} \right\rvert & = ((-2 \cdot 5 \cdot 2)+(1 \cdot -7 \cdot 1)+(4 \cdot 3 \cdot 6))-((-2 \cdot -7 \cdot 6)+(1 \cdot 3 \cdot 2)+(4 \cdot 5 \cdot 1)) \\
                                      & = ((-20)+(-7)+(72))-((84)+(6)+(20))                                                                                                     \\
                                      & = (45)-(110)                                                                                                                            \\
                                      & = -65
    \end{align*}
\end{proof}
\qdash

\question{18.}{Find all values of $\lambda$ for which $\det(A) = 0$.
    \[
        A = \begin{bmatrix}
            \lambda - 4 & 4       & 0         \\
            -1          & \lambda & 0         \\
            0           & 0       & \lambda-5
        \end{bmatrix}
    \]}
\begin{proof}[Work]
    \begin{align*}
        \left\lvert \begin{array}{ccc}
                        \lambda - 4 & 4       & 0         \\
                        -1          & \lambda & 0         \\
                        0           & 0       & \lambda-5
                    \end{array} \right\rvert & = (\lambda-5) \cdot (-1)^{3+3} \left\lvert \begin{array}{cc}
                                                                                              \lambda - 4 & 4       \\
                                                                                              -1          & \lambda
                                                                                          \end{array} \right\rvert = (\lambda-5) \cdot (\lambda(\lambda - 4) - (4 \cdot -1)) \\
                                             & = (\lambda-5) \cdot (\lambda^2 - 4\lambda + 4) = (\lambda-5) \cdot (\lambda-2)^2                                              \\ \\
        \lambda                              & = 5,~2
    \end{align*}
\end{proof}
\qdash

\question{21.}{Evaluate $\det(A)$ by a cofactor expansion along a row or column of your choice.
    \[
        A = \begin{bmatrix}
            -3 & 0 & 7 \\
            2  & 5 & 1 \\
            -1 & 0 & 5
        \end{bmatrix}
    \]}
\begin{proof}[Work]
    \begin{align*}
        \det(A) = 0 + 5 \cdot (-1)^{2+2} \left\lvert \begin{array}{cc}
                                                         -3 & 7 \\
                                                         -1 & 5
                                                     \end{array} \right\rvert + 0 = 5 \cdot 1 \cdot -8 = -40
    \end{align*}
\end{proof}
\qdash

\question{31.}{Evaluate the determinant of the given matrix by inspection.
    \[
        \begin{bmatrix}
            1 & 2 & 7  & -3 \\
            0 & 1 & -4 & 1  \\
            0 & 0 & 2  & 7  \\
            0 & 0 & 0  & 3
        \end{bmatrix}
    \]}
\begin{proof}[Work]
    \begin{align*}
        \left\lvert \begin{array}{cccc}
                        1 & 2 & 7  & -3 \\
                        0 & 1 & -4 & 1  \\
                        0 & 0 & 2  & 7  \\
                        0 & 0 & 0  & 3
                    \end{array} \right\rvert = 1 \cdot 1 \cdot 2 \cdot 3 = 6 &  & \text{via Theorem 1 of Lecture Notes 31}
    \end{align*}
\end{proof}
\qdash

\question{38.}{What is the maxiumum number of zeros that a $3 \times 3$ matrix can have without having a zero determinant? Explain your reasoning.}
\begin{proof}
    Consider $I = \begin{bmatrix}
            1 & 0 & 0 \\
            0 & 1 & 0 \\
            0 & 0 & 1
        \end{bmatrix}$.
    We know that $\det I = 1 \neq 0$. This means that 6 zeros can be achieved without having a zero determinant. \\
    If $A_{3 \times 3}$ has 7 zero entries, then $A$ has at most 2 non-zero entries. Since there are 3 rows, there is a row of all zeros. Expanding along this row to compute the determinant will produce an expansion with coefficients of all zeros, meaning that $\det A = 0$. Therefore, 6 zeros is the maximum number of zeros that a  $3 \times 3$ matrix can have without having a zero determinant.
\end{proof}
\qdash

\subsection*{2.2}

\question{14.}{Evaluate the determinant of the given matrix by reducing the matrix to row echelon form.
    \[
        \begin{bmatrix}
            1  & -2 & 3  & 1  \\
            5  & -9 & 6  & 3  \\
            -1 & 2  & -6 & -2 \\
            2  & 8  & 6  & 1
        \end{bmatrix}
    \]}
\begin{proof}[Work]
    \begin{align*}
         & \left\lvert \begin{array}{cccc}
                           1  & -2 & 3  & 1  \\
                           5  & -9 & 6  & 3  \\
                           -1 & 2  & -6 & -2 \\
                           2  & 8  & 6  & 1
                       \end{array} \right\rvert \overset{R_2 - 5R_1}{===}
        \left\lvert \begin{array}{cccc}
                        1  & -2 & 3  & 1  \\
                        0  & 1  & -9 & -2 \\
                        -1 & 2  & -6 & -2 \\
                        2  & 8  & 6  & 1
                    \end{array} \right\rvert \overset{R_3 + R_1}{===}
        \left\lvert \begin{array}{cccc}
                        1 & -2 & 3  & 1  \\
                        0 & 1  & -9 & -2 \\
                        0 & 0  & -3 & -1 \\
                        2 & 8  & 6  & 1
                    \end{array} \right\rvert \overset{R_4 - 2R_1}{===}                       \\
         & \left\lvert \begin{array}{cccc}
                           1 & -2 & 3  & 1  \\
                           0 & 1  & -9 & -2 \\
                           0 & 0  & -3 & -1 \\
                           0 & 12 & 0  & -1
                       \end{array} \right\rvert \overset{R_4 - 12R_2}{===}
        \left\lvert \begin{array}{cccc}
                        1 & -2 & 3   & 1  \\
                        0 & 1  & -9  & -2 \\
                        0 & 0  & -3  & -1 \\
                        0 & 0  & 108 & 23
                    \end{array} \right\rvert \overset{R_4 + 36R_2}{===}
        \left\lvert \begin{array}{cccc}
                        1 & -2 & 3  & 1   \\
                        0 & 1  & -9 & -2  \\
                        0 & 0  & -3 & -1  \\
                        0 & 0  & 0  & -13
                    \end{array} \right\rvert \overset{-\frac{1}{3}R_3}{===} -3               \\
         & \left\lvert \begin{array}{cccc}
                           1 & -2 & 3  & 1           \\
                           0 & 1  & -9 & -2          \\
                           0 & 0  & 1  & \frac{1}{3} \\
                           0 & 0  & 0  & -13
                       \end{array} \right\rvert \overset{-\frac{1}{13}R_4}{===} -13 \cdot -3
        \left\lvert \begin{array}{cccc}
                        1 & -2 & 3  & 1           \\
                        0 & 1  & -9 & -2          \\
                        0 & 0  & 1  & \frac{1}{3} \\
                        0 & 0  & 0  & 1
                    \end{array} \right\rvert = -13 \cdot -3 \cdot 1 = 39
    \end{align*}
\end{proof}
\qdash

\question{27.}{Evaluate the determinant given that
    \[
        \left\lvert \begin{array}{ccc}
            a & b & c \\
            d & e & f \\
            g & h & i
        \end{array} \right\rvert = -6.
    \]
    \[
        \left\lvert \begin{array}{ccc}
            -3a    & -3b    & -3c    \\
            d      & e      & f      \\
            g - 4d & h - 4e & i - 4f
        \end{array} \right\rvert
    \]}
\begin{proof}[Work]
    \begin{align*}
        \left\lvert\begin{array}{ccc}
                       -3a    & -3b    & -3c    \\
                       d      & e      & f      \\
                       g - 4d & h - 4e & i - 4f
                   \end{array} \right\rvert \overset{R_3 + 4R_2}{===}
        \left\lvert\begin{array}{ccc}
                       -3a & -3b & -3c \\
                       d   & e   & f   \\
                       g   & h   & i
                   \end{array} \right\rvert \overset{-\frac{1}{3}R_1}{===} -3
        \left\lvert\begin{array}{ccc}
                       a & b & c \\
                       d & e & f \\
                       g & h & i
                   \end{array} \right\rvert = -3 \cdot -6 = 18
    \end{align*}
\end{proof}
\qdash

\question{29.}{Use row reduction to show that ${\displaystyle \left\lvert\begin{array}{ccc}
                    1   & 1   & 1   \\
                    a   & b   & c   \\
                    a^2 & b^2 & c^2
                \end{array} \right\rvert = (b-a)(c-a)(c-b)}$}
\begin{proof}[Work]
    \begin{align*}
         & \left\lvert\begin{array}{ccc}
                          1   & 1   & 1   \\
                          a   & b   & c   \\
                          a^2 & b^2 & c^2
                      \end{array} \right\rvert \overset{R_2 - aR_1}{===}
        \left\lvert\begin{array}{ccc}
                       1   & 1   & 1   \\
                       0   & b-a & c-a \\
                       a^2 & b^2 & c^2
                   \end{array} \right\rvert \overset{R_3 - a^2R_1}{===}
        \left\lvert\begin{array}{ccc}
                       1 & 1       & 1       \\
                       0 & b-a     & c-a     \\
                       0 & b^2-a^2 & c^2-a^2
                   \end{array} \right\rvert \overset{R_3 - (b+a)R_2}{===}                         \\
         & \left\lvert\begin{array}{ccc}
                          1 & 1   & 1                    \\
                          0 & b-a & c-a                  \\
                          0 & 0   & c^2-a^2 - (b+a)(c-a)
                      \end{array} \right\rvert =
        \left\lvert\begin{array}{ccc}
                       1 & 1   & 1                       \\
                       0 & b-a & c-a                     \\
                       0 & 0   & (c+a)(c-a) - (b+a)(c-a)
                   \end{array} \right\rvert =                                              \\
         & \left\lvert\begin{array}{ccc}
                          1 & 1   & 1          \\
                          0 & b-a & c-a        \\
                          0 & 0   & (c-a)(c-b)
                      \end{array} \right\rvert = 1 \cdot (b-a) \cdot (c-a)(c-b) = (b-a)(c-a)(c-b)
    \end{align*}
\end{proof}
\qdash

\question{30.}{Confirm without evaluating the determinant directly:
    \[
        \left\lvert \begin{array}{ccc}
            a_1 + b_1t & a_2 + b_2t & a_3 + b_3t \\
            a_1t + b_1 & a_2t + b_2 & a_3t + b_3 \\
            c_1        & c_2        & c_3
        \end{array} \right\rvert = (1-t^2)
        \left\lvert \begin{array}{ccc}
            a_1 & a_2 & a_3 \\
            b_1 & b_2 & b_3 \\
            c_1 & c_2 & c_3
        \end{array} \right\rvert
    \]}
\begin{proof}[Work]
    \begin{align*}
         & \left\lvert \begin{array}{ccc}
                           a_1 + b_1t & a_2 + b_2t & a_3 + b_3t \\
                           a_1t + b_1 & a_2t + b_2 & a_3t + b_3 \\
                           c_1        & c_2        & c_3
                       \end{array} \right\rvert \overset{R_1 - R_2}{===}
        \left\lvert \begin{array}{ccc}
                        a_1 + b_1t - a_1t - b_1 & a_2 + b_2t - a_2t - b_2 & a_3 + b_3t - a_3t + b_3 \\
                        a_1t + b_1              & a_2t + b_2              & a_3t + b_3              \\
                        c_1                     & c_2                     & c_3
                    \end{array} \right\rvert =                   \\
         & \left\lvert \begin{array}{ccc}
                           a_1 - b_1 + b_1t - a_1t & a_2 - b_2 + b_2t - a_2t & a_3 + b_3 + b_3t - a_3t \\
                           a_1t + b_1              & a_2t + b_2              & a_3t + b_3              \\
                           c_1                     & c_2                     & c_3
                       \end{array} \right\rvert =                \\
         & \left\lvert \begin{array}{ccc}
                           -1(b_1 - a_1) + t(b_1 - a_1) & -1(b_2 - a_2) + t(b_2 - a_2) & -1(b_3 - a_3) + t(b_3 - a_3) \\
                           a_1t + b_1                   & a_2t + b_2                   & a_3t + b_3                   \\
                           c_1                          & c_2                          & c_3
                       \end{array} \right\rvert = \\
         & \left\lvert \begin{array}{ccc}
                           (t-1)(b_1 - a_1) & (t-1)(b_2 - a_2) & (t-1)(b_3 - a_3) \\
                           a_1t + b_1       & a_2t + b_2       & a_3t + b_3       \\
                           c_1              & c_2              & c_3
                       \end{array} \right\rvert =
        (t-1)
        \left\lvert \begin{array}{ccc}
                        b_1 - a_1  & b_2 - a_2  & b_3 - a_3  \\
                        a_1t + b_1 & a_2t + b_2 & a_3t + b_3 \\
                        c_1        & c_2        & c_3
                    \end{array} \right\rvert \overset{R_2 - R_1}{===}                                             \\
         & (t-1)
        \left\lvert \begin{array}{ccc}
                        b_1 - a_1              & b_2 - a_2              & b_3 - a_3              \\
                        a_1t + b_1 - b_1 + a_1 & a_2t + b_2 - b_2 + a_2 & a_3t + b_3 - b_3 + a_3 \\
                        c_1                    & c_2                    & c_3
                    \end{array} \right\rvert =                      \\
         & (t-1)
        \left\lvert \begin{array}{ccc}
                        b_1 - a_1  & b_2 - a_2  & b_3 - a_3  \\
                        a_1t + a_1 & a_2t + a_2 & a_3t + a_3 \\
                        c_1        & c_2        & c_3
                    \end{array} \right\rvert =
        (t-1)
        \left\lvert \begin{array}{ccc}
                        b_1 - a_1  & b_2 - a_2  & b_3 - a_3  \\
                        a_1(t + 1) & a_2(t + 1) & a_3(t + 1) \\
                        c_1        & c_2        & c_3
                    \end{array} \right\rvert =                                                          \\
         & (t-1)(t + 1)
        \left\lvert \begin{array}{ccc}
                        b_1 - a_1 & b_2 - a_2 & b_3 - a_3 \\
                        a_1       & a_2       & a_3       \\
                        c_1       & c_2       & c_3
                    \end{array} \right\rvert =
        (t^2-1)
        \left\lvert \begin{array}{ccc}
                        b_1 & b_2 & b_3 \\
                        a_1 & a_2 & a_3 \\
                        c_1 & c_2 & c_3
                    \end{array} \right\rvert =
        (1-t^2)
        \left\lvert \begin{array}{ccc}
                        a_1 & a_2 & a_3 \\
                        b_1 & b_2 & b_3 \\
                        c_1 & c_2 & c_3
                    \end{array} \right\rvert
    \end{align*}
\end{proof}
\qdash

\question{34.}{Find the determinant of the following matrix:
    \[
        \begin{bmatrix}
            a & b & b & b \\
            b & a & b & b \\
            b & b & a & b \\
            b & b & b & a
        \end{bmatrix}
    \]}
\begin{proof}[Work]
    \begin{align*}
        \left\lvert \begin{array}{cccc}
                        a & b & b & b \\
                        b & a & b & b \\
                        b & b & a & b \\
                        b & b & b & a
                    \end{array} \right\rvert = a^4 - 3b^4 -6a^2b^2 + 8ab^3
    \end{align*}
    For full derivation, see attached sheet.
\end{proof}
\qdash

\subsection*{2.3}

\question{19.}{Decide whether the matrix is invertible, and if so, use the adjoint method to find its inverse.
    \[
        A = \begin{bmatrix}
            2  & 5  & 5 \\
            -1 & -1 & 0 \\
            2  & 4  & 3
        \end{bmatrix}
    \]}
\begin{proof}[Work]
    \begin{align*}
        \det(A) & = 2 \cdot (1) \cdot (-1 \cdot 3 - 0 \cdot 4) + (-1) \cdot (-1) \cdot (5 \cdot 3 - 5 \cdot 4) + 2 \cdot (1) \cdot (5 \cdot 0 - 5 \cdot (-1)) \\
                & = 2(-3) + 1(-5) + 2(10) = -6 -5 + 10 = -1
    \end{align*}
    Since $\det(A) \neq 0$, by the Big Theorem, $A$ is invertible. To find the inverse via the adjoint method, first find the cofactors of A.
    \begin{align*}
        C_{11} & = (-3 - 0) = -3 & C_{12} & = -(-3-0) = 3 & C_{13} & = (-4+2) = -2 \\
        C_{21} & = -(0-5) = 5    & C_{22} & = (6-10) = -4 & C_{23} & = -(8-10) = 2 \\
        C_{31} & = (0+5) = 5     & C_{32} & = -(0+5) = -5 & C_{33} & = (-2+3) = 3
    \end{align*}
    According to Theorem 9 of Lecture Notes 32,
    \begin{align*}
        A^{-1} & = \frac{1}{\det(A)}\text{adj}(A) \\
        A^{-1} & = -1 \cdot \begin{bmatrix}
                                -3 & 3  & -2 \\
                                5  & -4 & 2  \\
                                5  & -5 & 3
                            \end{bmatrix}^T       \\
        A^{-1} & = -1 \cdot \begin{bmatrix}
                                -3 & 5  & 5  \\
                                3  & -4 & -5 \\
                                -2 & 2  & 3
                            \end{bmatrix}        \\
        A^{-1} & = \begin{bmatrix}
                       3  & -5 & -5 \\
                       -3 & 4  & 5  \\
                       2  & -2 & -3
                   \end{bmatrix}
    \end{align*}
\end{proof}
\qdash

\question{27.}{Solve by Cramer's rule, where it applies:
    \begin{align*}
        1x_1 - 3x_2 + 1x_3 & = 4  \\
        2x_1 - 1x_2 + 0x_3 & = -2 \\
        4x_1 + 0x_2 - 3x_3 & = 0
    \end{align*}}
\begin{proof}[Work]
    \begin{align*}
        \begin{bmatrix}
            1 & -3 & 1  \\
            2 & -1 & 0  \\
            4 & 0  & -3
        \end{bmatrix}
        \begin{bmatrix}
            x_1 \\ x_2 \\ x_3
        \end{bmatrix} =
        \begin{bmatrix}
            4 \\ -2 \\ 0
        \end{bmatrix}
    \end{align*}
    According to Cramer's rule, $x_1 = \frac{\det A_1}{\det A}, x_2 = \frac{\det A_2}{\det A}, \tand x_3 = \frac{\det A_3}{A}$, where $A_i$ is the matrix obtained by replacing the $i$-th column of $A$ by $\vec{b}$.
    \begin{align*}
        x_1 & = \frac{\det A_1}{\det A} = \frac{30}{-11} = -\frac{30}{11} \\
        x_2 & = \frac{\det A_2}{\det A} = \frac{38}{-11} = -\frac{38}{11} \\
        x_3 & = \frac{\det A_3}{\det A} = \frac{-40}{-11} = \frac{40}{11}
    \end{align*}
\end{proof}
\qdash

\question{33.}{Prove that if $\det(A) = 1$ and all the entries in $A$ are integers, then all the entries in $A^{-1}$ are integers.}
\begin{proof}
    Consider matrix $A$ where $\det(A) = 1$ and all the entries in $A$ are integers. Lets look at the adjoint formula for the inverse.
    \[
        A^{-1} = \frac{1}{\det(A)}\text{adj}(A)
    \]
    We know that $\det(A) = 1$, so $\frac{1}{\det(A)}$ is also 1. Since 1 is an integer, if all of entries of the adjacency matrix for $A$ are integers, then all of the entries for $A^{-1}$ will be integers. Lets look at the adjoint matrix formula.
    \[
        \text{adj}(A) = \begin{bmatrix}
            C_{11} & C_{12} & \cdots & C_{1n} \\
            C_{21} & C_{22} & \cdots & C_{2n} \\
            \vdots & \vdots & \ddots & \vdots \\
            C_{n1} & C_{n2} & \cdots & C_{nn}
        \end{bmatrix}^T
    \]
    If all of these cofactor entries of $\text{adj}(A)$ are integers, then all of the entries of $A^{-1}$ will be integers. Consider how these cofactor entries are calculated, and which operations they use. A quick inspection reveals that they use only the following operations; $+,-, \tand \cdot$. All of these operations are operations that keep integers as integers. This means that if all of the entries of $A$ are integers, then all of the cofactors of the adjacency matrix for $A$ will also be integers. Finally, this implies that all of the entries of $A^{-1}$ will also be integers, so long as $\det(A) = 1$.
\end{proof}
\qdash

\end{document}